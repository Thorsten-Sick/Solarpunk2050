\section{Norms}

Norms life in upgraded cities with added solar panels, green parks, small streams of water and lakes. Everything is quite idyllic but you can still see the original buildings from the 2020 under the green walls.
The society is coordinated by AIs (one per town) and the goal of the AI is to improve quality of life.
To measure that every Norm carries a life logger which is a communication device like a mobile phone with AR interface. This is also the main tools to do their magic.

Someone from 2020 would call them wizards. Using Apps on their life logger they can request services from the AI. From refreshments to building a house. Based on available templates the AI will coordinate factories and Norm contractors (very spezialised ones) to build the customized request.
Delivery is within minutes (when ordering a gun with pearl-style grip) to days (adding a house to the city).

This is not limited to physical things but can also be an App to organize a Party.

But there are limits:

\begin{itemize}
    \item Things cost money. And material things will additionally cost resource points
    \item Apps must be unlocked first. Most of them by succeeding in a tutorial. This can take up to 2 years for an architecture App
    \item This magic will only work in the control area of an AI. And the supply lines must be open. The App will indicate if this service is not available
    \item Some services will also require a health check or other additional certificates
    \item The general rank of the user must be high enough
\end{itemize}

This is why most Norms focus on 1-2 topics and the usual 40 Apps everyone uses to simplify life.

A visiting Solarpunk can get a life logger but will start on rank 0 (guest). Having only the basic Apps available (pizza service, ....) and no clue how to use them.

The real stars in the Norm world are those who can create new templates for new services. And amongst them only some geniuses can generate templates for highly complex products (like trains).
Those templates and their Apps unlock new wonders for the society.

Appearance: The norm world is (thanks to the templates) very uniform compared to the world of the Solarpunks or the Lost. Everything is mass produced with some added personalisation. That includes the solar upgrades to houses and the clothes.
Chaos in the towns is created by the plants and animals that were allowed to spread.

Mobility: Mass transport. But with added quality of life (the self driving tram has a bar and a big TV screen for entertainment. People are standing in groups chatting and enjoying the ride.)

\begin{warning}
    The details below are not up to date with the new concept - which makes them playable. It will be updated after playtesting and discussion.
\end{warning}

\subsection{Profile}

\begin{itemize}
    \item Class: Job promotion based on age
    \item Advancement in Society: Automatic
    \item Political System: Ruled by AI
    \item Form of society: city-states
    \item Conflicts: Deadly Ennui
    \item Appearance: off-the-shelf clothes, clothing is not exchanged fast fashion moderately quickly for ecological reasons. But there are many accessories with which one can present one's belonging and fashion knowledge. Often adapted to the current favorite star
    \item Taboos: Not conforming, not familiar with current pop culture
    \item Language: Flowery with many references to series
    \item Tendencies: Conformist. consumers
    \item Law: * Judge: The AI * Penalties: Deprivation of some privileges up to drug reprogramming
    \item Corruption: There is still something like that among the last remnants of the factory owners. Otherwise the society is happy.
    \item Weapons: None. Law enforcement officers have them, from tasers to simple firearms. But the crime rate is low. Attacks by the Lost are met with (late arriving) military
    \item Architecture: Ecologically converted cities with high-rise buildings. New construction areas with identical houses
    \item Vehicles: Technologically very modern, battery or hydrogen powered. Easy to use (especially in cities): self-driving e-cars, public transport
    \item Resources: There is a lack of creative and new things. The media industry always tries to use templates creating something new and often fails. On the other hand, it must not be too disruptive either
    \item Celebrations: Big parties and clubs. Pomp. You sometimes start thinking a week in advance about how to wants to be “individual” and stand out. . . .
    \item Success: There are few who stand out from the crowd. Often there are artificially constructed stars from series or music
    \item Drugs: To numb and forget the world. Gladly longer. Sick leave for drugs After effects are normal. Drugs are cheap in stores.
    \item Psychotherapy: Is planned from birth. Therapy places are available. If the AI detects a problem, the seat is assigned. It's not a big deal in society when someone goes into therapy .
    \item Media: Media production and entertainment is central to culture. A lot is consumed and perfection in the quality of the media is immensely important. News and information is there, but simplified enough for a 14 year old to understand.
    \item Education: Norms learn through Apps. Most Apps have a tutorial. There are educational Apps as well.
    \item Coordination: Hierarchical. Very classic company bosses and supervisors
    \item Nutrition: Highly processed branded food from corporations. Optimized for mass taste. Gladly too Eating out at chain restaurants or in the canteen
    \item Names: International fashion names. But rewritten to local syntax. To make it simpler to pronounce them. Tscharlien, Mischelle, Kewin
%    \item Gender: sometimes yes, sometimes no
\end{itemize}

They live in a golden cage. The door is open and hardly anyone wants to leave it.
Make up about 80\% of the population. They have done the minimum necessary to adapt to 2050. You live  a life as similar as the one before. However, they also do not benefit greatly from modern technology and society. Often consumers of industrial products. Are CO2 neutral because the structures around them ensure this. But spend disproportionate money by paying the structures for it. Their environment was configured by algorithms (aka AI) and optimized for the highest possible quality of life, which also leads to distortions. They work 25 hours a week (it has been found that this promotes happiness). Career jumps are about every 2 years and you hardly have to do anything for them. there is enough money. TV series are knitted from templates that A/B tests have determined that as many people as possible like them. Advertising steers consumption in the right direction. Life expectancy is about 90 years.

Where a kink in old age is the midlife crisis:
None of them are used to challenges. Everything is simulated. But at that age some want to test themselves again and then die from accidents in extreme sports. Solarpunks are a bit irritated. Mainly because their lifestyle is more adventurous and those freak accidents don't happen there.
Norms practically live in a well organized amusement park. The high quality of life requirements also ensure clear guidelines for the AI: playground instead of parking lot, drugs are ok - allowed and desired (with norms these are prescribed psychotropic drugs). . . Monitoring is permanent and
necessary. Not for safety, but to feed the AI. The norms are fine with that. And they always have their life loggers with them. Voluntarily.


\subsection{Education}
Frontal teaching is still established at Norms. Since the epidemics in the 1920s, however, digital media have also been used. Companies offer further training for employees, but this is strongly geared to the needs of the workplace.
Educational goals are clearly defined and quantifiable. Voluntary further training courses tend to be the exception. Many Norm parents have a problem with their children going to project activities at Solarpunks in addition to normal school lessons. But that doesn't stop all children. Which often leads to problems and accusations of kidnapping.

\subsection{Conflicts: Deadly Ennui}
By middle age, some norms realize that life doesn't hold too many surprises in store for them.
This causes a spike in death statistics between the ages of 40 and 60. Reasons for "death from midlife crisis" are:
\begin{itemize}
    \item suicides
    \item Taking up an extreme sport
    \item Be bold with excuses like "I saw all of that in Survival Shad's adventures," "The Solarpunks do that too", "doesn't look that difficult" Some survive. The slightly more flexible and intelligent norms also like to migrate to the solarpunks, where they are more than welcome (as long as they can adapt). This creates a brain drain that only makes the norms more stereotypical.
\end{itemize}

\subsection{Work and Education}
Labor is highly specialized. Norms learn their job for life and are then cogs in a complex system. Her level in her job is equal to the City Culture skill since the job is basically done in the city using the AI.
\begin{normtalk}[title=An example from the life of a Level 3 Architect.]
    \begin{itemize}
        \item \npcquote{Architect}{(Drags three apartment modules onto the grassy area in AR) Ok, what color should the light switches have?}
        \item \npcquote{Customer}{White}
        \item \npcquote{Architect}{(Changes setting). Finished. The AI checked the construction and calculated the costs. Euro and raw material costs are displayed, please confirm.}
        \item \npcquote{Customer}{(Confirmed)}
        \item \npcquote{Architect}{The AI organizes the construction and the 80 craft companies. The excavators will arrive in 1 hour, followed by the glass fiber reinforced concrete. Completion in 5 days by the carpet fitters and light switch fitters. Housewarming is included and invitations were just sent.}
    \end{itemize}
\end{normtalk}

\subsection{Police}
The police methods of the future are app based. The police and all potential helpers can send information, evidence and interrogations to a central police server via an app. There, an AI evaluates the data and shows live who is the most suspicious. The investigative activity is thus limited to collecting facts. And the players characters have the opportunity to contribute to the investigation. Of course, all other suspects are also involved and informed. This creates the potential for manipulation. Another problem with such algorithms are anomalies of all kinds. Things that are not common or unusual have a large effect on the verdict.

\subsection{training}

Since the player characters are strangers to the norm world and do not accompany a high rank in the app (probably just rank 0: guest) they are very limited in a Norm world. "Sorry, without a medical examination stored in your app you can not set the sauna temperature above 60°", "Sorry, you did not finish the tutorial"...