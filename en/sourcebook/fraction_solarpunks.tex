\section{Solarpunks}

The Solarpunks have taken the future into their own hands. They saved everyone and the planet and now draw their strength from this.
Hard-core Solarpunks make up only a small fraction of the population (and that can vary locally). But since they actively participate in life, they are also the part that leaves their mark on society.

Social: Sexually open (queer, trans, . . . ) and sometimes in polyamorous relationships. an entire community often identifies as one family and cares for the children (“It takes a village to raise children”).
The major disadvantage of this freedom, however, is the tension with more conservative sections of the population. Norms and Lost. Solarpunks generally use the gender asterisk and the person's preferred personal pronouns. Many Solarpunk communities are accessible as agrotourism for Norms and others to "experience" something.
Drugs: Light drugs are legal across the EU. Solarpunks like to mix their own experimental stuff or grow mushrooms or hemp at great expense. Good drug experts are celebrated like artists.
Redemption: Some more subversive solarpunk groups feel they need to do the Norms a favor by tearing them out of the rut of normality. To do this, they plan subversive art actions, which they carry out at night and in fog to "open your eyes". The reaction of the norms and the AI (which wants to restore normality) is mixed and very unpredictable.

\subsection{Profile}
\begin{itemize}
    \item Class: Meritocracy.
    \item Advancement in society: Successful and creative projects and campaigns
    \item Political System: Holocracy
    \item Form of society: extended family
    \item Conflicts: meritocracy bugs, urge for change
    \item Appearance: Self-tailored or fitted clothing. Very individual. Value is placed on practicability during projects.
    \item Taboos: idleness, stealing glory
    \item Language: Nerdy youth slang prevails
    \item Trends: progressive, creative, builder
    \item Law: * Judge: The group * Penalties: Community service to banishment from all Solarpunk communities
    \item Corruption: Hardly. If so, then your own project may be supported with resources.
    \item Weapons: Elegant weapons are preferred. Because it is practiced as a sport: swords, bows (high-tech) and firearms, . . .
    \item Architecture: creative fusion with nature and technology. Community oriented. Very individual.
    \item Vehicles: Technologically very modern and digital, battery or hydrogen powered. Lightweight and small. E-bikes, e-cargo bikes, quads, exoskeletons
    \item Resources: There is a lack of large-scale industry goods. You're always chasing technology that you can recycle because you don't want to buy anything new (for ideological reasons)
    \item Celebrations: creative and rather small. With art and culture. In meeting houses or in the central square. Camp fires. Stories. Presentations of new projects. Theatre. Homemade music. Drone ballet in the night sky.
    \item Success: inventors of crazy technology, people who have performed a risky action
    \item Drugs: Used to be creative, happy and open. Enhancing ones. To boom and long-lasting influence is frowned upon. Because it paralyzes projects. Experimentation with intelligently designed and bred drugs is welcome.
    \item Psychotherapy: There are many educated, wise and pastoral people. Therapy is more of a sideline ongoing mentoring thing and no one really notices.
    \item Media: There are many very topic-specific podcasts/video casts and the like. Each created by our own experts. Lectures are also held and recorded. An immense knowledge archive is available. Art and culture are also recorded and distributed decentrally by communities.
    \item Education: flipped classroom, hands-on learning. No age limit for courses, but at best required prior knowledge as a limitation. Prior knowledge is managed using a badge system: https://support.mozilla.org/en/products/open-badges/introduction-open-badges .Teachers are recognized for their explanatory skills in their field. But not necessarily full-time teachers.
    \item Coordination: angel system. Tasks are entered into a system. If someone has time, they call up the tasks and sort them according to their skills/interests. Then do it and get points https://engelsystem.en/ An order in the Engel system can also be the practical part of a further training measure and be rewarded with a badge.
    \item Diet: Homegrown meat, vegetables. Creative cuisine is very welcome. And good Chefs can become stars.
    \item Names: nicknames, abbreviations, chat names, names referring to peculiarities. The names can often change in the course of life and are usually chosen by the person himself. Or earned
    % \item Gendern: With the -y ending and 'das' (after Phettberg)
\end{itemize}

\subsection{Policy}
Communities are how Solarpunks organize themselves democratically. These are self-governing structures that, at best, form a kind of communal village. But locally fragmented communities also exist and thrive. The form of democracy is a holocracy (https://en.wikipedia.org/wiki/Holocracy). Everything is organized in circles. These are linked hierarchically. The lowest level is a single project. After that comes the community, the local cluster of communities, the EU
representation of communities and then the global circle.
The members of the circles are democratically elected. In addition, one elected member is sent to the neighboring circles. That way everyone is connected. Another important principle: The circles do NOT try to find the best solution to a problem, but rather prefer the most easily correctable variant. This allows more experiments to be tackled.
A community lives according to self-determined rules. The standard Solarpunk rules are their basis.
The basic set of rules for communities:
\begin{enumerate}
    \item Be excellent to each other!
    \item The community is democratic. Elections are held once a year
    \item These positions are to be filled: * Elder * Dispute mediator * Logistics officer * Quartermaster
    \item Nobody stays in the same office for more than 3 years
    \item The aim must be to fill the gender parity
    \item Refugees must be helped.
    \item The community must live and operate within natural limits. She must help others to do the same.
    \item Protective equipment must be worn in dangerous situations
    \item No biological experiments in the kitchen area!
    \item Who makes is right
    \item A competition between communities is a matter of honor
    \item Friday is pancake day
\end{enumerate}

The own community in which the characters live is itself a protagonist and is welcome to receive a character sheet. Growing it and making it more connected is a potential goal for players. But you also get direct benefits from new skills/ equipment from the community.
In addition, there are also regional customs between communities. Are known:
\begin{itemize}
    \item Sharing your own projects as a sign of trust. Two communities in the Black Forest exchanged the sourdoughs they had cultivated and optimized over the years. At first glance, this might seem banal. But they gave the other group years of work and a cultural uniqueness of their own. And in a reproducible form.
    \item Plant seeds as a welcome gift. Here, of course, attention is paid to special features and quality. 
\end{itemize}
Of course, this is also rooted in the "grow and let grow" of the solar punks.
For GM: These rules intentionally have some ambiguity. The story can then be hung up with that.
\begin{enumerate}
    \item Who is eligible to vote? Can you sabotage the election?
    \item What exactly do the offices do? Can one person hold multiple offices?
    \item And what if no good successor is found?
    \item How much leeway is ? How many genders are there?
    \item Fled from what ? The law ? How to help ? Also refugees from hostile groups?
    \item Purely theoretically: If you burn down a nasty industrial plant in self-defense, you have to compensate for the CO2. And whether others accept this help. . . 
    \item Protective equipment in the workshop is good. Here, however, it was forgotten that one must also be able to handle the devices. Especially with those that have been tweaked and customized by the community. . . 
    \item  Self-explanatory. But what if you have bred new brewer's yeast? When are they no longer experimental? 
    \item When someone proves that something can be done. Is he right? . . but can cause problems with very enthusiastic slobs can who challenge can't listen others
    \item to Communities compete. The winner gets clear reputation. The loser can also get points for a Grand Commendation to the winners. Both are archived in annals. The exact form of the competition is defined between the communities. => Start of many adventures
\end{enumerate}

\subsection{Holacracy}
Holacracy is actually a simple democracy with voting in small groups (that can be project groups, teams, communities). Since each group is networked with others, they send a kind of diplomat. This is why someone from the community leadership sits in the project group. Information can be exchanged quickly. And everyone's interests are represented. But that can also lead to problems.

\begin{solartalk}[]
    \begin{itemize}
        \item \npcquote{Wheels}{I see the Lost camping out in the woods down there. They also have the standard hostages with them. I'll show you the map right away. Ask your circles what they think of our “get in and get out” plan.}
        \item \npcquote{Gemstone}{The ecos say it's breeding season. I'll add the nests to the map. No fights within 20 meters, they say.}
        \item \npcquote{Net}{The Norms are currently running an adventure series. The culture exchange circle says we can expect more positive reactions if we conform to their script when performing.}
        \item \npcquote{Les}{Man-at-Arms wants to know how his net launcher works. Could we film that?}
    \end{itemize}
\end{solartalk}

. . It is also relevant for a holocracy that no attempt is made to find the optimal solution to a problem.
But one that you can easily changed should it prove wrong.

\subsection{Meritocracy}
Advancement among the Solarpunks is achieved through successfully completed projects. This meritocracy is a hierarchy based on recognition.

Bugs in Meritocracy:

\begin{itemize}
    \item Of course, newcomers can't not have many achievements to show for themselves
    \item People with a lack of ability (or a disability) stay on the lowest rungs - as long as the society does not recognize the effort, regardless of the success
    \item Fame can be stolen, foreign projects appropriated
    \item Different communities have different focuses (arts, technology or plants) and thus find it difficult to assess foreign work
\end{itemize}

\subsection{Education}
At Solarpunks learning is freer. Lifelong learning is the order of the day. Knowledge is exchanged between people and communities. Nomad teachers travel through the country every several months and teach interested people (children and adults) new things. Communities offer courses in their specialization (“Hydrogen synthesis using algae, 4 weekends”, “Brewing beer, original ancient Egyptian recipe. From a historian/beer brewer”). There is no clear educational plan. Knowledge and skills are highly valued. Norms are also welcome. But they rarely take advantage of this offer.

\subsection{Relationships}
Diversity is the norm for Solarpunks.
Relationships are common in all varieties, especially among Solarpunks. From hetero-monogamous to polyamorous. The children's parents feel responsible for their upbringing, even if there are 5 parents. In many communities, however, it is such a common custom that everyone is responsible for the upbringing of the children that irritated children have to be asked several times “Who are your REAL parents?” The individual determines their own gender identity. Anything else
would be weird.
Note the difference between "Frequent" and "Normal". Solarpunks are no more gay than the general population. But it's totally normal for them to show it openly. That's why it can appear to outsiders who are lost that there are disproportionately homosexuals here. Simply because of their freedom.
Sometimes, though, a non-heterosexual Lost may join the Solarpunks just to be themselves. The person will still find it very difficult to adapt to the other culture.

\subsection{Die Walz}
When Solarpunks get stuck in their personal development, they take to the road. Similar to how journeymen used to be craftsmen (In Germany "Die Walz"). However, there are some changes to the customs of the time:

\begin{itemize}
    \item You can do this as a group
    \item The duration is flexible
    \item You can go anywhere - and back again
    \item Normally, Solarpunks do this several times in their lives. As soon as they feel like it They travel to distant communities. Get to know new technologies and ways of living there. However, the rules for traveling Solarpunks are:
    \item The travelers are to be received in a friendly manner and included in the group
    \item In return, the travelers help with any problems that arise
    \item Knowledge and experience are exchanged in both directions
    \item There is a big party to welcome and say goodbye
\end{itemize}

As you can quickly see: For player characters, this is a great way to get to new adventure locations or to introduce new characters to the group.

\subsection{Solar Nomads}
Solarpunks But mobile. Are constantly traveling between cities. In their e-caravans. With survival gear. They are usually the first to help in disasters. They are important to the Solarpunk communities as they exchange culture and knowledge between them.


\subsection{Children}
Children are allowed to participate in all safe activities within their ability. Sometimes they are assigned a mentor. Alternatively, they can also help in the control center (coordinate and process tickets there) or be deployed with drones (recycling, first aid, reconnaissance, . . . ). The drones here can be moving, flying or swimming.
Children also always have a voice when voting on future topics. Because it's about their future.
In the case of critical votes, they can even block with a veto. This is done by the so-called children's council.

\subsection{Furries}
Furries are often a normal part of a solarpunk community. People in anthropomorphic animal costumes (called fursuits). The people in the fursuits are often known to others as the "Fursona" only^.
Already in 2020 there were impressive costumes. But technical progress has greatly improved the ability to express oneself. Animated eyes, ears and tails as the focal point.
With enough technical progress, the suits are not only a hindrance (heavy, hot, little overview) but have also become an advantage. Monitors in the head provide all-around and wider range of vision, power and dexterity enhancers provide superhuman movement and strength. Dragon scales for armor. Metal claws. And then there's the wacky ones with the dragon breath. . . .
All of this in a fluffy costume (except for dragon or insect furries - they're not fluffy).
Judgments and prejudices about various Sonas: 

\begin{itemize}
    \item Dragons: Egocentric and overconfident, likes to collect stuff. Dragons come in powerful and big or small and derpy.
    \item Foxes: Of the loose variety (see Zootopia). Gladly horny too.
    \item Huskies: Foxes, but with more drama
    \item Dogs: Loyal and helpful
    \item Wolf: More mature than dogs
    \item Cats (domestic cat, tiger, puma): Solitary, self-determined, only do what they want. If you force them to do something, you only get minimal effort. Rather reserved and like to tinker with their projects alone.
\end{itemize}

In Solarpunk communities, furries like to build their own neighborhoods. If only for practical reasons (door handles are installed instead of knobs and the like). But also for aesthetic reasons.
Many furries do not take their heads off in public. Thus, their human identity is often unknown outside of a narrow circle.