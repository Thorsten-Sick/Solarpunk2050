% Checked by Editor
\section{Lost}
\label{sec:Lost}
\index{Lost|(}

The Lost are the faction that did not join the others on the "Dirty Road to Eden". They realised early on that there would be some horrors involved in achieving a Solarpunk utopia. And they did not want to pay the price. This faction is the only one that does not actively deny what has happened.

Their approach is to live in and from the ruins of the Lemmings' civilisation. They reuse, recycle and upcycle the old technology. This is what their clothes, tools, vehicles and shelters are made of. Old technology reused in a creative way. Most of it looks shaggy.

Living among the ruins they have developed fighting and survival skills and are experts in exploration.
Many are historians, collecting ancient documents for their secret "Alexandria" project.

Daily life: The Lost are constantly struggling to survive. Either by gathering food in the wilderness, in the ruins or by repairing old technology. Some tend to farms. Others run shops or try to keep truck stops alive to provide some sort of safe travel through the wilderness. Their internal trading system is a \hyperref[sec:Barter]{barter} system, so they trade and items for other items, food, information or diesel.
The Lost gain Resource Points like everyone else. But they tend to spend them on the privilege of burning fuel. Therefore, they are forced to search for resources in the ruins to upcycle or recycle (for Resource Points).

Their reliance on old technology makes them the only faction without drones. They are also the only faction with trained animals.

\begin{itemize}
    \item Dogs for searching, rescuing, hunting and pulling carts
    \item Horses for riding and transport
    \item Rats for sniffing for substances, testing food and rescues
    \item Birds (eagles, falcons, owls for hunting or scouting
    \item Dolphins and seals.
\end{itemize}

Some of the Lost almost developed an almost mystical relationship with their animals. At least according to their Norm friends who grew up in towns.

The Lost do not necessarily trust the Pioneers or the Norms. Because they still remember the "Dirty Road to Eden" pretty well.

\subsection{Shakespeare battle}
\index{Shakespeare battle}
\label{sec:Shakespeare battle}

Despite the rough first and second impression they make on outsiders Lost are educated and value history and old books a lot. The battle consists of one person starting to play a scene from an old theater play and handing over to another person who will have to continue without failing. This is a drinking game and will turn into a chaotic situation.

(If player characters are part of the battle-crowd. GM: Bring a copy of a page of a Shakespeare play so it can be played at the table).

\subsection{Skill: Bushcraft}
\index{Bushcraft}
\label{sec:Bushcraft skill}

The Lost know their way around the wilderness and ruins. They can up-cycle and recycle old technology. They build camps, hunt, gather food, train animals and scavenge ruins for tools.

This is all covered by the skill Bushcraft skill.

\subsection{Typical Lost protagonists}

\begin{itemize}
    \item Scrap collectors
    \item Indiana Jones style adventurers
    \item Lost ranger and animal expert
    \item Survival specialist
    \item Librarian
\end{itemize}

\subsection{Traditions}

\subsubsection{Guests and weapons}
\label{sec:Lost guests and weapons}
As survival experts helping people and providing shelter is a sacred tradition. People who come in good faith will be given shelter, food and medical care (Lost medicine is mostly rustic first aid).

But before entering a Lost camp guests are searched for weapons. If they do not have any they are given some simple ones to use ones to use for the duration of their visit. They can also receive basic weapons training. Lost expect guests to help defend their camp if it is attacked by critters or other humans. This is why guests are armed.

If the guests are not trustworthy, they will not receive the ammo for their weapon - until their help is needed to defend the camp.

\subsubsection{Libraries}

The first impression other factions have of the Lost when they first meet them is of an uneducated bunch of survivalists and fighters. Once they get to know some of the Lost they learn of their true mission: to protect the past. They collect old books (from phone books to the Luther Bible) and art (including every day objects to old paintings).
These are kept in the mobile library that most Lost families carry with them (a steel box in the boot or a special armoured bus).

Books that are not yet in Alexandria are sent there on a pilgrimage. This is a paper chase adventure with missions at each stop. The official goal is spiritual experience and growth, which may lead to some of the participants becoming Librarians of Alexandria. The unofficial goal is to hide the current location of Alexandria from anyone following the group.
Groups travelling to Alexandria will be secretly monitored by experienced Librarians. Anyone following these groups will be dealt with.

For most of the year, Librarians live with a family. Only when Alexandria is formed will they be called and leave the family for a few months.

Alexandria is a library built out of shipping containers. To protect it, it only exists for a few weeks a year. Its location is a secret known only to the Librarians. Pilgrims are led there by the Paper Chase. The containers contain the library's books and index.

When a group of pilgrims arrives, they hand over the book, which is then indexed. Skilled forgers will make exact duplicates, which will be sealed with the Library's seal and returned to the pilgrims. Each pilgrim can also choose a book from the library. One copy will be mass-produced (using an ancient printing press) and sent back with the pilgrims to their family library.
In this way, knowledge is spread and a decentralised back-up is established.

All the Lost know of the Library. But only a few outsiders - those who have saved the life of a Lost and found a book not yet in Alexandria - are asked to join a group of pilgrims under a vow of secrecy.

\subsubsection{Librarians}

A pilgrim can apply to become a Librarian. As well as being able to quote 50 books, a Librarian must be able to demonstrate special skills that the library needs. These range from forger, fighter (to protect the library), treasure hunter, builder (to rebuild the library once a year) and cook (to support the pilgrims' gathering).
Part of the initiation rite is to tattoo as much text as possible from the new librarian's favourite book on his or her body. People who know (this is mostly Lost) can recognise a Librarian by the letters tattooed on their skin.
Skin that cannot be hidden will be kept free of text.

\subsection{Economy: Barter}
\label{sec:Barter}
\index{Barter}
The Lost spend most of their resource points on diesel. This is one of the reasons why they loot old ruins for materials and technology that they can upcycle. This way they do not have to spend RPs on products.

They also have an internal barter system where they trade goods for goods (a chicken for this hammer...).
It is much more complicated than money, but they lack trust in other systems and it is transparent. So it works for them. And most of the time they trade within the same family anyway.
And you can use the goods while you wait to trade them (eggs from the chicken).

\subsection{Food}
\label{sec: lost food}
\index{Food!Lost}

Lost are hunters and gatherers. They cook whatever they find. Ingredients are animals, vegetables, herbs and cans from raided ruins. The kitchen is a camp fire and some tarpaulins around it to collect the gathered food.

A good Lost chef is good at improvising. The same dish will never taste exactly the same because the ingredients will vary. The cook's goal is to get as many nutritients into people as possible. Starting with the essential ones like fat and other calories.

Cooking is mostly done for the whole camp including guests. Those could also be recruited by the chef for simple tasks.

A very typical Lost dish is "Exhaust bread" which is baked ont he exhaust pipe of a moving truck.

\subsection{Diesel}

They are burning Diesel in their quest for independence and living the "good old times". For that they spend all their Resource Points that represent the resources a person can use without harming the environment. But spending those points also restricts them to depending on looted and recycled material. They can not afford anything new. And they do not want to.
The other factions know that and don't see the social contract broken. But they just don't understand that decission.

% TODO: Add example stunts
\index{Lost|)}