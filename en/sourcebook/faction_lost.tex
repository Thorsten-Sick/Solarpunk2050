% Checked by Editor
\section{Lost}
\label{sec:Lost}
\index{Lost|(}

The Lost are the faction that did not join the others on the "Dirty Road to Eden". They realised early on that there would be some horrors involved in achieving a solarpunk utopia. And they were not willing to pay the price. This faction is the only one that does not actively deny what has happened.

Their approach is to live in and from the ruins of the Lemmings' civilisation. They reuse, recycle and upcycle the old technology. This is what their clothes, tools, vehicles and shelters are made of. Old technology reused in creative ways. Most of it looks shaggy.

Living among the ruins, they have developed fighting and survival skills and are experts at exploration.
Many are historians, collecting ancient documents for their secret "Alexandria" project.

Daily life: The Lost are constantly fighting for survival. Either by gathering food in the wilderness or in the ruins, or by repairing old technology. Some tend farms. Others run shops or try to keep truck stops alive to provide some sort of safe travel through the wilderness. Their internal trading system is a \hyperref[sec:Barter]{barter} system, so they trade items for other items, food, information or fuel.
The Lost gain resource points like everyone else. But they tend to spend them on the privilege of burning fuel. This forces them to scavenge the ruins for resources to upcycle or recycle (for Resource Points).

Their reliance on old technology makes them the only faction without drones. They are also the only faction to have trained animals.

\begin{itemize}
    \item Search, rescue, hunting and cart pulling dogs
    \item Horses for riding and transport
    \item Rats for substance detection, food testing and rescue
    \item Birds (eagles, falcons, owls for hunting or reconnaissance)
    \item Dolphins and seals.
\end{itemize}

Some of the Lost developed an almost mystical relationship with their animals. At least according to their Norm friends who grew up in cities.

The Lost do not necessarily trust the Pioneers or the Norms. They remember the "Dirty Road to Eden" well.

\subsection{Shakespeare battle}
\index{Shakespeare battle}
\label{sec:Shakespeare battle}

Despite the rough first and second impressions they make on outsiders, the Lost are educated and hold history and old books in high esteem. The battle consists of one person starting to act out a scene from an old play and handing over to another person who must continue without fail. This is a drinking game and will turn into a chaotic situation.

(If the players' characters are part of the fighting crowd. GM: Bring a copy of a page from a Shakespeare play to play at the table).

\subsection{Skill: Bushcraft}
\index{Bushcraft}
\label{sec:Bushcraft skill}

The Lost know their way around wilderness and ruins. They can up-cycle and recycle old technology. They build camps, hunt, gather food, train animals and scavenge ruins for tools.

All of this is covered by the Bushcraft skill.

\subsection{Typical Lost protagonists}

\begin{itemize}
    \item Scrap collectors
    \item Indiana Jones-style adventurer
    \item Lost ranger and animal expert
    \item Survival specialist
    \item Librarian
\end{itemize}

\subsection{Traditions}

\subsubsection{Guests and weapons}
\label{sec:Lost guests and weapons}
As survival experts, helping people and providing shelter is a sacred tradition. People who come in good faith are given shelter, food and medical care (Lost medicine is mostly rustic first aid).

But before entering a Lost camp, guests are searched for weapons. If they do not have any, they will be given some basic ones to use for the duration of their visit. They can also receive basic weapons training. The Lost expect their guests to help defend their camp if it is attacked by critters or other humans. Guests are therefore armed.

If guests are not trustworthy, they will not be given ammunition for their weapon - until their help is needed to defend the camp.

\subsubsection{Libraries}

The first impression other factions have of the Lost when they first meet them is that they are an uneducated bunch of survivalists and fighters. Once they get to know some of the Lost, they learn of their true mission: to protect the past. They collect old books (from phone books to the Luther Bible) and art (from everyday objects to old paintings).
These are stored in the mobile library that most Lost families carry with them (a steel box in the boot or a special armoured bus).

Books that are not yet in Alexandria are sent there on a pilgrimage. It is a paper-chase adventure with missions at each stop. The official aim is spiritual experience and growth, which may lead to some of the participants becoming librarians of Alexandria. The unofficial goal is to hide the current location of Alexandria from anyone following the group.
Groups travelling to Alexandria will be secretly monitored by experienced Librarians. Anyone following these groups will be dealt with.

For most of the year, Librarians live with a family. Only when Alexandria is established will they be called upon to leave the family for a few months.

Alexandria is a library built out of shipping containers. To protect it, it only exists for a few weeks a year. Its location is a secret known only to the librarians. Pilgrims are led there by the Paper Chase. The containers contain the library's books and index.

When a group of pilgrims arrives, they hand over the book, which is then indexed. Skilled forgers make exact duplicates, which are sealed with the library's seal and returned to the pilgrims. Each pilgrim can also choose a book from the library. One copy will be mass-produced (using an ancient printing press) and returned with the pilgrims to their family library.
In this way, knowledge spreads and a decentralised backup is created.

All the Lost know of the Library. But only a few outsiders - those who have saved the life of a Lost and found a book not yet in Alexandria - are asked to join a group of pilgrims under a vow of secrecy.

\subsubsection{Librarians}

A pilgrim can apply to become a librarian. As well as being able to quote from 50 books, a librarian must be able to demonstrate special skills that the library needs. These range from forger, fighter (to protect the library), treasure hunter, builder (to rebuild the library once a year) and cook (to help with the pilgrims' gathering).
Part of the initiation ritual is to tattoo as much text as possible from the new librarian's favourite book on his or her body. People who know (this is mostly Lost) can recognise a Librarian by the letters tattooed on their skin.
Skin that cannot be hidden will remain free of text.

\subsection{Economy: Barter}
\label{sec:Barter}
\index{Barter}
The Lost spend most of their resource points on diesel. This is one of the reasons why they loot old ruins for materials and technology that they can up-cycle. This way they do not have to spend RPs on products.

They also have an internal barter system where they trade goods for goods (a chicken for this hammer...).
It is much more complicated than money, but they lack trust in other systems and it is transparent. So it works for them. And most of the time they trade within the same family anyway.
And you can use the goods while you wait to trade them (eggs from the chicken).

\subsection{Food}
\label{sec: lost food}
\index{Food!Lost}

Lost are the hunters and gatherers. They cook whatever they find. The ingredients are animals, vegetables, herbs and tins from looted ruins. The kitchen is a campfire and a few tarps around it to collect the food they gather.

A good Lost cook is good at improvising. The same dish will never taste the same because the ingredients will vary. The cook's goal is to get as many nutrients into the people as possible. Starting with the essentials like fat and other calories.

Cooking is usually done for the whole camp, including guests. They can also be recruited by the cook for simple tasks.

A very typical Lost dish is "Exhaust Bread", which is baked on the exhaust pipe of a moving truck.

\subsection{Diesel}

They burn diesel in their quest for independence and the "good old days". They spend all their Resource Points, which represent the resources a person can use without harming the environment. But spending these points also forces them to rely on looted and recycled materials. They cannot afford anything new. And they do not want to.
The other factions know this and don't see the social contract being broken. But they just don't understand the decision.

\subsection{Law}
\label{sec: lost law}
\index{Law!Lost}

\subsubsection{Investigation}

%% Elders are called together for the criminal case
%% Elders decide who will investigate

\subsubsection{Jurisdiction}

%% Elders receive info and judge

\subsubsection{Punishment}

%% People will be kicked out of the family. Can return after a Quest has been finished. This one is decided by the Elders. In a way of a "Hero journey" or "Finding my true self" journey from books

% TODO: Add example stunts




%% TODO: Add Flea markets as gathering
\index{Lost|)}