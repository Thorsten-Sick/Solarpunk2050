\chapter{Player stop reading}

Dear players: You are done reading. The next part contains adventures and surprises I do not want to spoil. If you want to be game master: go on


\section{Adventure structure}

There are some tricks to give the adventure some Solarpunk feeling. A positive, inclusive, optimistic one.

\subsection{Protagonists}

The players are not the heroes but the protagonists. Many adventures are solved by using the Comunity resources.
Or by building a community first. Solving NPCs problems and bringing them together and enable/empower them.

The "Die Hard" style hero will have a negative impact on the total feeling.

\subsection{From the utopia}

Adventures start in a Solarpunk setting. A party, people building together.

\subsection{To a better utopia}

The end of adventures should also be solarpunk style: This can be a party, the construction of a new building for the community, visiting some awesome spots in nature... This is the reward for a sucessful adventure.

\subsection{Challenges}

During the adventure there are challenges. This setting contains lots of them. Pick one or two. Most of the time the Solarpunk utopia is not achieved yet when the protagonists enter the stage or the utopia went out of balance. A backlash from the past (the "Dirty road to Eden") can also cause trouble. Or friction between fractions.

\subsection{Solarpunk style solutions}

Building a community to help tackling the problem is a Solarpunk-esqe solution. Building things, fixing things. Helping people. The adventure should contain some of those elements to make the adventure feel Solarpunky.

\subsection{Topics}

Every adventure should have several topics to explore. This could be 
\begin{itemize}
    \item differences between the factions
    \item philosophies
    \item approaches to agriculture
    \item energy
    \item mistakes of the past
    \item dark secrets of the Dirty Road to Eden
\end{itemize}

If you as a GM manage to start discussions between the players after the game you did it right. In this game there are no dragons to slay and brag about later. The more relvant topics gravitate towards the different point of views and different approaches.
