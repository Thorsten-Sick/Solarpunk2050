\chapter{Philosophies}

In addition to fractions the philosophies of the people have been challenged during the "Dirty road to Eden" phase. In the 2050 several of them are common - but not evenly distributed amongst the fractions. They may even have a specific flavour depending on the fraction applying the philosophy.

\begin{enumerate}
\item The Anarchists / The Social Experiment - When faced with everything that happened in their lifetimes, The Anarchists say no to any rigid hierarchy and structure. They promote mesh networks and technologies, distributed systems and manufacturing, support - but - not - dependence. They're trying out different new governance systems in different places and outposts, sometimes physically, sometimes online. One person wants to connect a neural network to a vast fungal network to create an oracle or an advisor, another preaches pure do-ocracy.

\item The Technologists / The Transhumanists - The problem of the Exponential Age is that we misunderstood the science. We abandoned cybernetics and governance, ignored the environmental sciences, but we shall no more. Wielding the knowledge we gained with the understanding of the priorities we can heal the planet and rebuild The Great Projects even greater, but sustainable this time. We can have the orbital elevator without polluting the ocean or the orbits, we can recreate the internet like it once was, but even more beautiful, allowing even the most remote village to join any cultural event or a university. If we find out that humans will fall in the same traps as before, we can change them with science. Brains can be rewired, traits modified. Being a functional part of the nature is more important than being pure in some way. Look at what pure did to the planet.

\item The Spiritualists / The Luddites - Our mistake was abandoning The Mother Earth as our spiritual mother first. We forgot her to the point where we were blind to hurting her, where we didn't even see what sins against her we're committing. We must do better. We stopped causing her pain, now we can start tending to her wounds and begging her for forgiveness. Maybe, in a few millennia, she will accept us as children again. We can only hope for that - and in that hope, abandon all that was superficial, all that was unneeded, all that caused our hubris. If we find any artifacts of the days gone, we should never fall for them again, destroy if possible. Even if some fool would say they can save lives, we know that they will destroy much more than our lives - our mother-given souls.

\item The Academia / The Curators - The only way not to repeat the mistakes of our past is to remember them, to teach them to all the future generations. We need to be more careful, more responsible, double- and triple- checking our every step. No more hurray at the magic of lead or single use plastics. No more killing the whales or burning the coal. We need to find and catalog everything about or past - and be very mindful about our plans forward, even at the cost of speed. We lost SO MUCH of our past, so many cultures, so much wisdom wiped out never to be learned. We weep with the colonized tribes of Africa and Asia, we see the mass graves of people who tried to rise and change the world before us, we remember them.

\item The Rescuers / The Healers - Everyone is shocked and traumatized and everyone has their own coping mechanism. Some look into the past, some into the future, but only we are looking at the now: the billions suffering, confused as we are. We should plan and dream, but right now we should help everyone who's still alive. Tomorrow we will find a better way forward, but there are so many cities, towns and villages without a stable source of water. Bringing it to them is the most important, even if we use the ruins of the old to do so. We know not to start the mines and the chimneys again, but a lot of carbon is already here, in the short cycle, isn't it? We can use the excavators with the last of the diesels, we can run the generators on the toxic batteries just to keep the hospital running. We'll dispose of them responsibly, but first and foremost, we'll help whoever's alive.

\end{enumerate}