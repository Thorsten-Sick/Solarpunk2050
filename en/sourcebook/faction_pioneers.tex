\section{Pioneers}
\label{sec:Pioneers}
\index{Pioneers|(}

The Pioneers took the future into their own hands. They saved everyone and the planet, and now they draw strength from that.

Hardcore Pioneers are only a small part of the population (and this can vary locally). But because they actively participate in life, they are the part that leaves its mark on society.

Their main goal is to develop a modern lifestyle where humanity, nature, technology and spirituality are in balance and much more advanced.

Their communities are a place to experiment with new ways of living, energy production, society, spirituality and food.

Most of the objects are prototypes and experiments. Almost everything is unique and built from scratch.

Sometimes they succeed in their experiments. Sometimes they fail horribly.


Daily life: Pioneers spend their time doing what they love. Tinkering for 10 or more hours a day is not uncommon. They often change projects once a month. If someone runs out of projects, tasks can be found in the local computer system, where anyone can post things that need to be done. There is no payment, but there is an unofficial currency: respect. This can be earned through achievements or contributions to the community. This \hyperref[sec:meritocracy]{Meritocracy} results in people with high levels of respect being able to attract more people and resources to their next project, or being the centre of attraction.
Going on an adventure to serve the community could also be a source of this respect.

\subsection{Pragmatic vs ecstatic}

Whenever a group of Pioneers falls in love with a project, they tend to focus on it completely and go into "ecstatic" mode. This project is going to be the greatest implementation you can imagine. With everyone contributing.

Other projects tend to just get done with minimal focus. They fall into the "pragmatic" level.

If someone has a great idea and can persuade others to join in, an ecstatic project can be in its first prototype stage within a few hours.

All Pioneer communities will have at least one project in ecstatic mode. Only very large ones will be able to have 4-5 parallel ecstatic projects running in parallel.

Examples for ecstatic/pragmatic projects are below.

\subsection{Cyberware}
\index{Cyberware}
\label{sec:Cyberware Pioneers}

Pioneers never manage to get the complex infrastructure for building and implanting cyberware up and running. It requires hundreds of specialists and is just boring. This is a task for the \hyperref[sec:Cyberware Norm]{Norm} Society.

But they love to tinker with cyberware that is already installed. Add a few extra sensors, overwrite the firmware. This can unlock new features and also introduce glitches.


\subsection{Skill: Prototyping}
\index{Prototyping}
\label{sec:Prototyping skill}

Pioneer technology is unique. They find intelligent solutions to problems. They tend to build prototypes that will never be used in mass production.

These prototypes are built quickly, solve a problem, use available technology and will never get a safety certificate.
Often they are not meant to last.
If a pioneer has to build the same technology a second time, it will be built with additional features or a different approach.

Use the Prototyping skill for this type of dirty engineering.

If non-renewable resources are needed, also roll for Resources.

\subsection{Economy: Meritocracy}
\label{sec:meritocracy}
\index{Meritocracy}

In the Pioneer world, your reputation is the key that unlocks doors. People who trust you will give you access to resources, join your projects or simply believe you.
Anyone can earn a reputation by building projects, organising parties, surviving an adventure, or simply being the best person to talk to when someone needs a problem solved.

\subsection{Appearance}

People are constantly improving their outfits. This includes hairstyles, clothes, tool belts and tattoos.

\subsubsection{Ecstatic}

All the clothes are unique. They adopt to the environment, display the mood of their wearer by light or are concept clothes (Made from mushrooms - because I can).

\subsubsection{Pragmatic}

DIY clothes based on concepts from other Pioneer groups, with modern fabric.

\subsection{Housing and sleeping}


\subsubsection{Ecstatic}

A creative blend of nature and technology. From Hobbit caves to grown, plant-based houses. No two houses are alike. Each one is a prototype.
Or buildings based on statics optimised by a genetic algorithm. Looks like the concrete and steel grew like a living organism.

\subsubsection{Pragmatic}

Pioneers can sleep on cushions near their workbench, in hammocks, in stacked sleeping capsules. They do not care, as long as the distance to their project is short.

\subsection{Communication}


\subsubsection{Ecstatic}

Each Pioneer community is constantly improving its communication technology. Using holo-projectors, bone-conducting audio, tactile displays. Whatever is fancy.
This often leads to compatibility issues with other communities. Applying a few updates to your personal computing device will solve the problem - most of the time.

\subsubsection{Pragmatic}

Laptops. The ability to code is the key feature.

\subsection{Food}
\label{sec: pioneer food}
\index{Food!Pioneers}

There are two types of food you can find in a Pioneer community.

\subsubsection{Ecstatic}

Be ready to experience new tastes. Just sign up - you will be randomly selected for the test or the control group.

Pioneers like to experiment with food. They first invented hydroponic gardens, which are now used in Norm's hives. They also invented in-vitro meat. But they were never interested in scaling it up. Instead, they experimented with new cell cultures and tissues. Pre-seasoned meat is also a thing.
Insects piqued their interest (because they grow quickly and the project cycle time is short). Yeasts - which can be genetically modified to produce a variety of different flavours and substances.

It is hard to eat the same dish twice when visiting a pioneer community. They are constantly improving their recipes.

In practice, food is free because there is always a kitchen experiment going on that needs testers and tasters.

Pioneer food is always unusual and surprising. There is a risk that the experiment will go wrong, which could result in terrible food or some health risks.

\subsubsection{Pragmatic}

Tinker food is easy to prepare and eat. It is meant to be eaten while a Pioneer is hacking in a flow that can last 20 or more hours. Typical Tinker food is Flavour Balls. Pea-sized balls of any flavour. These dried balls just need to be watered and grow to table tennis ball size. One is sufficient for the next 4-6 hours. And contains plenty of guarana.
Typically, a pioneer will put one in their mouth and sip coffee to let it grow in their mouth. It is not recommended to take more than one.

\subsection{Tech level}

Experimental technology. Things that were prototypes in 2020. Nothing boring

\subsection{Music}
\index{Music!Pioneers}

\subsubsection{Ecstatic}

Pioneers love music called "nature core". These songs go from idyllic to noisy/extracting and back to idyllic again. But the most unique feature is that the songs are generated by algorithms fed by listener behaviour.
The song is different every time you listen to it. But the algorithm used to generate it still makes it different. So musicians encode these algorithms and advertise them. At rave parties, the dancers and party-goers are the "musicians".

\subsubsection{Pragmatic}

Techno-style music playing on a loop. Maybe a few months ago someone made a light show that synchronised with the music.

\subsection{Mobility}

All mobility is electric and individual. Most Pioneers use a construction kit for rapid prototyping new vehicles. Those vehicles are called Modular Individual Vehicles or \hyperref[sec:MIV]{MIVs}. Walker, roller, swimmer or flyer. Anything is possible.

\subsubsection{Ecstatic}

There is a workshop that spits out new designs every day. These can range from specialised one-person vehicles to "10 people + a crane and amphibious with a gyro-stabilised coffee machine".

\subsubsection{Pragmatic}

Someone once built some \hyperref[sec:MIV]{MIVs} based on standard blueprints, and they are still in use. No one has picked up the thread since.

\subsection{Opinion: Eat the Rich Festivals}
\index{Eat the Rich Festivals!Pioneers}

"I was there. I wanted to work for the billionaires and get my brain implant. There was a party while I waited for my number. But I guess there were too many people applying. I never got the implant. But the food was good. Now that I am older, I do not know if I would try it again. But maybe I should try to recreate the recipe for the sausage. Yes, I know what they say was in it. But I don't believe it."

\subsection{Law}
\label{sec: pioneer law}
\index{Law!Pioneers}

\subsubsection{Important Pioneer laws}

\begin{itemize}
\item{Never commit Resource Point fraud}
\item{Never murder anyone}
\item{Never sabotage anyone's project}
\item{Always share your knowledge - but you can keep 8 of your inventions/innovations a secret}   % 8 = 2³
\end{itemize}

\subsubsection{Investigation}

In a very uncoordinated way, everyone will be running around collecting facts. These will be entered into a wiki.

\subsubsection{Jurisdiction}

Judgement is a democratic exercise. Anyone can vote digitally after reading the wiki.

\subsubsection{Punishment}

Punishment depends on the crime and the percentage of votes for "guilty".

The harshest punishment is forced sharing of technology and innovation secrets. This can go up to all 8. After that, the guilty must innovate to rebuild a stock of uniqueness.

The second harshest punishment is to leave the community and travel to other communities to spend time there and get enough "has been a perfect contributor to society" signatures. Doing specific projects there or contributing to society for a few months will do the trick. Depending on the verdict, the offender will need more or less signatures to return. In the end, the offender will have learned a lot and will have the chance to be welcomed back into their old society.

\section{Proof of skill}

Skill is highly valued among the Pioneers. And sometimes a Pioneer will decide that a certain item or knowledge should only be available to an extremely skilled or clever person.
This Pioneer will then build or create challenges in the form of puzzles, escape rooms or the like.
Lost and Norms find this terribly annoying.
Pioneers love the challenge.

\index{Pioneers|)}