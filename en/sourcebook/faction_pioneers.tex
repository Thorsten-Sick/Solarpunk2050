\section{Pioneers}
\label{sec:Pioneers}
\index{Pioneers|(}

The Pioneers took the future into their own hands. They saved everyone and the planet, and now they draw strength from that.

Hardcore Pioneers are only a small part of the population (and this can vary locally). But because they actively participate in life, they are the part that leaves its mark on society.

Their main goal is to develop a modern lifestyle where humanity, nature, technology and spirituality are in balance and much more advanced.

Their communities are a place to experiment with new ways of living, energy production, society, spirituality and food.

Most of the objects are prototypes and experiments. Almost everything is unique and built from scratch.

Sometimes they succeed in their experiments. Sometimes they fail horribly.


Daily life: Pioneers spend their time doing what they love. Tinkering for 10 or more hours a day is not uncommon. They often change projects once a month. If someone runs out of projects, tasks can be found in the local computer system, where anyone can post things that need to be done. There is no payment, but there is an unofficial currency: respect. This can be earned through achievements or contributions to the community. This \hyperref[sec:meritocracy]{Meritocracy} results in people with high levels of respect being able to attract more people and resources to their next project, or being the centre of attraction.
Going on an adventure to serve the community could also be a source of this respect.

\subsection{Cyberware}
\index{Cyberware}
\label{sec:Cyberware Pioneers}

Pioneers never manage to get the complex infrastructure for building and implanting cyberware up and running. It requires hundreds of specialists and is just boring. This is a task for the \hyperref[sec:Cyberware Norm]{Norm} Society.

But they love to tinker with cyberware that is already installed. Add a few extra sensors, overwrite the firmware. This can unlock new features and also introduce glitches.


\subsection{Skill: Prototyping}
\index{Prototyping}
\label{sec:Prototyping skill}

Pioneer technology is unique. They find intelligent solutions to problems. They tend to build prototypes that will never be used in mass production.

These prototypes are built quickly, solve a problem, use available technology and will never get a safety certificate.
Often they are not meant to last.
If a pioneer has to build the same technology a second time, it will be built with additional features or a different approach.

Use the Prototyping skill for this type of dirty engineering.

If non-renewable resources are needed, also roll for Resources.

\subsection{Economy: Meritocracy}
\label{sec:meritocracy}
\index{Meritocracy}

In the Pioneer world, your reputation is the key that unlocks doors. People who trust you will give you access to resources, join your projects or simply believe you.
Anyone can earn a reputation by building projects, organising parties, surviving an adventure, or simply being the best person to talk to when someone needs a problem solved.

\subsection{Food}
\label{sec: pioneer food}
\index{Food!Pioneers}

There are two types of food you can find in a Pioneer community.

\subsubsection{Party food}

Pioneers like to experiment with food. They first invented hydroponic gardens, which are now used in Norm's hives. They also invented in-vitro meat. But they were never interested in scaling it up. Instead, they experimented with new cell cultures and tissues. Pre-seasoned meat is also a thing.
Insects piqued their interest (because they grow quickly and the project cycle time is short). Yeasts - which can be genetically modified to produce a variety of different flavours and substances.

It is hard to eat the same dish twice when visiting a pioneer community. They are constantly improving their recipes.

In practice, food is free because there is always a kitchen experiment going on that needs testers and tasters.

Pioneer food is always unusual and surprising. There is a risk that the experiment will go wrong, which could result in terrible food or some health risks.

\subsubsection{Tinker food}

Tinker food is easy to prepare and eat. It is meant to be eaten while a Pioneer is hacking in a flow that can last 20 or more hours. Typical Tinker food is Flavour Balls. Pea-sized balls of any flavour. These dried balls just need to be watered and grow to table tennis ball size. One is sufficient for the next 4-6 hours. And contains plenty of guarana.
Typically, a pioneer will put one in their mouth and sip coffee to let it grow in their mouth. It is not recommended to take more than one.

\subsection{Law}
\label{sec: pioneer law}
\index{Law!Pioneers}

\subsubsection{Investigation}

%% Everyone runs around and collects stuff. Data will be stored in a Wiki

\subsubsection{Jurisdiction}

%% Everyone has to judge in a secret and digital way

\subsubsection{Punishment}

%% depending on the percentage of guilty votes the punishment can be reduced

%% Punishment is having to leave the community and travel to other communities to spend time there and get enough "was a perfect contributor to the society" signatures
%% Doing specific projects there or contributing to the society for some months will do the trick. Depending on the verdict the criminal needs more or less signatures to return.
\index{Pioneers|)}