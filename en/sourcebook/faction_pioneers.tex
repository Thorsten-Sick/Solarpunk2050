\section{Pioneers}
\label{sec:Pioneers}
\index{Pioneers|(}

The Pioneers have taken the future into their own hands. They saved everyone and the planet and now draw their strength from this.

Hard-core Pioneers make up only a small fraction of the population (and that can vary locally). But since they actively participate in life, they are also the part that leaves their mark on society.

Their main goal is to develop a modern lifestyle where humans, nature, technology and spirituality is balanced and much more advanced.

Their communities are a place to experiment with new ways of housing, energy production, society, spirituality and food.

Most objects are prototypes and experiments. Almost everything is unique and built from scratch.

Sometimes they succeed with their experiments. Sometimes they fail horribly.


Daily life: Pioneers spend doing what they love. Tinkering for 10 or more hours per day is not uncommon. Quite often they change projects once a month. If someone runs out of projects tasks can be found in the local computer system where everyone can place things that need to be done. There is no payment but an unofficial currency: respect. This can be earned by achievements or contributions to the community. This \hyperref[sec:meritocracy]{Meritocracy} results in people whit high levels of respect to attract more people and resources to their next project or be the centre of attraction.
Going on an adventure to serve the community could be a source for this respect as well.

\subsection{Cyberware}
\index{Cyberware}
\label{sec:Cyberware Pioneers}

Pioneers never manage to get the complex infrastructure running to build and implant Cyberware. It needs hundreds of specialists and is just boring. This is a task for the \hyperref[sec:Cyberware Norm]{Norm} society.

But they love to tinker with already installed Cyberware. Add some extra sensors, overwrite the Firmware. This can unlock new features and will also introduce glitches.


\subsection{Skill: Prototyping}

Pioneer technology is unique. They find smart solutions for problems. They tend to build prototypes that will never be used as mass produced.

Those prototypes are built fast, fix a problem, use available technology and will never get a certificate for safety.
Quite often they are not meant to last.
If a pioneer will have to build the same technology a second time it will be built with additional features or a different approach.

For this kind of dirty engineering, use the skill "Prototyping".

If non-renewable resources are requires also roll for "Resources"

\subsection{Economy: Meritocracy}
\label{sec:meritocracy}
\index{Meritocracy}

In the pioneer world your reputation is the key that unlocks doors. People who trust you will give you access to resources, join your projects or just believe you.
Everyone can build a reputation by building projects, organising parties, surviving an adventure or just be the best person to talk to when someone needs a problem solved.

\subsection{Food}
\label{sec: pioneer food}
\index{Food!Pioneers}

Pioneers love to experiment with food. They first invented the hydroponic gardens now in use in Norm hives. The also invented the in-vitro meat. But they were never interested into scaling it up. Instead they experimented with new cell cultures and tissues. Pre-spiced meat is also a thing.
Insects got their interest (as they are fast to grow and the project roundtrip time is short). Yeasts - which can be genetically modified to produce a large variety of different flavours and substances.

It is hard to eat the same dish twice when visiting a Pioneer community. They are constantly improving the receipe.

In pratice food is free because there is always a kitchen-experiment that needs testers and tasters.

Pioneer food is always fancy and surprising. There is a risk the experiment went wrong which could result in horrible food or some health risks.

\index{Pioneers|)}