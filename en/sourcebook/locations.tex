\chapter{Locations}

Large parts of the world have been destroyed or damaged by climate catastrophes. But small bubbles, towns and settlements have fought back and found their own way to survive.

\section{Lake of constance region}

A lake between the old countries of Germany, Austria and Switzerland.
In the old days it was rural, tourist, agricultural and a bit high-tech.

\subsection{Altshausen}
\label{subsec:Altshausen}
\index{Location!Altshausen}

In 2020, Altshausen has hot springs on a hill above the old town. Due to heavy fighting in the area, the springs have been unleashed and the hot mineral water has been flowing downhill ever since. The new inhabitants - mostly Lost - use it to their advantage.
They channel it into pools and gardens. Their greenhouses grow winter and summer vegetables.
All this is relatively low-tech, but some Pioneers are helping them keep the water flowing. For the look: Mix a medieval town with Venice and Pamukkale. Add some greenhouses and small onsen pools.


\subsection{Weingarten}
\label{subsec:Weingarten}
\index{Location!Weingarten}

At first glance, Weingarten is a lost city. It is stuck in the 1980s and can be visited. In reality, the few hundred inhabitants live in a kind of re-enactment slash theme park. It is not self-sustaining (which is why any Lost would hate to be associated with Weingarten).
There are only about 200 people left in Weingarten. Each of them plays several roles to keep the park alive.
It can only survive by trading with every visitor, trying to extract as much food, money, resource points and help as possible from everyone who stays there. In return, they offer an authentic 80s feeling...
As the neighbouring town of Ravensburg is fighting against electrosmog at night, there is a large fence between the towns to "protect" Ravensburg from electromagnetic waves.
Some people from Ravensburg like to spend the night in Weingarten - at least Weingarten has electricity at night.

Still the fence will block the Hive radio waves from reaching Weingarten.

Some noticable items and events from the 80s:

\begin{itemize}
    \item Cars
    \item Traffic jams
    \item Video rental shops
    \item Smoking
    \item Bonanza Bikes (\href{https://de.wikipedia.org/wiki/Bonanzarad}{Bonanza Rad}})
    \item Aerobics
    \item No digital money, but Deutsche Mark
    \item Helmut Kohl election posters
    \item Mettigel made out of raw minced meat (\href{https://de.wikipedia.org/wiki/Mettigel}{Mettigel})
    \item Russian eggs
\end{itemize}

\subsubsection{Pioneers in Weingarten}

There was a pioneer community in Weingarten. All of them left because they have not been accepted any more. The area where the university was is now a park. But there are rumours they left a hint for other Pioneers who wanted to follow them.
Details are in the adventure \hyperref[ch:the name of the rose]{The Name of the Rose}

\subsection{Ravensburg}
\label{subsec:Ravensburg}
\index{Location!Ravensburg}

Ravensburg is a Norm hive.
In the early 2020s, fears of electrosmog led the city to turn off the free wifi at night. This only got worse over the years. In 2050, Ravensburg switches off all electricity at night.
Some outskirt villages were turned into factories and facilities for processes where 24/7 power support is essential.

Situated on the river "Schussen", Ravensburg has built a park on the banks. To give people shelter during the summer heat waves. There are trees, water gardens and water playgrounds. Perfect for spending a hot day with the family.
When the river floods, this area is evacuated and flooded.

On the sports side: Ravensburg is famous for its parkour team and the tracks across the medieval roofs.



\subsection{Meckenbeuren}
\label{subsec:Meckenbeuren}
\index{Location!Meckenbeuren}

Meckenbeuren lies between Ravensburg and Friedrichshafen. The trains never stop at Meckenbeuren. Instead, protective armour plates are lowered over the windows and the train speeds up. As the train passes through Meckenbeuren, gunshots can be heard.

It is a mixture of Escape from New York and a rural farm town. And it is very likely that sooner or later someone will have to be rescued.

\subsection{Friedrichshafen}
\label{subsec:Friedrichshafen}
\index{Location!Friedrichshafen}

Friedrichshafen is located on Lake Constance. This Norm town is famous for its hobby group of historical pilots and dogfighters.

There are two leagues: The VR League, where pilots fly simulated aircraft. And the real league, where pilots fly WW2 planes like the Spitfire.

The planes have electric motors and an autopilot that flies them to the 'graveyard airport' as soon as the plane registers a simulated hit.
The pilot can choose to parachute out of the plane to skip the respawn (which is expensive in terms of league points), or stay in the automated plane and fly back to the graveyard.

Lost shudder at the lack of historical accuracy. All teams can fly all types of aircraft. Plus: the bad guy team is based on Nazi zombie B-movies and dressed like them.

But this city is a valid source of characters with special flying skills.

\subsection{Waterworld}
\label{subsec:Waterworld}
\index{Location!Waterworld}

A floating Pioneer city somewhere on Lake Constance. The different rafts are used to grow plants, sleep, build new things, generate electricity or clean water. Transport is provided by various water vehicles.

%% TODO Salem

%% TODO Überlingen

%% Todo Venice

%% TODO Alexandria