\chapter{World}

The world of 2050 is a world being rebuilt after several disasters caused by climate change and other human mistakes have shattered it. It is not yet a solarpunk world. It is a world shattered by catastrophes. With bubbles of functional utopias of different styles.
Each faction has its own style of utopia. And even there there is a lot of variety.
And all these utopias are missing something. The mission of Solarpunk is to learn from them and combine them. And then to share the benefits of these utopias.

This is the mission of the protagonists, who leave their utopian bubble to help others. And maybe even face the reality of different approaches to utopia.

\section{Wilderness}

Large areas are uninhabited and have reverted to wilderness. Some old cities still exist - ruins that have not yet been claimed.

Humans have found several new ways of life, the \hyperref[sec:Norms]{Norms}, \hyperref[sec:Lost]{Lost} and \hyperref[sec:Pioneers]{Pioneers}.

Their settlements are the seeds from which a new Solarpunk world can grow.

\section{Climate manipulation}

\section{Relics}
\index{Relic}
\label{sec:Relic}
Relics are things or organisations from the past. They do not fit into the new world and will soon be dismantled. But as with all relics, there is a long list of tasks to be done to dismantle them. And some just soldier on until their turn comes. Or they fight back.

\section{UN}
\index{UN}
\label{sec:UN}

The UN is the only central organisation left after the disaster. They are the good guys, leading the rescue and reconstruction efforts, and they also introduced the Resource Point system. After the collapse of nations, the UN had to be reinvented. As the organisation it should have always been.

Its missions:

\begin{itemize}
    \item Rebuilding global communications with drones, zeppelins and fibre optics
    \item Build an economy based on recycling with \hyperref[sec:Resource Points]{Resource Points}
    \item Find and destroy \hyperref[sec:Relic]{Relics}
    \item Reconnect settlements lost to civilisation during the catastrophes
\end{itemize}

The UN has a few bases around the world. But most of its efforts are franchised and specialists are hired for specific types of work.

That is what is publicly known. The truth behind it:

After disasters hit the globe and the UN became more important than ever, the senior UN officials (who chose their jobs for money and glory) quietly left their seats. The UN was kindly taken over by the lower ranks (who joined because they believed in the mission) and volunteers. The best functioning UN headquarters are in the countries formerly known as "development countries". Which in fact turned out to be the countries that were not stagnating.

These volunteers run the relief operations and are available by video conference. If anyone asks for higher ranks, they will get a reply along the lines of "sorry, full schedule", or a signed letter or email (both fake) if they wish: ....

If anyone is able to find one of the former higher ranks, they will accept the credits and still claim to be responsible for all those successful projects.

\subsection{Equipment}

The UN has some specialised equipment and personnel at its bases.

\subsubsection{Cargo Zeppelins}
\label{sec:UN Cargo Zeppelins}
\index{UN! Cargo Zeppelin}
Large cargo zeppelins. Can carry up to 4 shipping containers of cargo. The solar cells in their hull make them black, their electric engines silent. This has the side effect of making them mobile bases for secret missions. But the original idea is to use them as floating power generators: They rise up to 500 metres, are moored to the ground and generate solar power for the ground part of the mission. In addition, the engines can run in reverse to generate wind power.

\subsubsection{Away team}
\label{sec:UN away team}
\index{UN! Away team}
The away team can disembark from the zeppelins using their e-paragliders and prepare the ground for the landing of the cargo or build an anchor for the zeppelin. These people are some of the best trained in the UN.

\section{Resource points}
\index{Resource Points}
\label{sec:Resource Points}
Resource points are the main currency accepted by all factions. To prevent abuse of the resource, each person receives a limited number of resource points from the UN each year. They are required to obtain any non-renewable material-based item. The allotted amount is enough for a normal lifestyle. But not enough to build a brewery or other large structures (problem for pioneers). Or just enough to buy diesel for the cars - and nothing else (Lost). There are two ways to get more of these points:

\begin{itemize}
    \item Recycling items. Bring your old phone to the store and get a new one. If you want to do more than just replace your phone: Large items or those made from rare materials will give you more resource points. The items you recycle must not belong to anyone! This is one of the main reasons to embark on an adventure to the ruins of the Lemmings.
    \item Solve problems for the UN: The UN pays in resource points (they have their own budget for this) or in cash.
\end{itemize}

These two additional ways of earning Resource Points are the only ways of transferring them. When you recycle or complete a task, you will be asked where you want to transfer your points. You can pay someone, donate them, or support your community or one of your projects. This is the only time you can control the flow of Resource Points.

The Resource Point system is run by the UN. But everyone in their right mind has bought into it after witnessing the disasters. Cheating with Resource Points is considered as bad as cannibalism.



\section{Kessler Syndrome}
\index{Kessler Syndrome}
\label{sec: Kessler Syndrome}
A mass crash of satellites has made Earth's orbit inaccessible. There are no more weather satellites, GPS or mapping satellites. No communication.
This is called "Kessler Syndrome". Leaving the Earth is very risky and no one tries it anymore.

Instead, people use high-altitude drones, balloons and zeppelins. These only cover a small area and have to be launched deliberately. But it is better than nothing.