\chapter{Gear}

Players want equipment for their characters. Everyone wants to understand the technology available in the world.
%% \section{Philosophy}

%% \section{Housing}

%% \section{Food}

%% \section{Transportation}

\subsection{Pioneers: MIVs modular individual vehicles}

Pioneers like to experiment with new designs. To support rapid prototyping, many of their vehicles are based on a modular framework. The chassis is made up of struts and angle connectors. Engines and transmissions are flexible, as are batteries and hydrogen fuel cells. Steering is often drive-by-wire with microcontroller support. Much of the behaviour can be tuned by software.

Based on this framework, anyone can build their own vehicle within a few hours. Based on their taste and requirements. Number of passengers, acceleration, off-road capability, range, payload... All this can be customised.

It is so simple that even children can do it.

Pioneers interested in vehicles tend towards more drastic experiments (those listed are still considered entry-level ....):

\begin{itemize}
    \item AI controlled suspension
    \item Liquid nitrogen cooled superconductors
    \item Drone swarms to map the road ahead
    \item Chip tuning
    \item Ultralight carbon frame
    \item Solar panels for continuous charging
\end{itemize}


\begin{pioneertalk}[title=Inshallah 2]
    \begin{itemize}
        \item \npcquote{Pioneer}{This vehicle, the Inshallah 2, is made of something like ... Lego ... but for real projects}
        \item \npcquote{Lost}{There was this cold steam coming out from under my seat!}
        \item \npcquote{Pioneer}{This is just the pressure valve for the liquid nitrogen tank. We need it for the superconductors. Whatever. It can go at least 200 km/h off-road.}
        \item \npcquote{Norm}{You built your first Inshallah together with your brother, you said?}
        \item \npcquote{Pioneer}{Yes, but this is the first vehicle I have built on my own}
        \item \npcquote{Norm}{What happened to the Inshallah 1? And where is your brother?}
        \item \npcquote{Pioneer}{We are running out of time. Please fasten your seat belts.}
    \end{itemize}
\end{pioneertalk}

\subsection{Pioneers: Spray on PV}

Spray on photovoltaic is a simple method of creating murals that generate electricity.
Using three different spray cans (Spray on Conductor, Spray on Insulator and Spray on PV Coating), anyone can turn a wall into a power plant in minutes. Using stencils and different colours for the pv top layer, this power plant can also become a work of art.

\subsection{Norms: Trains}

Travelling across the continent is done by train. Trains have been improved enormously to make them more comfortable.

A train carriage is autonomous. It can uncouple at any time when there are two parallel tracks and move to another train or to a station. This means that no one has to rush to catch a train; they can just get on a carriage at any time and it will try to catch the next train going in the right direction.
Trains can split up en route and go to different destinations.
A standard carriage for people to travel in already has: seating (which can be converted to a sleeping position) - and tables, vending machines (free food and coffee), a cinema screen and a washroom (including shower).

People can book their own wagon through the Hive Controller. For a few premium credits, they can add special carriages to the train for recreation (and allow others to use them, too). All these personalisation options and quality of life carriages make the train feel more like a hotel.

Bonus cars:

\begin{itemize}
    \item 4-5 stars restaurant
    \item Spa
    \item Library + coffee shop
    \item A proper cinema
    \item Adventure playground for children, different themes (pirates, princess + castle, mad scientist lab, ...)
    \item Gym
    \item Garage for individual mobility + workshop
\end{itemize}

Sometimes a wagon carrying urgent goods is added to the train. But there are special trains for large, heavy and slow goods.

Norms have made standard journeys free. It is a basic right.

\subsection{Norms: Buses}

Similar to trains, buses are available in Hive cities - use your Hive Controller to define your destination, it will find a bus going there and coordinate a rendezvous.

\begin{normtalk}[title=This bus is haunted]
    \begin{itemize}
        \item \npcquote{Norm}{Stop ! Do not get on this bus. Wait three minutes and ours will arrive.}
        \item \npcquote{Lost}{Why? If I read my Hive Controller correctly, it will take us to our destination.}
        \item \npcquote{Norm}{Yes, it does. But there is this 'family' icon. Click on 'details' and you can see it: This bus has a playground for 30 children - 3 left and seats for 10 adults - 4 left. Trust me, you do not want to go in there.}
        \item \npcquote{Norm}{Also: The next bus has a restaurant. I have already reserved a table. I'll send you the menu.}
        \item \npcquote{Norm}{<Sends menu>}
        \item \npcquote{Norm}{Is half an hour enough, or should I extend the journey ?}
    \end{itemize}
\end{normtalk}

%% \section{Communication}

%% \section{Navigation}

%% \section{Cyberware}

%% \section{Furries}