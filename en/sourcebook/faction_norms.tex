\section{Norms}
\label{sec:Norms}
%% TODO : Add hedonistic sustainability https://de.wikipedia.org/wiki/Amager_Bakke
\index{Norms|(}

Norms live in retrofitted cities with added solar panels, green parks, small streams of water and lakes. Everything is quite idyllic but you can still see the original buildings from the 2020 under the green walls.
As modern Norm tech is built based on templates, many items, clothes, fashion and architecture looks identical. Those templates can be personalised. And smart people do that. But they will never achieve the creative chaos the Pioneers have.

The cities are known as "hive cities", the Norm society is also quite often called a "hive". The reason for this is the tight and close coordination this society has.

This society is coordinated by AIs (one per town) with the goal to improve quality of life for the inhabitants.
To measure this quality of life every Norm carries a "hive controller". A communication device with AR interface. This is also the main tool to do their magic: "hive control".

Someone from 2020 would call them wizards. Using Augmented Reality Apps on their hive controller they can request services from the AI. These range from ordering refreshments to building a house. 
Based on available templates the AI will coordinate factories and Norm contractors (very specialised ones) to build and deliver the customized request by drone.
Delivery is within minutes (when ordering a gun with pearl-style grip) to days (adding a house to the city) as long as the Norm user is within city limits.

This skill is not limited to physical things but can also be an App to organize a Party for 500 people.

But there are limits:

\begin{itemize}
    \item Extraordinary things cost money. And material things will additionally cost resource points
    \item Apps must be unlocked first by succeeding in a tutorial. This can take up to 2 years for a complex architecture App
    \item This magic will only work in the control area of an AI. And the supply lines must be open. The App will indicate if a service is not available
    \item Some risky services will also require a health check or other additional certificates    
\end{itemize}

AI controlled hives are meshed. And a traveller can get similar services in most of the connected AI hives.

It is time consuming to finish the tutorials and requirements for the most complex Apps - most Norms focus on 1-2 topics and the usual 40 Apps everyone uses to simplify life.

A visiting Pioneer or Lost can get a hive controller but will start on rank 0 (toddler). Having only the basic Apps available (pizza service, ....) and no clue how to use them. Finishing the basic tutorials will require quite some time.

The real stars in the Norm world are those who can create new templates for new services. And amongst them only some geniuses can generate templates for highly complex products (like trains).
Those templates and their Apps unlock new wonders for the society and make the creators famous.

Appearance: The norm world is (thanks to the templates) uniform compared to the world of the Pioneers or the Lost. Everything is mass produced with some added personalisation. That includes the solar upgrades to houses and the clothes.

Despite that, some dynamics in the towns is created by the plants and animals that were allowed to spread.

Mobility: Mass transport. But with added quality of life (the self driving tram has a bar and a big TV screen for entertainment. People are standing in groups chatting and enjoying the ride.)

Daily life: Work is optional. All \hyperref[sec:basic income]{basic goods are available for free}. If someone choses to work 25h per week he will get paid in local Euro or other local currency. This can be spend on premium products and services. Jobs are very specific (like drone pilot for food deliveries, barista or medi-tech). People tend to have at least 1-2 hobbies they are very skilled at in addition to their job.

Norm society depends on a storage-building-sized AI controllers with communication interfaces in the city.

To spread Norm society to new towns the Norms normally start with a shipping container containing AI, communication interfaces and power. These "seed hives" have limited capability but are sufficient to bootstrap a new society. The AI in there is created from an empty template and can adopt to the situation. When set up the AI will establish a radio link to a partner city hive and start creating an inventory of local services and people with a hive controller in their area.

A Norm on expedition can still have some benefits by using a "personal hive" which is just a long range communication interface to the next city hive. With all the negative effects like: lag, broken connection, slow delivery...

\subsection{Hive controller}
\index{Hive controller}
\label{sec:Hive controller}

The Hive controller is a head gear used for communication. It also has an AR interface which projects menus in front of the user. Those can be controlled by touch. Using this Hive controller the user is part of the social and technological network of a Norm hive.
The used can communicate, plan, organize and order things.
The Hive controller has a range of sensors that can measure the environment or the health of the user. That way it can guide the user to a more healthy life style.

It also handles payment for premium services.

\subsection{Cyberware}
\index{Cyberware}
\label{sec:Cyberware Norm}

Norms have a very sophisicated medical system. Anyone can get medical help and therapy for free. Cyberware replacement until limbs are grown back is an option.
By law Cyberware is limited to the power a healthy human has. This is to prevent people from using Cyberware as an enhancement.

This medical excellency can only be achieved by a granular devision of labour and a bunch of specialists.
To boring for a Pioneer, to complex for the Lost. But people of those factions can also get treatment in a Norm town. Maybe some help is needed to overcome the cultural differences. But there will always be someone who does that - for a price.

There are rumours Pioneers hack already implanted \hyperref[sec: Cyberware Pioneers]{Cyberware} to remove the limitations. But this would be illegal.



\subsection{The AI}

The AI as center of the hive network is in most cities the same type of AI. It is not self aware. Its skills at planning are "high human" and at its core it is a statistics engine combined with a  bunch of algorithms. Nothing magical. But as a central project manager with a city sized "team" of specialists and experts it can achieve miracles.

It is also a force multiplyer for local behaviour and traditions. As it optimizes for "quality of life" it will support anything the local people deem positive and dampen anything they do not like. The cultures of the Norm cities can develop into different directions after a few years of AI control.

There are rumours some hives upgraded their AI to something more powerful and maybe self aware thanks to the help of some pioneers. But those rumours are very likely based on non-standard hive controller interfaces and a very strange local culture.

\subsection{Skill: Hive control}

Being close to a hive and having a hive controller enables a Norm to request services from the Hive. The special skill "Hive control" is used for that.

The goal that needs to be rolled depends on the complexity of the task the Norm wants to achieve.

Sometimes and additional resource roll is necessary to check if the Norm has the Resource Points  for the required non-renewable resources.

If the hive-control dice roll fails, the reason could be:
\begin{itemize}
    \item User is required to complete some more tutorials first
    \item The App got an update....
    \item The hive is busy and this request can not be queued yet
    \item Basic components must be shipped from another hive first
    \item Hive controller down (which is an emergency !)
    \item You just got a task for your day to day job...
\end{itemize}

AI controlled areas can be in 4 quality ranges:

\begin{itemize}
    \item City hive: no penalty
    \item Seed hive (AI size: a shipping container): -2 on throw, slow delivery
    \item Personal hive (just a good communication interface to the next hive): -4, slow delivery
    \item No connection: Sorry.    
\end{itemize}

Most people in a Norm society chose to have a job. This will keep the busy about 25 hours per week. Those jobs are cordinated by the AI (a society wide project management software). It can happen that the Norm character adventuring somewhere gets a job ticket assigned at any time....

\subsection{Economy: Basic income}
\label{sec:basic income}
\index{Basic income}

The norms only did minory adjustments to the monetary system. The big thing is: Everyone has a kind of basic income: What you need for a living is free.
You only pay money for premium things. That way people with a job can get some motivation out of that.

Every person (not only Norms, but also Pioneers and Lost) can just take the basic goods in a reasonable amout. This covers:

\begin{itemize}
    \item Food
    \item Drinks
    \item Clothing (basic)
    \item Shelter
    \item Public transportation
    \item Cultural participation
\end{itemize}

From all those goods there is also a "premium" version you pay for. This version is somehow extra fancy. As a reference: Free to play games where you can buy extra gimmicks without any game effect that was common in the year 2020.

The absic income is not transfered to your bank account ! Instead there are shelves/vending machines and other opportunities in the cities to just grab what you need.

Example for premium: Close to the free coffee vending machine there could be a real person barista brewing awesome coffe and offering a nice chat for some money you earned at your job.

This basic income is possible because renewable energy is unlimited, production is automated and logistics as well.

\subsection{Food}
\label{sec: norm food}
\index{Food!Norms}
\index{Ceres}

Food is produced industrialised and automated in a so called "Ceres" (Ceres is the roman goddess of agriculture, but no Norm will care where the name is from). Old factories were transformed into those agriculture factories. Plants are grown here in in-house vertical gardens tended by robots. Meat is produced in-vitro where just energy and nutritients are added to a cell culture for it to grow into a steak. Lots of food is also bar-sized and the igredients are produced by algea and yeast grown in large tanks.
Everything is tasty but according to Pioneers this food "lack the special something".
This food is free and can be either gotten from automated kiosks or cantinas. But each Ceres has also an attached restaurant where you can get it directly from a robot. A Ceres is by design open for visitors and everyone can confirm that the food is grown in a hygienic way.

If you want premium food you can get that at a restaurant (which quite often shares parts of the kitchen with a free cantina). Here the same ingredients are cooked by chefs. Spiced properly and served in the combination with the perfect wine. With a Hive controller you can even define the matching table cloth and a scented candle to stimulate the appetite.
Sometimes you will meet a salad-sauce sommelier who can help you defined the perfect sauce for your dish. The receipe will be stored in your Hive controller. Of course this service is premium but worth the investment.

Among the food variants the factions offer Norm food is the least spectacular and most boring one. On the positive side: There is zero risk

% TODO: Add example stunts
\index{Norms|)}