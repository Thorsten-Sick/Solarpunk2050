\section{Norms}
\label{sec:Norms}
%% TODO : Add hedonistic sustainability https://de.wikipedia.org/wiki/Amager_Bakke
\index{Norms|(}

Norms live in retrofitted cities with added solar panels, green parks, small streams and lakes. It is all quite idyllic, but you can still see the original buildings from the year 2020 beneath the green walls.
Because modern Norm tech is built using templates, many items, clothing, fashion and architecture look identical. These templates can be personalised. And smart people do. But they will never achieve the creative chaos of the Pioneers.

The cities are known as 'hive cities', and Norm society is often referred to as a 'hive'. The reason for this is the tight and close coordination of this society.

This society is coordinated by AIs (one per city) with the aim of improving the quality of life for the inhabitants.
To measure this quality of life, each norm carries a "hive controller". A communication device with an AR interface. This is also the main tool they use to work their magic: "hive control".

Someone from 2020 would call them magicians. Using augmented reality apps on their hive controllers, they can request services from the AI. These range from ordering refreshments to building a house.
Based on available templates, the AI will coordinate factories and Norm contractors (very specialised ones) to build and deliver the customised request by drone.
Delivery can take anywhere from a few minutes (ordering a pearl-gripped weapon) to a few days (adding a house to the city), as long as the Norm user is within the city limits.

This ability is not limited to physical things, but can also be an app to organise a party for 500 people.

But there are limits:

\begin{itemize}
    \item Extraordinary things cost money. And material things also cost resource points
    \item Apps must first be unlocked by successfully completing a tutorial. This can take up to 2 years for a complex architecture App.
    \item This magic will only work in the control area of an AI. And the supply lines must be open. The app will indicate when a service is unavailable
    \item Some high-risk services may also require a medical, age or other additional certification. This includes drinking alcohol and caffeinated drinks.
\end{itemize}

AI-controlled hives are networked. And a traveller can get similar services in most connected AI Hives.

It takes time to complete the tutorials and requirements for the most complex apps - most norms focus on 1-2 topics and the usual 40 apps everyone uses to make life easier.

A visiting Pioneer or Lost can get a hive controller, but will start at rank 0 (infant). They will only have the basic apps (pizza service, ....) and no idea how to use them. Completing the basic tutorials will take some time.

The real stars of the Norm world are those who can create new templates for new services. And only a few geniuses can create templates for highly complex products (like trains).
These templates and their apps unlock new wonders for society and make their creators famous.

Appearance: Compared to the world of the Pioneers or the Lost, the world of the Norms is uniform (thanks to the templates). Everything is mass-produced with some added personalisation. This includes solar upgrades to houses and clothing.

However, there is a certain dynamism to the cities, due to the plants and animals that have been allowed to proliferate.

Mobility: Mass transport. But with added quality of life (the self-driving tram has a bar and a big TV screen for entertainment. People stand in groups chatting and enjoying the ride).

Daily life: Work is optional. All \hyperref[sec:basic income]{basic goods are available for free}. If someone chooses to work 25 hours a week, they will be paid in local euros or other local currency. This can be spent on premium products and services. Jobs are very specific (like drone pilot for food delivery, barista or med-tech). People tend to have at least 1-2 hobbies that they are very good at in addition to their job.

Norm society depends on a warehouse sized AI controller with communication interfaces in the city.

To spread Norm society to new cities, the Norms usually start with a shipping container containing AI, communication interfaces and power. These "seed hives" have limited capabilities, but are sufficient to bootstrap a new society. The AI inside is created from an empty template and can adapt to the situation. When set up, the AI will establish a radio link to a partner city's hive and begin building an inventory of local services and people with a hive controller in their area.

A Norm on expedition can still have some advantages by using a 'personal hive', which is just a long range communication interface to the nearest city hive. With all the negative effects of: lag, broken connection, slow delivery...

\subsection{Hive controller}
\index{Hive controller}
\label{sec:Hive controller}

The Hive Controller is a headset used for communication. It also has an AR interface that projects menus in front of the user. These can be controlled by touch. This device makes the user part of the social and technological network of a Norm Hive.
It is used to communicate, plan, organise and order things.
The Hive Controller has a number of sensors that can measure the environment or the user's health. In this way it can guide the user towards a healthier lifestyle.

It also handles payment for premium services.

A special skill is required to operate it. These are acquired and improved by Norms by completing the tutorials in the apps the Hive Controller has. More and more functions are unlocked, starting from the Norm's childhood. Obviously: When a Pioneer or Lost receives a Hive Controller, they must start with the most basic tutorials.
Sometimes applications are updated with new features. This may also mean that users will need to complete another tutorial before they can continue to use that application.

\subsection{Cyberware}
\index{Cyberware}
\label{sec:Cyberware Norm}

The Norms have a very advanced medical system. Anyone can receive free medical care and therapy. Cyberware replacement until limbs grow back is an option.
By law, cyberware is limited to the power of a healthy human. This is to prevent people from using Cyberware as an enhancement.

This medical excellence can only be achieved by a granular division of labour and a bunch of specialists.
Too boring for a Pioneer, too complex for the Lost. But people from these factions can also be treated in a Norm city. It may take some help to overcome the cultural differences. But there will always be someone who will - for a price.

There are rumours that the Pioneers hack already implanted \hyperref[sec:Cyberware Pioneers]{Cyberware} to remove the restrictions. But that would be illegal.


\subsection{The AI}

The AI at the centre of the hive is the same type of AI in most cities. It is not self-aware. Its planning skills are "high human" and at its core it is a statistical engine combined with a bunch of algorithms. Nothing magical. But as a central project manager with a "team" of specialists and experts the size of a city, it can work wonders.

It is also a force multiplier for local behaviour and traditions. As it optimises for "quality of life", it will support everything that local people see as positive and dampen everything they do not like. The cultures of Norm cities may evolve in different directions after a few years of AI control.

There are rumours that some hives, with the help of some Pioneers, have upgraded their AI to something more powerful and perhaps self-aware. But these rumours are most likely based on non-standard hive controller interfaces and a very strange local culture.

\subsection{Library of things}

Each hive has a library of items that Norms can borrow. They usually select them from the Hive Controller menu and have them delivered by drone. These items are not personalised. \textbf{Of course, they have to return them on time, unmodified and undamaged.} (unmodified is tricky if you have a Pioneer in your party).

\subsection{Skill: Hive control}
\index{Hive control}
\label{sec:Hive control skill}

Being near a hive and having a hive controller allows a Norm to request services from the hive. This is done using the Hive Control special ability.

The target that needs to be rolled depends on the complexity of the task the norm is trying to accomplish.

Sometimes an additional resource roll is needed to check if the norm has the resource points for the non-renewable resources they need. Most of the time the hive will try to help by allowing the user to borrow a similar item from the hive's library.

If the hive control die roll fails, this may be the reason:
\begin{itemize}
    \item User must complete some tutorials first
    \item The app has been updated ....
    \item The hive is busy and this request cannot be queued yet.
    \item Basic components must first be shipped from another hive
    \item Hive controller down (this is an emergency!)
    \item You have just been given a task for your daily work...
\end{itemize}

AI controlled areas can be in 4 quality ranges:

\begin{itemize}
    \item City hive: no penalty
    \item Seed Hive (AI size: one shipping container) -2 to throw, slow delivery
    \item Personal hive (just a good communication interface to the nearest hive): -4, slow delivery
    \item No connection: Sorry.
\end{itemize}

Most people in a Norm society choose to have a job. This will keep them busy for about 25 hours a week. These jobs are coordinated by the AI (a society-wide project management software). It can happen that the Norm character who is adventuring somewhere will be assigned a job ticket at any time: ....

\subsection{Economy: Basic income}
\label{sec:basic income}
\index{Basic income}

The Norms have done little more than tweak the monetary system. The big thing is: Everyone has a kind of basic income: What you need to live is free.
You only pay for premium items. That way, people with jobs can get some motivation out of it.

Every person (not only Norms, but also Pioneers and Lost) can just take the basic goods in a reasonable amount. This includes:

\begin{itemize}
    \item Food
    \item Drinks
    \item Clothing (basic)
    \item Shelter
    \item Public transportation
    \item Cultural participation
\end{itemize}

There is also a 'premium' version of all these goods that you have to pay for. This version is somehow extra fancy. As a reference: Free-to-play games where you can buy extra gadgets without any gameplay effect. That's common in 2020.

The basic income is not paid into your bank account! Instead, there are shelves/vending machines and other places in the cities where you can just grab what you need.

Example of a premium: Near the free coffee machine there could be a real person barista making great coffee and offering a nice chat for some money you earned at your job.

This basic income is possible because renewable energy is unlimited, production is automated and so is logistics.

\subsection{Food}
\label{sec: norm food}
\index{Food!Norms}
\index{Ceres}

Food production is industrialised and automated in a so-called "Ceres" (Ceres is the Roman goddess of agriculture, but no Norm will care where the name comes from). Old factories have been converted into these agricultural factories. Plants are grown in in-house vertical gardens, tended by robots.
Meat is produced in vitro, where a cell culture is simply given energy and nutrients to grow into a steak. Much of the food is also bar-sized and the ingredients are produced by algae and yeast grown in large tanks.
Dairy products are also produced in tanks from cell cultures grown in the lab. Starting with milk (cow, goat and tiger flavours are common), which is the base product for joghurt - and which is made from milk with microbiotics as in the old days.
Fruit from large plants is also grown from bioengineered cells - which produce fruit pulp in a tank. This pulp - mango in this case - is then pressed through a mango extruder to form a rod - a few hundred metres of mango. This is then sliced, dried or diced into yoghurt and that's it. Tiger Mango Yoghurt is hip in 2050.

Everything is tasty, but according to the Pioneers, this food "lacks that special something".
This food is free and can be obtained from either automated kiosks or cantinas. But each Ceres also has an attached restaurant where you can get it directly from a robot. A Ceres is designed to be open to visitors, and anyone can verify that the food has been grown in a hygienic way.

If you want to eat premium food, you can go to a restaurant (which often shares parts of the kitchen with a free cantina). Here the same ingredients are cooked by chefs. Properly seasoned and served with the perfect wine. With a Hive controller, you can even specify the matching tablecloth and a scented candle to whet your appetite.
Sometimes you will meet a salad sauce sommelier who can help you define the perfect sauce for your dish. The recipe is stored in your Hive Controller. This is a premium service, of course, but well worth the investment.

Of the food options offered by the factions, Norm Food is the least spectacular and most boring. But tasty. On the plus side: There is zero risk.

\subsection{Law}
\label{sec: norm law}
\index{Law!Norms}

\subsubsection{Investigation}

%% AI already knows. Needs independent verification with help of AI "Maybe you should look for blood patterns under the stairs"

\subsubsection{Jurisdiction}

%% AI judges

\subsubsection{Punishment}

%% Hive restrictions, "Probation officer" App to guide the person back to the society. Unlocking Hive Apps in this process.




% TODO: Add example stunts
\index{Norms|)}