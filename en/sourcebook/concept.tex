\chapter{Concept}

This chapter is a background chapter to explain some of my reasoning. And also define what this setting is, is not and why it is.

Some of the considerations are based on the Solarpunk philosophy, but others stem from the fact that
you have to be able to quickly and easily write stories with conflicts. That bites. I'll try to list my principles here. For me and others who want to work creatively - or invent adventures for their own group.

\begin{itemize}
\item Solarpunk is a positive utopia
\item Another important topic is “empowerment”. You can do a lot in your environment - if you want to.
\item People who can create are the protagonists
\item The society of the solar punks must be as open/chaotic and free as possible (firstly because of the “punk”, secondly
because of the positive utopia)
\item Every good on earth is limited. Except solar energy and creativity.
\item In order to experience adventure, there must still be problems despite the utopia. Therefore is not the whole society
solar punks
\item Fate is good for starting new story elements using aspects. Combat-heavy systems rather: Nope
\item The RPG project could fail if the story ideas ran out because the solarpunk society was too perfect
is for problems
\item A campaign is best oriented towards building a community and acquiring new buildings, technologies and
members. These are represented by Extras and NPCs. Adventures are required to acquire these. And under
Acquire: You don't have to own it. Finding a friend who will lend you a laser cutter is worth a lot.
\item This setting also contains many plot hooks, all of which are negative. To keep the positive mood should
not all are used at the same time!
\item There should be no heroes. But protagonists
\item They find everything they need to solve the problems on site
\item In particular, they build a community to solve problems
\item Fish out of water adventures: By having three different factions and their bioms the GM can create this kind of adventures where characters have to solve problems in regions where they have weaknesses (a Norm in the wilderness, ...) 
\end{itemize}


\section{The Solarpunk Manifesto}

https://www.re-des.org/a-solarpunk-manifesto/
(also in German: https://www.re-des.org/ein-solarpunk-manifest-deutsch/)

\section{Chobani Advertiser Movies}
Some commercials pretty much define the world: https://www.thelineanimation.com/work/chobani

\section{World Building}
The framework of the world is complete. And I just got the tip to read the following blog posts about solarpunk world building:
https://alpakawolken.de/category/solarpunk/
Came a little late, would have made things a lot easier for me. Leave them and let yourself be enriched. I do now too.

\section{Shift action to dilemmas}
I'm more of a method actor/dilemma lover myself. But the rules here are written with Action in the Spotlight. Because it reads
better, because it picks up players better. And also very easy to play (after a long day at work).
Still, the solarpunk world is bursting with drama. Three groups with at least comprehensible goals and mindset, potentially changing
cooperation with them. Dark sides even with the solar punks (meritocracy is nothing for people without skills).
So dear game master: If you're feeling a thoughtful mood in your group: Use that. Because the set of rules isn't Fate and not
Dungeon Slayer for nothing (nothing against Dungeon Slayer after a hard day's work).

\section{Robin's Laws of Good Game Mastering}

Robin divides the players into 7 classes. Not every game system hits everyone equally well. But it is helpful if there are at least
initial ideas of what each type of player likes.
\begin{itemize}
\item Method Actor: The confusion between the factions allows for interesting interactions with many shades of gray instead of
black and white
\item Storyteller: Complex stories can be told with the factions. There are enough seeds for epic ones
stories
\item Power Gamer: With the expansion of the community, hiring new people, new workshops and technologies you can make it
a lot of fun
\item Tactician: Can operate the angel system and plan tasks. Either with notes at the table or even
digitally simulated
\item Butt Kicker: The mutated animals were installed for the. plagues that can be eliminated
\item Specialist: Building blocks to recreate a popular class and its traits should be in place
\item Casual Gamers: Well. . .
\end{itemize}

\section{FAQs}
\begin{itemize}
    \item Does this game have a hidden agenda? Answer: No, it is not hidden
\end{itemize}