\chapter{Economy}

\section{Resource points}
\label{sec:Resource points}
Resource points are the main currency accepted by all factions. To avoid abuse of the nature every person gets a limited amount of them each year from the UN. They are required to get any non-renewable material based object. The ones appointed are enough for a normal life style. But not enough to build a brewery. The only way to get more of those points is by recycling objects. You bring your old phone back to the shop and get qa new one. if you want more than just to replace your phone: Big objects or those made of rare materials return more resource points. This is one of the main reasons to start an adventure visiting the ruins of the Lemmings. 

\begin{reason}[]
    Scarce resources (those resource points) are a good story hook. You can use it for adventures like expeditions, for loot, or crime stories where criminals try to trick the system to get more resource points. This is also the reason why they can not be traded. It would be to easy that way to trade them for other resources like money.
\end{reason}

\section{Lost: Barter}
\label{sec:Barter}
The Lost spend most of their resource points on gasoline. This is one of the reasons they loot old ruins for material and technology they can upcylce. This way they will not have to spend RPs on products.

In addition to that they have an internal barter system where they trade goods for goods (one chcken for this hammer...).
It is much more complicated than with money but they lack trust in other systems and it is transparent. So it works for them. And most of the time they trade within the same family anyway.
And you can use the goods while you wait to be able to trade them (eggs fromt he chicken).


\section{Norms: Basic income}
\label{sec:basic income}

The norms only did minory adjustments to the monetary system. The big thing is: Everyone has a kind of basic income: What you need for a living is free.
You only pay money for premium things. That way people with a job can get some motivation out of that.

Every person (not only Norms, but also Pioneers and Lost) can just take the basic goods in a reasonable amout. This covers:

\begin{itemize}
    \item Food
    \item Drinks
    \item Clothing (basic)
    \item Shelter
    \item Public transportation
    \item Cultural participation
\end{itemize}

From all those goods there is also a "premium" version you pay for. This version is somehow extra fancy. As a reference: Free to play games where you can buy extra gimmicks without any game effect that was common in the year 2020.

The absic income is not transfered to your bank account ! Instead there are shelves/vending machines and other opportunities in the cities to just grab what you need.

Example for premium: Close to the free coffee vending machine there could be a real person barista brewing awesome coffe and offering a nice chat for some money you earned at your job.

This basic income is possible because renewable energy is unlimited, production is automated and logistics as well.

\begin{reason}[]
    Reason to use the free shelves in the game instead money transfered to the characters account: If just resource points are automatically refreshed and the character is buying stuff with a price tag it will not be part of the game world. This way the player can decide to grab something from the free shelve or pay for something premium. That way the basic income can be experienced. In a real world economy it would be smarter to just transfer a certaina mount of money to someones account and have them decide what to spend that on.
\end{reason}

\section{Pioneers: Meritocracy}
\label{sec:meritocracy}

In the pioneer world your reputation is the key that unlocks doors. People who trust you will give you access to resources, join your projects or just believe you.
Everyone can build a reputation by building projects, organising parties, surviving an adventure or just be the best person to talk to when someone needs a problem solved.
