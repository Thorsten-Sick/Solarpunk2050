\chapter{The World Destroying Machine}
\label{ch:the world destroying machine}

A simple Hello World-style adventure with pre-created characters to play at cons or whenever you need a quick session.

Following the tradition, it is a special kind of "rats in the cellar" that seems to exist for almost all RPGs.

In this game, the pre-made characters are all from the Pioneer group. If, for some reason, a player cannot relate to the pre-made characters, you could try to create a Lost or Norm character and include it. This character could be a relative of one of the Pioneers and join the party at the start.

\begin{sidebarBox}[title=Dirty Road to Eden]

People living before 2025 are called "Lemmings" in 2050. So named because of their self-destructive habits. After 2025, more and more people doubted the wisdom of self-destruction and took action. This led to a 2050 where humanity was saved and could survive in thriving automated eco-cities, pioneer communities and lost camps amidst the wilderness and ruins of the old civilisation. But the road to this new and bright future was dirty. Not everyone could be saved. Some cities had to be sacrificed. Many hard choices had to be made in order to combat the disasters caused by climate change.

\end{sidebarBox}

\section{Topics}

This adventure covers some typical Solarpunk themes. As a game to play with Solarpunk beginners or even roleplaying game beginners, it can be used as a first step in a tutorial.

It also offers

\begin{itemize}
\item Introduction to the basics of the Solarpunk 2050 world
\item Character interaction: Players must balance interests to earn Fate Points
\item Culture Clash: All three cultures are represented. Collaboration can be critical to success
\item The mission starts with no weapons. Pioneers can make them or get help from NPCs.
\item Introducing the mistakes of the "Lemmings" (us) that lead to devastation
\end{itemize}

\section{Summary}

The map of the adventure is linear, but the protagonists can still go back and forth. To find allies, trade for tools and prepare for the final challenge.

Although the map is linear, there are several ways to solve the challenges, making the adventure flexible. The player's choices and solutions are sandbox style.

The linear order is:

\begin{itemize}
\item Players learn about the Pioneer philosophy at a Pioneer Party in their community.
\item Mission: Find the world destroying machine (= coal power plant) and recover raw materials to build a brewery.
\item The protagonists meet some of the Lost.
\item Upon entry into the world-destroying machine, you will meet Norms, who are recording a series of films.
\item The final boss of this adventure makes its first appearance: A mutant hamster
\item Search the world-destroying machine, solve puzzles, build weapons, deal with the hamster
\item The final party at the construction of the brewery, or dealing with the consequences of their decisions.
\end{itemize}

\section{Getting Started}

\begin{sidebarBox}[title=Pioneers]
\hyperref[sec:Pioneers]{Pioneers} are a group of hyper-inventive people who live in self-built, eco-friendly, high-tech communities. Most of the technology in use today is based on their concepts. During the Dirty Road to Eden they were the (uncoordinated) main drivers of the revolution. Today they either do not talk about this phase or they call it necessary for the safety of humanity (which it was). Most of the time they do not care about the past but focus on the future - many details have already been forgotten. Pioneers love their creative society but are very individualistic and everyone has their own pet projects.
Pioneers are very forward looking. To be interested in the past is a waste of time. That is why no one cares what a world-destroying machine is. No one among the Pioneers can answer that. If the protagonists asked the Lost, they would get the answer.
\end{sidebarBox}

It's a big outdoor Pioneer party. The community has gathered. There is homemade music and the usual LED and laser spectacle. As well as the usual garden grown food, a special drink is being served: a schnapps glass for everyone with a new beer to try.
It is brewed with home-made genetically modified yeast and a new brewing process. Delicious. And it glows thanks to bioluminescence. Unfortunately, the quantity is limited: the current labs and brewing equipment can no longer cope. They need to be expanded. And for that, the Community needs Resource Points.

\begin{sidebarBox}[title=Resource points]
\hyperref[sec:Resource Points]{Resource Points} are the main currency. To prevent abuse, each person receives a limited amount of them each year. They are required to obtain any non-renewable, material-based item. They are enough for a normal lifestyle. But not enough to build a brewery. The only way to get more of these points is to recycle items. Large objects or those made from rare materials will give you more resource points. This is one of the main reasons to embark on an adventure to the ruins of the Lemmings.
\end{sidebarBox}

Luckily, a "world-destroying machine" (a coal-fired power station - but this is never mentioned), buried in one of the many disasters, was found after another flood removed half of a hill. An auction was held for salvage rights, and the Pioneers won the right to enter first.
The party was given 4 (or number of players - 1) UN-issued salvage tags to attach to items to be salvaged. Once attached, they cannot be removed without heavy equipment. It is up to the protagonists to decide what is most valuable to them. Besides the tags, you can take as much as you can carry.
Other groups (who would enter the area much later) were given more salvage tags as compensation. Going first with fewer tags was a conscious decision by the Pioneer community.

\begin{sidebarBox}[title=Salvage tags]
Salvage tags are stickers with a small power source, computer and radio transmitter. They become inseparable (crazy glue) from an object and identify it as salvage. At the end of the adventure, specialists (NPCs) will arrive with heavy equipment to cut, drag and recycle objects. And add the points to the Pioneer Community account.
These tags have a display and a microcontroller. They activate at the time programmed into them. You cannot attach them before that time. This is why the different teams - who have different time slots for recovery - are not in direct competition. They are auctioned off by the United Nations, which organises auctions to distribute them.
\end{sidebarBox}

\emph{Salvage Tags are a game system designed to improve the flow of the game. It is a sort of "bag of holding". Without them, characters would be carrying 300 tonnes of power generators.}

% TODO: DOT graphviz goes here

\section{Party}

Topic of the scene:
\begin{itemize}
\item Characters get to know each other
\item Players test rules
\item \textbf{And especially: get a taste of the Solarpunk feeling}
\end{itemize}

The Pioneers have an evening party outside on the village square. This time, something big has been announced. To pass the time (and learn the rules), the protagonists can take part in one of the many activities.
Everything is decorated with coloured lights. Scarves and bunting everywhere. People are standing in groups or dancing. In the middle of the festival ground is a large pillar, the lower part of which is currently lit in green.
Announcement from the elders: "Today we have some news. The first one: Dorothea has children! (Display of a video screen with a live shot of a nest of chicks in the forest). <Frenetic cheering>. Quiet, please! We have just placed the volume column in the centre of the festival area because of the breeding season.
It monitors the microphones scattered around the forest.
As always: If it turns red, please turn down the volume. The music systems will do this automatically. This year, the Children's 5th Drone Squadron has vowed to protect the clutches by keeping cats, martens and other predators away from the nests in a large perimeter. (Illuminated quadrocopters fly in formation over the festival, one of the drones quickly breaks formation, dives elegantly into the punch bowl and immediately rejoins the formation) <children cheer>.
The second announcement will be in an hour.
"

After the first announcement, the characters can enjoy themselves at the party. This is to learn the rules:

\begin{itemize}
\item Juggling workshop (participation)
\item More relaxed: Gardening and conversation with local NPCs.
\item The children race their drones through the trees. Pilots fix broken drones themselves (participate, help with repairs, avoid drones, get them out of the trees)
\item E-motor challenge: Everyone has a drink. Afterwards they try to build a working motor from scratch (participation, medical help for drunk people)
\item Party organisation: Everyone who is interested takes turns playing music and lighting (Organise music and lighting)
\end{itemize}

Just before the announcement in the evening, everyone is given a shot glass of locally brewed beer. The eldest: "This beer was brewed with our own engineered yeast. The team around 'The Barrel' made it possible (cheers). As you can see, the beer glows in the dark and tastes great. But without a large bio-lab and a proper brewery, we can't make more. . and we lack the resource points to build them. The good news is: The UN has given us salvage rights to an ancient world-destroying machine. It was buried in a disaster. And a new disaster has just removed half of the hill above it. Let's salvage heavy machinery and rare metals, and secure resource points through recycling! That will give us our brewery lab!"

"The Barrel" can then answer people's most important questions during the festival:
\begin{itemize}
\item "Do you glow after drinking?" (No)
\item "Does the pee glow?" (Yes)
\item "How long does the pee glow?" (a few days )
\item "Can you make glowing lemonade for children?" (Yes)
\end{itemize}

The protagonists set off, first by train (e-bikes and quads are in the goods wagon). Then they drive into a relatively new patch of forest, growing on land that was flooded 20 years ago.


\section{Camp of the Lost}

Scene themes:

\begin{itemize}
\item Meet the faction of the Lost
\item Your first encounter with a mutated giant hamster
\item You can acquire weapons (steal, buy)
\item You could ask the Lost for help
\end{itemize}

\begin{sidebarBox}[title=The Lost]
The \hyperref[sec:Lost]{Lost} are survival experts, fighters and historians. They travel the land in search of remnants of pre-2020 "Lemmings technology". They reject new technology, but are very good at reusing and upcycling old technology. Their camps look a bit ragged, but are very practical. They are a bit rough around the edges compared to the "Lifestyle" Norms and the "Hyperactive/Hypercreative" Pioneers. When the Dirty Road to Eden began to transform the way of life of 2020 into what we have now, they saw that there was a high price to pay. And they decided, on ethical grounds, not to participate in that transformation.
\end{sidebarBox}

The protagonists arrive in a forest. In front of the entrance is the Lost Camp. Heavy diesel cars are parked with their engines running. Oil is burning in oil pans. Tents are made of old tarpaulins. Everything is makeshift, built from the remains of the past. But it is practical and a decent camp.

Plus this: A giant hamster (the size of a bear) on a grill.

\begin{sidebarBox}[title=Failed CCS experiment: Hamster]
More educated people will know that the \hyperref[sec:Giant Hamster]{giant hamster} is a failed carbon capture and storage experiment from the Dirty Road to Eden era. The Lost know. And they hunt this dangerous animal wherever they can. This animal is dangerous because it is genetically programmed to drag chunks of protein (even kicking and screaming) down into its underground storage facility. The original plan was to remove carbon from the surface by breeding these beasts....
\end{sidebarBox}

Someone is making potato salad and setting up the picnic benches. Music is playing.
The speakers are out of tune and at least 20 years old. But that doesn't bother anyone here. In the background, someone is shooting at beer cans with a shotgun (that's their leader, Caligula). Old books are stacked on a table.
The Lost got 10 salvage tags themselves at the auction. That's more than the Pioneers have. But that's also why they're the second to enter the ruins. The tags are not active yet - they will be activated in 12 hours and then they can start salvaging. Until the tags are active, the Lost want to party
here in the forest. So the Lost are no competition if the players are reasonably quick.

Behaviour: Tease the Pioneers and ask them not to take "diesel tanks, generators or anything else", as this technology belongs to the Lost. If the Pioneers join in the teasing and prove themselves worthy, they can be invited for a quick "hamster, salad and beer".

After that, the Pioneers' salvage tags will be activated and they can start descending into the world-destroying machine through the newly found entrance.

\begin{npcBox}[title=Caligula]

    \begin{aspects}
    \item \aspect[High Concept]{Small budget Indiana Jones}
    \item \aspect[Trouble]{Alcohol fuelled}
    \end{aspects}

    \begin{skills}
    \item \nskill{Academics}{3}
    \item \nskill{Athletics}{2}
    \item \nskill{Burglary}{0}
    \item \nskill{Contacts}{0}
    \item \nskill{Crafts}{1}
    \item \nskill{Deceive}{0}
    \item \nskill{Drive}{1}
    \item \nskill{Empathy}{0}
    \item \nskill{Fight}{1}
    \item \nskill{Investigate}{0}
    \item \nskill{Lore}{2}
    \item \nskill{Notice}{2}
    \item \nskill{Physique}{0}
    \item \nskill{Provoke}{3}
    \item \nskill{Rapport}{0}
    \item \nskill{Resources}{0}
    \item \nskill{Shoot}{4}
    \item \nskill{Stealth}{0}
    \item \nskill{Will}{1}
    \end{skills}

    \begin{stunts}
    \item \stunt{Tuning}{Gets a +2 to shooting whenever he uses a weapon that he has recently tuned in a training session that lasts 1 hour.}
    \end{stunts}

    \begin{stressSection}
    \stressLine{\stress{1}\stress{1}\stress{1}}{\stress{1}\stress{1}\stress{1}\stress{1}}
    \end{stressSection}
    \begin{tabularx}{\textwidth}{ XX }
    \end{tabularx}

    \begin{consequences}
    \item \consequence{2}
    \item \consequence{4}
    \item \consequence{6}
    \end{consequences}

    \begin{npcDescription}
    Caligula leads a small family of scavengers. They travel the wilds, searching for treasure in ancient ruins. Whatever useful things they find, they reuse and recycle in creative ways.
    He is ready to fight if he has to. So is his family. But they would all appreciate a discussion about ancient artefacts and sites. A stranger's first impression is of a redneck with a gun.
    \end{npcDescription}

    \end{npcBox}

He is not alone, but his "family" consists of about 10 people who can use a weapon and are good at scouting ruins. They do not care as much about nature as the Pioneers. They are constantly fighting and struggling with the forces of nature and the wilderness. Their approach to nature is more... pragmatic.

If the protagonists make friends with the Lost, they could gain:

\begin{itemize}
    \item Weapons and people who can use them
    \item The insight that the World Destroying machine is a coal power plant. Including a rough sketch of the map
    \item Maybe learn that Norms arrived 2 days ago. "Looked strange. But they always look strange. Not prepared for the ruins. Even less prepared than you are"
\end{itemize}

\section{Battleground}

Goal of the scene:

\begin{itemize}
\item You meet the Norms for the first time.
\item Learn: The world destroying machine is absurdly designed. Almost dull and boring
\end{itemize}

\begin{sidebarBox}[title=Norms]

80 percent of the people in 2050 are \hyperref[sec:Norms]{Norms}. They live in automated eco-cities. Governed by AIs that set all parameters for maximum quality of life and happiness. Society is highly cooperative. Most people have a 25h/week job that is highly specialised. The AI plans projects to coordinate these specialists in an incredible dance to achieve great projects.

Norms all carry a Hive Controller. This device provides them with apps and an AR interface where they can simply request things from the AI and Society. And it will be done - magic!
While the Norms enjoy their hobbies, they will never reach the solo capabilities of the Pioneers. They will always need to be close to the AI and a functioning Society. If the requirements are not met, some applications will indicate this and will be unavailable.

Focused on the now, they do not care about the past or the dirty road to Eden. Everything is fine now. It must have been worth the price.

In this region, all Norm characters have limited benefits from the app, as this region is only covered by the small AI they brought with them in a shipping container. The social network is also small. Almost everything they need has to be delivered by drones from the nearest town (1 hour extra). For Pioneers, this can still feel like magic.

\end{sidebarBox}

The characters enter a corridor through a crooked metal hatch on the side of a hill. Through the hatch: a corridor. The walls are white - but now musty. The floor is linoleum.
White plastic cupboards, devoid of any personality, line the corridors. Many doors (white, plastic with a wood look) branch off to the right and left. On them are signs with the names of the people whose offices used to be there. Behind the doors: rubble and mud.

Soon the protagonists find a simulated accident. It looks realistic: A Norm actor (Delta Awesome) lies under a foam H-beam (which looks like steel). A hidden camera man (Kevin) films him screaming. The hero of the reality soap should have appeared at any moment to 'rescue' him. Instead, the protagonists (real pioneers) come to the rescue.

The actor 'Delta Awesome' continues to act and 'Kevin' continues to film while the Pioneers begin the rescue. They soon learn that there was no real danger.

After the misunderstanding has been cleared up and everyone is waiting impatiently for the hero actor "Theophil Tierlieb", you can hear some screams coming down the aisle.

A quick glimpse: The expected hero, the actor in the role of \textbf{"Theophil Tierlieb"}, is being pulled into a pipe by a giant bear-sized hamster. These pipes seem to run through the whole world-destroying machine.

Unfortunately, the pipes are almost impossible for a human to crawl through (being dragged unconscious by a monster seems to take up less space, and the hamster itself is built for tunnels and pipes). Eventually, the pipe will break under the strain. Drones could follow the beast. Following the pipes is possible, but tricky. Some of them go through walls.

The protagonists need a map. And maybe weapons. As a Pioneer, you improvise as you go.

At the end of the corridor, the protagonists find a large hall lined with marble. The official entrance hall and the Museum of the World Destroying Machine.

\begin{npcBox}[title=Kevin\, Camera]

    \begin{aspects}
    \item \aspect[High Concept]{Camera for action}
    \item \aspect[Trouble]{Finding good action scenes}
    \end{aspects}

    \begin{skills}
    \item \nskill{Academics}{1}
    \item \nskill{Athletics}{1}
    \item \nskill{Burglary}{0}
    \item \nskill{Contacts}{2}
    \item \nskill{Crafts Filming}{3}
    \item \nskill{Deceive}{0}
    \item \nskill{Drive}{1}
    \item \nskill{Empathy}{3}
    \item \nskill{Fight}{0}
    \item \nskill{Investigate}{0}
    \item \nskill{Lore}{0}
    \item \nskill{Notice}{4}
    \item \nskill{Physique}{0}
    \item \nskill{Provoke}{0}
    \item \nskill{Rapport}{2}
    \item \nskill{Resources}{0}
    \item \nskill{Shoot}{1}
    \item \nskill{Stealth}{2}
    \item \nskill{Will}{0}
    \end{skills}

    \begin{stunts}
    \item \stunt{App based Filming}{Gets a +2 on filming action scenes when in range of an AI to support him}
    \end{stunts}

    \begin{stressSection}
    \stressLine{\stress{1}\stress{1}\stress{1}}{\stress{1}\stress{1}\stress{1}}
    \end{stressSection}
    \begin{tabularx}{\textwidth}{ XX }
    \end{tabularx}

    \begin{consequences}
    \item \consequence{2}
    \item \consequence{4}
    \item \consequence{6}
    \end{consequences}

    \begin{npcDescription}
    Kevin loves entertaining audiences. His skills with the camera and AI-based editing help him do this. If he becomes a friend, he can boost the Pioneer team's publicity (whether they want it or not). More importantly, though, are his Notice skills.
    "I've known I wanted to be a cameraman since the AI recommended me for the job when I was 10."


    He has no access to weapons and cannot order them from the app. "Sorry, I haven't done the weapons tutorial yet, should I?"

    \end{npcDescription}

\end{npcBox}


\begin{npcBox}[title=Delta Awesome - acting the victim]

    \begin{aspects}
    \item \aspect[High Concept]{Method acting actor}
    \item \aspect[Trouble]{Observe and copy the real Pioneers}
    \item \aspect[Aspect]{Always stay in character}
    \end{aspects}

    \begin{skills}
    \item \nskill{Academics}{0}
    \item \nskill{Athletics}{2}
    \item \nskill{Burglary}{0}
    \item \nskill{Contacts}{3}
    \item \nskill{Crafts (Acting)}{4}
    \item \nskill{Deceive}{2}
    \item \nskill{Drive}{0}
    \item \nskill{Empathy}{0}
    \item \nskill{Fight}{0}
    \item \nskill{Investigate}{0}
    \item \nskill{Lore}{1}
    \item \nskill{Notice}{1}
    \item \nskill{Physique}{2}
    \item \nskill{Provoke}{0}
    \item \nskill{Rapport}{3}
    \item \nskill{Resources}{1}
    \item \nskill{Shoot}{0}
    \item \nskill{Stealth}{0}
    \item \nskill{Will}{1}
    \end{skills}

    \begin{stunts}
    \item \stunt{Acting}{Can use Craft/Acting to convince people to join his heroic mission}
    \end{stunts}

    \begin{stressSection}
    \stressLine{\stress{1}\stress{1}\stress{1}\stress{1}}{\stress{1}\stress{1}\stress{1}\stress{1}}
    \end{stressSection}
    \begin{tabularx}{\textwidth}{ XX }
    \end{tabularx}

    \begin{consequences}
    \item \consequence{2}
    \item \consequence{4}
    \item \consequence{6}
    \end{consequences}

    \begin{npcDescription}
    Delta Awesome is the character's name. His role is that of an experienced Pioneer expert. But he does not live up to it. He insists on method acting and has to stay in the role (otherwise it will take him 2 hours to get back into it). He will also be constantly trying to improve the role by observing and copying the Pioneers.

    Delta Awesome's gadgets are useless props. In the film they are always exactly what he needs.

    He has trained for his role and has developed some real muscles. Which can come in handy. That and his acting/charisma-fuelled ability to convince people to help him. But in order to benefit from this, he has to be convinced first.
    \end{npcDescription}

\end{npcBox}


\section{Exhibition}

Topics of the scene:
\begin{itemize}
\item First clear indications of coal power (when researching the exhibition)
\item Socialize with the Norms
\item Find out where the pipes lead (on models and plans)
\item You can find many kilograms of protein paste here
\end{itemize}


Cameraman Kevin and Delta Awesome quickly lead the protagonists to the "headquarters". A former museum (also a film location). Catering is set up here. The Norms plan to accommodate 500 fans of the series after filming. With 10 extra seats for VIPs. The party location is currently being prepared.

There is an old museum where school classes can learn about coal power from very nice models.
Everything is nicely done. With a mascot. The mineral collection is also interesting, with a huge geode that might interest Disco.

In the catering area, there is a food designer (Cherie) who makes real-looking mealworms out of protein paste for the Pioneer food shots.  That way the VIPs can feel like solar punks without having to eat mealworms.

The food designer can control a 3D food printer via an app and could also make a protein-based fake body to lure the hamster.

According to the food designer, the others are deeper into the world-destroying machine, preparing it for filming. Haven't heard from them in a while. (Info: They've been hoarded). Access is through a steel door which is locked.

Someone with historical knowledge (books) can work out that the heaviest part here is probably the coal-fired generator with flywheel. This can be found further down the plant.

Cherie has a key to the door leading deeper into the plant. It could be stolen, she could be persuaded, the lock could be picked, or the door could be welded open.


\begin{npcBox}[title=Cherie]

    \begin{aspects}
    \item \aspect[High Concept]{Food artist}
    \item \aspect[Trouble]{Want to be my friend ?}
    \item \aspect[Aspect]{Food must be tasty and beautiful}
    \end{aspects}

    \begin{skills}
    \item \nskill{Academics}{1}
    \item \nskill{Athletics}{1}
    \item \nskill{Burglary}{0}
    \item \nskill{Contacts}{2}
    \item \nskill{Crafts (Food artist)}{4}
    \item \nskill{Deceive}{1}
    \item \nskill{Drive}{0}
    \item \nskill{Empathy}{1}
    \item \nskill{Fight}{0}
    \item \nskill{Investigate}{0}
    \item \nskill{Lore}{0}
    \item \nskill{Notice}{3}
    \item \nskill{Physique}{0}
    \item \nskill{Provoke}{0}
    \item \nskill{Rapport}{3}
    \item \nskill{Resources}{2}
    \item \nskill{Shoot}{0}
    \item \nskill{Stealth}{0}
    \item \nskill{Will}{2}
    \end{skills}

    \begin{stressSection}
    \stressLine{\stress{1}\stress{1}\stress{1}}{\stress{1}\stress{1}\stress{1}\stress{1}}
    \end{stressSection}
    \begin{tabularx}{\textwidth}{ XX }
    \end{tabularx}

    \begin{consequences}
    \item \consequence{2}
    \item \consequence{4}
    \item \consequence{6}
    \end{consequences}

    \begin{npcDescription}
    Does catering and simulates a Pioneer world for spectators and guests.

    She loves to chat at work and is positive and upbeat. A challenge such as "build a body using your 3D printer and protein" would be accepted with glee.

    Maybe that would distract the hamster?

    The monster next door worries her a lot because she grew up in a very safe environment - a Norm town.

    \end{npcDescription}

\end{npcBox}



\section{Coal Bunker}

Topics of the scene:
\begin{itemize}
\item Overcome technical problems
\item Can build weapons
\item Show dirtiness of World Destroying Machine
\end{itemize}

Problems:

\begin{itemize}
\item Dry coal dust (explosive)
\item Dark black water below, with oil film
\item The Norms built the SFX stuff. In particular, cables through the water and prepared pyrotechnics
\item Some of the old processes still seem to be working. The norms have been wildly hooking up batteries and motors in the hope of bringing things to life. Looks good on film but could be a death trap.
\end{itemize}

Weapon Material:

\begin{itemize}
\item Coal dust (potato cannon, pipe bombs)
\item Pipes from handrails
\item Explosives from the SFX
\end{itemize}

After a dirty corridor, the protagonists enter a huge hall. Coal wagons full of coal were delivered here on rails. Some of them are still here. Derailed and crushed by the disaster that happened many years ago. Here the coal was checked for quality, crushed into pellets and dust and transported down the hall on belts. Much of this can still be seen here - but in a sad state of disrepair.
Everything is rusted. Coal dust hangs in the air (and is explosive!). There are black puddles on the floor (oil and coal).
At least it's clear where the Norms went. They left behind batteries, lights and pyrotechnics, and their trail runs diagonally across the area. This obstacle course could explode at any time. It will require careful navigation, some parkour, metal-cutting skills and disarming explosives. Lots of skill tosses.

The coal conveyor belt leads to the next room, where the protagonists will want to go.


\section{Walkways}

Topics of the scene:
\begin{itemize}
\item Overcome obstacles
\item Demonstrate the devastation and grandeur of the world-destroying machine
\item Can build weapons
\end{itemize}

The protagonists have to climb over catwalks and through large running ventilation fans.
These are eerily backlit and a fog machine creates an eerie look. The SFX people were here. I'm sure it will look great in the film.
Greenish glowing dust puffs (mutated) grow on the floor below. Anyone with any knowledge of ecology would know that the spores are psychoactive. The director is lying next to the mushrooms. Unconscious thanks to the psychoactive mushrooms.
A make-up area is set up below. This is where filming is planned.

Problems:
\begin{itemize}
\item Broken metal walkways
\item Pipe labyrinths (in which hamsters move)
\item Mutated mushrooms, the director must be rescued
\end{itemize}

Weapon material:

\begin{itemize}
\item Sharp blades from ventilation (Swords)
\item Pieces of pipe (spears, pipe bomb, potato cannon)
\item Psychoactive mushrooms (wear protective gear when harvesting !)
\end{itemize}


The conveyor belt leads to the combustion chamber (which is not accessible). Next door is the generator room. There is the nest.
In this room you can already see the steam pipes leading there.

\begin{npcBox}[title=Lucien\, Director]

    \begin{aspects}
    \item \aspect[High Concept]{Film director with a skill for blockbusters}
    \item \aspect[Trouble]{There is a story in there}
    \end{aspects}

    \begin{skills}
    \item \nskill{Academics}{1}
    \item \nskill{Athletics}{0}
    \item \nskill{Burglary}{0}
    \item \nskill{Contacts}{3}
    \item \nskill{Crafts (Film director)}{4}
    \item \nskill{Deceive}{0}
    \item \nskill{Drive}{0}
    \item \nskill{Empathy}{1}
    \item \nskill{Fight}{0}
    \item \nskill{Investigate}{1}
    \item \nskill{Lore}{0}
    \item \nskill{Notice}{2}
    \item \nskill{Physique}{0}
    \item \nskill{Provoke}{0}
    \item \nskill{Rapport}{2}
    \item \nskill{Resources}{3}
    \item \nskill{Shoot}{0}
    \item \nskill{Stealth}{1}
    \item \nskill{Will}{2}
    \end{skills}

    \begin{stressSection}
    \stressLine{\stress{1}\stress{1}\stress{1}}{\stress{1}\stress{1}\stress{1}\stress{1}}
    \end{stressSection}
    \begin{tabularx}{\textwidth}{ XX }
    \end{tabularx}

    \begin{consequences}
    \item \consequence{2}
    \item \consequence{4}
    \item \consequence{6}
    \end{consequences}

    \begin{npcDescription}
    Is stoned when found and will not recover before the end of the story. Could be interesting as a friend at the party.
    \end{npcDescription}

\end{npcBox}

\section{Nest}

Topics of the scene:
\begin{itemize}
\item Final Battle
\end{itemize}

All kinds of organic material can be found in the nest. From old sacks of potatoes to dead animals (hunted dogs and wild boars).
It's confusing and full of the remains of an ancient civilisation.
The hamster has dragged the lifeless Norm onto the pile and he will die here soon.
A particular treasure here is the 4 large generators with the heavy, massive flywheel. This is the treasure that can be marked for recovery (either after the hamster dies or by sneaking in).

Solution ideas:
\begin{itemize}
\item You could make the hamster overeat with fake protein to make it fall asleep (Bio knowledge to trigger the "eat now" reflex)
\item Or intoxicate her with the psychoactive mushrooms (Bio Skills, Weapon Technique)
\item Or kill her (combat)
\item Or fetch the Lost for help (Social Interaction)
\item Sneak in and rescue the injured, including secretly planting salvage tags.
\item Dazzle the hamster by drones, pyro, SFX...
\end{itemize}


\begin{npcBox}[title=Hamster]

    \begin{aspects}
    \item \aspect[High Concept]{Fluffy killer machine on a CCS mission}
    \item \aspect[Trouble]{Damned by the genes}
    \item \aspect[Aspect]{Always hungry for protein}
    \end{aspects}

    \begin{skills}
    \item \nskill{Academics}{0}
    \item \nskill{Athletics}{2}
    \item \nskill{Burglary}{0}
    \item \nskill{Contacts}{0}
    \item \nskill{Crafts (nest)}{1}
    \item \nskill{Deceive}{2}
    \item \nskill{Drive}{0}
    \item \nskill{Empathy}{0}
    \item \nskill{Fight}{3}
    \item \nskill{Investigate}{0}
    \item \nskill{Lore}{0}
    \item \nskill{Notice}{3}
    \item \nskill{Physique}{4}
    \item \nskill{Provoke}{1}
    \item \nskill{Rapport}{0}
    \item \nskill{Resources}{0}
    \item \nskill{Shoot}{0}
    \item \nskill{Stealth}{2}
    \item \nskill{Will}{0}
    \end{skills}

    \begin{stunts}
    \item \stunt{Through the pipes}{Using \textbf{notice} the hamster can enter a pipe and emerge 1 round later at a tactical spot anywhere else in the room gaining an advantage for the attack (+2). By being better at \textbf{notice} the player characters can find out where the hamster is moving and negate the effect.}
    \end{stunts}

    \begin{stressSection}
    \stressLine{\stress{1}\stress{1}\stress{1}\stress{1}\stress{1}\stress{1}}{\stress{1}\stress{1}\stress{1}}
    \end{stressSection}
    \begin{tabularx}{\textwidth}{ XX }
    \end{tabularx}

    \begin{consequences}
    \item \consequence{2}
    \item \consequence{4}
    \item \consequence{6}
    \end{consequences}

    \begin{npcDescription}
    The hamster is bear-sized and can be quite aggressive when attacked. Her normal goal is to harvest proteins and drag them down here. And store them. The protagonists can also find this out when searching: The hamster is female and has 4 young hamsters in a nest that she protects.
    \end{npcDescription}

\end{npcBox}


\begin{npcBox}[title=Theophil Tierlieb]

    \begin{aspects}
    \item \aspect[High Concept]{Actor in role of "Theophil Tierlieb" - animal whisperer}
    \item \aspect[Trouble]{Animals hate the real me}
    \item \aspect[Aspect]{Hamster chow}
    \end{aspects}

    \begin{skills}
    \item \nskill{Academics}{0}
    \item \nskill{Athletics}{1}
    \item \nskill{Burglary}{0}
    \item \nskill{Contacts}{3}
    \item \nskill{Crafts (Acting)}{4}
    \item \nskill{Deceive}{1}
    \item \nskill{Drive}{0}
    \item \nskill{Empathy}{3}
    \item \nskill{Fight}{1}
    \item \nskill{Investigate}{0}
    \item \nskill{Lore}{2}
    \item \nskill{Notice}{2}
    \item \nskill{Physique}{2}
    \item \nskill{Provoke}{1}
    \item \nskill{Rapport}{0}
    \item \nskill{Resources}{0}
    \item \nskill{Shoot}{0}
    \item \nskill{Stealth}{0}
    \item \nskill{Will}{0}
    \end{skills}

    \begin{stressSection}
    \stressLine{\markedstress{1}\markedstress{1}\markedstress{1}\markedstress{1}}{\stress{1}\stress{1}\stress{1}}
    \end{stressSection}
    \begin{tabularx}{\textwidth}{ XX }
    \end{tabularx}

    \begin{consequences}
    \item \consequence{2} Headache
    \item \consequence{4} Broken bones
    \item \consequence{6} Strong bleeding
    \end{consequences}

    \begin{npcDescription}
    He plays an animal whisperer, but learns after 2 episodes that the animals hate him, so he suffers for 3 seasons until he is attacked by a real monster hamster. But the audience loved his character.
    \end{npcDescription}

\end{npcBox}


\section{The real end boss}

After the battle, the protagonists find 4 tiny monster hamsters (tiny = the size of a dog). The children of the mother they just killed. They are old enough to survive without their mother if someone takes care of them. They have not attacked humans. But they might in the future. Now it is up to the protagonists to decide:

\begin{itemize}
    \item Take them to the Community and care for them ?
    \item Kill them
    \item Leave them to their fate ?
    \item Sell them to the Lost (so they can be fattened and later be slaughtered) ?
\end{itemize}

You see, the real final boss is a dilemma, and you should make it extra dramatic. If possible, have a 5 minute discussion between the characters to find the best way. Make it clear that the resource points earned will not be enough for a monster hamster cage and a brewery.

\section{Victory Party}

Topics:

\begin{itemize}
\item Conclude the adventure and to celebrate.
\item Shows the consequences of their decisions
\end{itemize}

A few days later. The resource points were exchanged for resources. These arrive in the community and a brewery can be built. Its construction is part of a party with music, food and drink.

Friends you have made are invited. They will play a part in the celebrations.

If they have rescued the tiny monster hamster, they must first build a giant hamster cage. Including a wheel and pipes running through the community. Perhaps the resource points will not be enough for the brewery and the cage?

\section{Player characters}

\begin{itemize}
\item Books: Scholar, wants to salvage historical things
\item Curly: Acrobat, wants to recover something funny
\item The Barrel: Brewer, wants to salvage objects as large as possible because of resource points - the brewery wants them.
\item Disco: Bard, wants to salvage beautiful things
\item Spark: Tech, wants to recover technology
\item Primrose: Ecology, wants to save nature
\end{itemize}

%%%%%%%%%% Books
\newpage
\begin{npcBox}[title=Books]

    \begin{aspects}
    \item \aspect[High Concept]{Scholar - think first then act}
    \item \aspect[Trouble]{We don't do anything without a plan}
    \item \aspect[Relationship]{I wrote an article, could you please review it ?}
    \item \aspect[Aspect]{The more I know the better I perform}
    \end{aspects}

    \begin{skills}
    \item \nskill{Academics}{4}
    \item \nskill{Athletics}{2}
    \item \nskill{Burglary}{0}
    \item \nskill{Contacts}{0}
    \item \nskill{Crafts}{1}
    \item \nskill{Deceive}{0}
    \item \nskill{Drive}{1}
    \item \nskill{Empathy}{0}
    \item \nskill{Fight}{2}
    \item \nskill{Investigate}{3}
    \item \nskill{Lore}{0}
    \item \nskill{Notice}{3}
    \item \nskill{Physique}{0}
    \item \nskill{Provoke}{0}
    \item \nskill{Rapport}{2}
    \item \nskill{Resources}{0}
    \item \nskill{Shoot}{1}
    \item \nskill{Stealth}{0}
    \item \nskill{Will}{1}
    \end{skills}

    \begin{stunts}
    \item \stunt{E-Book}{While I have my treasured e-book, I get +2 when I use Academics}
    \end{stunts}

    \begin{stressSection}
    \stressLine{\stress{1}\stress{1}\stress{1}}{\stress{1}\stress{1}\stress{1}\stress{1}}
    \end{stressSection}
    \begin{tabularx}{\textwidth}{ XX }
    \end{tabularx}

    \begin{consequences}
    \item \consequence{2}
    \item \consequence{4}
    \item \consequence{6}
    \end{consequences}

    \begin{npcDescription}
    Hunter of knowledge. Wears a jacket with stains on the elbows. Hair turns grey. Shows a sense of fashion by wearing an elegant hat.
    \end{npcDescription}


    \begin{equipment}
    \item Light source (OLED film: battery operated, can be cut to size and glued on. Colour controllable)
    \item Food (Mama Salsa's famous mealworm buns in the bento box)
    \item First aid kit
    \item Mobile computers, headphones, communication via radio (mesh network)
    \end{equipment}

\end{npcBox}

%%%%%%%%%%% Curly
\newpage
\begin{npcBox}[title=Curly]

    \begin{aspects}
    \item \aspect[High Concept]{Childish climbing acrobat}
    \item \aspect[Trouble]{Secret fan of bad Norm TV series}
    \item \aspect[Relationship]{Always looking for a role model in the group}
    \item \aspect[Aspect]{Let's see if I can do something fun with it. . .}
    \end{aspects}

    \begin{skills}
    \item \nskill{Academics}{0}
    \item \nskill{Athletics}{4}
    \item \nskill{Burglary}{2}
    \item \nskill{Contacts}{0}
    \item \nskill{Crafts}{1}
    \item \nskill{Deceive}{2}
    \item \nskill{Drive}{0}
    \item \nskill{Empathy}{0}
    \item \nskill{Fight}{3}
    \item \nskill{Investigate}{0}
    \item \nskill{Lore}{0}
    \item \nskill{Notice}{2}
    \item \nskill{Physique}{1}
    \item \nskill{Provoke}{0}
    \item \nskill{Rapport}{1}
    \item \nskill{Resources}{0}
    \item \nskill{Shoot}{1}
    \item \nskill{Stealth}{3}
    \item \nskill{Will}{0}
    \end{skills}

    \begin{stunts}
    \item \stunt{E-Tail}{With my furry balance tail, I get +2 on Acrobatics when balancing}
    \end{stunts}

    \begin{stressSection}
    \stressLine{\stress{1}\stress{1}\stress{1}\stress{1}}{\stress{1}\stress{1}\stress{1}}
    \end{stressSection}
    \begin{tabularx}{\textwidth}{ XX }
    \end{tabularx}

    \begin{consequences}
    \item \consequence{2}
    \item \consequence{4}
    \item \consequence{6}
    \end{consequences}

    \begin{npcDescription}
    Shaved head, tattooed with a scale pattern. Wearing a homemade balance suit (with balancing tail).
    Upper body is naked, showing Bruce Lee style muscles. Curly wears cargo pants (to carry things around) and acrobat shoes with uncovered toes for better grip.
    \end{npcDescription}


    \begin{equipment}
    \item Light source (OLED film: battery operated, can be cut to size and glued on. Colour controllable)
    \item Food (Mama Salsa's famous mealworm buns in the bento box)
    \item First aid kit
    \item Mobile computers, headphones, communication via radio (mesh network)
    \item A stripped down light exoskeleton with a balancing tail
    \item Climbing rope
    \end{equipment}
\end{npcBox}


%%%%%%% The Barrel

\newpage
\begin{npcBox}[title=The Barrel]

    \begin{aspects}
    \item \aspect[High Concept]{Strong listener}
    \item \aspect[Trouble]{Likes to talk to microorganisms (yeasts)}
    \item \aspect[Relationship]{Active counseling. "How are you with that?"}
    \item \aspect[Aspect]{Driving force of the beer project}
    \end{aspects}

    \begin{skills}
    \item \nskill{Academics (genetics)}{3}
    \item \nskill{Athletics}{1}
    \item \nskill{Burglary}{0}
    \item \nskill{Contacts}{0}
    \item \nskill{Crafts}{3}
    \item \nskill{Deceive}{0}
    \item \nskill{Drive}{1}
    \item \nskill{Empathy}{4}
    \item \nskill{Fight}{2}
    \item \nskill{Investigate}{0}
    \item \nskill{Lore}{0}
    \item \nskill{Notice}{2}
    \item \nskill{Physique}{2}
    \item \nskill{Provoke}{0}
    \item \nskill{Rapport}{1}
    \item \nskill{Resources}{0}
    \item \nskill{Shoot}{0}
    \item \nskill{Stealth}{0}
    \item \nskill{Will}{1}
    \end{skills}

    \begin{stunts}
    \item \stunt{Empathy}{Because I'm highly empathic, using empathy to help someone gives me a +2. Unfortunately, his problems won't let me go for
    some time.}
    \end{stunts}

    \begin{stressSection}
    \stressLine{\stress{1}\stress{1}\stress{1}\stress{1}}{\stress{1}\stress{1}\stress{1}\stress{1}}
    \end{stressSection}
    \begin{tabularx}{\textwidth}{ XX }
    \end{tabularx}

    \begin{consequences}
    \item \consequence{2}
    \item \consequence{4}
    \item \consequence{6}
    \end{consequences}

    \begin{npcDescription}
    Comfortable, bearish, strong. Interested in optimising the art of brewing and willing to read up on genetic engineering. Normally dressed in a flannel shirt, 3/4 trousers and a small beer belly. Has not shaved for the last 3 days.
    \end{npcDescription}


    \begin{equipment}
    \item Light source (OLED film: battery operated, can be cut to size and glued on. Colour controllable)
    \item Food (Mama Salsa's famous mealworm buns in the bento box)
    \item First aid kit
    \item Mobile computers, headphones, communication via radio (mesh network)
    \item Gene laboratory in a suitcase
    \item 2 bottles of glowing beer (prototype)
    \end{equipment}
\end{npcBox}


%%%%%%%% Disco
\newpage
\begin{npcBox}[title=Disco]

    \begin{aspects}
    \item \aspect[High Concept]{Fighter for the colourful lights and the eternal party}
    \item \aspect[Trouble]{Uncomfortable in serious situations}
    \item \aspect[Relationship]{Wants everyone to be happy and to get along}
    \item \aspect[Aspect]{Looking for beautiful things}
    \end{aspects}

    \begin{skills}
    \item \nskill{Academics}{0}
    \item \nskill{Athletics}{1}
    \item \nskill{Burglary}{0}
    \item \nskill{Contacts}{2}
    \item \nskill{Crafts (SFX)}{3}
    \item \nskill{Deceive}{1}
    \item \nskill{Drive}{1}
    \item \nskill{Empathy}{3}
    \item \nskill{Fight}{0}
    \item \nskill{Investigate}{0}
    \item \nskill{Lore}{0}
    \item \nskill{Notice}{2}
    \item \nskill{Physique}{1}
    \item \nskill{Provoke}{0}
    \item \nskill{Rapport}{4}
    \item \nskill{Resources}{0}
    \item \nskill{Shoot}{2}
    \item \nskill{Stealth}{0}
    \item \nskill{Will}{0}
    \end{skills}

    \begin{stunts}
    \item \stunt{Disco !}{Because I'm Disco Artist, I get +2 when I use Craft (SFX) to draw or manipulate attention or mood of people or creatures using my disco systems.}
    \end{stunts}

    \begin{stressSection}
    \stressLine{\stress{1}\stress{1}\stress{1}\stress{1}}{\stress{1}\stress{1}\stress{1}}
    \end{stressSection}
    \begin{tabularx}{\textwidth}{ XX }
    \end{tabularx}

    \begin{consequences}
    \item \consequence{2}
    \item \consequence{4}
    \item \consequence{6}
    \end{consequences}

    \begin{npcDescription}
    Fidgety colourful party kid. Clothing has expanded over the years to include more and more quirky accessories
    \end{npcDescription}


    \begin{equipment}
    \item Light source (OLED film: battery operated, can be cut to size and glued on. Colour controllable)
    \item Food (Mama Salsa's famous mealworm buns in the bento box)
    \item First aid kit
    \item Mobile computers, headphones, communication via radio (mesh network)
    \item A dozen mini lighting drones for festivals
    \item Disco sound equipment (loudspeakers, recordings, microphones, all wirelessly connected)
    \end{equipment}
\end{npcBox}

%%%%%%%%% Primrose
\newpage
\begin{npcBox}[title=Primrose]

    \begin{aspects}
    \item \aspect[High Concept]{Lovable and pacifist eco-terrorist}
    \item \aspect[Trouble]{Nature does best without man}
    \item \aspect[Relationship]{Nature is great. People are OK too.}
    \item \aspect[Aspect]{Finding new nature and conserving it}
    \end{aspects}

    \begin{skills}
    \item \nskill{Academics (Biology and Ecology)}{4}
    \item \nskill{Athletics}{3}
    \item \nskill{Burglary}{2}
    \item \nskill{Contacts}{0}
    \item \nskill{Crafts (Explosives)}{2}
    \item \nskill{Deceive}{1}
    \item \nskill{Drive}{0}
    \item \nskill{Empathy}{1}
    \item \nskill{Fight}{0}
    \item \nskill{Investigate}{0}
    \item \nskill{Lore}{0}
    \item \nskill{Notice}{1}
    \item \nskill{Physique}{0}
    \item \nskill{Provoke}{0}
    \item \nskill{Rapport}{2}
    \item \nskill{Resources}{0}
    \item \nskill{Shoot}{0}
    \item \nskill{Stealth}{3}
    \item \nskill{Will}{1}
    \end{skills}

    \begin{stunts}
    \item \stunt{Do Drugs}{Because I'm experienced Eco-Terrorist, I get +2 when I use Academics (Biology) to use psychoactive substances to manipulate moods.}
    \end{stunts}

    \begin{stressSection}
    \stressLine{\stress{1}\stress{1}\stress{1}}{\stress{1}\stress{1}\stress{1}\stress{1}}
    \end{stressSection}
    \begin{tabularx}{\textwidth}{ XX }
    \end{tabularx}

    \begin{consequences}
    \item \consequence{2}
    \item \consequence{4}
    \item \consequence{6}
    \end{consequences}

    \begin{npcDescription}
    The clothes are visible eco-textiles - something that is not necessary with today's technology. It is a conscious choice.
    Primerose is an eco-hippie with rasta hair and a self-crocheted shirt.
    \end{npcDescription}


    \begin{equipment}
    \item Light source (OLED film: battery operated, can be cut to size and glued on. Colour controllable)
    \item Food (Mama Salsa's famous mealworm buns in the bento box)
    \item First aid kit
    \item Mobile computers, headphones, communication via radio (mesh network)
    \item Lock picking set
    \item Little biolab in a box
    \end{equipment}
\end{npcBox}


%%%%%%%% Spark
\newpage
\begin{npcBox}[title=Spark]

    \begin{aspects}
    \item \aspect[High Concept]{A punk in action - with tools}
    \item \aspect[Trouble]{Wants to learn new tricks from old technology}
    \item \aspect[Relationship]{Motivate others to tinker}
    \item \aspect[Aspect]{If it's small: shake it, if it's big: kick it}
    \end{aspects}

    \begin{skills}
    \item \nskill{Academics (Engineering)}{3}
    \item \nskill{Athletics}{1}
    \item \nskill{Burglary}{0}
    \item \nskill{Contacts}{0}
    \item \nskill{Crafts}{4}
    \item \nskill{Deceive}{0}
    \item \nskill{Drive}{2}
    \item \nskill{Empathy}{0}
    \item \nskill{Fight}{2}
    \item \nskill{Investigate}{0}
    \item \nskill{Lore}{0}
    \item \nskill{Notice}{1}
    \item \nskill{Physique}{3}
    \item \nskill{Provoke}{0}
    \item \nskill{Rapport}{0}
    \item \nskill{Resources}{2}
    \item \nskill{Shoot}{1}
    \item \nskill{Stealth}{0}
    \item \nskill{Will}{1}
    \end{skills}

    \begin{stunts}
    \item \stunt{McGyver genes}{Because I have McGyver gear and genes, I get +2 when I use crafting to screw something together in a hurry. Immediately after the successful use of the improvised hack it will probably fail spectacularly.}
    \end{stunts}

    \begin{stressSection}
    \stressLine{\stress{1}\stress{1}\stress{1}\stress{1}\stress{1}\stress{1}}{\stress{1}\stress{1}\stress{1}\stress{1}}
    \end{stressSection}
    \begin{tabularx}{\textwidth}{ XX }
    \end{tabularx}

    \begin{consequences}
    \item \consequence{2}
    \item \consequence{4}
    \item \consequence{6}
    \end{consequences}

    \begin{npcDescription}
    "Straw hat" woven from scraps of cable, other pieces of machinery woven into clothing.
    \end{npcDescription}


    \begin{equipment}
    \item Light source (OLED film: battery operated, can be cut to size and glued on. Colour controllable)
    \item Food (Mama Salsa's famous mealworm buns in the bento box)
    \item First aid kit
    \item Mobile computers, headphones, communication via radio (mesh network)
    \item Laser welder (no weapon)
    \item Duct tape, WD40 and Swiss army knife
    \end{equipment}
\end{npcBox}

\newpage

\section{Impact !}

All these adventures have an impact on the world. And especially in Solarpunk it is important to make this impact visible. Please write down the decisions the players have made, the friendships they have forged. The results they achieved for the Pioneer community. And then decide which of these things can appear in other stories.
The characters in this adventure were only meant for this adventure. But they had a lasting effect.

\begin{itemize}
    \item Will there be a special episode about the killer hamster ?
    \item Will a limited quantity of the glowing beer be available as a super-premium offer in Norm bars?
    \item Will you be able to visit the Pioneer Community? Will there be a giant hamster cage with several tiny monster hamsters?
\end{itemize}

Write down your ideas now and use them in the next few sessions.
