\chapter{The world destroying machine}

A simple hello-world style adventure with pre-created characters to play at cons or any time you need a short session.
Following the tradition it is a special kind of "Rats in the cellar" which seems to exist for almost all RPGs.

\begin{sidebarBox}[title=Dirty Road to Eden]

The people living before 2020 are called "Lemmings" in the year 2050. Named that way thanks to their self destructive habits. After 2020 more and more people doubted the wisdom of self destruction and took action. This led to a 2050 where mankind was safed and could survive in prospering automated eco-towns, Solarpunk communities and Lost camps in the middle of the wilderness and ruins of the old civilsation. But the road to this new and bright future was dirty. Not everyone could be safed. Some towns had to be sacrificed. Lots of hard decissions made fighting climate change induced disasters. 

\end{sidebarBox}

\section{Topics}

This adventure covers some typical topics. As a game to play with Solarpunk beginners or even Role Playing Game beginners it can be used as first steps in a tutorial.

It also offers

\begin{itemize}
\item Introduction to basics of the Solarpunk 2050 world
\item Character Interaction: Players must balance interests to earn Fate Points
\item Culture Clash: All three cultures are represented. Cooperation can be essential to success
\item The mission starts without weapons. Solarpunks can build them or get help from NPCs
\item Introducing the mistakes of the "Lemmings" (us) that lead to devastation
\end{itemize}

\section{Summary}

The adventure is linear. Even the map is. But the protagonists can always go back and forth. Find allies, trade for tools and prepare for the last challenge.

While the the map is linear there are several options to solve the last challenge which makes the adventure flexible. The player deccissions and the possible solutions are sandboxed.

The linear order is:

\begin{itemize}
\item Players get to know the Solarpunk philosophy at a Solarpunk Party in their Community
\item Task: Search the world destroying machine (=coal power plant) and recover raw materials to build a brewery
\item Protagonists meet some Lost.
\item After entering the World Destroying Machine they will meet Norms recording a series
\item The boss of this adventure has a first appearance: A mutated hamster
\item Search the World Destroying Machine, solve some problems, build weapons (like in the movie Predator), pose
the hamster
\item Closing party at the construction of the brewery
\end{itemize}

\section{Getting Started}

\begin{sidebarBox}[title=Solarpunks]
Solarpunks are a group of hyper inventive people of living in self build eco friendly high tech communities. Most of the currently used technology is based on their concepts. During the Dirty Road to Eden they have been the (uncoordinated) main driver of the revolution. Today they either do not talk about that phase or flag it as necessary to safe mankind (which it was). Most of the time they do not care about the past but focus on the future - many details have aleady been forgotten. Solarpunks love their creative society but are very individualistic and everyone has their own pet projects.
\end{sidebarBox}

It's a big outdoor Solarpunk party. The congregation has gathered. There is homemade music and the usual LED and laser spectacle. There is a schnapps glass for everyone with a new beer to taste.
Self-generated yeast and a new brewing process. Delicious. And glows thanks to bioluminescence. Unfortunately, the quantity is limited: the current laboratories and brewery devices can no longer cope. That should be expanded. And for that you need resource points.
By luck, a "world destruction machine" (a coal-fired power plant) that had been buried in one of the many catastrophes nearby was half flushed. You can go in there. And the Solarpunks won the right of first rescue in a competition.
The party receives 4 (or number of players-1) EU-issued salvage tags to stick on objects to be salvaged. Once attached, these cannot be removed without heavy equipment. Let them decide what is most valuable to them. Otherwise you can take as much as you can carry.

Salvage tags:
Salvage tags are sticker with small energy source and computer and radio transmitter. They become inextricably linked to an object and identify it as salvage. After the adventure, specialists (NPCs) will arrive with heavy equipment that will cut, haul, and recycle objects.

\emph{Salvage tags are a game system to make the flow better. This is a kind of "bag of holding". Without those tags the characters would be carrying 30 tons of power generators with them.}

\section{Procedure}
Eventually, players will find alternative routes through the world destruction machine. But here are the stages in the
standard order and who should get the spotlight there.


% TODO: DOT here

\section{Party 1}
Topic of the scene: 
\begin{itemize}
\item Characters get to know each other
\item Players test rules
\item \bf{And especially: get a taste of the Solarpunk feeling}
\end{itemize}

The Solarpunks have an evening party outside on the community fairground. Something big is announced this time. To pass
the time (and learn the rules) you can take part in one of the many activities. All is decorated with colored lights. Scarves and
pennants hang everywhere. People stand around in groups or dance. In the middle is a large pillar, the lower part of which is
currently illuminated in green.
Announcement from the elders: “Today we have some news. The first: Dorothea has offspring! (Display of a video screen
with live switch to a nest with chicks in the forest). <Frenetic cheers>. Quiet please! We have now put up the volume column
again in the breeding season. She monitors the microphones distributed in the forest.
As always: If the red is, please turn down the volume. The music systems do this automatically. This year, the Children's 5th
Drone Squadron vowed to protect the clutches by keeping cats, martens and other predators out in a large perimeter around
them. (Illuminated quadrocopters fly in formation over the festival, one of the drones quickly veers out of formation, dips
elegantly into the punch bowl and immediately rejoins the formation) <Children cheer>.
The second announcement is in an hour.
"
\begin{itemize}
\item Juggling workshop (participation)
\item Drones race the kids around through the trees. The pilots repair broken drones themselves (participation, help with repairs, dodge drones, get them out of the trees)
\item E-motor challenge: Everyone drinks a schnapps. Then the wire is wrapped around the electric motor core (participate, medical help drunken people)
\item Party organization: Everyone who is interested takes turns playing the music and lighting (orgnize music and lighting)
\end{itemize}

At the announcement in the evening everyone receives a shot cup of locally brewed beer. The eldest: “This is our own yeast
breeding. The team around 'Das Fass' made it possible (jubilation). As you can see, the beer glows in the dark and tastes
great. But without a large bio laboratory with a brewery, you can't produce more. . . and we lack the resource pointsBut
for we
that.
have salvage rights to an ancient world-destroying machine. That just got cleared. Was buried since a disaster. Let's salvage
heavy machinery and rare metals, and secure resource points through recycling! That will give us our brewery laboratory!”
The barrel is then allowed to answer people's most important questions during the festival: "Does one glow when one has
drunk this?" (No), "Does the pee glow?" (Yes), "How long does the pee glow?" (a few days ), "Can you also brew glowing
lemonade for children?" (Yes)
Then we set off, first by train (e-bikes and quads are in the goods wagon). Then drive into a relatively new patch of forest
growing on land that was flooded 20 years ago.


\section{Lost}

Topics of the scene:

\begin{itemize}
\item You get to know the faction of the lost
\item You have the first encounter with a mutant giant hamster
\item You can acquire weapons (steal, buy)
\item One could ask the lost for support
\end{itemize}

\begin{sidebarBox}[title=The Lost]
The Lost are survival experts, fighters and historicans. They travel the country looking for ruins of "Lemmings technology" from before 2040. They reject new technology but are very skilled in reusing and upcycling old technology. Their camps look a bit ragged but is very practical. They are a bit rough compared to the "lifestyle" Norms and the "hyperactive/hypercreative" Solarpunks. When the Dirty Road to Eden started to transorm the 2020 way of life into what we have now they saw that there is a high priec to pay. And decided to not join that transformation out of ethical reasons.
\end{sidebarBox}

You are in a forest. The lost camp in front of the entrance in the machine. Heavy diesel cars stand with their engines
running. Oil burns in oil pans. Above: A giant hamster on a rotisserie. Enough as with all 10 losers will be fed.
Someone is making potato salad and setting up the picnic benches. music blares.
The speakers are misadjusted and it's just a din. But that doesn't bother anyone here. In the background someone
is shooting at beer cans with a shotgun. The lost got 10 salvage tags themselves in the auction. That's more than the
Solarpunks have. That's why they're second. The tags are not active yet.
But will be activated in 12 hours and then they can start salvaging. Until the tags are active, the lost want to party
here in the woods. The lost are therefore no competition if the players proceed reasonably quickly.
Behaviour: They bully the solarpunks and just threaten them not to take "diesel tanks, generators or something" with
them, they belong to the lost.
After that's done, the Solarpunks' salvage tags activate and they're allowed to begin descending through the newly
found entrance into the World Destroying Machine

\begin{npcBox}[title=Caligula]

    \begin{aspects}
    \item \aspect[High Concept]{Small budget Indiana Jones}
    \item \aspect[Trouble]{Alcohol fueled}    
    \end{aspects}
    
    \begin{skills}
    \item \skill{4} Shoot
    \item \skill{3} Academics
    \item \skill{3} Provoke
    \end{skills}
    
    \begin{stunts}
    \item \stunt{Tuning}{Gets a +2 on shooting whenever using a gun he has recently tuned in a 1 hour practice session}
    \end{stunts}
    
    \begin{stressSection}
    \stressLine{\stress{1}\stress{1}\stress{1}}{\stress{1}\stress{1}\stress{1}}
    \end{stressSection}
    \begin{tabularx}{\textwidth}{ XX }
    \end{tabularx}
    
    \begin{consequences}
    \item \consequence{2}
    \item \consequence{4}
    \item \consequence{6}
    \end{consequences}
    
    \begin{npcDescription}
    Caligula leads a small family of ruin raiders. They travel the wildlands and look for treasure in old ruins. Whatever useful things they find they reuse in creative ways.
    He is ready to fight if he must. But would appreciate a discussion about old artifacts and sites a lot more. The first impression a stranger will see will still be the redneck with the gun.
    \end{npcDescription}
    
    \end{npcBox}

He is not alone but his "family" is about 10 people who can use a weapon and are skilled in ruins.

If the protagonists make friends with the Lost they could gain:

\begin{itemize}
    \item Weapons and people who can use them
    \item The insight that the World Destroying machine is a coal power plant. Including a rough sketch of the map
    \item Maybe learn that Norms arrived 2 days ago. "Looked strange. But they always look strange. Not prepared for the ruins. Even less prepared than you are"
\end{itemize}

\section{Battleground}
Goal of the scene:
\begin{itemize}
\item You meet the norms for the first time. Is drawn into an adventure.
\item The world destruction machine is absurdly engineered. Almost dull and boring
\item Knowledge: You need weapons
\end{itemize}

\begin{sidebarBox}[title=Norms]

80 percent of the people in 2050 are Norms. They live in automated eco-towns. Governed by AIs tuning all the parameters to achieve maximum quality of life and happiness. The society is highly cooperative. Most people have a 25h/week job that is highly specialized. The AI plans projects to coordinating those specialists in a incredible dance to achieve big projects.
Norms all carry a life-logger with them. This device offers them Apps and an AR interface where they can just request things from the AI and the society. And it will be done - magic !
While the Norms enjoy their hobbies they will never achieve the solo skills Solarpuks have. They are always dependend on veing close to the AI and a working society. If the requirements are not met some Apps may indicate that and will be unavailable.

Focussing on the now they do not care about the past or the Dirty Road to Eden. Everything is fine now. So it must have been worth the price.

\end{sidebarBox}

The characters enter a corridor through a crooked hatch. The walls are white - but musty now. The floor is linoleum.
White plastic cupboards devoid of any personality line the aisles. Many doors (white, plastic in wood look) branch off
to the right and left. Signs on them with the names of the people whose office this used to be. Behind the doors:
rubble and mud.


Soon you will find a simulated accident. A standard actor lies under a foam H-beam. A hidden cameraman films him
screaming. Actually, the hero of the reality soap should appear at any time. Instead, real solarpunks come to the rescue.
Both professionals continue and take up the rescue operation.
After the misunderstanding has been cleared up and everyone is impatiently waiting for the hero actor * Theophil Tierlieb*,
you can hear some screams down the aisle. Take a quick look: The expected hero, actor in the role of "Theophil Tierlieb" is
being pulled into a pipe by a huge hamster. These pipes seem to run through the whole world destruction machine.
Unfortunately, the pipe is almost impossible for a human to crawl through (being pulled unconscious by a monster seems to
take up less space). At some point the pipe will also break due to the stress. drones would go. In general you have a problem
to follow the monster and the victim quickly and the pipes end in walls. You need a map.
Characters should find that they have no weapons. But whatever. As a solarpunk, you improvise on the go.
Surely with social skills they could borrow the weapons from the Lost too! But not too much, because they want to come in a
few hours and not be unarmed.
At the end of the corridor, the protagonists find a large hall lined with marble. That was the Museum of the World Destruction
Machine.
NPC: Kevin
Kewin is the cameraman on the production. Likes to talk and always wants the best scenes in the can. If not stopped, he will
accompany the players.
"I've known I wanted to be a cameraman since the AI recommended me for the job when I was 10."
Concept: cameraman who gives everything for a good scene
Aspect: Likes to talk
• Perception: 4
• Craft (Movie): 3
• Charisma: 3
• Stealth: 2
NPC: Delta Awesome
Delta Awesome is the actor's role name beneath the simulated debris. His role is that of a skilled solarpunk expert. But he
does not do justice to this. But since he insists on method acting and has to stay in role (otherwise it will take 2 hours until he
comes in again) it gets exhausting at first. But it is easy to put it down at a buffet.
Equipped is Delta Awesome with useless tool props. But in the film they are always exactly the ones he needs. In reality it
doesn't help.
Concept: Delta Awesome, Solarpunk and Hero
Aspect: Always stay in character
• Craft (acting): 4
• Contacts: 3
• Charisma: 3
• Deception: 2


\section{Exhibition}
Goals of the scene:
\begin{itemize}
\item First clear indications of coal power (when researching the exhibition)
\item Social with the norms
\item Find out where the pipes go (on models and plans)
\item You can find many kilograms of protein paste here. It can attract the hamster or be put under food anesthesia
\end{itemize}
put. It works even better with the addition of anesthetics. But that doesn't exist here.
The cameraman and Delta Awesome quickly lead the protagonists to the "headquarters". A former museum (also film
location). There catering is set up (outside the cameras). The Norms plan to accommodate 500 fans of the series there
after filming. With 10 seats for VIPs. That's why it's being prepared.
Here is an old museum in which school classes could learn something about coal power from very beautiful models.
Everything is done nicely. Well sold. With mascot. Also interesting is the mineral collection, with a huge geode, which might
interest Disco.
This is the norm headquarters. They are setting up for a VIP party. Some spectators won a backstage pass and will arrive
tomorrow (so after the protagonists).
At the catering there is a food designer (Scherie) who makes real-looking mealworms out of protein paste for the Solarpunk
food shots. So that the VIPs can feel like solar punks but don't have to eat mealworms.
The paste is made from mealworms. It's just not clear to them - but it says on the packaging.
According to the food designer, the others are deeper into the world-destroying machine to set it up for filming. Haven't
heard from them in a while. (Info: They were hoarded). Access is through a steel door. Closed.
Someone with historical knowledge (Books) can figure out that the heaviest part here is probably the coal plant generator
with flywheel. And that it has to go deeper.
If you steal the food designer's key (or persuade her), pick the lock or weld open the door, you can penetrate deeper into
the facility.
NPC: Sherie
Does catering and simulates a solarpunk world for spectators and guests. So also props and make-up artists
Concept: I build the simulation as a makeup artist
• Craft (mask and prop): 4
• Charisma: 3
• Deception: 2


\section{Coal Bunker}
Goals of the scene:
\begin{itemize}
\item Overcome technical problems
\item Can build weapons
\item Show dirtiness of world destruction machine
\end{itemize}
problems:
\begin{itemize}
\item Dry coal dust (explosive) • Dark black
water below, with oil film
\item The norms built the SFX stuff. In particular, cables through the water and prepared pyrotechnics
\item Some of the old processes still seem to be running. The norms have been wildly hooking up batteries and motors in hopes of bringing
things to life. Looks good on film.
\end{itemize}
Weapon Material:
\begin{itemize}
\item Coal dust (potato cannon, pipe bombs)
\end{itemize}
Following a dirty corridor, the protagonists enter a huge hall. Goods wagons full of coal were delivered here on rails. Checked for quality,
ground, further transported. Much of it can still be seen here.
Massively rusted. The coal dust hangs in the air (explosive). There are black puddles on the floor.
At least it's clear where the norms went. They left behind batteries, lights and pyrotechnics and the trail runs diagonally through the area. But
in the terrain, the technology is massively negligent. Could blow up at any time.
Next room is the one where the conveyor belt leads.

\section{Walkways}
Goals of the scene:
\begin{itemize}
\item Overcome obstacles
\item Demonstrate the desolation and grandeur of the world-destroying machine
\item Can build weapons
\end{itemize}
You have to climb over catwalks and through big ventilation fans
The ventilation runs and is spooky backlit. The norms have connected a battery. . . .I'm sure it looks great in the movie.
Greenish glowing dust puffs (mutated) grow on the ground below. Someone with eco knowledge would know that the spores are psychoactive.
The director lies slurring by the mushrooms.
A make-up opportunity is set up below. The shooting here is already planned.
problems:
\begin{itemize}
\item Broken metal webs
\item Pipe labyrinths (in which hamsters move)
\item Mutated mushrooms, the director must be rescued 
\end{itemize}

weapon material

\begin{itemize}
\item Sharp Blades of Ventilation (Swords)
\item Pieces of pipe (spears, pipe bomb, potato cannon)
\item Psychoactive mushrooms
\end{itemize}

The conveyor belt leads to the combustion chamber (Not accessible). Next door is the generator room. There is the nest.
Here you can already see pipes that lead there.
NPC: Tscharli, director
director in distress
Concept: director of an endless series
• Craft (Director): 4
• Contacts: 3
• Resources: 3

\section{Nest}
Goals of the scene:
\begin{itemize}
\item Final Battle
\end{itemize}
All kinds of organic material can be found in the nest. Ranging from old sacks of potatoes to dead animals (hunted dogs and wild
boars).
It's confused and full of debris from ancient civilization.
The hamster himself dragged the lifeless Norm onto the heap and he will die here.
A special treasure here is the large generator with the heavy, massive flywheel.
Solution ideas:
\begin{itemize}
\item By the way, you could make the hamster overeat with a tub of protein porridge so that he falls asleep (Bio
Knowledge)
\item Or intoxicate with the Psychoactive Mushrooms (Bio Skills, Weapon Technique)
\item Or kill (combat)
\item Or fetch the lost (Social Interaction)
\item Sneak up and rescue the injured by secretly attaching salvage tags
\end{itemize}
NPC: Hamster
Concept: Fluffy killer machine on a CCS mission
Dilemma: Damned by the genes
Aspect: Always hungry for protein
The hamster is programmed to pull chunks of protein down underground - into its nest. He does.
Even if the chunks fight back. Coal itself does not interest him.
• Athletics: 3
• Charisma: 2

• Power: 4
• Fight: 3
• Perception: 2
Stress, physical: 6
Stress, mental: 3
NPC: Theophil Tierlieb
actor in role.
Concept: Actor in the role of "Theophil Tierlieb"
Aspect: Friend of all animals (i.e. the role).
Aspect: Just severely injured hamster chow
• Craft (acting): 4
• Contacts: 3
• Charisma: 3
• Deception: 2

\section{The real end boss}

After the fight the protagonists will find 4 tiny monster hamsters (tiny = the size of a dog). Ths kids of the mother they just killed. They are old enough to survive without monther if someone cares for them. They did not attack people. But could do that in the future. Now it is up to the protagonists to decide:

\begin{itemize}
    \item Take them to the Community and care for them ?
    \item Kill them
    \item Leave them to their fate ?
    \item Sell them to the Lost (so they can be fattened and later be slaughtered) ?
\end{itemize}

You see, the real end boss is a dilemma and you should make it extra dramatic.

\section{Victory Party}

Task:

\begin{itemize}
\item Serves to conclude the adventure and to celebrate.
\item Shows the consequences of their decissions
\end{itemize}
A few days later. The resource points were exchanged for raw materials. Those arrived at the Community and a brewery can be built. Building it is part of a party with music, food and drinks.

Friends they made will be invited. They will play a role in the celebration.

If they rescued the tiny monster hamster they will have to build a giant hamster cage first. Including a wheel and pipes.