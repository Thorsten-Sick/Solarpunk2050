\chapter{Die Weltvernichtungsmaschine}
\label{ch:the world destroying machine}

Dieses Abenteuer eignet sich hervorragend für Cons mit ungefähr 4h Spielzeit.

Der Tradition folgend ist es ein "Ratten im Keller" Startabenteuer - wie es für fast alle Rollenspiele existiert.

Alle Charaktere sind aus der Pioneer Fraktion. Die Lost und Norm Fraktionen tauchen als NSCs auf.

\begin{sidebarBox}[title=Der dreckige Weg nach Eden]

Die Menschen, die vor 2050 lebten sind die sogenannten "Lemminge". Diesen Namen haben sie wegen ihrer selbstzerstörerischen Lebensweise erhalten. Ab 2025 zweifelten die Leute zunehmend an der Weisheit, sich selbstzerstörerisch zu verhalten und sind in Aktion getreten. Diese führte in das Jahr 2050 in dem die Menschheit sich gerettet hat und in Lebenswerten automatisierten Norm-Städten, in Pioneer Wissenschaftscommunities oder in naturnahen Lost Camps - zwischen Wildnis und Ruinen - leben.
Aber der Weg zu dieser neuen und glänzenden Zukunft war dreckig. Nicht jeder Konnte gerettet werden. Städte wurden geopfert. Harte Entscheidungen mussten getroffen werden im Kampf gegen die Desaster, die durch den Klimawandel ausgelöst wurden. Und die Revolution ist noch im vollen Gang.

\end{sidebarBox}

\section{Themen}

Dieses Abenteuer deckt typische Solarpunk Themen ab. Es kann gut mit Solarpunk- oder Rollenspiel Neulingen genutzt werden, da es im Tutorial Stil aufgebaut ist.

Es bietet

\begin{itemize}
\item Einführung in die Solarpunk 2050 Welt
\item Charakter Interaktion: Die Charaktere müssen ihre Interessen ausbalancieren, um Fate Punkte zu erhalten
\item Culture Clash: Alle drei Kulturen sind in diesem Abenteuer vorhanden. Kooperation ist wichtig für den Erfolg.
\item Die Mission startet ohne Waffen. Die Protagonisten können aber welche basteln oder von NSCs erwerben.
\item Einführung in die Fehler der Lemminge (also wir), die zur Verheerung geführt haben.
\end{itemize}

\section{Abenteuer Zusammenfassung}

Die Karte des Abenteuers ist linear, aber die Protagonisten können jederzeit vor- und zurück reisen. Um Verbündete zu finden, Gegenstände einzuhandeln und sich auf die letzte Herausforderung vorzubereiten.

Obwohl die Karte linear ist, gibt es viele Möglichkeiten, die Herausforderungen anzugehen. Das Abenteuer ist absichtlich flexibel gehalten. Die Entscheidungen der Charaktere und die Lösungswege sind im Sandbox Stil !

Die lineare Ordnung ist:

\begin{itemize}
\item Die Spieler lernen die Pioneer Philosophie kennen auf einer Pioneer Party in ihrer Gemeinschaft.
\item Mission: Finde die Weltvernichtungsmaschine (den Spielern nicht bekannt: Es ist ein Kohlekraftwerk) und bergt recycelbares Rohmaterial, um eine Brauerei zu bauen.
\item Die Protagonisten treffen im Wald auf die Lost.
\item Nach dem Betreten der Weltvernichtungsmaschine treffen sie auf Norms, die gerade eine Folge für eine Film Serie über die Abenteuer fiktiver Pioneers drehen.
\item Erster Auftritt des End-Bosses: Ein mutierter Riesenhamster entführt einen der Norm Schauspieler
\item Durchsuche die Weltvernichtungsmaschine, löse Puzzle, baue Waffen und finde eine Lösung für den Hamster
\item Die Protagonisten stellen sich einem schweren Dilemma - der Monster Hamster ist Mama
\item Nach Erfolg gibt es eine Abschlussparty beim Errichten der neuen Brauerei. Auch muss man mit den Konsequenzen der eigenen Entscheidungen klar kommen
\end{itemize}

\section{Los geht es - mit einer Party: Die Übersicht}

\begin{sidebarBox}[title=Pioneers]
Pioneers sind eine Gruppe hyperinnovativer Leute mit selbst gebauten Öko freundlichen Hightech Communities. Ein Großteil der von den Norms genutzten Technologien basiert auf deren Konzepte. Während dem "Dreckigen Weg nach Eden" waren sie der unkoordinierte Treiber der Revolution. Heutzutage reden sie nicht über diese Phase oder sie nennen sie "Nötig für die Sicherheit der Menschheit" (was sie auch war). Die meiste Zeit kümmern sie sich nicht um die Vergangenheit, sondern fokussieren sich auf die Zukunft - so wurden viele Details bereits vergessen. Pioneers lieben ihre kreative Gesellschaft. Sind aber sehr individualistisch und jeder hat sein eigenes Lieblingsprojekt am laufen. Da sie nur nach vorne schauen, halten sie den Rückblick in die Vergangenheit meist für Zeitverschwendung. Aus diesem Grund weiß auch niemand, was die Weltvernichtungsmaschine ist. Und es kümmert sie auch nicht. Niemand wird helfen können. Wenn die Protagonisten aber die Lost fragen würden, könnten sie eine Antwort erhalten.
\end{sidebarBox}

Es findet eine große Outdoor Pioneer Party statt. Die Gemeinschaft hat sich versammelt. Selbst gemachte Musik läuft aus den Boxen mit dem üblichen LED und Laser Spektakel. Zum Essen gibt es selbst gezogenes Gemüse - und ein spezielles Getränk wird serviert:

Ein Schnaps Glas gefüllt mit einem neuen Bier. "Das Fass" hat mittels Gentechnik modifizierten Hefen und einem neuen Brauprozess ein köstliches Bier erzeugt. Und es leuchtet dank Biolumineszenz. Unglücklicherweise ist die Menge sehr limitiert. Das aktuelle Labor und die Brauausstattung reichen nicht für größere Mengen. Die Brauerei muss wachsen. Und dazu benötigt die Gemeinschaft Ressourcen Punkte.

\begin{sidebarBox}[title=Ressourcen Punkte]
Ressourcen Punkte sind die Hauptwährung. Um Missbrauch zu verhindern, erhält jede Person eine feste Anzahl dieser Punkte pro Jahr von der UN. Diese Punkte werden benötigt, um nicht-erneuerbare Rohstoffe und darauf basierte Gegenstände zu erwerben. Sie genügen für einen normalen Lebensstil. Aber reichen nicht aus, um eine Brauerei zu gründen. Der einzige Weg, mehr von diesen Punkten zu bekommen ist es, Objekte zu recyceln. Große Objekte oder solche aus seltenen Materialien generieren mehr Ressourcen punkte. Dies ist einer der Hauptgründe, Missionen zu den Ruinen der Lemminge zu starten.
\end{sidebarBox}

Glücklicherweise wurde eine "Weltvernichtungsmaschine" (ein Kohlekraftwerk - aber das wird nie erwähnt) frei gespült. Sie wurde vor vielen Jahren bei einem der Naturkatastrophen verschüttet. Nachdem eine erneute Katastrophe die Hälfte eines Berges weg spülte ist sie wieder entdeckt worden.

Es fand eine Auktion um Bergungsrechte statt. Und die Pioneers dürfen ihr Güter zuerst bergen. Dies wird freudig auf der Party verkündet. Und die Protagonisten haben die Möglichkeit, zur Bergungstruppe zu gehören.

Die Bergungstruppe erhält 4 (oder Anzahl Mitglieder -1) Bergungs-Tags der UN. Diese werden an die zu bergenden Objekte geklebt und können danach nicht mehr entfernt werden. Die Protagonisten können entscheiden, was es wert zu bergen und recyceln ist. Neben den Tags können sie auch so viel bergen, wie sie tragen können.
Andere Gruppen haben sich auch beworben. Diese dürfen aber erst später das Gebiet betreten und bergen, was übrig ist. Dafür haben sie mehr Bergungs Tags erhalten. Die Pioneer Gruppe hat sich bewusst entschieden, als erste Gruppe da rein zu gehen - und dafür weniger Tags zu erhalten.

\begin{sidebarBox}[title=Bergungs Tags]
Die Bergungs Tags sind kleine Aufkleber mit einer Energie Quelle, einem Computer und Funk. Sie können nach dem Aufkleben nicht mehr von einem Objekt entfernt werden und dienen dazu, Beute zu markieren. Am Ende des Abenteuers werden Recyclingspezialisten (NSCs) mit schweren Equipment kommen und die markierten Objekte zerlegen, abtransportieren und recyceln. Die Ressourcen Punkte werden dann der Pioneer Community zugesprochen.
Diese Tags haben ein Display und einen Mikrocontroller. Sie aktivieren sich erst aber der programmierten Zeit. Vor dieser Zeit können sie nicht an den Objekten befestigt werden. Darum sind die Teams - denen verschiedene Zeit Slots zugewiesen wurden - nicht in direkter Konkurrenz bei der Bergung.
Die United Nations haben diese Tags in Auktionen vergeben. Und sie gewähren hinterher auch die Ressourcen Punkte.
\end{sidebarBox}

\emph{Die Bergungs Tags sind ein Spielsystem um die Fluss des Spiels zu verbessern. Sie sind eine Art "bag of holding". Ohne sie müssten die Charaktere 300 Tonnen Generatoren mit sich durch das Abenteuer schleppen.}

% TODO: DOT graphviz goes here

\section{Party}

Inhalt diese Szene:
\begin{itemize}
\item Die Charaktere lernen sich kennen
\item Die Spieler lernen die Spielregeln
\item \textbf{Und ganz wichtig: Das Solarpunk-Gefühl aufbauen}
\end{itemize}

Die Pioneers haben eine Abend-Party draußen auf dem Dorfplatz. Diesmal wird etwas großes verkündet werden. Um bis dahin die Zeit zu verbringen (und die Regeln zu lernen) können die Protagonisten bei vielen der Aktivitäten auf der Party teilnehmen.

Alles ist mit bunten Lichtern dekoriert. Überall Schals und Wimpel. Leute stehen in Gruppen zusammen oder tanzen. In der Mitte des Platzes ist eine große Säule. Deren unterer Teil ist gerade grün beleuchtet.

Eine Ankündigung der Ältesten: "Heute haben wir einige Neuigkeiten. Die Erste: Dorothea hat Kinder !"

Eine Live Schalte wird auf einem großen Bildschirm eingeblendet: ein Fasanennest im nahen Wald. Mit kleinen Fasanen Küken. \textit{Frenetischer Jubel} "Leise Bitte ! Wir haben gerade die Lautstärke Säule hier auf dem Dorfplatz aufgebaut. Weil Brutsaison ist. Sie zeigt die Lautstärke auf den Mikrofonen an, die wir im Wald verteilt haben. Wie jedes Jahr: Wenn sie in den roten Bereich wechselt - bitte etwas leiser sein. Das Musiksystem macht das automatisch. Wie jedes Jahr hat das 5. Drohnen Squadron der Kinder geschworen, die Küken zu schützen, indem sie Katzen, Marder und andere Räuber von den Gelegen fernhalten".

Jetzt fliegt ein Schwarm beleuchteter Quadrocopter in Formation über das Fest. Eine der Drohnen bricht aus der Formation aus, dippt kurz in die Punsch Schüssel und schließt sich wieder an. Die Kinder-piloten Lachen und jubeln.

"So, jetzt feiert erst mal, die zweite große Ankündigung wird in einer Stunde sein".

Jetzt können sich die Charaktere die Zeit auf der Party vertreiben. Diese dient dazu, die Regeln zu lernen und damit sich die Charaktere begegnen.

\begin{itemize}
\item Jonglier Workshop
\item Etwas entspannter: Gemeinsames Gärtnern an den Hochbeeten und Konversation mit lokalen NSCs
\item Die Kinder rasen mit ihren Drohnen durch die Baumwipfel. Manche reparieren ihre Drohnen nach Unfällen. Hier kann man: teilnehmen, bei Reparaturen helfen, Drohnen ausweichen, Drohnen aus Bäumen oder dem Punsch holen
\item E-Motor Herausforderung: Jeder trinkt einen Schnaps. Danach versucht man aus Schrottteilen, einen funktionierenden E-Motor zu bauen. Hier kann man: Teilnehmen, oder Betrunkenen helfen
\item Party Organisation: Jeder, der Interesse hat, kann Aufgaben übernehmen. Cocktails mischen, Musik auflegen oder noch schnell eine Lichtshow programmieren.
\end{itemize}

Direkt vor der Ankündigung des Abends bekommt jeder ein Schnapsglas mit lokal gebrautem Bier. Die Älteste: "Dieses Bier wurde mit unserer eigenen, genmodifizierten Hefe gebraut. Das Team rund um 'Dem Fass' haben es möglich gemacht. (alle jubeln). Wie ihr sehen könnt, leuchtet das Bier im Dunkeln und es schmeckt großartig. Aber ohne großem Bio Labor und einer größeren Brauerei können wir nicht mehr produzieren. Und uns fehlen die Ressourcen Punkte dafür. Die gute Nachricht ist: Die UN gab uns Bergungsrechte bei einer alten Weltvernichtungsmaschine. Sie wurde bei einer Naturkatastrophe verschüttet. Und eine neue Katastrophe hat gerade den halben Berg über ihr weggespült. Lasst und dort die schweren Maschinen und seltenen Metalle bergen. Mit den Ressourcen Punkten, die wir durch das Recycling erhalten können wir unser Brauerei Labor aufbauen !"

"Das Fass" (evtl. ein Spielercharakter) kann auf dem Fest die wichtigsten Fragen beantworten:
\begin{itemize}
\item "Leuchtet man nach dem Trinken selber ?" (Nein)
\item "Leuchtet das Pipi?" (Ja)
\item "Wie lange leuchtet das Pipi?" (Einige Tage)
\item "Kannst du auch leuchtende Limo für die Kinder machen?" (Ja)
\end{itemize}

Nach der Vergabe der Mission brechen die Protagonisten am nächsten Tag auf zur Weltvernichtungsmaschine.

Zuerst per Zug (E-Bikes und E-Quads sind im Güterwagon). Danach fahren sie durch ein relativ neues und wildes Waldstück. Das auf Land wächst, das vor 20 Jahren überflutet wurde.

\section{Camp der Lost}

Inhalt dieser Szene:

\begin{itemize}
\item Die Pioneers treffen die Fraktion der Lost
\item Erstes Treffen mit einem mutierten Riesenhamster
\item Man könnte Waffen besorgen (stehlen, oder kaufen)
\item Man könnte die Lost um Hilfe bitten
\end{itemize}

\begin{sidebarBox}[title=Die Lost]
Die \hyperref[sec:Lost]{Lost} sind Überlebensexperten, Kämpfer und Historiker. Sie reisen durchs Land und suchen nach alter Lemming Technologie. Sie lehnen jede Technologie mit Mikrocontrollern ab, sind aber Experten im wiederverwenden und upcyclen alter Technologie. Ihre Camps sehen etwas mitgenommen aus. Verglichen mit den "Lifestyle" Norms oder den "Hyperaktiven/Hyperkreativen" Pioneers wirken sie wild und rückständig.
Damals, als der "Dreckige Weg nach Eden" begann, die Welt zu transformieren, sahen sie, dass ein hoher Preis zu zahlen ist. Und sie beschlossen - aus ethischen Gründen - nicht an dieser Transformation teilzunehmen.
Die Lost verwenden alle ihnen zustehenden Ressourcen Punkte um Diesel zu kaufen. Aus diesem Grund sind sie auf Streifzüge in den Ruinen angewiesen.
\end{sidebarBox}

Die Protagonisten kommen im Wald an. Vor dem Eingang zur Weltvernichtungsmaschine befindet sich ein Lost Camp. Schwere Diesel Autos parken dort mit laufendem Motor. Holz und Öl brennen in Fässern. Aus alten Planen wurden Zelte errichtet. Alles ist sehr temporär, gebaut aus Material der Vergangenheit. Aber es ist ein praktisches und ordentliches Camp.

In der Mitte rotiert ein Bären großes Tier über einem Grill. Die Lost könnten erklären, dass es ein hier gefangener mutierter Riesenhamster ist.

\begin{sidebarBox}[title=Fehlgeschlagenes CCS Experiment: Monster Hamster]
Gebildete Leute könnten wissen, dass die riesigen Hamster ein fehlgeschlagenes Carbon Capture and Storage Experiment sind aus der Ära des "Dreckigen Wegs nach Eden". Die Lost wissen das. Und jagen dieses gefährliche Tier, wo sie nur können. Der Hamster ist genetisch dazu programmiert, Proteinbrocken (auch um sich schlagende und schreiende) in ihre unterirdischen Hamsterbauten zu zerren. Der Ursprüngliche Plan was es, mittels dieser Hamster Kohlenstoff von der Erdoberfläche zu entfernen.
\end{sidebarBox}

Jemand macht Kartoffelsalat und baut Picknick Bänke auf. Musik spielt. Die Lautsprecher sind verzerrt und mindestens 20 Jahre alt. Aber das scheint niemanden zu interessieren. Im Hintergrund schießt jemand auf Bierdosen mit seiner Schrotflinte. Das ist deren Anführer - Caligula.Auf einem Tisch sind alte Bücher gestapelt.
Die Lost haben selbst 10 Bergungs-Tags von der Auktion bekommen. Mehr als die Pioneers haben. Aber deshalb dürfen sie auch erst als zweite in die Ruine. Die Tags werden sich erst in 12 Stunden aktivieren. Dann können die Lost beginnen zu plündern. Bis dahin feiern sie hier in diesem Wald. Solange die Protagonisten halbwegs zügig vorangehen, werden die Lost keine Konkurrenz sein.

Verhalten: Wenn sie angesprochen werden, ziehen sie die Pioneers auf ihre rustikale Art auf. Und verbieten ihnen, so Dinge wie Diesel, Dieselgeneratoren und so aus der Ruine zu nehmen. Das gehört den Lost. Wenn die Pioneers bei den Frotzeleien mitmachen und sich als würdig erweisen, werden sie zu Hamster, Salat und Bier eingeladen.

Kurz danach aktivieren sich die Bergungs Tags der Pioneers und sie können in die Ruinen der Weltvernichtungsmaschine hinabsteigen.

\begin{npcBox}[title=Caligula]

    \begin{aspects}
    \item \aspect[High Concept]{Wühltisch Indiana Jones}
    \item \aspect[Trouble]{Alkoholbetrieben}
    \end{aspects}

    \begin{skills}
    \item \nskill{Bildung}{3}
    \item \nskill{Athletik}{2}
    \item \nskill{Diebeskunst}{0}
    \item \nskill{Kontakte}{0}
    \item \nskill{Handwerk}{1}
    \item \nskill{Täuschung}{0}
    \item \nskill{Fahren}{1}
    \item \nskill{Charisma}{0}
    \item \nskill{Kämpfen}{1}
    \item \nskill{Nachforschung}{0}
    \item \nskill{Spezialwissen}{2}
    \item \nskill{Wahrnehmung}{2}
    \item \nskill{Kraft}{0}
    \item \nskill{Provozieren}{3}
    \item \nskill{Empathie}{0}
    \item \nskill{Ressourcen}{0}
    \item \nskill{Schießen}{4}
    \item \nskill{Heimlichkeit}{0}
    \item \nskill{Wille}{1}
    \end{skills}

    \begin{stunts}
    \item \stunt{Tuning}{Bekommt +2 auf Schießen wenn er eine Waffe nutzt, die auf einer 1h Trainingssitzung kalibriert und eingestellt wurde.}
    \end{stunts}

    \begin{stressSection}
    \stressLine{\stress{1}\stress{1}\stress{1}}{\stress{1}\stress{1}\stress{1}\stress{1}}
    \end{stressSection}
    \begin{tabularx}{\textwidth}{ XX }
    \end{tabularx}

    \begin{consequences}
    \item \consequence{2}
    \item \consequence{4}
    \item \consequence{6}
    \end{consequences}

    \begin{npcDescription}
    Caligula führt eine kleine Familie von Ruinenplünderern. Sie reisen durch die Wildnis, immer auf der Suche nach den Schätzen der Lemminge. All die nützlichen Dinge, die sie finden werden kreativ wiederverwendet oder recycled.
    Wenn nötig ist er bereit zu kämpfen. Aber er und seine Familie würden lieber über alte Artefakte, Bücher und Ruinen diskutieren. Der erste Eindruck eines Fremden von ihm ist aber der eines Rednecks mit Gewehr.
    \end{npcDescription}

    \end{npcBox}

Er ist nicht alleine, sondern reist mit seiner "Familie" von ca. 10 Leuten. Sie alle sind bewaffnet und gut darin, Ruinen zu scouten. Sie haben eine deutlich pragmatischere Einstellung zur Natur als die Pioneers. Denn sie sind ständig damit beschäftigt sich mit den Kräften der Natur zu arrangieren und sie teils zu bekämpfen um zu überleben.

Wenn die Protagonisten die Lost als Freunde gewinnen, erhalten sie evtl.

\begin{itemize}
    \item Waffen und Leute, die sie benutzen können
    \item Die Erkenntnis, dass die Weltvernichtungsmaschine ein Kohlekraftwerk ist. Evtl. sogar mit einer groben Karte eines solchen
    \item Norms kamen bereits vor 2 Tagen hier an. "Das sah seltsam aus. Die waren nicht vorbereitet für Ruinen. Sogar noch weniger als ihr."
\end{itemize}

\section{Battleground}

Ziel der Szene

\begin{itemize}
\item Man trifft zum ersten mal Norms
\item Man erfährt: Die Weltvernichtungsmaschine ist fast schon langweilig und trostlos entworfen. Was ist das ?
\end{itemize}

\begin{sidebarBox}[title=Norms]

80 Prozent der Leute im Jahr 2050 sind \hyperref[sec:Norms]{Norms}. Sie leben in automatisierten Öko Städten. Diese werden durch  AIs automatisiert. Deren Parameter sind darauf ausgelegt, die maximale Lebensqualität und das maximale Glück sicher zu stellen. Die Gesellschaft ist hochgradig kooperativ. Die Leute müssen nicht arbeiten, suchen sich aber sehr gerne einen Job als Hobby. Die Planung größerer Projekte wird von der KI (und Apps) übernommen und so können große Projekte durchgeführt werden.

Alle Norms tragen einen Hive Controller. Dieses am Kopf getragene Gerät bietet ihnen Apps und ein AR interface, mit dem sie einfach Dinge bei der AI oder in der Hive Gesellschaft bestellen können. Für Pioneers und besonders Lost wirkt es wie Magie.

Während die Norms ihre Jobs durch Kooperation perfekt erfüllen, sind sie anders als Pioneers alleine oder ohne Verbindung zum Hive verloren und fast hilflos.

Sie sind sehr auf das Jetzt fokussiert. Sie kümmern sich weniger um die Vergangenheit oder um Pläne für die ferne Zukunft. Jetzt ist alles perfekt.

In der Region der Weltvernichtungsmaschine haben alle Norms nur limitierten Nutzen durch den Hive Controller und die Apps. Der nächste AI Knoten ist ein gerade so in Funkreichweite stehender Schiffscontainer. Das soziale Netzwerk der dort arbeitenden Norms ist klein. Und wenn man etwas bestellt muss es erst aus der nächsten Stadt per Drohne geliefert werden - 1 Stunde Flug. Doch selbst das kann sich für Außenstehende wie Magie anfühlen. Während die Norms an den Mängeln verzweifeln und teils Entzugserscheinungen haben.

\end{sidebarBox}

Die Charaktere können die Weltvernichtungsmaschine durch eine verbogene Metalltüre betreten und stehen danach in einem langen, weißen Korridor. Alles ist muffig. Der Boden besteht aus Linoleum. Im Gang stehen weiße Schränke ohne Persönlichkeit. Nach Links und rechts führen viele weiße Türen (Plastik mit Holz Optik), An den Türen sind Schilder mit den Namen denen diese Büros gehörten. Hinter den Türen ist Schlamm, Geröll und darin verbackenes Büromaterial.

Einige Biegungen den Gang entlang stoßen die Charaktere auf einen simulierten Unfall. Doch zuerst sieht er realistisch aus. Ein Norm Schauspieler (Delta Awesome) liegt unter einem H-Stahlträger (Schaumstoff, sieht aber nach Stahl aus) eingeklemmt. Dieser Stürzte von der Decke. Ein in der Ecke versteckter Kameramann filmt seine Schreie. Nach deren Plan sollte der Held der Reality Soap jetzt jederzeit von der anderen Seite des Ganges kommen und ihn retten. Unerwartet kommen die Protagonisten - echte Pioneers - zur Rettung.

Geistesgegenwärtig schauspielert Delta Awesome weiter während der Rettung. Und der Kameramann Kevin filmt die Rettungsaktion. Diese stellen sehr bald fest, dass der Träger falsch ist und keine echte Gefahr bestand.

Nachdem sich die Verwirrung gelegt hat und die Situation erklärt ist wartet jeder auf den Helden, der eigentlich zur Rettung hätte schreiten sollen. Theophil Tierlieb. Doch dieser kommt nicht. Statt dessen hört man einen markerschütternden Schrei von weiter innen in der Maschine.

Ein vorsichtiger Blick hinter die nächste Biegung enthüllt: Der erwartete Held, der Schauspieler in der Rolle des \textbf{"Theophil Tierlieb"} wird gerade bewusstlos und blutend von einem bärengroßen Monsterhamster in eine der großen Rohre unter der Decke gezogen. Diese Rohre durchziehen große Teile der Weltvernichtungsmaschine.

Unglücklicherweise sind die Rohre zu klein, als dass Menschen durch krabbeln können. Sie sind aber offensichtlich groß genug, dass eine bewusstlose Person von einem Hamster - dessen natürliches Habitat Tunnel und Rohre sind - hindurch gezogen werden kann. Auch werden die Rohre bei zu starker Belastung brechen. Drohnen könnten dem Biest folgen. Den Rohren selbst aber zu Fuß zu folgen ist möglich, wenn auch schwierig. Oft führen sie durch die Wände.

Die Protagonisten brauchen eine Karte. Und vielleicht Waffen. Als Pioneer improvisiert man.

Am Ende des Korridors finden die Pioneers eine große Halle, die mit Marmor ausgekleidet ist. Dies ist die offzielle Eingangshalle zur Weltvernichtungsmaschine. Und beinhaltet sogar ein kleines Museum.

\begin{npcBox}[title=Kevin\, Kamera]

    \begin{aspects}
    \item \aspect[High Concept]{Kamera und Action}
    \item \aspect[Trouble]{Riskiert für gute Action Szenen alles}
    \end{aspects}

    \begin{skills}
    \item \nskill{Bildung}{1}
    \item \nskill{Athletik}{1}
    \item \nskill{Diebeskunst}{0}
    \item \nskill{Kontakte}{2}
    \item \nskill{Handwerk Filmen}{3}
    \item \nskill{Täuschung}{0}
    \item \nskill{Fahren}{1}
    \item \nskill{Charisma}{3}
    \item \nskill{Kämpfen}{0}
    \item \nskill{Nachforschung}{0}
    \item \nskill{Spezialwissen}{0}
    \item \nskill{Wahrnehmung}{4}
    \item \nskill{Kraft}{0}
    \item \nskill{Provozieren}{0}
    \item \nskill{Empathie}{2}
    \item \nskill{Ressourcen}{0}
    \item \nskill{Schießen}{1}
    \item \nskill{Heimlichkeit}{2}
    \item \nskill{Wille}{0}
    \end{skills}

    \begin{stunts}
    \item \stunt{App basiertes Filmen}{Bekommt +2 beim filmen, wenn eine AI in Reichweite ihn unterstützt}
    \end{stunts}

    \begin{stressSection}
    \stressLine{\stress{1}\stress{1}\stress{1}}{\stress{1}\stress{1}\stress{1}}
    \end{stressSection}
    \begin{tabularx}{\textwidth}{ XX }
    \end{tabularx}

    \begin{consequences}
    \item \consequence{2}
    \item \consequence{4}
    \item \consequence{6}
    \end{consequences}

    \begin{npcDescription}
    Kevin liebt es, die Zuschauer zu unterhalten. Seine Fähigkeiten mit der Kamera und dem AI basierten Schneiden der Filme helfen ihm dabei. Sollte er ein Freund werden, kann er die Publicity des Pioneer Teams boosten - ob sie es wollen oder nicht. Wichtiger sind aber seine Wahrnehmungs Fähigkeiten.
    "Ich wusste sofort, dass ich Kameramann werden wollte, als mich die KI damals als ich 10 wurde dafür vorschlug"

    Er hat keinen Zugriff auf Waffen und kann auch keine bestellen. "Sorry, ich habe das Waffen Tutorial nicht abgeschlossen. Ich kann nichts bestellen".

    \end{npcDescription}

\end{npcBox}


\begin{npcBox}[title=Delta Awesome - spielt das Opfer]

    \begin{aspects}
    \item \aspect[High Concept]{Method acting Schauspieler}
    \item \aspect[Trouble]{Beobachtet und kopiert die echten Pioneers}
    \item \aspect[Aspect]{Bleibt immer in Character}
    \end{aspects}

    \begin{skills}
    \item \nskill{Bildung}{0}
    \item \nskill{Athletik}{2}
    \item \nskill{Diebeskunst}{0}
    \item \nskill{Kontakte}{3}
    \item \nskill{Handwerk (Schauspielern)}{4}
    \item \nskill{Täuschung}{2}
    \item \nskill{Fahren}{0}
    \item \nskill{Charisma}{0}
    \item \nskill{Kämpfen}{0}
    \item \nskill{Nachforschung}{0}
    \item \nskill{Spezialwissen}{1}
    \item \nskill{Wahrnehmung}{1}
    \item \nskill{Kraft}{2}
    \item \nskill{Provozieren}{0}
    \item \nskill{Empathie}{3}
    \item \nskill{Ressourcen}{1}
    \item \nskill{Schießen}{0}
    \item \nskill{Heimlichkeit}{0}
    \item \nskill{Wille}{1}
    \end{skills}

    \begin{stunts}
    \item \stunt{Acting}{Kann Leute mittels Schauspielern überzeugen, bei seiner heroischen Mission dabei zu sein.}
    \end{stunts}

    \begin{stressSection}
    \stressLine{\stress{1}\stress{1}\stress{1}\stress{1}}{\stress{1}\stress{1}\stress{1}\stress{1}}
    \end{stressSection}
    \begin{tabularx}{\textwidth}{ XX }
    \end{tabularx}

    \begin{consequences}
    \item \consequence{2}
    \item \consequence{4}
    \item \consequence{6}
    \end{consequences}

    \begin{npcDescription}
    Delta Awesome ist der Charakter Name. Seine Rolle ist die eines erfahrenen Pioneer Experten. Aber er hat in der Realität keinerlei der nötigen Fähigkeiten. Er besteht auf Method acting und will in der Rolle bleiben (andernfalls braucht er wieder 2h um rein zu kommen). Auch wird er permanent seiner Rolle verbessern wollen, indem er die Pioneers beobachtet und kopiert.

    Delta Awesomes mitgeführte Gadgets sind nutzlose Film Props. In der Serie stellen sie aber immer genau das dar, was er in der Szene gerade braucht.

    Da er für seine Rolle trainiert hat, hat er echte Muskeln entwickelt - das könnte nützlich sein. Die Muskeln und seine Fähigkeiten durch Charisma und Schauspielern Leute zu überzeugen können sehr nützlich werden. Aber dafür muss er erst von der Mission überzeugt werden.
    \end{npcDescription}

\end{npcBox}


\section{Die Ausstellung}

Inhalt der Szene:

\begin{itemize}
\item Erster klarer Beweis für ein Kohlekraftwerk - wenn man die Ausstellung anschaut
\item Socializing mit den Norms
\item Herausfinden, wohin die Rohre führen (in den Modellen der Ausstellung und den Plänen)
\item Hier findet man mehrere Kilogramm der Proteinpaste, sie könnte nützlich werden
\end{itemize}

Der Kameramann Kevin und Delta Awesome führen die Protagonisten schnell zum "Hauptquartier". Ein altes Museum und jetzt eine Film Location. Hier ist Catering aufgebaut. Nach dem Filmen dieser Folge planen die Norms hier 500 Norm-Fans der Serie zu empfangen. Und 10 VIPs. Gerade wird die Party Location von Cherie aufgebaut.

Im alten Museum, das am Rand der Halle steht konnten Schulklassen alles über Kohlekraft lernen. Hilfreiche Pläne und Modelle gibt es genug.

Alles hier ist schön gebaut. Und es gibt ein Maskottchen. Die Mineraliensammlung ist auch sehr interessant. Besonders die große Geode dürfte Disco interessieren.

In der Catering Ecke bereitet die Food Designerin Cherie real aussehende Mehlwurm Sandwiches aus 3D gedruckter Protein Paste zu. Diese werden für einige Aufnahmen benötigt und werden auch zum Empfang hinterher gereicht. So muss kein Norm echte Mehlwürmer essen.

Cherie kann den 3D Drucker per App steuern und praktisch jede Form und Geschmacksrichtung drucken. Auch ein falscher Menschenkörper als Köder für den Hamster wäre denkbar.

Laut ihr ist der Rest der Filmcrew tiefer in die Weltvernichtungsmaschine unterwegs, um weitere Drehs vorzubereiten. Sie hat schon länger nichts von ihnen gehört. Der Zugang dazu geht durch eine abgeschlossene Stahltüre.

Jemand mit historischem Wissen (Books) könnte erkennen, dass der wertvollste Teil der ganzen Anlage wohl der Stromgenerator selbst ist - besonders schwer ist dort wohl das Schwungrad. Diese können weit tiefer in der Anlage gefunden werden.

Cherie hat einen Schlüssel zu der Türe, die tiefer hineinführt. Man kann ihn stehlen, sie überzeugen, das Schloß knacken oder die Türe aufschweißen.


\begin{npcBox}[title=Cherie]

    \begin{aspects}
    \item \aspect[High Concept]{Essens-Künstlerin}
    \item \aspect[Trouble]{Willst du mein Freund sein ?}
    \item \aspect[Aspect]{Essen muss lecker und wunderschön sein}
    \end{aspects}

    \begin{skills}
    \item \nskill{Bildung}{1}
    \item \nskill{Athletik}{1}
    \item \nskill{Diebeskunst}{0}
    \item \nskill{Kontakte}{2}
    \item \nskill{Handwerk (Kochkunst)}{4}
    \item \nskill{Täuschung}{1}
    \item \nskill{Fahren}{0}
    \item \nskill{Charisma}{1}
    \item \nskill{Kämpfen}{0}
    \item \nskill{Nachforschung}{0}
    \item \nskill{Spezialwissen}{0}
    \item \nskill{Wahrnehmung}{3}
    \item \nskill{Kraft}{0}
    \item \nskill{Provozieren}{0}
    \item \nskill{Empathie}{3}
    \item \nskill{Ressourcen}{2}
    \item \nskill{Schießen}{0}
    \item \nskill{Heimlichkeit}{0}
    \item \nskill{Wille}{2}
    \end{skills}

    \begin{stressSection}
    \stressLine{\stress{1}\stress{1}\stress{1}}{\stress{1}\stress{1}\stress{1}\stress{1}}
    \end{stressSection}
    \begin{tabularx}{\textwidth}{ XX }
    \end{tabularx}

    \begin{consequences}
    \item \consequence{2}
    \item \consequence{4}
    \item \consequence{6}
    \end{consequences}

    \begin{npcDescription}
    Macht das Catering und simuliert eine Pioneer Welt für die Fans der Serie und andere Gäste (die werden in ein paar Stunden eintreffen).
    Sie liebt es, bei der Arbeit zu reden und ist immer positiv und fröhlich. Eine Herausforderung wie "Drucke mal aus der Protein Paste mit deinem 3D Drucker einen Körper" würde sie mit Freude akzeptieren. Evtl. würde das den Hamster ablenken.

    Das Monster nebenan macht ihr aber sehr viel Angst, sobald sie davon erfährt. Sie wuchs in einer maximal behüteten Norm Stadt auf.

    \end{npcDescription}

\end{npcBox}



\section{Der Kohlebunker}

Inhalt der Szene:
\begin{itemize}
\item Überwinden technischer Probleme
\item Man kann Waffen bauen
\item Man erfährt die Dreckigkeit der Weltvernichtungsmaschine an der eigenen Haut
\end{itemize}

Probleme:

\begin{itemize}
\item Viel explosiver trockener Kohlestaub
\item Große Pfützen dunkles schwarzes Wasser - mit Ölfilm
\item Die Norms haben Pyrotechnik und anderes Spezialeffekt Material aufgestellt. Beängstigend: Die Kabel im Wasser und Pyrotechnik naher der Kohle
\item Einige der alten Maschinen wurden von den Norms für den Film wiederbelebt. Das sieht sicher gut auf dem Film aus - kann aber auch eine Todesfalle sein.
\end{itemize}

Waffenmaterial:

\begin{itemize}
\item Kohlestaub (für Kartoffelkanonen, Rohrbomben)
\item Rohre von den brüchigen Geländern
\item Sprengstoffe von den Spezialeffekten
\end{itemize}


Nach einem dreckigen Korridor kommen die Protagonisten in eine riesige Halle. Auf Schienen stehen und liegen Kohlewagen. Manche wurden in der Katastrophe damals auch zerquetscht und schwer beschädigt. Hier wurde die gelieferte Kohle auf Qualität geprüft. Zermahlen zu Pallets und Staub und mit Förderbändern die Halle entlang transportiert. Vieles davon kann man hier noch sehen. Aber in erbärmlichem Zustand.

Alles ist verrostet. Kohlestaub hängt in der Luft (und ist explosiv). Auf dem Boden sind schwarze Pfützen aus Kohle, Öl und Wasser.

Wenigstens ist offensichtlich, wohin die Norms gegangen sind. Sie ließen Batterien, Lichter und Pyrotechnik zurück. Verstreut im ganzen Gebiet. Um alles schnell zu entschärfen wäre Parkour, Navigation, Metallschneider und Sprengstoffe Entschärfen nötig. Also viele Würfe auf die buntesten Fähigkeiten.

Das Kohleförderband führt in den nächsten Raum. Hier wollen die Protagonisten hin.

\section{Laufstege}

Inhalt der Szene:
\begin{itemize}
\item Hindernisse überwinden
\item Aufzeigen der Zerstörung, aber auch der immensen Größe der Weltvernichtungsmaschine
\item Waffen bauen
\end{itemize}

Die Protagonisten müssen über Stege und durch riesige laufende Ventilationslüfter klettern.

Bei diesen sorgt die Beleuchtung von hinten und die laufende Nebelmaschine für ein gruseliges Aussehen. Die Spezialeffekt Leute waren bereits da. Auf dem Film sieht das sicher großartig aus.

Mutierte grün leuchtende Bovist Pilze wachsen auf dem Boden unterhalb des Laufstegs. Mit etwas Biologie Wissen weiß man, dass die Sporen psychoaktiv sind. Der Film Direktor liegt bewusstlos neben den Pilzen. Er hat die psychoaktive Wirkung direkt erfahren. Unten ist bereits ein Makeup Tisch aufgebaut. Bald soll hier gefilmt werden.

Probleme:
\begin{itemize}
\item Bröckelige Laufstege
\item Man sieht unter der Decke das Labyrinth aus Rohren - durch das sich der Hamster bewegt.
\item Mutierte Pilze - der Direktor muss gerettet werden.
\end{itemize}

Waffenmaterial:

\begin{itemize}
\item Scharfe Klingen aus der Lüftung
\item Rohrstücke (für Rohrbomben, Speere, eine Kartoffelkanone)
\item Psychoaktive Pilze (beim Ernten an passende Schutzkleidung denken !)
\end{itemize}

Das Laufband führt zu einer Brennkammer (die ist gerade nicht zugänglich). Der nächste danach Raum ist der Generatorraum. Hier ist das Hamsternest.
In diesem Raum sieht man bereits die Dampf Rohre, die dorthin führen.

\begin{npcBox}[title=Lucien\, Director]

    \begin{aspects}
    \item \aspect[High Concept]{Film Director mit einem Händchen für Blockbuster}
    \item \aspect[Trouble]{Darin ist doch eine Geschichte....}
    \end{aspects}

    \begin{skills}
    \item \nskill{Bildung}{1}
    \item  \nskill{Athletik}{0}
    \item \nskill{Diebeskunst}{0}
    \item \nskill{Kontakte}{3}
    \item \nskill{Handwerk (Film Director)}{4}
    \item \nskill{Täuschung}{0}
    \item \nskill{Fahren}{0}
    \item \nskill{Charisma}{1}
    \item \nskill{Kämpfen}{0}
    \item \nskill{Nachforschung}{1}
    \item \nskill{Spezialwissen}{0}
    \item \nskill{Wahrnehmung}{2}
    \item \nskill{Kraft}{0}
    \item \nskill{Provozieren}{0}
    \item \nskill{Empathie}{2}
    \item \nskill{Ressourcen}{3}
    \item \nskill{Schießen}{0}
    \item \nskill{Heimlichkeit}{1}
    \item \nskill{Wille}{2}
    \end{skills}

    \begin{stressSection}
    \stressLine{\stress{1}\stress{1}\stress{1}}{\stress{1}\stress{1}\stress{1}\stress{1}}
    \end{stressSection}
    \begin{tabularx}{\textwidth}{ XX }
    \end{tabularx}

    \begin{consequences}
    \item \consequence{2}
    \item \consequence{4}
    \item \consequence{6}
    \end{consequences}

    \begin{npcDescription}
    Er ist stoned wenn man ihn findet und wird sich nicht bis zum Ende der Geschichte erholen. Könnte während der Party aber ein interessanter Freund sein.
    \end{npcDescription}

\end{npcBox}

\section{Nest}

Inhalt der Szene:
\begin{itemize}
\item Endkampf
\end{itemize}

Im Nest können alle Arten von organischem Material gefunden werden. Von alten Kartoffelsäcken bis zu toten Tieren (gejagte Hunde und Wildschweine)

Es ist verwirrend und voller Artefakte der alten Zivilisation.

Der Hamster hat den bewusstlosen Norm auf den Gipfel des Haufens gezerrt. Hier wird er bald sterben.
Der besondere Schatz hier sind aber vier riesige Generatoren mit massiven Schwungrädern. Das ist der Schatz, den man für die Bergung markieren kann (wenn man noch Bergungsmarker hat). Aber vorher muss man am Hamster vorbei.

Ideen für Lösungen:
\begin{itemize}
\item Mann kann dafür sorgen, dass sich der Hamster an dem Protein aus dem 3D Drucker überfrisst (Bio Wissen, um die Nahrungsmenge perfekt zu portionieren)
\item Oder ihn vergiften mit den psychoaktiven Pilzen (Waffen oder Bio Skills)
\item Oder den Hamster im Kampf töten
\item Oder die Lost zu Hilfe holen (Soziale Skills)
\item Hinein schleichen, den Verletzten retten, die Bergungsmarker ankleben und raus
\item Den Hamster mit Drohnen, Pyro- und Spezialeffekten verwirren, ...
\end{itemize}


\begin{npcBox}[title=Hamster]

    \begin{aspects}
    \item \aspect[High Concept]{Flauschige Killermaschine auf einer CCS Mission}
    \item \aspect[Trouble]{Von den Genen verdammt}
    \item \aspect[Aspect]{Immer hungrig auf Protein}
    \end{aspects}

    \begin{skills}
    \item \nskill{Bildung}{0}
    \item \nskill{Athletik}{2}
    \item \nskill{Diebeskunst}{0}
    \item \nskill{Kontakte}{0}
    \item \nskill{Handwerk (Nestbau)}{1}
    \item \nskill{Täuschung}{2}
    \item \nskill{Fahren}{0}
    \item \nskill{Charisma}{0}
    \item \nskill{Kämpfen}{3}
    \item \nskill{Nachforschung}{0}
    \item \nskill{Spezialwissen}{0}
    \item \nskill{Wahrnehmung}{3}
    \item \nskill{Kraft}{4}
    \item \nskill{Provozieren}{1}
    \item \nskill{Empathie}{0}
    \item \nskill{Ressourcen}{0}
    \item \nskill{Schießen}{0}
    \item \nskill{Heimlichkeit}{2}
    \item \nskill{Wille}{0}
    \end{skills}

    \begin{stunts}
    \item \stunt{Durch die Rohre}{Mittels \textbf{Wahrnehmung} kann der Hamster in eine der vielen Rohröffnungen kriechen und eine Runde später an einem taktisch perfekten Punkt wieder aus dem Rohr kommen. Der Hamster erhält so einen Vorteil von +2 auf den nächsten Angriff. Indem man bei   \textbf{Wahrnehmung} besser als der Hamster ist, findet man seinen geplanten Weg heraus und negiert den Effekt.}
    \end{stunts}

    \begin{stressSection}
    \stressLine{\stress{1}\stress{1}\stress{1}\stress{1}\stress{1}\stress{1}}{\stress{1}\stress{1}\stress{1}}
    \end{stressSection}
    \begin{tabularx}{\textwidth}{ XX }
    \end{tabularx}

    \begin{consequences}
    \item \consequence{2}
    \item \consequence{4}
    \item \consequence{6}
    \end{consequences}

    \begin{npcDescription}
    Der Hamster ist Bärengroß und kann sehr aggressive werden. Ihr normales Ziel ist es, Proteine zu finden und hier nach unten zu zerren um sie zu lagern. Sollten die Protagonisten den Raum durchsuchen, finden sie heraus. Der Hamster ist weiblich und hat 4 Baby Hamster in einem Nest unter einem Generator, die sie beschützt.
    \end{npcDescription}

\end{npcBox}


\begin{npcBox}[title=Theophil Tierlieb]

    \begin{aspects}
    \item \aspect[High Concept]{Schauspieler in der Rolle des "Theophil Tierlieb" - ein Tierflüsterer}
    \item \aspect[Trouble]{In Wahrheit hassen mich Tiere}
    \item \aspect[Aspect]{Hamsterfutter}
    \end{aspects}

    \begin{skills}
    \item \nskill{Bildung}{0}
    \item \nskill{Athletik}{1}
    \item \nskill{Diebeskunst}{0}
    \item \nskill{Kontakte}{3}
    \item \nskill{Handwerk (Schauspielern)}{4}
    \item \nskill{Täuschung}{1}
    \item \nskill{Fahren}{0}
    \item \nskill{Charisma}{3}
    \item \nskill{Kämpfen}{1}
    \item \nskill{Nachforschung}{0}
    \item \nskill{Spezialwissen}{2}
    \item \nskill{Wahrnehmung}{2}
    \item \nskill{Kraft}{2}
    \item \nskill{Provozieren}{1}
    \item \nskill{Empathie}{0}
    \item \nskill{Ressourcen}{0}
    \item \nskill{Schießen}{0}
    \item \nskill{Heimlichkeit}{0}
    \item \nskill{Wille}{0}
    \end{skills}

    \begin{stressSection}
    \stressLine{\markedstress{1}\markedstress{1}\markedstress{1}\markedstress{1}}{\stress{1}\stress{1}\stress{1}}
    \end{stressSection}
    \begin{tabularx}{\textwidth}{ XX }
    \end{tabularx}

    \begin{consequences}
    \item \consequence{2} Kopfschmerzen
    \item \consequence{4} Gebrochene Knochen
    \item \consequence{6} Starke Blutungen
    \end{consequences}

    \begin{npcDescription}
    Er spielt einen Tierflüsterer. Aber nach 2 Episoden fand man heraus, dass Tiere ihn hassen. Also leider er als Schauspieler drei Staffeln lang, bis er von einem echten Monster Hamster angegriffen wird.....aber die Zuschauer lieben seinen Charakter.
    \end{npcDescription}

\end{npcBox}


\section{Der wirkliche Endboss}

Nach dem Kampf finden die Protagonisten vier kleine Monster Hamster (klein = die Größe eines Hundes). Sie sind die Kinder der Mutter, die sie gerade getötet haben. Sie sind alt genug, um ohne ihre Mutter zu überleben - wenn sich jemand um sie kümmert. Bisher haben sie noch keine Menschen angegriffen - könnten das aber in der Zukunft tun. Jetzt müssen die Protagonisten entscheiden:

\begin{itemize}
    \item Nehmen wir sie mit zur Pioneer Community und kümmern uns um sie ?
    \item Töten wir sie ?
    \item Überlassen wir sie ihrem Schicksal ?
    \item Verkaufen wir sie an die Lost ? Dort würden sie wahrscheinlich gemästet und später getötet.
\end{itemize}

Man sieht gleich. Der wahre Endboss ist ein Dilemma. Und man sollte es sehr dramatisch ausschmücken. Am besten ist eine 5 minütige Diskussion zwischen den Charakteren. Es muss klar sein: Dies Ressoucenpunkte, die man verdient hat reichen entweder für die Brauerei oder die Artgerechte Haltung in einem Monsterkäfig.

Hat die Mutter überlebt (weil sie z.B. mit den Pilzen betäubt wurde) ist es möglich, die wilden Kreaturen hier zu lassen und alle Menschen aus dem Gebiet zu evakuieren. Das reduziert evtl. die Beute.

\section{Siegesparty}

Inhalt:

\begin{itemize}
\item Abschluss des Abenteuers mit einer Feier
\item Den konsequenzen der Entscheidungen ins Gesicht sehen
\end{itemize}

Ein paar Tage später sind die Ressourcen Punkte gegen reale Ressource in Form von Baumaterial getauscht. Diese Materialien kommen an der Pioneer Community a und die Brauerei kann gebaut werden. Die Errichtung findet im Pioneer Stil mit einer großen Bauparty statt. Musik, Essen und Getränke sind bereit und jeder packt an.

Freunde, die man auf dem Abenteuer gemacht hat sind eingeladen. Sie werden ein teil der Feier sein.

Wurden die kleinen Monsterhamster gerettet, muss zuerst ein Monsterkäfig (inklusive Laufrad und Röhren) gebaut werden. Und wahrscheinlich reichen die Ressourcenpunkte nicht für Käfig und Brauerei....

\section{Spielercharaktere}

\begin{itemize}
\item Books: Gelehrter, will alte Dinge retten
\item Curly: Akrobat, will irgend etwas lustiges finden
\item Das Fass: Brauer, will so große Objekte wie möglich bergen, um an Ressourcen Punkte für die Brauerei zu kommen
\item Disco: Barde, will schöne Dinge bergen
\item Spark: Techniker, will spannende Technologie finden
\item Primel: Ökologe, will die Natur retten
\end{itemize}

Alle Charaktere können einfach umgegendered werden - und haben absichtlich genderneutrale Spitznamen.

%%%%%%%%%% Books
\newpage
\begin{npcBox}[title=Books]

    \begin{aspects}
    \item \aspect[High Concept]{Gelehrter - erst denken, dann handeln}
    \item \aspect[Trouble]{Ohne plan machen wir gar nichts}
    \item \aspect[Relationship]{Ich habe einen Artikel geschrieben, würdest du ihn bitte reviewen ?}
    \item \aspect[Aspect]{Umso mehr ich weiß, umso besser die Ergebnnisse}
    \end{aspects}

    \begin{skills}
    \item \nskill{Bildung}{4}
    \item \nskill{Athletik}{2}
    \item \nskill{Diebeskunst}{0}
    \item \nskill{Kontakte}{0}
    \item \nskill{Handwerk}{1}
    \item \nskill{Täuschung}{0}
    \item \nskill{Fahren}{1}
    \item \nskill{Charisma}{0}
    \item \nskill{Kämpfen}{2}
    \item \nskill{Nachforschung}{3}
    \item \nskill{Spezialwissen}{0}
    \item \nskill{Wahrnehmung}{3}
    \item \nskill{Kraft}{0}
    \item \nskill{Provozieren}{0}
    \item \nskill{Empathie}{2}
    \item \nskill{Ressourcen}{0}
    \item \nskill{Schießen}{1}
    \item \nskill{Heimlichkeit}{0}
    \item \nskill{Wille}{1}
    \end{skills}

    \begin{stunts}
    \item \stunt{E-Book}{Solange ich mein geliebtes E-book habe, bekomme ich  +2 auf Bildung}
    \end{stunts}

    \begin{stressSection}
    \stressLine{\stress{1}\stress{1}\stress{1}}{\stress{1}\stress{1}\stress{1}\stress{1}}
    \end{stressSection}
    \begin{tabularx}{\textwidth}{ XX }
    \end{tabularx}

    \begin{consequences}
    \item \consequence{2}
    \item \consequence{4}
    \item \consequence{6}
    \end{consequences}

    \begin{npcDescription}
    Jäger des Wissens. Trägt ein Jackett mit Flicken am Ellbogen. Seine Haare werden langsam grau. Sein Sinn für Geschmack drückt sich durch seinen eleganten Hut aus.
    \end{npcDescription}


    \begin{equipment}
    \item Lichtquelle (OLED Folie: Batteriebetrieben, kann in jede Größe und Form geschnitten werden und angeklebt. Die Farbe ist programmierbar)
    \item Essen (Mama Salsa's berühmtes Mehlwurm Brötchen in der Bento Box)
    \item Erste Hilfe Ausstattung
    \item Funkvernetzte Mobilcomputer, Kopfhörer, Radio Mesh taugliche KommunikationMobile computers, headphones, communication via radio (mesh network)
    \end{equipment}

\end{npcBox}

%%%%%%%%%%% Curly
\newpage
\begin{npcBox}[title=Curly]

    \begin{aspects}
    \item \aspect[High Concept]{Kindischer Kletter Akrobat}
    \item \aspect[Trouble]{Geheimer Fan einer schlechten Norm Fernsehserie über Pioneers}
    \item \aspect[Relationship]{Sucht in der Gruppe immer nach Rollenmodellen}
    \item \aspect[Aspect]{Lass mal schauen, ob ich damit was lustiges machen kann}
    \end{aspects}

    \begin{skills}
    \item \nskill{Bildung}{0}
    \item \nskill{Athletik}{4}
    \item \nskill{Diebeskunst}{2}
    \item \nskill{Kontakte}{0}
    \item \nskill{Handwerk}{1}
    \item \nskill{Täuschung}{2}
    \item \nskill{Fahren}{0}
    \item \nskill{Charisma}{0}
    \item \nskill{Kämpfen}{3}
    \item \nskill{Nachforschung}{0}
    \item \nskill{Spezialwissen}{0}
    \item \nskill{Wahrnehmung}{2}
    \item \nskill{Kraft}{1}
    \item \nskill{Provozieren}{0}
    \item \nskill{Empathie}{1}
    \item \nskill{Ressourcen}{0}
    \item \nskill{Schießen}{1}
    \item \nskill{Heimlichkeit}{3}
    \item \nskill{Wille}{0}
    \end{skills}

    \begin{stunts}
    \item \stunt{E-Balanceschwanz}{Dank meines Furry Balanceschwanzes habe ich +2 auf Akrobatik beim balancieren}
    \end{stunts}

    \begin{stressSection}
    \stressLine{\stress{1}\stress{1}\stress{1}\stress{1}}{\stress{1}\stress{1}\stress{1}}
    \end{stressSection}
    \begin{tabularx}{\textwidth}{ XX }
    \end{tabularx}

    \begin{consequences}
    \item \consequence{2}
    \item \consequence{4}
    \item \consequence{6}
    \end{consequences}

    \begin{npcDescription}
    Rasierter Kopf, mit Schuppenmuster tätowiert. Trägt einen selbstgemachten Balanceanzug (mit Balanceschwanz).
    Der Oberkörper ist nackt und zeigt Muskeln im Bruce Lee Stil. Curly trägt Cargo Pants um Dinge herumzuschleppen und Akrobatenschuhe mit offenen Zehen für den besseren Halt.
    \end{npcDescription}


    \begin{equipment}
    \item Lichtquelle (OLED Folie: Batteriebetrieben, kann in jede Größe und Form geschnitten werden und angeklebt. Die Farbe ist programmierbar)
    \item Essen (Mama Salsa's berühmtes Mehlwurm Brötchen in der Bento Box)
    \item Erste Hilfe Ausstattung
    \item Funkvernetzte Mobilcomputer, Kopfhörer, Radio Mesh taugliche KommunikationMobile computers, headphones, communication via radio (mesh network)
    \item Ein einfaches Exoskelet mit Balance-Schwanz
    \item Kletterausstattung mit Seil
    \end{equipment}
\end{npcBox}


%%%%%%% The Barrel

\newpage
\begin{npcBox}[title=Das Fass]

    \begin{aspects}
    \item \aspect[High Concept]{Guter Zuhörer}
    \item \aspect[Trouble]{Liebt es von Hefen und Mikroorganismen zu reden}
    \item \aspect[Relationship]{Aktives Zuhören "Wie geht es dir damit?"}
    \item \aspect[Aspect]{Treibende Kraft des Bier Projekts}
    \end{aspects}

    \begin{skills}
    \item \nskill{Bildung (Genetik)}{3}
    \item \nskill{Athletik}{1}
    \item \nskill{Diebeskunst}{0}
    \item \nskill{Kontakte}{0}
    \item \nskill{Handwerk}{3}
    \item \nskill{Täuschung}{0}
    \item \nskill{Fahren}{1}
    \item \nskill{Charisma}{4}
    \item \nskill{Kämpfen}{2}
    \item \nskill{Nachforschung}{0}
    \item \nskill{Spezialwissen}{0}
    \item \nskill{Wahrnehmung}{2}
    \item \nskill{Kraft}{2}
    \item \nskill{Provozieren}{0}
    \item \nskill{Empathie}{1}
    \item \nskill{Ressourcen}{0}
    \item \nskill{Schießen}{0}
    \item \nskill{Heimlichkeit}{0}
    \item \nskill{Wille}{1}
    \end{skills}

    \begin{stunts}
    \item \stunt{Empathie}{Weil ich sehr empathisch bin, bekomme ich +2 auf Empathie. Unglücklicherweise lässt mich das Problem danach lange nicht los.}
    \end{stunts}

    \begin{stressSection}
    \stressLine{\stress{1}\stress{1}\stress{1}\stress{1}}{\stress{1}\stress{1}\stress{1}\stress{1}}
    \end{stressSection}
    \begin{tabularx}{\textwidth}{ XX }
    \end{tabularx}

    \begin{consequences}
    \item \consequence{2}
    \item \consequence{4}
    \item \consequence{6}
    \end{consequences}

    \begin{npcDescription}
    Kuschelig, Bärig, Stark. Will die Kunst des Brauens meistern und setzt dabei auf Gentechnik. Trägt ein Flanellhemd, 2/4 Hosen und hat einen kleinen Bierbauch. Hat sich seit mindestens 3 Tagen nicht rasiert.
    \end{npcDescription}


    \begin{equipment}
    \item Lichtquelle (OLED Folie: Batteriebetrieben, kann in jede Größe und Form geschnitten werden und angeklebt. Die Farbe ist programmierbar)
    \item Essen (Mama Salsa's berühmtes Mehlwurm Brötchen in der Bento Box)
    \item Erste Hilfe Ausstattung
    \item Funkvernetzte Mobilcomputer, Kopfhörer, Radio Mesh taugliche Kommunikation
    \item Gen Labor im Koffer
    \item 2 Flaschen leuchtendes Bier (ein Prototyp)
    \end{equipment}
\end{npcBox}


%%%%%%%% Disco
\newpage
\begin{npcBox}[title=Disco]

    \begin{aspects}
    \item \aspect[High Concept]{Kämpfer für bunte Lichter und die ewige Party}
    \item \aspect[Trouble]{Fühlt sich in ernsten Situationen unwohl}
    \item \aspect[Relationship]{Wünscht sich, dass alle miteinander klar kommen}
    \item \aspect[Aspect]{Schaut nach schönen Dingen}
    \end{aspects}

    \begin{skills}
    \item \nskill{Bildung}{0}
    \item \nskill{Athletik}{1}
    \item \nskill{Diebeskunst}{0}
    \item \nskill{Kontakte}{2}
    \item \nskill{Handwerk (SFX)}{3}
    \item \nskill{Täuschung}{1}
    \item \nskill{Fahren}{1}
    \item \nskill{Charisma}{3}
    \item \nskill{Kämpfen}{0}
    \item \nskill{Nachforschung}{0}
    \item \nskill{Spezialwissen}{0}
    \item \nskill{Wahrnehmung}{2}
    \item \nskill{Kraft}{1}
    \item \nskill{Provozieren}{0}
    \item \nskill{Empathie}{4}
    \item \nskill{Ressourcen}{0}
    \item \nskill{Schießen}{2}
    \item \nskill{Heimlichkeit}{0}
    \item \nskill{Wille}{0}
    \end{skills}

    \begin{stunts}
    \item \stunt{Disco !}{Weil ich ein Disco Künstler bin bekomme ich +2 wenn ich mittels Handwerk (SFX) die Stimmung von Menschen oder Tieren beeinflussen will.}
    \end{stunts}

    \begin{stressSection}
    \stressLine{\stress{1}\stress{1}\stress{1}\stress{1}}{\stress{1}\stress{1}\stress{1}}
    \end{stressSection}
    \begin{tabularx}{\textwidth}{ XX }
    \end{tabularx}

    \begin{consequences}
    \item \consequence{2}
    \item \consequence{4}
    \item \consequence{6}
    \end{consequences}

    \begin{npcDescription}
    Zappeliges buntes Party Kind. Die Kleidung ist über die letzten Jahre um immer mehr seltsame Accessoirs angewachsen.
    \end{npcDescription}


    \begin{equipment}
    \item Lichtquelle (OLED Folie: Batteriebetrieben, kann in jede Größe und Form geschnitten werden und angeklebt. Die Farbe ist programmierbar)
    \item Essen (Mama Salsa's berühmtes Mehlwurm Brötchen in der Bento Box)
    \item Erste Hilfe Ausstattung
    \item Funkvernetzte Mobilcomputer, Kopfhörer, Radio Mesh taugliche Kommunikation
    \item Ein dutzend mini Leucht Drohnen für Festivals
    \item Disco Sound Ausstattung (Lautsprecher, Aufnahme, Mikrofone, alles kabellos verbunden)
    \end{equipment}
\end{npcBox}

%%%%%%%%% Primrose
\newpage
\begin{npcBox}[title=Primel]

    \begin{aspects}
    \item \aspect[High Concept]{Liebenswerter und pazifistischer Öko Terrorist}
    \item \aspect[Trouble]{Die Natur kommt ohne Menschen besser klar}
    \item \aspect[Relationship]{Die Natur ist großartig. Menschen sind auch ok.}
    \item \aspect[Aspect]{Neue Natur finden und sie bewahren}
    \end{aspects}

    \begin{skills}
    \item \nskill{Bildung (Biologie und Ökologie)}{4}
    \item \nskill{Athletik}{3}
    \item \nskill{Diebeskunst}{2}
    \item \nskill{Kontakte}{0}
    \item \nskill{Handwerk (Sprengstoffe)}{2}
    \item \nskill{Täuschung}{1}
    \item \nskill{Fahren}{0}
    \item \nskill{Charisma}{1}
    \item \nskill{Kämpfen}{0}
    \item \nskill{Nachforschung}{0}
    \item \nskill{Spezialwissen}{0}
    \item \nskill{Wahrnehmung}{1}
    \item \nskill{Kraft}{0}
    \item \nskill{Provozieren}{0}
    \item \nskill{Empathie}{2}
    \item \nskill{Ressourcen}{0}
    \item \nskill{Schießen}{0}
    \item \nskill{Heimlichkeit}{3}
    \item \nskill{Wille}{1}
    \end{skills}

    \begin{stunts}
    \item \stunt{Do Drugs}{Weil ich ein erfahrener Öko Terrorist bin, bekomme ich +2 wenn ich mittels Bildung(Biologie) psychoaktive Substanzen nutzen.}
    \end{stunts}

    \begin{stressSection}
    \stressLine{\stress{1}\stress{1}\stress{1}}{\stress{1}\stress{1}\stress{1}\stress{1}}
    \end{stressSection}
    \begin{tabularx}{\textwidth}{ XX }
    \end{tabularx}

    \begin{consequences}
    \item \consequence{2}
    \item \consequence{4}
    \item \consequence{6}
    \end{consequences}

    \begin{npcDescription}
    Die Kleidung besteht aus Öko Textilien - das wäre mit heutiger Technologie nicht nötig. Es ist eine bewusste Entscheidung. Primel ist ein Öko Hippie mit Rasta Haaren und selbstgehäkeltem Shirt.
    \end{npcDescription}


    \begin{equipment}
    \item Lichtquelle (OLED Folie: Batteriebetrieben, kann in jede Größe und Form geschnitten werden und angeklebt. Die Farbe ist programmierbar)
    \item Essen (Mama Salsa's berühmtes Mehlwurm Brötchen in der Bento Box)
    \item Erste Hilfe Ausstattung
    \item Funkvernetzte Mobilcomputer, Kopfhörer, Radio Mesh taugliche Kommunikation
    \item Schlossknacker Set
    \item Ein kleines Biolabor in der Schachtel
    \end{equipment}
\end{npcBox}


%%%%%%%% Spark
\newpage
\begin{npcBox}[title=Spark]

    \begin{aspects}
    \item \aspect[High Concept]{Ein Punk in Aktion - mit Werkzeug}
    \item \aspect[Trouble]{Will neue Tricks aus alter Technologie lernen}
    \item \aspect[Relationship]{Ander zum basteln motivieren}
    \item \aspect[Aspect]{Wenns klein ist - schütteln. Wenns groß ist: treten}
    \end{aspects}

    \begin{skills}
    \item \nskill{Bildung (Ingenieur)}{3}
    \item \nskill{Athletik}{1}
    \item \nskill{Diebeskunst}{0}
    \item \nskill{Kontakte}{0}
    \item \nskill{Handwerk}{4}
    \item \nskill{Täuschung}{0}
    \item \nskill{Fahren}{2}
    \item \nskill{Charisma}{0}
    \item \nskill{Kämpfen}{2}
    \item \nskill{Nachforschung}{0}
    \item \nskill{Spezialwissen}{0}
    \item \nskill{Wahrnehmung}{1}
    \item \nskill{Kraft}{3}
    \item \nskill{Provozieren}{0}
    \item \nskill{Empathie}{0}
    \item \nskill{Ressourcen}{2}
    \item \nskill{Schießen}{1}
    \item \nskill{Heimlichkeit}{0}
    \item \nskill{Wille}{1}
    \end{skills}

    \begin{stunts}
    \item \stunt{McGyver Gene}{Weil ich MacGyver Gene und Ausstattung habe, bekomme ich +2 wenn ich Handwerk nutze um schnell was zu basteln. Kurz nach dem erfolgreichen Gebrauch wird es wahrscheinlich spektakulär auseinanderfallen.}
    \end{stunts}

    \begin{stressSection}
    \stressLine{\stress{1}\stress{1}\stress{1}\stress{1}\stress{1}\stress{1}}{\stress{1}\stress{1}\stress{1}\stress{1}}
    \end{stressSection}
    \begin{tabularx}{\textwidth}{ XX }
    \end{tabularx}

    \begin{consequences}
    \item \consequence{2}
    \item \consequence{4}
    \item \consequence{6}
    \end{consequences}

    \begin{npcDescription}
    "Stroh Hut" aus alten Kabeln gewoben. Maschinenteile in die Kleidung eingewoben.
    \end{npcDescription}


    \begin{equipment}
    \item Lichtquelle (OLED Folie: Batteriebetrieben, kann in jede Größe und Form geschnitten werden und angeklebt. Die Farbe ist programmierbar)
    \item Essen (Mama Salsa's berühmtes Mehlwurm Brötchen in der Bento Box)
    \item Erste Hilfe Ausstattung
    \item Funkvernetzte Mobilcomputer, Kopfhörer, Radio Mesh taugliche Kommunikation
    \item Laser Schweißgerät (keine Waffe)
    \item Panzerband, WD40 und ein Schweizer Taschenmesser
    \end{equipment}
\end{npcBox}

\newpage

\section{Impact !}

Alle diese Abenteuer haben einen Einfluss auf die Welt. Und gerade bei Solarpunk ist es wichtig, den Einfluss sichtbar zu machen. Schreib die Entscheidungen auf, die die Charaktere machten. Die Freundschaften, die sie pflegten. Die Erfolge für ihre Pioneer Community. Und lasse dies in Folgeabenteuer wieder anklingen.
Die Charaktere in diesem Abenteuer wurden extra für das Abenteuer gebaut. Aber ihre Handlungen haben einen bleibenden Effekt.

\begin{itemize}
    \item Werden die Ereignisse um den Killer Hamster in der nächsten Staffel der Serie auftauchen ?
    \item Wird es das glühende Bier in Norm Bars schaffen ?
    \item Werden andere Charaktere die Pioneer Community besuchen ? Wird es dort den Hamsterkäfig geben ? Mit niedlichen Mini Monstern ?
\end{itemize}

Schreibe die Ideen jetzt auf und lasse es in die nächsten Sessions einfließen.
