\chapter{Die Weltvernichtungsmaschine}
\label{ch:the world destroying machine}

Dieses Abenteuer eignet sich hervorragend für Cons mit ungefähr 4h Spielzeit.

Der Tradition folgend ist es ein "Ratten im Keller" Startabenteuer - wie es für fast alle Rollenspiele existiert.

Alle Charaktere sind aus der Pioneer Fraktion. Die Lost und Norm Fraktionen tauchen als NSCs auf.

\begin{sidebarBox}[title=Der dreckige Weg nach Eden]

Die Menschen, die vor 2050 lebten sind die sogenannten "Lemminge". Diesen Namen haben sie wegen ihrer selbstzerstörerischen Lebensweise erhalten. Ab 2025 zweifelten die Leute zunehmend an der Weisheit, sich selbstzerstörerisch zu verhalten und sind in Aktion getreten. Diese führte in das Jahr 2050 in dem die Menschheit sich gerettet hat und in Lebenswerten automatisierten Norm-Städten, in Pioneer Wissehschafts Communities oder in naturnahen Lost camps - zwischen Wildniss und Ruinen - leben.
Aber der Weg zu dieser neuen und glänzenden Zukunft war dreckig. Nicht jeder Konnte gerettet werden. Städte wurden geopfert. Harte Entscheidungen mussten getroffen werden im Kampf gegen die Desaster, die durch den Klimawandel ausgelöst wurden. Und die Revolution ist noch im vollen Gang.

\end{sidebarBox}

\section{Themen}

Diees Abenteuer deckt typische Solarpunk Themen ab. Es kann gut mit Solarpunk- oder Rollenspiel Neulingen genutzt werden, da es im Tutorial Stil aufgebaut ist.

Es bietet

\begin{itemize}
\item Einführung in die Solarpunk 2050 Welt
\item Charakter Interaktion: Die Charaktere müssen ihre Interessen ausbalancieren, um Fate Punkte zu erhalten
\item Culture Clash: Alle drei Kulturen sind in diesem Abenteuer vorhanden. Kooperation ist wichtig für den Erfolg.
\item Die Mission startet ohne Waffen. Die Protagonisten können aber welche basteln oder von NSCs erwerben.
\item Einführung in die Fehler der Lemminge (also wir), die zur Verherrung geführt haben.
\end{itemize}

\section{Abenteuer Zusammenfassung}

Die Karte des Abenteuers ist linear, aber die Protagonisten können jederzeit vor- und zurück reisen. Um Verbündete zu finden, Gegenstände einzuhandeln und sich auf die letzte Herausforderung vorzubereiten.

Obwohl die Karte linear ist, gibt es viele Möglichkeiten, die Herausforderungen anzugehen. Das Abenteuer ist absichtlich flexibel gehalten. Die Entscheidungen der Charaktere und die Lösungswege sind im Sandbox-Stil !

Die lineare Ordnung ist:

\begin{itemize}
\item Die Spieler lernen die Pioneer Philosophie kennen auf einer Pioneer Party in ihrer Gemeinschaft.
\item Mission: Finde die Weltvernichtungsmaschine (den Spielern nicht bekannt: Es ist ein Kohlekraftwerk) und bergt recyclebares Rohmaterial, um eine Brauerei zu bauen.
\item Die Protagonisten treffen im Wald auf die Lost.
\item Nach dem Betreten der Weltvernichtungsmaschine treffen sie auf Norms, die gerade eine Folge für eine Film Serie über die Abenteuer fiktiver Pioneers drehen.
\item Erster Auftritt des End-Bosses: Ein mutierter Riesenhamster enfrührt einen der Norm Schauspieler
\item Durchsuche die Weltvernichtungsmaschine, löse Puzzle, baue Waffen und finde eine Lösung für den Hamster
\item Die Protagonisten stellen sich einem schweren Dilemma - der Monster Hamster ist Mama
\item Nach Erfolg gibt es eine Abschlussparty beim Errichten der neuen Brauerei. Auch muss man mit den Konsequenzen der eigenen Entscheidungen klar kommen
\end{itemize}

\section{Los geht es - mit einer Party: Die Übersicht}

\begin{sidebarBox}[title=Pioneers]
\hyperref[sec:Pioneers]{Pioneers} sind eine Gruppe hyper innovativer Leute mit selbstgebauten öko freundlichen high tech Communities. Ein Großteil der von den Norms genutzten Technologien basiert auf deren Konzepte. Während dem "Dreckigen Weg nach Eden" waren sie der unkoordinierte Treiber der Revolution. Heutzutage reden sie nicht über diese Phase oder sie nennen sie "Nötig für die Sicherheit der Menschheit" (was sie auch war). Die meiste Zeit kümmern sie sich nicht um die Vergangenheit, sondern fokussieren sich auf die Zukunft - so wurden viele Details bereits vergessen. Pioneers lieben ihre kreative Gesellschaft. Sind aber sehr individualistisch und jeder hat sein eigenes Lieblingsprojekt am laufen. Da sie nur nach vorne schauen, halten sie den Rückblick in die Vergangenheit meist für Zeitverschwendung. Aus diesem Grund weiss auch niemand, was die Weltvernichtungsmaschine ist. Und es kümmert sie auch nicht. Niemand wird helfen können. Wenn die Protagonisten aber die Lost fragen würden, könnten sie eine Antwort erhalten.
\end{sidebarBox}

Es findet eine große Outdoor Pioneer arty statt. Die Gemeinschaft hat sich versammelt. Selbstgemachte Musik läuft aus den Boxen mit dem üblichen LED und Laser Spektakel. Zum Essen gibt es selbstgezogenes Gemüse - und ein spezielles Getränk wird serviert:

Ein Schnapps Glas gefüllt mit einem neuen Bier. "Das Fass" hat mittels Gentechnik modifizierten Hefen und einem neuen Brauprozess ein köstliches Bier erzeugt. Und es leuchtet dank Biolumineszenz. Unglücklicherweise ist die Menge sehr limitiert. Das aktuelle Labor und die Brauausstattung reichen nicht für größere Mengen. Die Brauerei muss wachsen. Und dazu benötigt die Gemeinschaft Ressourcenpunkte.

\begin{sidebarBox}[title=Resourcenpunkte]
\hyperref[sec:Resource Points]{Resource Points} sind die Hauptwährung. Um Missbrauch zu verhindern, erhält jede Person eine feste Anzahl dieser Punkte pro Jahr von der UN. Diese Punkte werden benötigt, um nicht-erneuerbare Rohstoffe und darauf basierte Gegnstände zu erwerben. Sie genügen für einen normalen Lebensstil. Aber reichen nicht aus, um eine Brauerei zu gründen. Der einzige Weg, mehr von diesen Punkten zu bekommen ist es, Objekte zu recyclen. Große Objekte oder solche aus seltenen Materialien generieren mehr Ressourcenpunkte. Dies ist einer der Hauptgründe, Missionen zu den Ruinen der Lemminge zu starten.
\end{sidebarBox}

Glücklicherweise wurde eine "Weltvernichtungsmaschine" (ein Kohlekraftwerk - aber das wird nie erwähnt) freigespült. Sie wurde vor vielen Jahren bei einem der Naturkatastrophen verschüttet. Nachdem eine erneute Katastrophe die Hälfe eines Berges wegspülte ist sie wiederentdeckt worden.

Es fand eine Auktion um Bergungsrechte statt. Und die Pioneers dürfen ihr Güter zuerst bergen. Dies wird freudig auf der Party verkündet. Und die Protagonisten haben die Möglichkeit, zur Bergungstruppe zu gehören.

Die Bergungstruppe erhält 4 (oder Anzahl Mitglieder -1) Bergungs-Tags der UN. Diese werden an die zu bergenden Objekte geklebt und können danach nicht mehr entfernt werden. Die Protagonisten können entscheiden, was es wert zu bergen und recyclen ist. Neben den Tags können sie auch so viel bergen, wie sie tragen können.
Andere Gruppen haben sich auch beworben. Diese dürfen aber erst später das Gebiet betreten und bergen, was übrig ist. Dafür haben sie mehr Bergungstags erhalten. Die Pioneer Gruppe hat sich bewusst entschieden, als erste Gruppe da rein zu gehen - und dafür weniger Tags zu erhalten.

\begin{sidebarBox}[title=Bergungs-Tags]
Die Bergungs Tags sind kleine Aufkleber mit einer Energie Quelle, einem computer und Funk. Sie können nach dem Aufkleben nicht mehr von einem Objekt entfernt werden und dienen dazu, Beute zu markieren. Am Ende des Abenteuers werden Recyclingspezialisten (NPCs) mit schweren Equipment kommen und die markierten Objekte zerlegen, abtransportieren und recyclen. Die Ressourcen Punkte werden dann der Pioneer Community zugesprochen.
Diese Tags haben ein Display und einen Microcontroller. Sie aktivieren sich erst aber der programmierten Zeit. Vor dieser Zeit können sie nicht an den Objekten befestigt werden. Darum sind die Teams - denen verschiedene Zeit Slots zugewiesen wurden - nicht in direkter Konkurrenz bei der Bergung.
Die United Nations haben diese Tags in Auktionen vergeben. Und sie gewähren hinterher auch die Ressourcen Punkte.
\end{sidebarBox}

\emph{Die Bergungs Tags sidn ein Spielsystem um die Fluss des Spiels zu verbessern. Sie sind eine Art "bag of holding". Ohne sie müssten die Charaktere 300 Tonnen Generatoren mit sich durch das Abenteuer schleppen.}

% TODO: DOT graphviz goes here

\section{Party}

Inhalt diese Szene:
\begin{itemize}
\item Die Charaktere lernen sich kennen
\item Die Spieler lernen die Spielregeln
\item \textbf{Und ganz wichtig: Das Solarpunk-Gefühl aufbauen}
\end{itemize}

Die Pioneers haben eine Abend-Party draußen auf dem Dorfplatz. Diesmal wird etwas großes verkündet werden. Um bis dahin die Zeit zu verbringen (und die Regeln zu lernen) können die Protagonisten bei vielen der Aktivitäten auf der Party teilnehmen.

Alles ist mit bunten Lichtern dekoriert. Überall Schals und Wimpel. Leute stehen in Gruppen zusammen oder tanzen. In der Mitte des Platzes ist eine große Säule. Deren unterer Teil ist gerade grün beleuchtet.

Eine Ankündigung der Ältesten: "Heute haben wir einige Neuigkeiten. Die Erste: Dorothea hat Kinder !"

Eine Live Schalte wird auf einem großen Bildschirm eingeblendet: ein Fasanennest im nahen Wald. Mit kleinen Fasanen Küken. \it{Frenetischer Jubel} "Leise Bitte ! Wir haben gerade die Lautstärke Säule hier auf dem Dorfplatz aufgebaut. Weil Brutsaison ist. Sie zeigt die Lautstärke auf den Mikrofonen an, die wir im Wald verteilt haben. Wie jedes Jahr: Wenn sie in den roten Bereich wechselt - bitte etwas leiser sein. Das Musiksystem macht das automatisch. Wie jedes Jahr hat das 5. Drohnen Squadron der Kinder geschworen, die Küken zu schützen, indem sie Katzen, Marder und andere Räuber von den Gelegen fernhalten".

Jetzt fliegt ein Schwarm beleuchteter Quadrocopter in Formation über das Fest. Eine der Drohnen bricht aus der Formation aus, dippt kurz in die Punsch Schüssel und schließt sich wieder an. Die Kinder-piloten Lachen und jubeln.

"So, jetzt feiert erstmal, die zweite große Ankündigung wird in einer Stunde sein".

Jetzt können sich die Charaktere die Zeit auf der Party vertreiben. Diese dient dazu, die Regeln zu lernen und damit sich die Charaktere begegnen.

\begin{itemize}
\item Jonglier Workshop
\item Etwas entspannter: Gemeinsames Gärtnern an den Hochbeeten und Konversation mit lokalen NSCs
\item Die Kinder rasen mit ihren Drohnen durch die Baumwipfel. Manche reparieren ihre Drohnen nach crashes. Hier kann man: teilnehmen, bei Reperaturen helfen, Drohnen ausweichen, Drohnen aus Bäumen  oder dem Punsch holen
\item E-Motor Herausforderun: Jeder trinkt einen Schnapps. Danach versucht man aus Schrottteilen, einen funktionierenden E-Motor zu bauen. Hier kann man: Teilnehmen, oder Betrunkenen helfen
\item Party organisation: Jeder, der Interesse hat, kann Aufgaben übernehmen. Cocktails mischen, Musik auflegen oder noch schnell eine Lichtshow programmieren.
\end{itemize}

Direkt vor der Ankündigung des Abends bekommt jeder ein Schnapsglas mit lokal gebrautem Bier. Die Älteste: "Dieses Bier wurde mit unserer eigenen, genmodifizierten Hefe gebraut. Das Team rund um 'Dem Fass' haben es möglich gemacht. (alle jubeln). Wie ihr sehen könnt, leuchtet das Bier im Dunkeln und es schmeckt großartig. Aber ohne großem Bio Labor und einer größeren Brauerei können wir nicht mehr produzieren. Und uns fehlen die Ressourcen Punke dafür. Die gute Nachricht ist: Die UN gab uns Bergungsrechte bei einer alten Weltvernichtungsmaschine. Sie wurde bei einer Naturkatastrophe verschüttet. Und eine neue Katastrophe hat gerade den halben Berg über ihr weggespült. Lasst und dort die schweren Maschinen und seltenen Metalle bergen. Mit den Ressourcen Punkten, die wir durch das Recycling erhalten können wir unser Brauerei Labor aufbauen !"

"Das Fass" (evtl. ein Spielercharakter) kann auf dem Fest die wichtigsten Fragen beantworten:
\begin{itemize}
\item "Leuchtet man nach dem Trinken selber ?" (Nein)
\item "Leuchtet das Pipi?" (Ja)
\item "Wie lange leuchtet das Pipi?" (Einige Tage)
\item "Kannst du auch leuchtende Limo für die Kinder machen?" (Ja)
\end{itemize}

Nach der Vergabe der Mission brechen die Protagonisten am nächsten Tag auf zur Weltvernichtungsmaschine.

Zuerst per Zug (E-Bikes und E-Quads sind im Güterwagon). Danach fahren sie durch ein relativ neues und wildes Waldstück. Das auf Land wächst, das vor 20 Jahren überflutet wurde.

\section{Camp der Lost}

Inhalt dieser Szene:

\begin{itemize}
\item Die Pioneers treffen die Fraktion der Lost
\item Erstes Treffen mit einem mutierten Riesenhamster
\item Man könnte Waffen besorgen (stehlen, oder kaufen)
\item Man könnte die Lost um Hilfe bitten
\end{itemize}

\begin{sidebarBox}[title=Die Lost]
Die \hyperref[sec:Lost]{Lost} sine Überlebensexperten, Kämpfer und Historiker. Sie reisen durchs Land und suchen nach alter Lemming Technologie. Sie lehnen jede Technologie mit Microcontrollern ab, sind aber Experten im wiederverwenden und upcyclen alter Technologie. Ihre Camps sehen etwas mitgenommena aus. Verglichen mit den "Lifestyle" Norms oder den "Hypceraktiven/Hyperkreativen" Pioneers wirken sie wild und rückständig.
Damals, als der "Dreckige Weg nach Eden" begann, die Welt zu transformieren, sahen sie, dass ein hoher Preis zu zahlen ist. Und sie beschlossen - aus ethischen Gründen - nicht an dieser Transformation teilzunehmen.
Die Lost verwenden alle ihnen zustehenden Ressourcen Punkte um Diesel zu kaufen. Aus diesem Grund sind sie auf Streifzüge in den Ruinen angewiesen.
\end{sidebarBox}

Die Protagonisten kommen im Wald an. Vor dem Eingang zur Weltvernichtungsmaschine befindet sich ein Lost Camp. Schwere Diesel Autos parken dort mit laufendem Motor. Holz und Öl brennen in Fässern. Aus alten Planen wurden Zelte errichtet. Alles ist sehr temporär, gebaut aus Material der Vergangenheit. Aber es ist ein praktische sund ordentliches Camp.

In der Mitte rotiert ein Bärengroßes Tier über einem Grill. Die Lost könnten erklären, dass es ein hier gefangener mutierter Riesenhamster ist.

\begin{sidebarBox}[title=Fehlgeschlagenes CCS Experiment: Monster Hamster]
Gebildete Leute könnten wissen, dass die riesigen Hamster ein fehlgeschlagenes Carbon Capture and Storage Experiment sind aus der Ära des "Dreckigen Wegs nach Eden". Die Lost wissen das. Sie jagen diees gefährliche Tier, wo sie nur können. Der Hamster ist genetisch programmiert, Proteinbrocken (auch um sich schlagende und schreiende) in ihre Untergrund Hamsterbauten zu zerren. Der Ursprüngliche Plan was es, Kohlenstoff von der Erdoberfläche zu entfernen, mittels dieser Hamster....
\end{sidebarBox}

Jemand macht Kartoffelsalat und baut Picknick Bänke auf. Musik spielt. Die Lautsprecher sind verzerrt und mindestens 20 Jahre alt. Aber das scheint niemanden zu interessieren. Im Hintergrund schießt jemand auf Bierdosen mit seiner Schrotflinte. Das ist deren Anführer - Caligula.Auf einem Tisch sind alte Bücher gestapelt.
Die Lost haben selbst 10 Bergungs-Tags von der Auktion bekommen. Mehr als die Pioneers haben. Aber deshalb dürfen sie auch erst als zweite in die Ruine. Die Tags werden sich erst in 12 Stunden aktivieren. Dann können die Lost beginnen zu plündern. Bis dahin feiern sie hier in diesem Wald. Solange die Protagonisten halbwegs zügig vorangehen, werden die Lost keine Konkurrenz sein.

Verhalten: Wenn sie angesprochen werden, ziehen sie die Pioneers auf ihre rustikale Art auf. Und verbieten ihnen, so Dinge wie Diesel, Dieselgeneratoren und so aus der Ruine zu nehmen. Das gehört den Lost. Wenn die Pioneers bei den Frotzeleien mitmachen und sich als würdig erweisen, werden sie zu Hamster, Salat und Bier eingeladen.

Kurz danach aktivieren sich die Bergungs Tags der Pioneers und sie können in die Ruinen der Weltvernichtungsmaschine hinabsteigen.

\begin{npcBox}[title=Caligula]

    \begin{aspects}
    \item \aspect[High Concept]{Small budget Indiana Jones}
    \item \aspect[Trouble]{Alcohol fuelled}
    \end{aspects}

    \begin{skills}
    \item \nskill{Bildung}{3}
    \item \nskill{Athletik}{2}
    \item \nskill{Diebeskunst}{0}
    \item \nskill{Kontakte}{0}
    \item \nskill{Handwerk}{1}
    \item \nskill{Täuschung}{0}
    \item \nskill{Fahren}{1}
    \item \nskill{Charisma}{0}
    \item \nskill{Kämpfen}{1}
    \item \nskill{Nachforschung}{0}
    \item \nskill{Spezialwissen}{2}
    \item \nskill{Wahrnehmung}{2}
    \item \nskill{Kraft}{0}
    \item \nskill{Provozieren}{3}
    \item \nskill{Empathie}{0}
    \item \nskill{Ressourcen}{0}
    \item \nskill{Schießen}{4}
    \item \nskill{Heimlichkeit}{0}
    \item \nskill{Wille}{1}
    \end{skills}

    \begin{stunts}
    \item \stunt{Tuning}{Gets a +2 to shooting whenever he uses a weapon that he has recently tuned in a training session that lasts 1 hour.}
    \end{stunts}

    \begin{stressSection}
    \stressLine{\stress{1}\stress{1}\stress{1}}{\stress{1}\stress{1}\stress{1}\stress{1}}
    \end{stressSection}
    \begin{tabularx}{\textwidth}{ XX }
    \end{tabularx}

    \begin{consequences}
    \item \consequence{2}
    \item \consequence{4}
    \item \consequence{6}
    \end{consequences}

    \begin{npcDescription}
    Caligula leads a small family of scavengers. They travel the wilds, searching for treasure in ancient ruins. Whatever useful things they find, they reuse and recycle in creative ways.
    He is ready to fight if he has to. So is his family. But they would all appreciate a discussion about ancient artefacts and sites. A stranger's first impression is of a redneck with a gun.
    \end{npcDescription}

    \end{npcBox}

He is not alone, but his "family" consists of about 10 people who can use a weapon and are good at scouting ruins. They do not care as much about nature as the Pioneers. They are constantly fighting and struggling with the forces of nature and the wilderness. Their approach to nature is more... pragmatic.

If the protagonists make friends with the Lost, they could gain:

\begin{itemize}
    \item Weapons and people who can use them
    \item The insight that the World Destroying machine is a coal power plant. Including a rough sketch of the map
    \item Maybe learn that Norms arrived 2 days ago. "Looked strange. But they always look strange. Not prepared for the ruins. Even less prepared than you are"
\end{itemize}

\section{Battleground}

Goal of the scene:

\begin{itemize}
\item You meet the Norms for the first time.
\item Learn: The world destroying machine is absurdly designed. Almost dull and boring
\end{itemize}

\begin{sidebarBox}[title=Norms]

80 percent of the people in 2050 are \hyperref[sec:Norms]{Norms}. They live in automated eco-cities. Governed by AIs that set all parameters for maximum quality of life and happiness. Society is highly cooperative. Most people have a 25h/week job that is highly specialised. The AI plans projects to coordinate these specialists in an incredible dance to achieve great projects.

Norms all carry a Hive Controller. This device provides them with apps and an AR interface where they can simply request things from the AI and Society. And it will be done - magic!
While the Norms enjoy their hobbies, they will never reach the solo capabilities of the Pioneers. They will always need to be close to the AI and a functioning Society. If the requirements are not met, some applications will indicate this and will be unavailable.

Focused on the now, they do not care about the past or the dirty road to Eden. Everything is fine now. It must have been worth the price.

In this region, all Norm characters have limited benefits from the app, as this region is only covered by the small AI they brought with them in a shipping container. The social network is also small. Almost everything they need has to be delivered by drones from the nearest town (1 hour extra). For Pioneers, this can still feel like magic.

\end{sidebarBox}

The characters enter a corridor through a crooked metal hatch on the side of a hill. Through the hatch: a corridor. The walls are white - but now musty. The floor is linoleum.
White plastic cupboards, devoid of any personality, line the corridors. Many doors (white, plastic with a wood look) branch off to the right and left. On them are signs with the names of the people whose offices used to be there. Behind the doors: rubble and mud.

Soon the protagonists find a simulated accident. It looks realistic: A Norm actor (Delta Awesome) lies under a foam H-beam (which looks like steel). A hidden camera man (Kevin) films him screaming. The hero of the reality soap should have appeared at any moment to 'rescue' him. Instead, the protagonists (real pioneers) come to the rescue.

The actor 'Delta Awesome' continues to act and 'Kevin' continues to film while the Pioneers begin the rescue. They soon learn that there was no real danger.

After the misunderstanding has been cleared up and everyone is waiting impatiently for the hero actor "Theophil Tierlieb", you can hear some screams coming down the aisle.

A quick glimpse: The expected hero, the actor in the role of \textbf{"Theophil Tierlieb"}, is being pulled into a pipe by a giant bear-sized hamster. These pipes seem to run through the whole world-destroying machine.

Unfortunately, the pipes are almost impossible for a human to crawl through (being dragged unconscious by a monster seems to take up less space, and the hamster itself is built for tunnels and pipes). Eventually, the pipe will break under the strain. Drones could follow the beast. Following the pipes is possible, but tricky. Some of them go through walls.

The protagonists need a map. And maybe weapons. As a Pioneer, you improvise as you go.

At the end of the corridor, the protagonists find a large hall lined with marble. The official entrance hall and the Museum of the World Destroying Machine.

\begin{npcBox}[title=Kevin\, Camera]

    \begin{aspects}
    \item \aspect[High Concept]{Camera for action}
    \item \aspect[Trouble]{Finding good action scenes}
    \end{aspects}

    \begin{skills}
    \item \nskill{Bildung}{1}
    \item \nskill{Athletik}{1}
    \item \nskill{Diebeskunst}{0}
    \item \nskill{Kontakte}{2}
    \item \nskill{Handwerk Filmen}{3}
    \item \nskill{Täuschung}{0}
    \item \nskill{Fahren}{1}
    \item \nskill{Charisma}{3}
    \item \nskill{Kämpfen}{0}
    \item \nskill{Nachforschung}{0}
    \item \nskill{Spezialwissen}{0}
    \item \nskill{Wahrnehmung}{4}
    \item \nskill{Kraft}{0}
    \item \nskill{Provozieren}{0}
    \item \nskill{Empathie}{2}
    \item \nskill{Ressourcen}{0}
    \item \nskill{Schießen}{1}
    \item \nskill{Heimlichkeit}{2}
    \item \nskill{Wille}{0}
    \end{skills}

    \begin{stunts}
    \item \stunt{App based Filming}{Gets a +2 on filming action scenes when in range of an AI to support him}
    \end{stunts}

    \begin{stressSection}
    \stressLine{\stress{1}\stress{1}\stress{1}}{\stress{1}\stress{1}\stress{1}}
    \end{stressSection}
    \begin{tabularx}{\textwidth}{ XX }
    \end{tabularx}

    \begin{consequences}
    \item \consequence{2}
    \item \consequence{4}
    \item \consequence{6}
    \end{consequences}

    \begin{npcDescription}
    Kevin loves entertaining audiences. His skills with the camera and AI-based editing help him do this. If he becomes a friend, he can boost the Pioneer team's publicity (whether they want it or not). More importantly, though, are his Notice skills.
    "I've known I wanted to be a cameraman since the AI recommended me for the job when I was 10."


    He has no access to weapons and cannot order them from the app. "Sorry, I haven't done the weapons tutorial yet, should I?"

    \end{npcDescription}

\end{npcBox}


\begin{npcBox}[title=Delta Awesome - acting the victim]

    \begin{aspects}
    \item \aspect[High Concept]{Method acting actor}
    \item \aspect[Trouble]{Observe and copy the real Pioneers}
    \item \aspect[Aspect]{Always stay in character}
    \end{aspects}

    \begin{skills}
    \item \nskill{Bildung}{0}
    \item \nskill{Athletik}{2}
    \item \nskill{Diebeskunst}{0}
    \item \nskill{Kontakte}{3}
    \item \nskill{Handwerk (Schauspielern)}{4}
    \item \nskill{Täuschung}{2}
    \item \nskill{Fahren}{0}
    \item \nskill{Charisma}{0}
    \item \nskill{Kämpfen}{0}
    \item \nskill{Nachforschung}{0}
    \item \nskill{Spezialwissen}{1}
    \item \nskill{Wahrnehmung}{1}
    \item \nskill{Kraft}{2}
    \item \nskill{Provozieren}{0}
    \item \nskill{Empathie}{3}
    \item \nskill{Ressourcen}{1}
    \item \nskill{Schießen}{0}
    \item \nskill{Heimlichkeit}{0}
    \item \nskill{Wille}{1}
    \end{skills}

    \begin{stunts}
    \item \stunt{Acting}{Can use Craft/Acting to convince people to join his heroic mission}
    \end{stunts}

    \begin{stressSection}
    \stressLine{\stress{1}\stress{1}\stress{1}\stress{1}}{\stress{1}\stress{1}\stress{1}\stress{1}}
    \end{stressSection}
    \begin{tabularx}{\textwidth}{ XX }
    \end{tabularx}

    \begin{consequences}
    \item \consequence{2}
    \item \consequence{4}
    \item \consequence{6}
    \end{consequences}

    \begin{npcDescription}
    Delta Awesome is the character's name. His role is that of an experienced Pioneer expert. But he does not live up to it. He insists on method acting and has to stay in the role (otherwise it will take him 2 hours to get back into it). He will also be constantly trying to improve the role by observing and copying the Pioneers.

    Delta Awesome's gadgets are useless props. In the film they are always exactly what he needs.

    He has trained for his role and has developed some real muscles. Which can come in handy. That and his acting/charisma-fuelled ability to convince people to help him. But in order to benefit from this, he has to be convinced first.
    \end{npcDescription}

\end{npcBox}


\section{Exhibition}

Topics of the scene:
\begin{itemize}
\item First clear indications of coal power (when researching the exhibition)
\item Socialize with the Norms
\item Find out where the pipes lead (on models and plans)
\item You can find many kilograms of protein paste here
\end{itemize}


Cameraman Kevin and Delta Awesome quickly lead the protagonists to the "headquarters". A former museum (also a film location). Catering is set up here. The Norms plan to accommodate 500 fans of the series after filming. With 10 extra seats for VIPs. The party location is currently being prepared.

There is an old museum where school classes can learn about coal power from very nice models.
Everything is nicely done. With a mascot. The mineral collection is also interesting, with a huge geode that might interest Disco.

In the catering area, there is a food designer (Cherie) who makes real-looking mealworms out of protein paste for the Pioneer food shots.  That way the VIPs can feel like solar punks without having to eat mealworms.

The food designer can control a 3D food printer via an app and could also make a protein-based fake body to lure the hamster.

According to the food designer, the others are deeper into the world-destroying machine, preparing it for filming. Haven't heard from them in a while. (Info: They've been hoarded). Access is through a steel door which is locked.

Someone with historical knowledge (books) can work out that the heaviest part here is probably the coal-fired generator with flywheel. This can be found further down the plant.

Cherie has a key to the door leading deeper into the plant. It could be stolen, she could be persuaded, the lock could be picked, or the door could be welded open.


\begin{npcBox}[title=Cherie]

    \begin{aspects}
    \item \aspect[High Concept]{Food artist}
    \item \aspect[Trouble]{Want to be my friend ?}
    \item \aspect[Aspect]{Food must be tasty and beautiful}
    \end{aspects}

    \begin{skills}
    \item \nskill{Bildung}{1}
    \item \nskill{Athletik}{1}
    \item \nskill{Diebeskunst}{0}
    \item \nskill{Kontakte}{2}
    \item \nskill{Handwerk (Kochkunst)}{4}
    \item \nskill{Täuschung}{1}
    \item \nskill{Fahren}{0}
    \item \nskill{Charisma}{1}
    \item \nskill{Kämpfen}{0}
    \item \nskill{Nachforschung}{0}
    \item \nskill{Spezialwissen}{0}
    \item \nskill{Wahrnehmung}{3}
    \item \nskill{Kraft}{0}
    \item \nskill{Provozieren}{0}
    \item \nskill{Empathie}{3}
    \item \nskill{Ressourcen}{2}
    \item \nskill{Schießen}{0}
    \item \nskill{Heimlichkeit}{0}
    \item \nskill{Wille}{2}
    \end{skills}

    \begin{stressSection}
    \stressLine{\stress{1}\stress{1}\stress{1}}{\stress{1}\stress{1}\stress{1}\stress{1}}
    \end{stressSection}
    \begin{tabularx}{\textwidth}{ XX }
    \end{tabularx}

    \begin{consequences}
    \item \consequence{2}
    \item \consequence{4}
    \item \consequence{6}
    \end{consequences}

    \begin{npcDescription}
    Does catering and simulates a Pioneer world for spectators and guests.

    She loves to chat at work and is positive and upbeat. A challenge such as "build a body using your 3D printer and protein" would be accepted with glee.

    Maybe that would distract the hamster?

    The monster next door worries her a lot because she grew up in a very safe environment - a Norm town.

    \end{npcDescription}

\end{npcBox}



\section{Coal Bunker}

Topics of the scene:
\begin{itemize}
\item Overcome technical problems
\item Can build weapons
\item Show dirtiness of World Destroying Machine
\end{itemize}

Problems:

\begin{itemize}
\item Dry coal dust (explosive)
\item Dark black water below, with oil film
\item The Norms built the SFX stuff. In particular, cables through the water and prepared pyrotechnics
\item Some of the old processes still seem to be working. The norms have been wildly hooking up batteries and motors in the hope of bringing things to life. Looks good on film but could be a death trap.
\end{itemize}

Weapon Material:

\begin{itemize}
\item Coal dust (potato cannon, pipe bombs)
\item Pipes from handrails
\item Explosives from the SFX
\end{itemize}

After a dirty corridor, the protagonists enter a huge hall. Coal wagons full of coal were delivered here on rails. Some of them are still here. Derailed and crushed by the disaster that happened many years ago. Here the coal was checked for quality, crushed into pellets and dust and transported down the hall on belts. Much of this can still be seen here - but in a sad state of disrepair.
Everything is rusted. Coal dust hangs in the air (and is explosive!). There are black puddles on the floor (oil and coal).
At least it's clear where the Norms went. They left behind batteries, lights and pyrotechnics, and their trail runs diagonally across the area. This obstacle course could explode at any time. It will require careful navigation, some parkour, metal-cutting skills and disarming explosives. Lots of skill tosses.

The coal conveyor belt leads to the next room, where the protagonists will want to go.


\section{Walkways}

Topics of the scene:
\begin{itemize}
\item Overcome obstacles
\item Demonstrate the devastation and grandeur of the world-destroying machine
\item Can build weapons
\end{itemize}

The protagonists have to climb over catwalks and through large running ventilation fans.
These are eerily backlit and a fog machine creates an eerie look. The SFX people were here. I'm sure it will look great in the film.
Greenish glowing dust puffs (mutated) grow on the floor below. Anyone with any knowledge of ecology would know that the spores are psychoactive. The director is lying next to the mushrooms. Unconscious thanks to the psychoactive mushrooms.
A make-up area is set up below. This is where filming is planned.

Problems:
\begin{itemize}
\item Broken metal walkways
\item Pipe labyrinths (in which hamsters move)
\item Mutated mushrooms, the director must be rescued
\end{itemize}

Weapon material:

\begin{itemize}
\item Sharp blades from ventilation (Swords)
\item Pieces of pipe (spears, pipe bomb, potato cannon)
\item Psychoactive mushrooms (wear protective gear when harvesting !)
\end{itemize}


The conveyor belt leads to the combustion chamber (which is not accessible). Next door is the generator room. There is the nest.
In this room you can already see the steam pipes leading there.

\begin{npcBox}[title=Lucien\, Director]

    \begin{aspects}
    \item \aspect[High Concept]{Film director with a skill for blockbusters}
    \item \aspect[Trouble]{There is a story in there}
    \end{aspects}

    \begin{skills}
    \item \nskill{Bildung}{1}
    \item  \nskill{Athletik}{0}
    \item \nskill{Diebeskunst}{0}
    \item \nskill{Kontakte}{3}
    \item \nskill{Handwerk (Film director)}{4}
    \item \nskill{Täuschung}{0}
    \item \nskill{Fahren}{0}
    \item \nskill{Charisma}{1}
    \item \nskill{Kämpfen}{0}
    \item \nskill{Nachforschung}{1}
    \item \nskill{Spezialwissen}{0}
    \item \nskill{Wahrnehmung}{2}
    \item \nskill{Kraft}{0}
    \item \nskill{Provozieren}{0}
    \item \nskill{Empathie}{2}
    \item \nskill{Ressourcen}{3}
    \item \nskill{Schießen}{0}
    \item \nskill{Heimlichkeit}{1}
    \item \nskill{Wille}{2}
    \end{skills}

    \begin{stressSection}
    \stressLine{\stress{1}\stress{1}\stress{1}}{\stress{1}\stress{1}\stress{1}\stress{1}}
    \end{stressSection}
    \begin{tabularx}{\textwidth}{ XX }
    \end{tabularx}

    \begin{consequences}
    \item \consequence{2}
    \item \consequence{4}
    \item \consequence{6}
    \end{consequences}

    \begin{npcDescription}
    Is stoned when found and will not recover before the end of the story. Could be interesting as a friend at the party.
    \end{npcDescription}

\end{npcBox}

\section{Nest}

Topics of the scene:
\begin{itemize}
\item Final Battle
\end{itemize}

All kinds of organic material can be found in the nest. From old sacks of potatoes to dead animals (hunted dogs and wild boars).
It's confusing and full of the remains of an ancient civilisation.
The hamster has dragged the lifeless Norm onto the pile and he will die here soon.
A particular treasure here is the 4 large generators with the heavy, massive flywheel. This is the treasure that can be marked for recovery (either after the hamster dies or by sneaking in).

Solution ideas:
\begin{itemize}
\item You could make the hamster overeat with fake protein to make it fall asleep (Bio knowledge to trigger the "eat now" reflex)
\item Or intoxicate her with the psychoactive mushrooms (Bio Skills, Weapon Technique)
\item Or kill her (combat)
\item Or fetch the Lost for help (Social Interaction)
\item Sneak in and rescue the injured, including secretly planting salvage tags.
\item Dazzle the hamster by drones, pyro, SFX...
\end{itemize}


\begin{npcBox}[title=Hamster]

    \begin{aspects}
    \item \aspect[High Concept]{Fluffy killer machine on a CCS mission}
    \item \aspect[Trouble]{Damned by the genes}
    \item \aspect[Aspect]{Always hungry for protein}
    \end{aspects}

    \begin{skills}
    \item \nskill{Bildung}{0}
    \item  \nskill{Athletik}{2}
    \item \nskill{Diebeskunst}{0}
    \item \nskill{Kontakte}{0}
    \item \nskill{Handwerk (Nestbau)}{1}
    \item \nskill{Täuschung}{2}
    \item \nskill{Fahren}{0}
    \item \nskill{Charisma}{0}
    \item \nskill{Kämpfen}{3}
    \item \nskill{Nachforschung}{0}
    \item \nskill{Spezialwissen}{0}
    \item \nskill{Wahrnehmung}{3}
    \item \nskill{Kraft}{4}
    \item \nskill{Provozieren}{1}
    \item \nskill{Empathie}{0}
    \item \nskill{Ressourcen}{0}
    \item \nskill{Schießen}{0}
    \item \nskill{Heimlichkeit}{2}
    \item \nskill{Wille}{0}
    \end{skills}

    \begin{stunts}
    \item \stunt{Through the pipes}{Using \textbf{notice} the hamster can enter a pipe and emerge 1 round later at a tactical spot anywhere else in the room gaining an advantage for the attack (+2). By being better at \textbf{notice} the player characters can find out where the hamster is moving and negate the effect.}
    \end{stunts}

    \begin{stressSection}
    \stressLine{\stress{1}\stress{1}\stress{1}\stress{1}\stress{1}\stress{1}}{\stress{1}\stress{1}\stress{1}}
    \end{stressSection}
    \begin{tabularx}{\textwidth}{ XX }
    \end{tabularx}

    \begin{consequences}
    \item \consequence{2}
    \item \consequence{4}
    \item \consequence{6}
    \end{consequences}

    \begin{npcDescription}
    The hamster is bear-sized and can be quite aggressive when attacked. Her normal goal is to harvest proteins and drag them down here. And store them. The protagonists can also find this out when searching: The hamster is female and has 4 young hamsters in a nest that she protects.
    \end{npcDescription}

\end{npcBox}


\begin{npcBox}[title=Theophil Tierlieb]

    \begin{aspects}
    \item \aspect[High Concept]{Actor in role of "Theophil Tierlieb" - animal whisperer}
    \item \aspect[Trouble]{Animals hate the real me}
    \item \aspect[Aspect]{Hamster chow}
    \end{aspects}

    \begin{skills}
    \item \nskill{Bildung}{0}
    \item  \nskill{Athletik}{1}
    \item \nskill{Diebeskunst}{0}
    \item \nskill{Kontakte}{3}
    \item \nskill{Handwerk (Schauspielern)}{4}
    \item \nskill{Täuschung}{1}
    \item \nskill{Fahren}{0}
    \item \nskill{Charisma}{3}
    \item \nskill{Kämpfen}{1}
    \item \nskill{Nachforschung}{0}
    \item \nskill{Spezialwissen}{2}
    \item \nskill{Wahrnehmung}{2}
    \item \nskill{Kraft}{2}
    \item \nskill{Provozieren}{1}
    \item \nskill{Empathie}{0}
    \item \nskill{Ressourcen}{0}
    \item \nskill{Schießen}{0}
    \item \nskill{Heimlichkeit}{0}
    \item \nskill{Wille}{0}
    \end{skills}

    \begin{stressSection}
    \stressLine{\markedstress{1}\markedstress{1}\markedstress{1}\markedstress{1}}{\stress{1}\stress{1}\stress{1}}
    \end{stressSection}
    \begin{tabularx}{\textwidth}{ XX }
    \end{tabularx}

    \begin{consequences}
    \item \consequence{2} Headache
    \item \consequence{4} Broken bones
    \item \consequence{6} Strong bleeding
    \end{consequences}

    \begin{npcDescription}
    He plays an animal whisperer, but learns after 2 episodes that the animals hate him, so he suffers for 3 seasons until he is attacked by a real monster hamster. But the audience loved his character.
    \end{npcDescription}

\end{npcBox}


\section{The real end boss}

After the battle, the protagonists find 4 tiny monster hamsters (tiny = the size of a dog). The children of the mother they just killed. They are old enough to survive without their mother if someone takes care of them. They have not attacked humans. But they might in the future. Now it is up to the protagonists to decide:

\begin{itemize}
    \item Take them to the Community and care for them ?
    \item Kill them
    \item Leave them to their fate ?
    \item Sell them to the Lost (so they can be fattened and later be slaughtered) ?
\end{itemize}

You see, the real final boss is a dilemma, and you should make it extra dramatic. If possible, have a 5 minute discussion between the characters to find the best way. Make it clear that the resource points earned will not be enough for a monster hamster cage and a brewery.

If the mother survives (perhaps thanks to mushroom poisoning), one option might be to leave the wild creatures alone and evacuate the humans from the area. Limit the potential loot.

\section{Victory Party}

Topics:

\begin{itemize}
\item Conclude the adventure and to celebrate.
\item Shows the consequences of their decisions
\end{itemize}

A few days later. The resource points were exchanged for resources. These arrive in the community and a brewery can be built. Its construction is part of a party with music, food and drink.

Friends you have made are invited. They will play a part in the celebrations.

If they have rescued the tiny monster hamster, they must first build a giant hamster cage. Including a wheel and pipes running through the community. Perhaps the resource points will not be enough for the brewery and the cage?

\section{Player characters}

\begin{itemize}
\item Books: Scholar, wants to salvage historical things
\item Curly: Acrobat, wants to recover something funny
\item The Barrel: Brewer, wants to salvage objects as large as possible because of resource points - the brewery wants them.
\item Disco: Bard, wants to salvage beautiful things
\item Spark: Tech, wants to recover technology
\item Primrose: Ecology, wants to save nature
\end{itemize}

%%%%%%%%%% Books
\newpage
\begin{npcBox}[title=Books]

    \begin{aspects}
    \item \aspect[High Concept]{Scholar - think first then act}
    \item \aspect[Trouble]{We don't do anything without a plan}
    \item \aspect[Relationship]{I wrote an article, could you please review it ?}
    \item \aspect[Aspect]{The more I know the better I perform}
    \end{aspects}

    \begin{skills}
    \item \nskill{Bildung}{4}
    \item  \nskill{Athletik}{2}
    \item \nskill{Diebeskunst}{0}
    \item \nskill{Kontakte}{0}
    \item \nskill{Handwerk}{1}
    \item \nskill{Täuschung}{0}
    \item \nskill{Fahren}{1}
    \item \nskill{Charisma}{0}
    \item \nskill{Kämpfen}{2}
    \item \nskill{Nachforschung}{3}
    \item \nskill{Spezialwissen}{0}
    \item \nskill{Wahrnehmung}{3}
    \item \nskill{Kraft}{0}
    \item \nskill{Provozieren}{0}
    \item \nskill{Empathie}{2}
    \item \nskill{Ressourcen}{0}
    \item \nskill{Schießen}{1}
    \item \nskill{Heimlichkeit}{0}
    \item \nskill{Wille}{1}
    \end{skills}

    \begin{stunts}
    \item \stunt{E-Book}{While I have my treasured e-book, I get +2 when I use Academics}
    \end{stunts}

    \begin{stressSection}
    \stressLine{\stress{1}\stress{1}\stress{1}}{\stress{1}\stress{1}\stress{1}\stress{1}}
    \end{stressSection}
    \begin{tabularx}{\textwidth}{ XX }
    \end{tabularx}

    \begin{consequences}
    \item \consequence{2}
    \item \consequence{4}
    \item \consequence{6}
    \end{consequences}

    \begin{npcDescription}
    Hunter of knowledge. Wears a jacket with stains on the elbows. Hair turns grey. Shows a sense of fashion by wearing an elegant hat.
    \end{npcDescription}


    \begin{equipment}
    \item Light source (OLED film: battery operated, can be cut to size and glued on. Colour controllable)
    \item Food (Mama Salsa's famous mealworm buns in the bento box)
    \item First aid kit
    \item Mobile computers, headphones, communication via radio (mesh network)
    \end{equipment}

\end{npcBox}

%%%%%%%%%%% Curly
\newpage
\begin{npcBox}[title=Curly]

    \begin{aspects}
    \item \aspect[High Concept]{Childish climbing acrobat}
    \item \aspect[Trouble]{Secret fan of bad Norm TV series}
    \item \aspect[Relationship]{Always looking for a role model in the group}
    \item \aspect[Aspect]{Let's see if I can do something fun with it. . .}
    \end{aspects}

    \begin{skills}
    \item \nskill{Bildung}{0}
    \item  \nskill{Athletik}{4}
    \item \nskill{Diebeskunst}{2}
    \item \nskill{Kontakte}{0}
    \item \nskill{Handwerk}{1}
    \item \nskill{Täuschung}{2}
    \item \nskill{Fahren}{0}
    \item \nskill{Charisma}{0}
    \item \nskill{Kämpfen}{3}
    \item \nskill{Nachforschung}{0}
    \item \nskill{Spezialwissen}{0}
    \item \nskill{Wahrnehmung}{2}
    \item \nskill{Kraft}{1}
    \item \nskill{Provozieren}{0}
    \item \nskill{Empathie}{1}
    \item \nskill{Ressourcen}{0}
    \item \nskill{Schießen}{1}
    \item \nskill{Heimlichkeit}{3}
    \item \nskill{Wille}{0}
    \end{skills}

    \begin{stunts}
    \item \stunt{E-Tail}{With my furry balance tail, I get +2 on Acrobatics when balancing}
    \end{stunts}

    \begin{stressSection}
    \stressLine{\stress{1}\stress{1}\stress{1}\stress{1}}{\stress{1}\stress{1}\stress{1}}
    \end{stressSection}
    \begin{tabularx}{\textwidth}{ XX }
    \end{tabularx}

    \begin{consequences}
    \item \consequence{2}
    \item \consequence{4}
    \item \consequence{6}
    \end{consequences}

    \begin{npcDescription}
    Shaved head, tattooed with a scale pattern. Wearing a homemade balance suit (with balancing tail).
    Upper body is naked, showing Bruce Lee style muscles. Curly wears cargo pants (to carry things around) and acrobat shoes with uncovered toes for better grip.
    \end{npcDescription}


    \begin{equipment}
    \item Light source (OLED film: battery operated, can be cut to size and glued on. Colour controllable)
    \item Food (Mama Salsa's famous mealworm buns in the bento box)
    \item First aid kit
    \item Mobile computers, headphones, communication via radio (mesh network)
    \item A stripped down light exoskeleton with a balancing tail
    \item Climbing rope
    \end{equipment}
\end{npcBox}


%%%%%%% The Barrel

\newpage
\begin{npcBox}[title=The Barrel]

    \begin{aspects}
    \item \aspect[High Concept]{Strong listener}
    \item \aspect[Trouble]{Likes to talk to microorganisms (yeasts)}
    \item \aspect[Relationship]{Active counseling. "How are you with that?"}
    \item \aspect[Aspect]{Driving force of the beer project}
    \end{aspects}

    \begin{skills}
    \item \nskill{Academics (genetics)}{3}
    \item  \nskill{Athletik}{1}
    \item \nskill{Diebeskunst}{0}
    \item \nskill{Kontakte}{0}
    \item \nskill{Handwerk}{3}
    \item \nskill{Täuschung}{0}
    \item \nskill{Fahren}{1}
    \item \nskill{Charisma}{4}
    \item \nskill{Kämpfen}{2}
    \item \nskill{Nachforschung}{0}
    \item \nskill{Spezialwissen}{0}
    \item \nskill{Wahrnehmung}{2}
    \item \nskill{Kraft}{2}
    \item \nskill{Provozieren}{0}
    \item \nskill{Empathie}{1}
    \item \nskill{Ressourcen}{0}
    \item \nskill{Schießen}{0}
    \item \nskill{Heimlichkeit}{0}
    \item \nskill{Wille}{1}
    \end{skills}

    \begin{stunts}
    \item \stunt{Empathy}{Because I'm highly empathic, using empathy to help someone gives me a +2. Unfortunately, his problems won't let me go for
    some time.}
    \end{stunts}

    \begin{stressSection}
    \stressLine{\stress{1}\stress{1}\stress{1}\stress{1}}{\stress{1}\stress{1}\stress{1}\stress{1}}
    \end{stressSection}
    \begin{tabularx}{\textwidth}{ XX }
    \end{tabularx}

    \begin{consequences}
    \item \consequence{2}
    \item \consequence{4}
    \item \consequence{6}
    \end{consequences}

    \begin{npcDescription}
    Comfortable, bearish, strong. Interested in optimising the art of brewing and willing to read up on genetic engineering. Normally dressed in a flannel shirt, 3/4 trousers and a small beer belly. Has not shaved for the last 3 days.
    \end{npcDescription}


    \begin{equipment}
    \item Light source (OLED film: battery operated, can be cut to size and glued on. Colour controllable)
    \item Food (Mama Salsa's famous mealworm buns in the bento box)
    \item First aid kit
    \item Mobile computers, headphones, communication via radio (mesh network)
    \item Gene laboratory in a suitcase
    \item 2 bottles of glowing beer (prototype)
    \end{equipment}
\end{npcBox}


%%%%%%%% Disco
\newpage
\begin{npcBox}[title=Disco]

    \begin{aspects}
    \item \aspect[High Concept]{Fighter for the colourful lights and the eternal party}
    \item \aspect[Trouble]{Uncomfortable in serious situations}
    \item \aspect[Relationship]{Wants everyone to be happy and to get along}
    \item \aspect[Aspect]{Looking for beautiful things}
    \end{aspects}

    \begin{skills}
    \item \nskill{Bildung}{0}
    \item  \nskill{Athletik}{1}
    \item \nskill{Diebeskunst}{0}
    \item \nskill{Kontakte}{2}
    \item \nskill{Handwerk (SFX)}{3}
    \item \nskill{Täuschung}{1}
    \item \nskill{Fahren}{1}
    \item \nskill{Charisma}{3}
    \item \nskill{Kämpfen}{0}
    \item \nskill{Nachforschung}{0}
    \item \nskill{Spezialwissen}{0}
    \item \nskill{Wahrnehmung}{2}
    \item \nskill{Kraft}{1}
    \item \nskill{Provozieren}{0}
    \item \nskill{Empathie}{4}
    \item \nskill{Ressourcen}{0}
    \item \nskill{Schießen}{2}
    \item \nskill{Heimlichkeit}{0}
    \item \nskill{Wille}{0}
    \end{skills}

    \begin{stunts}
    \item \stunt{Disco !}{Because I'm Disco Artist, I get +2 when I use Craft (SFX) to draw or manipulate attention or mood of people or creatures using my disco systems.}
    \end{stunts}

    \begin{stressSection}
    \stressLine{\stress{1}\stress{1}\stress{1}\stress{1}}{\stress{1}\stress{1}\stress{1}}
    \end{stressSection}
    \begin{tabularx}{\textwidth}{ XX }
    \end{tabularx}

    \begin{consequences}
    \item \consequence{2}
    \item \consequence{4}
    \item \consequence{6}
    \end{consequences}

    \begin{npcDescription}
    Fidgety colourful party kid. Clothing has expanded over the years to include more and more quirky accessories
    \end{npcDescription}


    \begin{equipment}
    \item Light source (OLED film: battery operated, can be cut to size and glued on. Colour controllable)
    \item Food (Mama Salsa's famous mealworm buns in the bento box)
    \item First aid kit
    \item Mobile computers, headphones, communication via radio (mesh network)
    \item A dozen mini lighting drones for festivals
    \item Disco sound equipment (loudspeakers, recordings, microphones, all wirelessly connected)
    \end{equipment}
\end{npcBox}

%%%%%%%%% Primrose
\newpage
\begin{npcBox}[title=Primrose]

    \begin{aspects}
    \item \aspect[High Concept]{Lovable and pacifist eco-terrorist}
    \item \aspect[Trouble]{Nature does best without man}
    \item \aspect[Relationship]{Nature is great. People are OK too.}
    \item \aspect[Aspect]{Finding new nature and conserving it}
    \end{aspects}

    \begin{skills}
    \item \nskill{Academics (Biology and Ecology)}{4}
    \item  \nskill{Athletik}{3}
    \item \nskill{Diebeskunst}{2}
    \item \nskill{Kontakte}{0}
    \item \nskill{Handwerk (Sprengstoffe)}{2}
    \item \nskill{Täuschung}{1}
    \item \nskill{Fahren}{0}
    \item \nskill{Charisma}{1}
    \item \nskill{Kämpfen}{0}
    \item \nskill{Nachforschung}{0}
    \item \nskill{Spezialwissen}{0}
    \item \nskill{Wahrnehmung}{1}
    \item \nskill{Kraft}{0}
    \item \nskill{Provozieren}{0}
    \item \nskill{Empathie}{2}
    \item \nskill{Ressourcen}{0}
    \item \nskill{Schießen}{0}
    \item \nskill{Heimlichkeit}{3}
    \item \nskill{Wille}{1}
    \end{skills}

    \begin{stunts}
    \item \stunt{Do Drugs}{Because I'm experienced Eco-Terrorist, I get +2 when I use Academics (Biology) to use psychoactive substances to manipulate moods.}
    \end{stunts}

    \begin{stressSection}
    \stressLine{\stress{1}\stress{1}\stress{1}}{\stress{1}\stress{1}\stress{1}\stress{1}}
    \end{stressSection}
    \begin{tabularx}{\textwidth}{ XX }
    \end{tabularx}

    \begin{consequences}
    \item \consequence{2}
    \item \consequence{4}
    \item \consequence{6}
    \end{consequences}

    \begin{npcDescription}
    The clothes are visible eco-textiles - something that is not necessary with today's technology. It is a conscious choice.
    Primerose is an eco-hippie with rasta hair and a self-crocheted shirt.
    \end{npcDescription}


    \begin{equipment}
    \item Light source (OLED film: battery operated, can be cut to size and glued on. Colour controllable)
    \item Food (Mama Salsa's famous mealworm buns in the bento box)
    \item First aid kit
    \item Mobile computers, headphones, communication via radio (mesh network)
    \item Lock picking set
    \item Little biolab in a box
    \end{equipment}
\end{npcBox}


%%%%%%%% Spark
\newpage
\begin{npcBox}[title=Spark]

    \begin{aspects}
    \item \aspect[High Concept]{A punk in action - with tools}
    \item \aspect[Trouble]{Wants to learn new tricks from old technology}
    \item \aspect[Relationship]{Motivate others to tinker}
    \item \aspect[Aspect]{If it's small: shake it, if it's big: kick it}
    \end{aspects}

    \begin{skills}
    \item \nskill{Academics (Engineering)}{3}
    \item  \nskill{Athletik}{1}
    \item \nskill{Diebeskunst}{0}
    \item \nskill{Kontakte}{0}
    \item \nskill{Handwerk}{4}
    \item \nskill{Täuschung}{0}
    \item \nskill{Fahren}{2}
    \item \nskill{Charisma}{0}
    \item \nskill{Kämpfen}{2}
    \item \nskill{Nachforschung}{0}
    \item \nskill{Spezialwissen}{0}
    \item \nskill{Wahrnehmung}{1}
    \item \nskill{Kraft}{3}
    \item \nskill{Provozieren}{0}
    \item \nskill{Empathie}{0}
    \item \nskill{Ressourcen}{2}
    \item \nskill{Schießen}{1}
    \item \nskill{Heimlichkeit}{0}
    \item \nskill{Wille}{1}
    \end{skills}

    \begin{stunts}
    \item \stunt{McGyver genes}{Because I have McGyver gear and genes, I get +2 when I use crafting to screw something together in a hurry. Immediately after the successful use of the improvised hack it will probably fail spectacularly.}
    \end{stunts}

    \begin{stressSection}
    \stressLine{\stress{1}\stress{1}\stress{1}\stress{1}\stress{1}\stress{1}}{\stress{1}\stress{1}\stress{1}\stress{1}}
    \end{stressSection}
    \begin{tabularx}{\textwidth}{ XX }
    \end{tabularx}

    \begin{consequences}
    \item \consequence{2}
    \item \consequence{4}
    \item \consequence{6}
    \end{consequences}

    \begin{npcDescription}
    "Straw hat" woven from scraps of cable, other pieces of machinery woven into clothing.
    \end{npcDescription}


    \begin{equipment}
    \item Light source (OLED film: battery operated, can be cut to size and glued on. Colour controllable)
    \item Food (Mama Salsa's famous mealworm buns in the bento box)
    \item First aid kit
    \item Mobile computers, headphones, communication via radio (mesh network)
    \item Laser welder (no weapon)
    \item Duct tape, WD40 and Swiss army knife
    \end{equipment}
\end{npcBox}

\newpage

\section{Impact !}

All these adventures have an impact on the world. And especially in Solarpunk it is important to make this impact visible. Please write down the decisions the players have made, the friendships they have forged. The results they achieved for the Pioneer community. And then decide which of these things can appear in other stories.
The characters in this adventure were only meant for this adventure. But they had a lasting effect.

\begin{itemize}
    \item Will there be a special episode about the killer hamster ?
    \item Will a limited quantity of the glowing beer be available as a super-premium offer in Norm bars?
    \item Will you be able to visit the Pioneer Community? Will there be a giant hamster cage with several tiny monster hamsters?
\end{itemize}

Write down your ideas now and use them in the next few sessions.
