\chapter{Project lifeguard}
\label{ch:project lifeguard}

\section{Topic}

The aim of this adventure is to build a new community (or several: one for each faction), stop the evil plans of an old company (which is a relic) and give people new hope.

The protagonists can be of any faction. A mixed group is best.

The beginning of the adventure is well planned to get everything going smoothly. The core of the adventure (Albstadt) is a sandbox with characters, problems and locations.

Here players can start to help the people, build a community, fight the relic and make new friends. They work together - until they have to decide what kind of community they want to build:

\begin{itemize}
    \item Pioneer style
    \item Norm style
    \item Lost style
\end{itemize}

Their behaviour towards the NPCs and the style of their problem solving will tilt the NPCs towards one faction's philosophy or the other. The GM should take note of this and adjust the behaviour of the NPCs they play.

In addition, the NPCs may reveal some aspects of the main antagonists or the environment that can be used in the showdown.

The cities are based on real existing 2023 cities. For more details use Wikipedia and search for pictures.

This is a complete sandbox and I expect players to create a new and unique community with its own traditions.

\section{Summary}

There is a town (Albstadt in the Swabian Alps, Germany) that was lost in a disaster 25 years ago. The current inhabitants improvise to survive, but have never asked for help. About 20 years ago, they were contacted by Project Lifeguard, which sends them survival supplies. Most of it is quite cheap and even destroys their own small projects to become independent (e.g. cheap food kills local farming projects). About 2 years ago some people disappeared (a local dictator called Ash felt threatened). One person decides to escape to check what is going on outside Albstadt and sets the adventure in motion. And the first investigations will reveal the truth: Project Lifeguard does not exist.

The truth is: An old company is behind this project, using a cave system in Albstadt to dump toxic waste. Project Lifeguard is just a cover story and a cheap way to control the locals by making them dependent on continuous supplies.

The protagonists are hired by the UN to investigate. You have the chance to help the locals, build communities or city hives of your own design and uncover the evil plans of the \hyperref[sec:Relic]{relic} company.

\begin{sidebarBox}[title=Relics]
\hyperref[sec:Relic]{Relics} are objects or organisations from the past. They do not fit into the new world and will soon be dismantled. The list of relics to be dismantled is long. And some just keep on going until it is their turn. Or they fight back.
\end{sidebarBox}

\begin{sidebarBox}[title=UN]
The \hyperref[sec:UN]{UN} is the last big institution left in 2050. It organised the global rescue efforts and introduced the Resource Point system. They are the good guys.
But it is also a kind of "franchise". Decentralized small organisations are a key for local action. This is where player characters can get their jobs from !
It can pack a punch if a big effort is needed, but will not act without proof and a dire situation.
Local action is preferred.

The UN has some camps with heavy machines including VTOL (Vertical Take off and Landing) zeppelins and electric helicopters. They use it to rescue people in disasters. And can be called only after a detailed investigation by people on the ground - the exact type of the disaster and the local situation has to be known first. The next camp to Albstadt is in Stuttgart.
\end{sidebarBox}

\section{Getting started}

The protagonists can earn \hyperref[sec:Resource Points]{Resource Points} by helping the UN investigate Project Lifeguard and by helping the people of Albstadt by building a Lost Family, Pioneer Community or Norm City Hive (or all three) to integrate them back into civilisation.

The first meeting place is the nearest Norm city, the Sigmaringen Hive. Here you can meet the person who escaped from Albstadt and was picked up by the Lost.

\subsection{Scene one: UN meeting}

The player characters of all the factions have been summoned to the Norm city of Sigmaringen for a meeting of the UN. No one knows why. But they do know that the payment for helping the UN is Resource Points.

Norms use their AR interface called "\hyperref[sec:Hive controller]{Hive Controller}" to organise projects, chat and meet online.... To the non-Norms, people on the street seem to gesticulate with mysterious hand movements, interacting with something no Lost or Pioneer can see - they have no Hive Controller.

They will arrive independently (perhaps at the train station or the Zeppelin station next to it) and will meet the Norm characters at that location. Lost characters from their camp near Albstadt could have just walked there. Lost and Pioneers will not know how life in a Norm hive really works. Even if they live nearby.
Non-Norm guests will receive a Hive Controller as a gift. These devices are not yet configured and the new users have not been through any tutorials. The usability of these devices is limited to what a 2 year old Norm could get out of his first device. But at least the controllers have a few euros in the expense account to pay for premium services during the trip to the UN meeting.

Sigmaringen is a nice and relaxed town. Many streets have been replaced by parks after 2020. Tame deer walk around and can be fed (or steal anything that looks like food). Solar panels are attached to every available surface. Public transport works perfectly and is based on amphibious buses that can also use the Danube. There are vending machines in the city and in the public vehicles where people can get free food and other basic necessities. This is normal for the Norms and may seem strange to Lost and Pioneers.

Next to the train station is a large Zeppelin airport with several logistics companies using these airships to transport standard containers.

Sigmaringen is self-sufficient with a large \hyperref[sec: norm food]{Ceres} food production centre.

There is a castle on a small hill in the centre of Sigmaringen. It has been fitted with solar panels in recent years - but they are so well integrated that they are hardly noticeable. It is a 5 minute walk from the train station. Walking there, the non-Norms can experience a modern town, free public transport, free vending machines, deer, trees and a small stream that crosses the town streets.

The meeting room in the castle has been prepared for the UN conference. Someone got the "Emergency UN meeting props box" from the attic and decorated everything with large and small flags, put cards with the UN logo on the table, installed a UN logo on the big video screen in the room.

After all the characters have arrived, they are greeted by the city officials, who also have no idea why everyone is here (but are proud to be hosting a UN conference).

Perfect last chance for the characters to introduce themselves before the UN meeting starts.

After a short time, a UN official appears on the screen (from Addis Ababa, Ethiopia). Her name is Enku and she is an expert on relics.
The connection to Africa is via a few fibre-optic cables, a few high-flying repeater drones and other network hacks. Satellites no longer exist after the \hyperref[sec: Kessler Syndrome]{Kessler Syndrome}.

The characters are given a mission:
The next town "Albstadt" is a no-go area and a relic. It was destroyed when the caves underneath the area collapsed. Caused by low water levels over several years. A Lost Family camped nearby (the family of the characters in the meeting). Something strange happened there. But the characters can tell the story themselves.

GM: The next part is to tell the story of the three factions that brought them here. Divide the group and keep it short but interesting.

\subsection{Flashback for Lost: T minus 5 days}

Takeaways:

\begin{itemize}
    \item Kim fled from Albstadt. It is still inhabited! Find the refugee in the Lost Camp. Severely injured and unconscious.
    \item A professional fighter was hunting this person - obviously equipped with illegal cyberware (i.e. spare parts that enhance normal human abilities).
    \item Kim claimed that the Lifeguard project was helping the city. But life there is hard. Before losing consciousness.
\end{itemize}


The lost camp is between Sigmaringen and Albstadt in the wilderness. Close to Winterlingen. Albstadt and Sigmaringen are part of the Lost Hunting Ground. But they respect the Norms in Sigmaringen and will not be caught hunting there. And they know that the no-go area around Albstadt is very risky and avoid it. The camp is mobile and consists of diesel vehicles and tents. No Lost knows what happened in Albstadt 20 years ago. No one is allowed to go there. But they raid the outer ruins for things.

In the evenings, while partying, they eat "Risotto Rodenti" (which contains rats, squirrels and especially invasive species) and play \hyperref[sec:Shakespeare battle]{Shakespeare battle}.

While they are celebrating, the Lost guards on the perimeter stop a person fleeing from Albstadt. The person is wearing rags. A military cyborg is chasing the person.

The perimeter guards stop the Cyborg with gunfire. As soon as the Cyborg loses the fight, the Cyberware self-destructs, killing it. The Cyberware is obviously illegal.

(The player characters do not take part in the battle. But they can help later: First aid, examining the cyborg, ... So they know the essential information first hand).

The rescued person (called Kim) tells a strange story:

He was born in Albstadt. Which is almost 100 per cent destroyed. The collapsed area has made the whole town very vertical. Some ruins of buildings are still attached to rock spikes. Caves are now accessible where there were none before. People built makeshift bridges and waited for help. 10 years ago it came. \textbf{Project Lifeguard} came. The Cyborg is part of this project.

They come once a month and bring the bare minimum. Food, medicine, water, energy (a fuel cell in a cargo container and enough hydrogen for a month). The city is completely dependent on them (this is the plan of the Lifeguard project).

People began to disappear. That is why Kim fled. Kim soon loses consciousness and needs professional medical care. This can be found in Sigmaringen.

Kim is taken to a hospital in Sigmaringen. As soon as he regains consciousness, he will be questioned and will be able to attend the UN meeting via video call.

Kim's arrival has triggered a UN investigation into Project Lifeguard. See next section for the Norms.

\subsection{Flashback for Norms: T minus 3 days}

The Norm people in the room were asked by the UN to investigate a few days ago. Right after Kim was taken to hospital and the incident was reported. The only clue they got was "What is Project Lifeguard? They ship 3 containers a month with survival equipment".

Some of the investigation can only take place in AR mode, while the characters are enjoying their free latte macchiato with soy milk and cake somewhere on a Danube beach in Sigmaringen. Or they can break into the shipping terminal. Or scam the people who work there.

Other options: making phone calls, searching online databases, checking webcams and controlling drones.

You can learn:

\begin{itemize}
    \item The containers with the project logo are from Sigmaringen.
    \item Project Lifeguard is not registered anywhere!
    \item So officially these containers never started there.
    \item A company called "Dumpit waste management incorporate" has the same warehouse and is a subsidiary of "Cargo!" which does all kinds of shipping with Zeppelins.
    \item The 4 Dumpit containers are transported to Albstadt by Zeppelin drones.
    \item Only one is paid for by the institutions wishing to send the waste for recycling.
    \item Dumpit is the only company that ships containers in the same way - by passing Albstadt and sending them to a recycling plant in Luxembourg (this is the official version. In reality they just dump the waste in Albstadt. This has not yet been revealed)
\end{itemize}

When you enter the Cargo! container area, you will see that the containers all look the same with an ID code printed on them. The logo on the containers can be changed in seconds because it is just a giant multi-coloured e-paper. This is also the reason why you will not find a container with the Project Lifeguard logo on it, but it will literally appear in mid-air.

A Cargo Zeppelin can carry 4 containers and can be lowered with a crane. A standard trip to Luxembourg consists of 1 Dumpit waste container and 3 Project Lifeguard containers. They all start with the Dumpit logo.

\subsubsection{Zeppelin pilot Luke}

Luke is the Zeppelin pilot for Cargo! He has a strong suspicion that something is wrong with the Dumpit job. But he also knows that investigating and digging deeper will only confirm it, and then he will have to decide whether he wants to keep a good job or not. When approached on a personal level, a very good \textbf{Rapport} job might get some information out of him:
\begin{itemize}
    \item He never flies to Luxembourg
    \item There is a stop at Albstadt
    \item It is not a wasteland as expected. But he will not say more
    \item He starts with 4 Dumpit containers (not mentioning that 3 will be rebranded during the flight)
    \item He never mentions what happens to these containers in Albstadt. Not the 3 dropped at the supermarket, not the one emptied into the "bottomless pit".
\end{itemize}

Some of these topics make him uncomfortable. He will avoid them by talking about his favourite show, which he watches while flying his delivery routes.

GM: After realising that something was strange in Albstadt, the UN asked the local Pioneers for help. See next section.

\subsection{Flashback for Pioneers: T minus 2 days}

The Pioneer camp is located on the former university campus on a hill near Sigmaringen. The Pioneers occupied the abandoned university campus 10 years ago (the climate catastrophe forced everyone to flee). The boring cement buildings are just as they used to be, except for one thing:

\begin{itemize}
    \item The interior is now decorated with unicorns, or strange experiments can be found in the corridors.
    \item Outside and between the buildings, a high-tech shantytown has been built. From human-sized honeycombs glued to the wall for shelter to a small garden tended by robots.
    \item The laboratories are used, but not abused. This is the richness the Pioneers found there.
\end{itemize}

This university campus is a 30 minute walk from Sigmaringen (for the Lost). 5 minutes on an experimental Pioneer electric trike with AI assisted dampers - low risk. Or just call public transport and drive 15 minutes and have a coffee and a croissant (Norms).
The Pioneers think Sigmaringen is boring and don't usually visit it.

The mission: The UN needs a scout robot with construction skills in this region. And fast. The Pioneer community advertised their robot some time ago, so they get a call to send their robot and 1-2 operators.

The UN call gets everyone excited. So the community decides to throw a party and make some last minute fixes and improvements ....

These fixes are all introduced by slightly drunk crazy engineers (Pioneers ....) and harvest parts from other projects. The player characters can defend the robot's current features or move on. Each time you make a prototyping roll, the robot gains the listed features; if you fail, it also gets a new glitch. See the list below for ideas.

The robot starts as a semi-autonomous construction bot. The bot is table-sized, has 6 legs, a nozzle to spread a self-hardening concrete for building, and can run on a battery and a concrete supply of 6 bucket-sized containers. It is remote controlled.

Modifications suggested by the party people:

\begin{itemize}
    \item Gecko grip (on fail: will accidentally stick to people and other objects)
    \item Waterproof, 10 metres (on fail: building is flooded during test)
    \item Other material: wood pulp, plastic (on fail: ugly test object sticks to the floor and no one knows how to remove it)
    \item Can turn wood into pulp (fail: people keep feeding it furniture in their enthusiasm)
    \item Feed pipe for large projects (on fail: The community is now in possession of a large scale video character monument....)
    \item Autonomous option (on fail: starts printing strange objects when no one is looking)
    \item Ink cartridge (on fail: can now only print rainbow-coloured things)
\end{itemize}

Finally, the Pioneer kids will paint it with bright colours and add goggly eyes. This is not optional.

Then the delegation will take the robot to Sigmaringen to meet the other groups at the UN meeting.

GM: Go back to the joint UN meeting. Exchange information in the next section.

\subsection{UN meeting continued}

The UN meeting can then continue. Kim can be questioned, the protagonists are offered Resource Points in exchange for their help, ...., and are sent towards Albstadt. Perhaps with a short stop at the Lost Camp in between.

Some players will want to start investigating in Sigmaringen, now that the basic problem is known. Encourage them. They may find many answers. But not enough to satisfy the UN. They will have to do some investigation on the ground to confirm their suspicions.

\section{Areas}

\subsection{Sigmaringen Hive}

Sigmaringen is already described in the UN meeting section.

The protagonists can start their investigation in Sigmaringen. Here they will find a network of companies and shell companies, all belonging to "Cargo!

\subsection{Lost camp}

If the protagonists want to visit the Lost Camp between Albstadt and Sigmaringen, they will find:

\begin{itemize}
    \item Several diesel trucks
    \item Tents
    \item A whiteboard with a sector map (including Sigmaringen and Albstadt. With Sigmaringen marked "food" + "inhabited" and Albstadt marked "no-go" + "danger")
    \item A kitchen ready to cook rabbit stew and the deer a team of hunters shot "near Sigmaringen". The kitchen has camping-level hygiene standards. And eating real animals can confuse Norms.
    \item A library in a sheltered old school bus. Visitors can read the books, but are not allowed to touch them.
    \item The Hive Controllers will begin to lose radio contact with the Hive. This will make the Norms very nervous.
\end{itemize}

The sector map is a map that a Lost party will start to build as soon as they arrive somewhere. The area is divided into sectors, with the camp in the middle. Teams begin to explore what they can find in each sector: food sources, ruins, threats, and to complete the map.

When entering the camp with a Lost Guide, the guide must vouch for his friends. They will then be searched for weapons. If none are found, they will receive  \hyperref[sec:Lost guests and weapons]{easy to use weapons}.

\subsection{Forest: The no-go area close to Albstadt}

Near Albstadt there is a fence with many warning signs. The area looks like a normal forest. But if you look closely you can see 50m deep ditches starting just behind the fence. Closer to Albstadt these ditches can be up to 200m deep. So the danger is real.

The trick to crossing them is to build a kind of bridge to support the brittle forest floor. With the Pioneer robot, this is a piece of cake. But wooden planks and some testing of the ground can also help.


\subsubsection{Action scene}

Halfway to Albstadt, some wild boars will act as agents of chaos to create a life-threatening situation. The boars do not attack directly. However, if they are disturbed by the visitors, they will act rashly and trigger the real threat: Fallen trees (falling sideways or disappearing vertically into a new 50m ditch), broken forest floor, ...
The whole landscape becomes the enemy, and fighting the boars (to reduce the amount of chaos) or tinkering to build some stable ground or secure the remaining trees will be necessary to save the day.

\subsubsection{Part of the solution}

The forest can be part of the solution if the protagonists want to create a self-sufficient Albstadt. There is fertile soil for farming and tasty animals. Stability is an issue. But with good mapping and some structural repairs, parts of the forest can be made accessible.

\subsection{Albstadt}

From the edge of the forest, the protagonists can see a deep slope (about 200m deep) and flowing water at the bottom. The remains of Albstadt are on spires that reach up to normal ground level.

Albstadt was destroyed by a catastrophe. The karst area has collapsed. The landscape is now more vertical with cliffs than horizontal. Some buildings have been structurally secured to prevent them from falling completely into the ditches. Improvised bridges are everywhere. Caves have opened up. Some are filled with water and require diving equipment to reach the bottom (where some valuables from the disaster may have ended up).
The people here have done a good job of surviving in their shattered city. But they were afraid to ask for help. They were found by Project Lifeguard, but the help they sent made them even more dependent.

(For the cave system: see "Blautopf" on Wikipedia)

\subsubsection{General}

Albstadt is a maze of spires, with the tops of the spires at ground level. The canyon system is filled with water and caves (dry or filled with water) are everywhere.

Old buildings either survive at the top of the spires or hang off to the side. They are heavily damaged and secured with ropes.

Between the towers there are improvised suspension bridges and supply lines to the supermarket (electricity, water). All of this needs to be repaired or replaced.

At the start of the adventure, the Sigmaringen hive will drop a radio repeater near Albstadt. The Norms will have a connection to the Sigmaringen hive. No dice penalty, but the delivery time is +1h.

\subsubsection{Supermarket}

The supermarket is the largest surviving structure. It is controlled by Ash and his thugs. Ash is the local despot. The car park in front of the supermarket is where the cargo zeppelins drop off the containers. These are immediately secured by Ash and his goons.
They will store the goods in the supermarket (which is filled with shelves of wealth).

While the car park and the main building of the supermarket look intact, the entire back wall of the building is missing. It has fallen down the side of the tower. From a safety point of view, this is not a problem, as no one who lives there could climb up the steeple wall and get in that way. Or build a bridge or stairs with a robot - this is an option available to the protagonists.

The supermarket has many shelves with an epic amount of food and tools. A small but nice set of rooms where Ash lives and Ash's office. This is where he keeps all the dirty secrets he has collected about Dumpit.

\subsubsection{Part of the solution}

\textbf{Problem:} It is controlled by Ash and his thugs. He uses it to control the whole city through artificial scarcity.
\textbf{Solutions:} Get rid of Ash, blackmail him
\textbf{Benefits:} A building for storage, crafts, town centre, ....

Ash also has plenty of blackmail material to bring down Project Lifeguard and Dumpit.

\subsubsection{Leo's}

\textbf{Leo} tried to repair his building and spire as best he could. But he lacked materials and technology. After accidentally killing a person (throwing a stone to help Kim escape), he is in a depressed state. In his current mental state, he cannot help - but perhaps someone can help him out of it. And get him some tools!

\subsubsection{Part of the solution}

\textbf{Problem:} Leo does not have the technology to build what he wants properly.

\textbf{Solution:} Scale up, use Pioneer tech, order stuff from Sigmaringen

\textbf{Problem 2:} Leo is mentally unstable after accidentally killing someone

\textbf{Solution 2:} Leo needs therapy (long-term solution) or a call to action (and more therapy later)

\textbf{Benefits:} A first hanging garden, proper bridges and supply lines. A showcase for further improvements and a free Albstadt.

\subsubsection{Lara's}

\textbf{Lara} lives in an old house with a broken roof on top of a tower. Here she takes care of three poisoned people. The only special thing about her tower is a long ladder that leads down to a small dock she has built on the water, where she keeps her canoe.
She has three sick people in her house. They all went to the secret Dumpit Cave downriver and brought back a 'glow in the dark' talisman. A shard of radioactive material that was part of the waste dumped there. This shard caused radiation sickness. They still keep it around and Lara is unaware of the danger.
Lara is the one who knows the region. Especially the wilderness. If the patients are taken care of (maybe with Norm/Pioneer tech) she can show the people around. So sources for water, food and renewable energy can be found. Also places to prepare an ambush for the showdown.

\subsubsection{Part of the solution}

\textbf{Problem:} Lara has to look after the patients

\textbf{Solution:} Caring for patients with the right medical technology

\textbf{Benefits:} A proper map of the natural area with different sources of food, water and energy. And places to ambush

\subsubsection{Water level maze}

This is where \textbf{Lara} spends her time: 200m below the top of the towers. There are waterways with beaches, rapids, caves and large rocks in the water. Most of the water is drinkable (basic filtration is recommended). The animals and plants are edible. What the people living there do not know.

A small boat or canoes will be important for travel.

There are many dangerous caves. For natural reasons. Downstream there is a cave where toxic waste has been dumped. Staying there without protective equipment can be dangerous. Also: Do not eat the crabs there.

%% https://en.wikipedia.org/wiki/Orphan_source

\textbf{Problem:} Invasive American crabs eating away at local life

\textbf{Solutions:} Set traps, eat the crabs

\textbf{Benefits:} Food

\subsubsection{Sams bunker}

\textbf{Sam} is hiding in a bunker-like control structure. Built to hold 20 people, it was home to 3 until recently (the cyborg and Sam's colleague were killed). She is in shock at the death of her colleagues. She is also not happy with the project she is working on. Sam wanted to do something good, but it turns out that the project she signed up for is a fraud and is harming the people of Albstadt. Over the past few days, she has been collecting log data and notes of what has happened over the past few years (Sam arrived a few months ago, but the logs go back many years). This information can destroy Dumpit and everyone involved.

\textbf{Problem:} Sam is scared and hides there.

\textbf{Solutions:} Talk, produce some evidence that the situation has calmed down.

\textbf{Benefits:} Workshop, drones and information about Dumpit

% What is left ? Church as city assembly. An old super market as source for tools, a museum for local history (really old farm tools),
% People life in shanty town, makeshift tents
% How did the project replace essential things ? Power => Battery container, Food => Cans, Water => Bottles, Medicine => A doctor visits once a month
% Who is left ? Some people could build a sustainable life. Lifeguard is blocking that.
% How do they interact ? Some people love the dependency on PLG
% What can they find out ? Something is going on in the tunnels. The help is keeping people down.
% How can they rebuild ? Start gardening, build water cisterns,
% Who can be recruited & empowered ?
% How will PLG respond ? Sabotage and violence ? What before that ? Threats ?
% Supermarket owner gets gunned by PLG to defend the last resources
% Person who wants to do medic only gets basic material and everyone has to wait for the monthly visit of PLG for real help
% There is a spy for PLG, but this spy can be turned and used against PLG

\subsection{People at Albstadt}

There are about 200-300 people in Albstadt. Most of them suffer from learned helplessness, acquired over the last 20 years. They have even forgotten how to cook and just use the microwaves with the pre-prepared meals offered by Project Lifeguard. Some are very active, struggling to keep everyone alive. But Project Life~-guard tricked them into non-sustainable ways of struggling and building up more and more dependencies on Project Life~-guard. These people could - using their skills in a smarter and more coordinated way - build a non-dependent Pioneer Community/~-Lost Family or Norm Hive. Until then, they are a problem because they maintain the stupid structure tactically implemented by Project Lifeguard.

These are the NPCs to interact with.

\section{NPCs}

\newpage

\begin{npcBox}[title=Leo]

    \begin{aspects}
    \item \aspect[High Concept]{An architect}
    \item \aspect[Trouble]{I will hulk out if you make me angry}
    \item \aspect[Aspect]{I dream of building big, but the situation keeps me small}
    \end{aspects}

    \begin{skills}
        \item \nskill{Academics}{3}
        \item \nskill{Athletics}{1}
        \item \nskill{Burglary}{0}
        \item \nskill{Contacts}{1}
        \item \nskill{Crafts}{4}
        \item \nskill{Deceive}{0}
        \item \nskill{Drive}{0}
        \item \nskill{Empathy}{1}
        \item \nskill{Fight}{2}
        \item \nskill{Investigate}{0}
        \item \nskill{Lore}{1}
        \item \nskill{Notice}{0}
        \item \nskill{Physique}{3}
        \item \nskill{Provoke}{0}
        \item \nskill{Rapport}{0}
        \item \nskill{Resources}{2}
        \item \nskill{Shoot}{0}
        \item \nskill{Stealth}{0}
        \item \nskill{Will}{2}
     \end{skills}

    \begin{stunts}
    \item \stunt{Improvising}{Leo gets a +2 to crafting when building architecture with alternative materials or scrap.}
    \end{stunts}



    \begin{stressSection}
    \stressLine{\stress{1}\stress{1}\stress{1}\stress{1}\stress{1}\stress{1}}{\stress{1}\stress{1}\stress{1}\stress{1}}
    \end{stressSection}
    \begin{tabularx}{\textwidth}{ XX }
    \end{tabularx}

    \begin{consequences}
    \item \consequence{2}
    \item \consequence{4}
    \item \consequence{6}
    \end{consequences}

    \begin{npcDescription}
    An impulsive architect, he has a spire with a house on it. He is trying to cultivate the side of the rock by attaching small boxes for plants.
    He was part of the 5 minute riot that Kim escaped from. Being impulsive, he threw rocks at the three Project Lifeguard people. The cyborg followed Kim. A rock thrown by Leo hit an engineer, throwing him off a bridge and killing him. Sam, the third engineer, was hiding in their bunker. An antenna was destroyed by rocks thrown.
    After the rebellion turned bloody, Leo retreated. Shocked by his own actions.
    If this can be fixed, they can gain the support of a skilled architect who only lacks the tools to build fantastic vertical gardens and bridges.

    Leo is 60 years old and remembers the time before the disaster. He moved to Albstadt 2 years before the disaster. Leo studied engineering and architecture before moving there.

    \textbf{Location:} His own spire. With a small house (repaired). And a small hanging garden at the side of the spire (just some flower pots, he did not have the right tools for proper hanging gardens).

    \textbf{Problem:} Has to deal with killing someone. Maybe talking to Sam will help.

    \textbf{When convinced:} Can help build things (a real hanging garden, real suspension bridges, crab traps)
    \end{npcDescription}

\end{npcBox}

\newpage

\begin{npcBox}[title=Lara]

    \begin{aspects}
    \item \aspect[High Concept]{One Woman Rescue}
    \item \aspect[Trouble]{Willing to sacrifice for others}
    \item \aspect[Aspect]{Knowledge of the country where I spend my time}
    \end{aspects}

    \begin{skills}
        \item \nskill{Academics}{0}
        \item \nskill{Athletics}{2}
        \item \nskill{Burglary}{1}
        \item \nskill{Contacts}{0}
        \item \nskill{Crafts}{0}
        \item \nskill{Deceive}{0}
        \item \nskill{Drive}{3}
        \item \nskill{Empathy}{0}
        \item \nskill{Fight}{2}
        \item \nskill{Investigate}{2}
        \item \nskill{Lore}{0}
        \item \nskill{Notice}{4}
        \item \nskill{Physique}{3}
        \item \nskill{Provoke}{0}
        \item \nskill{Rapport}{0}
        \item \nskill{Resources}{0}
        \item \nskill{Shoot}{1}
        \item \nskill{Stealth}{1}
        \item \nskill{Will}{1}
     \end{skills}

    \begin{stunts}
    \item \stunt{Just standing here}{Lara can get a +2 to Notice when silently observing people. As long as she can use a cover story involving her trips into nature or helping people.}
    \end{stunts}


    \begin{stressSection}
    \stressLine{\stress{1}\stress{1}\stress{1}\stress{1}\stress{1}\stress{1}}{\stress{1}\stress{1}\stress{1}\stress{1}}
    \end{stressSection}
    \begin{tabularx}{\textwidth}{ XX }
    \end{tabularx}

    \begin{consequences}
    \item \consequence{2}
    \item \consequence{4}
    \item \consequence{6}
    \end{consequences}

    \begin{npcDescription}
    Lara is a friendly guide who knows everything about Albstadt. She wants to uncover the truth about the cursed and haunted cave, the so-called secret place where the Project dumps its toxic waste. It is not really haunted. But the people who went there soon died of disease.
    In order to carry out this investigation, the guide would have to abandon an important duty: taking care of the people she has rescued after accidents. They need medical care and leaving for 1-2 days is not an option.

    Lara is 30 years old. She has lived in Albstadt all her life and can barely remember the time before the accident.

    \textbf{Location:} At home (which needs repairing) looking after the injured. Some travel.

    \textbf{Problem:} Needs someone to look after the injured.

    \textbf{When convinced:} Can help you explore Albstadt and the Haunted Cave.
    \end{npcDescription}

\end{npcBox}

\newpage

\begin{npcBox}[title=Sam]

    \begin{aspects}
    \item \aspect[High Concept]{Scared Good Samaritan}
    \item \aspect[Trouble]{Are we the baddies ?}
    \item \aspect[Aspect]{Helper with evil masters}
    \end{aspects}

    \begin{skills}
        \item \nskill{Academics}{4}
        \item \nskill{Athletics}{1}
        \item \nskill{Burglary}{0}
        \item \nskill{Contacts}{0}
        \item \nskill{Crafts}{3}
        \item \nskill{Deceive}{0}
        \item \nskill{Drive}{1}
        \item \nskill{Empathy}{1}
        \item \nskill{Fight}{0}
        \item \nskill{Investigate}{0}
        \item \nskill{Lore}{1}
        \item \nskill{Notice}{2}
        \item \nskill{Physique}{0}
        \item \nskill{Provoke}{0}
        \item \nskill{Rapport}{3}
        \item \nskill{Resources}{2}
        \item \nskill{Shoot}{0}
        \item \nskill{Stealth}{0}
        \item \nskill{Will}{2}
     \end{skills}

    \begin{stunts}
    \item \stunt{I know my bunker}{Sam has spent the last few weeks studying the bunker and its computers. Sam gets a +2 to Academics when answering questions about the bunker.}
    \end{stunts}

    \begin{stressSection}
    \stressLine{\stress{1}\stress{1}\stress{1}}{\stress{1}\stress{1}\stress{1}\stress{1}}
    \end{stressSection}
    \begin{tabularx}{\textwidth}{ XX }
    \end{tabularx}

    \begin{consequences}
    \item \consequence{2}
    \item \consequence{4}
    \item \consequence{6}
    \end{consequences}

    \begin{npcDescription}
    Sam is the last of the three project members. She wants to help people, but is restricted by company policy. Sam knows nothing about the poison dump (apart from hints in the logs and a few observations). She was helpful at first, but is now scared after the short escalation, the "revolution". As an engineer and medic, she could look after the injured. If she were not afraid.

    Also: The antenna to HQ is damaged. Sam cannot contact the outside world to ask for help.

    Remember: There is some good in this NPC. But she works for the bad guys.

    With some persuasion she can become a powerful asset and friend. She helps with medicine and engineering. And also knows (or suspects) some of the inner workings of Dumpit.

    Sam is 25 years old.

    \textbf{Location:} The company bunker. Or explore the city via drones and make contact (the drones have speakers).

    \textbf{Problem:} Frightened by the attack. Also doubts that the company will help

    \textbf{When convinced:} Can provide medical assistance. Some technology and insights into Project Lifeguard and the company "Dumpit
    \end{npcDescription}

\end{npcBox}

\newpage

\begin{npcBox}[title=Ash]

    \begin{aspects}
    \item \aspect[High Concept]{Brute with power}
    \item \aspect[Trouble]{I am totally dependent on project Lifeguard}
    \item \aspect[Aspect]{Obey: I have the food, water and power}
    \item \aspect[Aspect]{A budget bond villain}
    \item \aspect[Aspect]{The art of the smart escape}
    \end{aspects}

    \begin{skills}
        \item \nskill{Academics}{0}
        \item \nskill{Athletics}{1}
        \item \nskill{Burglary}{1}
        \item \nskill{Contacts}{2}
        \item \nskill{Crafts}{0}
        \item \nskill{Deceive}{2}
        \item \nskill{Drive}{0}
        \item \nskill{Empathy}{0}
        \item \nskill{Fight}{3}
        \item \nskill{Investigate}{0}
        \item \nskill{Lore}{0}
        \item \nskill{Notice}{1}
        \item \nskill{Physique}{2}
        \item \nskill{Provoke}{3}
        \item \nskill{Rapport}{0}
        \item \nskill{Resources}{4}
        \item \nskill{Shoot}{2}
        \item \nskill{Stealth}{0}
        \item \nskill{Will}{1}
     \end{skills}

    \begin{stunts}
    \item \stunt{Prepared}{Can use Resources instead of Athletics when running away, as long as Ash has prepared the escape route with gadgets to help him escape.}
    \item \stunt{Brave henchmen}{Ash's henchmen protect him and would even sacrifice their lives (3 self-sacrifices per story).}
    \item \stunt{Broken civilians}{Once per story, Ash can get a +2 to Provoke to motivate the broken civilians of Albstadt to help him against their own self-interest.}
    \end{stunts}


    \begin{stressSection}
    \stressLine{\stress{1}\stress{1}\stress{1}\stress{1}}{\stress{1}\stress{1}\stress{1}\stress{1}}
    \end{stressSection}
    \begin{tabularx}{\textwidth}{ XX }
    \end{tabularx}

    \begin{consequences}
    \item \consequence{2}
    \item \consequence{4}
    \item \consequence{6}
    \end{consequences}

    \begin{npcDescription}
    The old supermarket is the central hub for food, resources and energy. Everything is delivered there. Ash, the owner of the supermarket, was recruited to distribute it. The power soon corrupted him. He now controls the town. And thanks to blackmail, he can also control Project Lifeguard's deliveries.

    He already has some dirty secrets that he uses to blackmail Dumpit. But to gain even more power over his masters (the Project), he wants to investigate what is in the secret waste dump that is poisoning people.

    Ash is now 45 years old and was 25 when Dumpit arrived. He immediately saw the opportunity.

    Ash is the real villain

    \textbf{Location:} Supermarket

    \textbf{Problem:} Dependent on the Project Lifeguard. Wants to blackmail them. Just in case.

    \textbf{When convinced:} Cannot be convinced. Just blackmailed, coerced, .... will then offer control over the entire settlement and the data on Dumpit.
    \end{npcDescription}

\end{npcBox}




\subsection{Project Lifeguard by Dumpit waste management inc.}

History:
\begin{enumerate}
    \item 25 years ago Albstadt was destroyed by the catastrophe (2025)
    \item People managed to survive
    \item 20 years ago, Dumpit discovered the settlement. It wanted to use the area for illegal waste disposal (2030)
    \item Dumpit launched a humanitarian aid project in parallel with the dump investigation.
    \item Dumpit kept the settlement secret so as not to jeopardise their illegal business.
    \item Ash soon found out about this illegal part of the project and began to blackmail Dumpit.
    \item Ash's aim was to get into a control position for the aid. Which he did. And became the de facto ruler of Albstadt.
    \item Dumpit inc is OK with this arrangement. Changing it is risky
\end{enumerate}

Project Lifeguard is a relic project - an old project that will be dismantled as soon as the UN has time. Being small and seemingly irrelevant, it is at the bottom of the backlog.
Dumpit waste management inc. officially recycles waste in Luxembourg. Its dirty secret: toxic waste is dumped into a natural chimney in the Albstadt area. Where it falls a few hundred metres and ends up in a natural cave filled with water. This cave can be entered from the side by experienced divers. People have done it and some of them died after a long illness. This is why the cave is called "haunted". But nobody in Albstadt connected the dots.

The reason is "Project Lifeguard". This is the cover story told to the people of Albstadt. Humanitarian aid sent there. But it is designed to make the people dependent. At the same time as a container of toxic waste is sent by helicopter, three containers are dropped off at the supermarket: One with food and water, one with electricity (hydrogen generators) and one with clothes and tools.
This is barely enough for the people there to survive. But it makes them dependent and undermines their efforts to grow their own food, clean the water or install generators.
The other reason they cannot abandon their aid project is that Ash has enough on them to blackmail the company. His power depends on the resources supplied by Project Lifeguard.

A small outpost (manned by three people) is there to monitor everything. But the guards in this outpost do not mix with the people of Albstadt. They use drones and cameras to keep an eye on them.

Before the adventure begins, an engineer is killed by Leo throwing a stone. The security cyborg followed the fleeing Kim and was killed by the Lost.
The last engineer and medic, Sam, is hiding in the bunker during the adventure.

One secret that is hard to spot are the two satellite dishes for communication.

The obvious one is aimed at a stratospheric relay zeppelin to communicate with the company.

The more obscure dish is always pointed at the moon.

\begin{sidebarBox}[title=Kessler Syndrome]

    \hyperref[sec: Kessler Syndrome]{The Kessler Syndrome} has killed all GPS satellites, Earth mapping and communications satellites. Earth orbit is inaccessible. The only fallback left is high-flying drones for mapping and communications relays.
\end{sidebarBox}


\begin{sidebarBox}[title=Moon base]

    There is no moon base. None that humans are aware of. Early in the Dirty Road to Eden phase, the billionaires secretly left Earth. For the Moon, Mars and some are even digging tunnels under the Earth. Nobody knows about it. Thanks to the Kessler Syndrome (caused by the last one to leave Earth), they could not be followed even if people realised what had happened.
    In their secret bases they continue to dream their capitalist dream and influence Earth. They fight to keep the relic corporations alive.

    But this will be dealt with in more detail in another adventure. It will also deal with the so-called "Eat the Rich Festivals". The only hint in this adventure that something big might be going on is the antenna pointed at the moon.
\end{sidebarBox}

Endgame: If enough of their evil plans are exposed, the UN can put this relic at the top of their "to be dismantled" list. Lawyers and security forces will raid the headquarters. The company's assets will be seized.

\section{Potential solutions}

There are many ways to solve the problem. If your players find one that is not listed here, please contact me. Whatever they find, it is likely that they will enjoy a showdown where they can be active. For this reason, Ash will not give up, but will try to escape. See "Ash flees".

\subsection{Call the UN}

With enough evidence, the characters could call the UN. The UN will fast-track the termination of Dumpit (by tomorrow) and send troops to secure the area and evacuate the people:

\begin{itemize}
    \item \hyperref[sec:UN away team]{Several UN away teams}
    \item 6 \hyperref[sec:UN Cargo Zeppelins]{UN Cargo Zeppelins}
\end{itemize}

Before the support can arrive, the landing zone (car park in front of the supermarket) must be cleared of containers. They could be pushed down the tower with heavy machinery. The protagonists will also have to secretly install electronic guidance systems there. Then they can watch the UN intervention from another tower.

\subsection{Walkaway}

By finding a safe route through the forest, the people can simply leave. If the characters can convince them that it is safe and that there is a world outside of Albstadt that would welcome them.

\subsection{Independence}

Teach the people how to grow and catch their own food, test the water for purity and build power sources. Ash's grip will break very quickly.

\subsection{More proof}

Enter the haunted cave and find more evidence of Dumpit's wrongdoings. Present it to the UN and they will immediately shut down the company (it was kept alive because it promised to recycle the waste, not dump it). No company. No Ash. But sooner or later Albstadt will run out of food.

\section{Showdown: Ash flees}
Even if they call in the UN and have a lot of firepower, the UN will be busy landing the zeppelins and Ash can slip through their nets. As soon as Ash realises that his plans are failing, he will use all the gadgets hidden in Albstadt to escape. Ash is clumsy and a budget Bond villain. But his thugs are willing to buy him some time by fighting back, and being the coward that he is, he prepares his escape.

* Zip lines between spires
* Zip lines down from spires
* Hideouts with tools
* Ruins with secret doors
* An electric stand up paddle board ready to take him out through the waterways.

Turn his final escape attempt into an epic race between the protagonists, who (hopefully) have a proper map and ways to get around Albstadt on both water and spire levels. And a clumsy Bond villain with the help of prepared gimmicks and his goons.

\section{Player characters}

These characters are full characters with small optimisations for this adventure. You can still use them in your campaign if you wish.

\newpage

\begin{npcBox}[title=Chris - a Norm adventure therapist]

    \begin{aspects}
    \item \aspect[High Concept]{Adventure therapist}
    \item \aspect[Trouble]{Adrenaline is the best way to start a therapy session}
    \item \aspect[Relationship]{I want to become your trusted friend}
    \item \aspect[Aspect]{Don't tell anyone: not being connected to the Hive doesn't bother me that much.}
    \item \aspect[Aspect]{Adrenaline junkie looking for a new thrill}
    \end{aspects}

    \begin{skills}
        \item \nskill{Academics}{1}
        \item \nskill{Athletics}{4}
        \item \nskill{Burglary}{0}
        \item \nskill{Contacts}{0}
        \item \nskill{Crafts}{0}
        \item \nskill{Deceive}{1}
        \item \nskill{Drive}{3}
        \item \nskill{Empathy}{3}
        \item \nskill{Fight}{0}
        \item \nskill{Investigate}{0}
        \item \nskill{Lore}{0}
        \item \nskill{Notice}{0}
        \item \nskill{Physique}{2}
        \item \nskill{Provoke}{0}
        \item \nskill{Rapport}{2}
        \item \nskill{Resources}{1}
        \item \nskill{Shoot}{2}
        \item \nskill{Stealth}{0}
        \item \nskill{Will}{1}
        \item \nskill{Hive control}{0}
     \end{skills}

    \begin{stunts}
    \item \stunt{Shock treatment}{When I start to build rapport with an adrenaline-fuelled person, I get +2}
    \item \stunt{First time is best time}{The first time Chris does a unique and extremly risky stunt for the very first time, he gets a +2 to Athletics.}
    \item \stunt{Reckless driver}{Driving a car with at least one frightened and screaming passenger gives Chris +2 to driving.}
    \end{stunts}

    \begin{stressSection}
    \stressLine{\stress{1}\stress{1}\stress{1}\stress{1}}{\stress{1}\stress{1}\stress{1}\stress{1}}
    \end{stressSection}
    \begin{tabularx}{\textwidth}{ XX }
    \end{tabularx}

    \begin{consequences}
    \item \consequence{2}
    \item \consequence{4}
    \item \consequence{6}
    \end{consequences}

    \begin{npcDescription}
    Chris is a Norm adventure therapist. Even in the very supportive and safe Norm society, people sometimes need therapy. And an adrenaline rush is a good place to start.
    So Chris invites people to some kind of (often simulated) adventure and extreme sports. And adds a therapy session after the adrenaline rush.

    Many Norms do not need much to get their adrenaline flowing, as their standard living environment is very controlled and safe.
    \end{npcDescription}

\end{npcBox}

\newpage

\begin{npcBox}[title=Stef - a Norm investigator]

    \begin{aspects}
    \item \aspect[High Concept]{Lover of mysteries}
    \item \aspect[Trouble]{Have to keep my neurons running}
    \item \aspect[Relationship]{I want to know you and your darkest secrets}
    \item \aspect[Aspect]{Oh, data stores !}
    \item \aspect[Aspect]{The Hive community is my home}
    \end{aspects}

    \begin{skills}
        \item \nskill{Academics}{2}
        \item \nskill{Athletics}{0}
        \item \nskill{Burglary}{1}
        \item \nskill{Contacts}{0}
        \item \nskill{Crafts}{0}
        \item \nskill{Deceive}{1}
        \item \nskill{Drive}{0}
        \item \nskill{Empathy}{0}
        \item \nskill{Fight}{0}
        \item \nskill{Investigate}{4}
        \item \nskill{Lore}{0}
        \item \nskill{Notice}{2}
        \item \nskill{Physique}{0}
        \item \nskill{Provoke}{0}
        \item \nskill{Rapport}{3}
        \item \nskill{Resources}{0}
        \item \nskill{Shoot}{1}
        \item \nskill{Stealth}{2}
        \item \nskill{Will}{1}
        \item \nskill{Hive control}{3}
     \end{skills}

    \begin{stunts}
    \item \stunt{Indexing databases}{Because I am an experienced investigator, I can find common entries in completely different databases or data sources and link them together - no matter how absurd the combination.}
    \item \stunt{Friends ?}{Stef gets a +2 to Rapport when he befriends a new person. But Stef will also feel the urge to help that person.}
    \item \stunt{People of the Hive}{Stef is very well connected in the Hive. He gets a +2 when he asks the Hive for a favour - as long as he has received at least 10 Likes in the last 2 hours.}
    \end{stunts}

    \begin{stressSection}
    \stressLine{\stress{1}\stress{1}\stress{1}}{\stress{1}\stress{1}\stress{1}\stress{1}}
    \end{stressSection}
    \begin{tabularx}{\textwidth}{ XX }
    \end{tabularx}

    \begin{consequences}
    \item \consequence{2}
    \item \consequence{4}
    \item \consequence{6}
    \end{consequences}

    \begin{npcDescription}
    Stef is a Norm investigator. Most of the time Stef sits in the park drinking coffee and poring over the Hive's data. Databases, logs, camera footage. Most cases can be solved this way. But sometimes drones or even a personal visit are needed. This is where lockpicking skills come into play.
    Stef is addicted to the Hive and loves to share his adventures with others.
    \end{npcDescription}

\end{npcBox}

\newpage

\begin{npcBox}[title=Gutenberg - a Lost trapper and cook]

    \begin{aspects}
    \item \aspect[High Concept]{Just eat what is troubling you}
    \item \aspect[Trouble]{Class instead of mass}
    \item \aspect[Relationship]{Spreading comfort}
    \item \aspect[Aspect]{This is a trap !}
    \item \aspect[Aspect]{Nature will feed you}
    \end{aspects}

    \begin{skills}
        \item \nskill{Academics}{0}
        \item \nskill{Athletics}{2}
        \item \nskill{Burglary}{0}
        \item \nskill{Contacts}{2}
        \item \nskill{Crafts}{4}
        \item \nskill{Deceive}{0}
        \item \nskill{Drive}{0}
        \item \nskill{Empathy}{0}
        \item \nskill{Fight}{1}
        \item \nskill{Investigate}{0}
        \item \nskill{Lore}{0}
        \item \nskill{Notice}{3}
        \item \nskill{Physique}{0}
        \item \nskill{Provoke}{0}
        \item \nskill{Rapport}{0}
        \item \nskill{Resources}{1}
        \item \nskill{Shoot}{2}
        \item \nskill{Stealth}{1}
        \item \nskill{Will}{1}
        \item \nskill{Bushcraft}{3}
     \end{skills}

    \begin{stunts}
    \item \stunt{Feeding dozens}{With a little help, I can turn whatever we find in the wilderness or in the ruins into a delicious meal for a dozen people.}
    \item \stunt{Wisdom of rats}{Because my two tame rats can sniff poison, I get a +2 to detect dangerous substances by watching my rats sniff them.}
    \item \stunt{This is a trap}{Get a +2 to Stealth when hiding a homemade trap.}
    \end{stunts}

    \begin{stressSection}
    \stressLine{\stress{1}\stress{1}\stress{1}}{\stress{1}\stress{1}\stress{1}\stress{1}}
    \end{stressSection}
    \begin{tabularx}{\textwidth}{ XX }
    \end{tabularx}

    \begin{consequences}
    \item \consequence{2}
    \item \consequence{4}
    \item \consequence{6}
    \end{consequences}

    \begin{npcDescription}
    Skilled at trapping and hunting animals, finding cans in ruins, plants and herbs in lost gardens and woods. Specialises in hunting invasive species. Can cook delicious meals from almost anything.
    Has two tame rats constantly climbing around his head and shoulders.
    \end{npcDescription}

\end{npcBox}


\newpage
\begin{npcBox}[title=Indiana - a Lost looter]

    \begin{aspects}
    \item \aspect[High Concept]{Ruins are my home}
    \item \aspect[Trouble]{Always more than I can carry}
    \item \aspect[Relationship]{Sometimes I need time for myself}
    \item \aspect[Aspect]{I prefer the parkour way}
    \item \aspect[Aspect]{The ancients (lemmings) are fascinating}
    \end{aspects}

    \begin{skills}
        \item \nskill{Academics}{3}
        \item \nskill{Athletics}{3}
        \item \nskill{Burglary}{2}
        \item \nskill{Contacts}{0}
        \item \nskill{Crafts}{0}
        \item \nskill{Deceive}{1}
        \item \nskill{Drive}{0}
        \item \nskill{Empathy}{0}
        \item \nskill{Fight}{2}
        \item \nskill{Investigate}{0}
        \item \nskill{Lore}{2}
        \item \nskill{Notice}{4}
        \item \nskill{Physique}{1}
        \item \nskill{Provoke}{0}
        \item \nskill{Rapport}{0}
        \item \nskill{Resources}{0}
        \item \nskill{Shoot}{0}
        \item \nskill{Stealth}{0}
        \item \nskill{Will}{1}
        \item \nskill{Bushcraft}{1}
     \end{skills}

    \begin{stunts}
    \item \stunt{Sneak n Loot}{If I see a valuable item that I want to loot, I can use Notice to sneak up and grab it.}
    \item \stunt{St. Bernards drafting dog}{I have a sure-footed drafting dog to carry the loot I find. Including a small cart and some bags for his back.}
    %% https://en.wikipedia.org/wiki/Drafting_dog
    \item \stunt{Knowing lemmings}{Knows pre-2020 Lemming technology and gets a +2 to Academics when identifying or using it.}
    \end{stunts}

    \begin{stressSection}
    \stressLine{\stress{1}\stress{1}\stress{1}\stress{1}}{\stress{1}\stress{1}\stress{1}\stress{1}}
    \end{stressSection}
    \begin{tabularx}{\textwidth}{ XX }
    \end{tabularx}

    \begin{consequences}
    \item \consequence{2}
    \item \consequence{4}
    \item \consequence{6}
    \end{consequences}

    \begin{npcDescription}
    Indiana loves the mysteries of the past. What did people eat in 2020? How did they live? Did they really have a 20-kilometre traffic jam?
    The ruins of the past are a wonderful and exciting adventure that lead to answers. And Indiana has the skills to survive them and find new insights in the loot and treasures.
    \end{npcDescription}

\end{npcBox}

\newpage
\begin{npcBox}[title=Static - a Pioneer Bionics architect]

    \begin{aspects}
    \item \aspect[High Concept]{I copy nature}
    \item \aspect[Trouble]{Technology must be art and look natural}
    \item \aspect[Relationship]{Look what it did ! Isn't it cute ?}
    \item \aspect[Aspect]{We should do that together}
    \item \aspect[Aspect]{I love to study nature}
    \end{aspects}

    \begin{skills}
        \item \nskill{Academics}{1}
        \item \nskill{Athletics}{2}
        \item \nskill{Burglary}{0}
        \item \nskill{Contacts}{0}
        \item \nskill{Crafts}{3}
        \item \nskill{Deceive}{0}
        \item \nskill{Drive}{1}
        \item \nskill{Empathy}{0}
        \item \nskill{Fight}{0}
        \item \nskill{Investigate}{0}
        \item \nskill{Lore}{2}
        \item \nskill{Notice}{3}
        \item \nskill{Physique}{0}
        \item \nskill{Provoke}{1}
        \item \nskill{Rapport}{0}
        \item \nskill{Resources}{2}
        \item \nskill{Shoot}{0}
        \item \nskill{Stealth}{1}
        \item \nskill{Will}{0}
        \item \nskill{Prototyping}{4}
     \end{skills}

    \begin{stunts}
    \item \stunt{Defies Gravity}{When building absurd constructions based on bionics, you get a +2 to Crafting. But the design must be physically borderline insane. Normal people will shake their heads and wonder how the magic trick works.}
    \item \stunt{Interesting....}{Gets a +2 to Notice when patiently observing nature for features to copy with technology. This could be animals, geology, behaviour, botany....}
    \item \stunt{Harvest}{Can gather raw materials for crafting from nature. Gets a +2 to Resources when doing so.}
    \end{stunts}

    \begin{stressSection}
    \stressLine{\stress{1}\stress{1}\stress{1}}{\stress{1}\stress{1}\stress{1}}
    \end{stressSection}
    \begin{tabularx}{\textwidth}{ XX }
    \end{tabularx}

    \begin{consequences}
    \item \consequence{2}
    \item \consequence{4}
    \item \consequence{6}
    \end{consequences}

    \begin{npcDescription}
    Nature is the best architect. Learn from it by copying its structures and designs.
    This leads Static to build great buildings that look like they grew where they are.
    The algorithms for this are home-grown. To help with the construction itself, Static uses a bot, which is a 3D concrete printer on 6 legs.
    \end{npcDescription}

    \begin{equipment}
        \item Light source (OLED film: battery operation, can be cut to size and glued on. Colour controllable)
        \item Mobile computers, headphones, communication via radio (mesh network)
        \item A building construction bot (6 legs, insect style, painted by kids, size of a table)
    \end{equipment}
\end{npcBox}

\newpage
\begin{npcBox}[title=Scriptit - a Pioneer Automationationeer]

    \begin{aspects}
    \item \aspect[High Concept]{Let my machines solve it: Automationeer}
    \item \aspect[Trouble]{This MUST be automated}
    \item \aspect[Relationship]{Hey ! Everyone join. This is more fun !}
    \item \aspect[Aspect]{I got a black belt in 3 martial arts styles !}
    \item \aspect[Aspect]{Maybe I will need this - collecting random things}
    \end{aspects}

    \begin{skills}
        \item \nskill{Academics}{2}
        \item \nskill{Athletics}{3}
        \item \nskill{Burglary}{0}
        \item \nskill{Contacts}{0}
        \item \nskill{Crafts: Automation}{4}
        \item \nskill{Deceive}{0}
        \item \nskill{Drive}{1}
        \item \nskill{Empathy}{0}
        \item \nskill{Fight}{2}
        \item \nskill{Investigate}{0}
        \item \nskill{Lore}{0}
        \item \nskill{Notice}{2}
        \item \nskill{Physique}{0}
        \item \nskill{Provoke}{0}
        \item \nskill{Rapport}{0}
        \item \nskill{Resources}{1}
        \item \nskill{Shoot}{1}
        \item \nskill{Stealth}{0}
        \item \nskill{Will}{1}
        \item \nskill{Prototyping}{3}
     \end{skills}

    \begin{stunts}
    \item \stunt{Scale it up}{Gets a +2 to Crafting when programming or building an automation that can produce incredible amounts of the desired product. It will be impossible to make "just one". After a trial run, there will be at least dozens of the product.}
    \item \stunt{Overconfident fighter}{Gets a +2 in melee in an overconfident mood. But automatically suffers 1 physical stress. Can be used once per fight.}
    \item \stunt{Witness me}{Gets a +2 in Athletics when climbing with at least 1 witness.}
    \end{stunts}

    \begin{stressSection}
    \stressLine{\stress{1}\stress{1}\stress{1}}{\stress{1}\stress{1}\stress{1}\stress{1}}
    \end{stressSection}
    \begin{tabularx}{\textwidth}{ XX }
    \end{tabularx}

    \begin{consequences}
    \item \consequence{2}
    \item \consequence{4}
    \item \consequence{6}
    \end{consequences}

    \begin{npcDescription}
    Scriptit would never do anything directly. If a machine, a computer or a script could do it automatically. Others have to be patient when Scriptit does something for the first time. But quite often it can be scaled up afterwards.

    \end{npcDescription}

    \begin{equipment}
        \item Light source (OLED film: battery operation, can be cut to size and glued on. Colour controllable)
        \item Mobile computers, headphones, communication via radio (mesh network)
        \item A suitcase sized robotics kit
    \end{equipment}
\end{npcBox}

\newpage
\section{GM cheat sheet player characters}

A short version of those characters for the GM.

\begin{npcBox}[title=Chris - a Norm adventure therapist]

    \begin{aspects}
    \item \aspect[High Concept]{Adventure therapist}
    \item \aspect[Trouble]{Adrenaline is the best way to start a therapy session}
    \item \aspect[Relationship]{I want to become your trusted friend}
    \item \aspect[Aspect]{Don't tell anyone: not being connected to the Hive doesn't bother me that much.}
    \item \aspect[Aspect]{Adrenaline junkie looking for a new thrill}
    \end{aspects}

    \begin{stunts}
    \item \stunt{Shock treatment}{When I start to build rapport with an adrenaline-fuelled person, I get +2}
    \item \stunt{First time is best time}{The first time Chris does a unique and extremly risky stunt for the very first time, he gets a +2 to Athletics.}
    \item \stunt{Reckless driver}{Driving a car with at least one frightened and screaming passenger gives Chris +2 to driving.}
    \end{stunts}
\end{npcBox}

\newpage

\begin{npcBox}[title=Stef - a Norm investigator]

    \begin{aspects}
    \item \aspect[High Concept]{Lover of mysteries}
    \item \aspect[Trouble]{Have to keep my neurons running}
    \item \aspect[Relationship]{I want to know you and your darkest secrets}
    \item \aspect[Aspect]{Oh, data stores !}
    \item \aspect[Aspect]{The Hive community is my home}
    \end{aspects}

    \begin{stunts}
    \item \stunt{Indexing databases}{Because I am an experienced investigator, I can find common entries in completely different databases or data sources and link them together - no matter how absurd the combination.}
    \item \stunt{Friends ?}{Stef gets a +2 to Rapport when he befriends a new person. But Stef will also feel the urge to help that person.}
    \item \stunt{People of the Hive}{Stef is very well connected in the Hive. He gets a +2 when he asks the Hive for a favour - as long as he has received at least 10 Likes in the last 2 hours.}
    \end{stunts}
\end{npcBox}

\newpage

\begin{npcBox}[title=Gutenberg - a Lost trapper and cook]

    \begin{aspects}
    \item \aspect[High Concept]{Just eat what is troubling you}
    \item \aspect[Trouble]{Class instead of mass}
    \item \aspect[Relationship]{Spreading comfort}
    \item \aspect[Aspect]{This is a trap !}
    \item \aspect[Aspect]{Nature will feed you}
    \end{aspects}

    \begin{stunts}
    \item \stunt{Feeding dozens}{With a little help, I can turn whatever we find in the wilderness or in the ruins into a delicious meal for a dozen people.}
    \item \stunt{Wisdom of rats}{Because my two tame rats can sniff poison, I get a +2 to detect dangerous substances by watching my rats sniff them.}
    \item \stunt{This is a trap}{Get a +2 to Stealth when hiding a homemade trap.}
    \end{stunts}
\end{npcBox}

\newpage

\begin{npcBox}[title=Indiana - a Lost looter]

    \begin{aspects}
    \item \aspect[High Concept]{Ruins are my home}
    \item \aspect[Trouble]{Always more than I can carry}
    \item \aspect[Relationship]{Sometimes I need time for myself}
    \item \aspect[Aspect]{I prefer the parkour way}
    \item \aspect[Aspect]{The ancients (lemmings) are fascinating}
    \end{aspects}

    \begin{stunts}
    \item \stunt{Sneak n Loot}{If I see a valuable item that I want to loot, I can use Notice to sneak up and grab it.}
    \item \stunt{St. Bernards drafting dog}{I have a sure-footed drafting dog to carry the loot I find. Including a small cart and some bags for his back.}
    %% https://en.wikipedia.org/wiki/Drafting_dog
    \item \stunt{Knowing lemmings}{Knows pre-2020 Lemming technology and gets a +2 to Academics when identifying or using it.}
    \end{stunts}
\end{npcBox}

\newpage

\begin{npcBox}[title=Static - a Pioneer Bionics architect]

    \begin{aspects}
    \item \aspect[High Concept]{I copy nature}
    \item \aspect[Trouble]{Technology must be art and look natural}
    \item \aspect[Relationship]{Look what it did ! Isn't it cute ?}
    \item \aspect[Aspect]{We should do that together}
    \item \aspect[Aspect]{I love to study nature}
    \end{aspects}

    \begin{stunts}
    \item \stunt{Defies Gravity}{When building absurd constructions based on bionics, you get a +2 to Crafting. But the design must be physically borderline insane. Normal people will shake their heads and wonder how the magic trick works.}
    \item \stunt{Interesting....}{Gets a +2 to Notice when patiently observing nature for features to copy with technology. This could be animals, geology, behaviour, botany....}
    \item \stunt{Harvest}{Can gather raw materials for crafting from nature. Gets a +2 to Resources when doing so.}
    \end{stunts}

\end{npcBox}

\newpage

\begin{npcBox}[title=Scriptit - a Pioneer Automationationeer]

    \begin{aspects}
    \item \aspect[High Concept]{Let my machines solve it: Automationeer}
    \item \aspect[Trouble]{This MUST be automated}
    \item \aspect[Relationship]{Hey ! Everyone join. This is more fun !}
    \item \aspect[Aspect]{I got a black belt in 3 martial arts styles !}
    \item \aspect[Aspect]{Maybe I will need this - collecting random things}
    \end{aspects}

    \begin{stunts}
    \item \stunt{Scale it up}{Gets a +2 to Crafting when programming or building an automation that can produce incredible amounts of the desired product. It will be impossible to make "just one". After a trial run, there will be at least dozens of the product.}
    \item \stunt{Overconfident fighter}{Gets a +2 in melee in an overconfident mood. But automatically suffers 1 physical stress. Can be used once per fight.}
    \item \stunt{Witness me}{Gets a +2 in Athletics when climbing with at least 1 witness.}
    \end{stunts}
\end{npcBox}