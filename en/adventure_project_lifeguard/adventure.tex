\chapter{Project lifeguard}
\label{ch:project lifeguard}

\section{Topic}

Goal is to build a new community (or several ones: one for each faction), stop the evil plans of an old company (which is a relic) and give people new hope.

The protagonists can be from any faction. A mixed group is best.

The start of the adventure is well planned to get everything running smoothly. The core of the adventure (Albstadt) is a sandbox with characters, trouble and locations.

Here the players can start helping the people, building a community, fight the relic and find new friends. They cooperate - up until they will have to decide which type of community they are about to build:

\begin{itemize}
    \item Pioneer style
    \item Norm style
    \item Lost style
\end{itemize}

Their behaviour towards the NPCs and the style of their solutions to the problems will topple the NPCs towards one faction's philosophy or the others. The GM should take note and adjust the behaviour of the played NPCs.

This is full sandbox and I hope there will be an interesting community with its own traditions built by the players at the end of the game.

\section{Summary}

There is a town (Albstadt from the Swabian Alps) which got lost during the Dirty Road to Eden. The currenty inhabitants improvise to survive but never called for help. About 10 years ago they got contacted by the "Project Lifeguard" which sends them survival items. Most of that is quite cheap and even breaks their own small projects to become indepentent (like: cheap food kills local farming projects). After 10 years some people vanished. Others assume this could be connected to the Project Lifeguard and finally travel outside of their settlement to a Norm town. Short investigation: This Project is unknown and does not exist.

Truth is: Behind this project is an old company that is using a tunnel system in Albstadt to dump toxic waste. The Project Lifeguard is just a  cover story and a cheap way to buy the locals. All help sent is ment to build dependencies and control the local population.

But so far no one knows. The Protagonists will be hired by the UN to investigate. They have a chance to help the locals, build communities or city hives in their fashion and uncover the evil plans of the relic.

\begin{sidebarBox}[title=Relics]
Relics are objects or organisations from the past. They do not fit into the new world and will be dismantled soon. But as there are any relics there is a long list of todos for dismantling. And some just soldier on until it is their turn. Or they fight back.
\end{sidebarBox}

\section{Getting started}

The UN started a project and funded it. A group of people can earn Resource Points if they help investigate the Project Lifeguard and help the people of Albstadt by building a Lost family, a pioneer community or a Norm city hive (or all three of them) to integrate them back into civilisation.

First meeting is in the closest Norm town, the Sigmaringen Hive. The protagonists are invited. They can also meet the messenger who finally made a trip to civilisation and is a bit lost.

\subsection{Scene one}

The player characters from all factions where called to the Norm town Sigmaringen for a UN meeting. No one knows yet why. But they know: payment for helping the UN is resource points.

They arrive independently and will meet at the location.

Sigmaringen is a nice and relaxed town. Many roads have been replaced by parks after 2020. Tame deer can be fed. Solar panels are attached to every available surface. Public transport is working flawless. There are many shelves where people can get free food and other thing of basic need. This is normal for the Norms and may seem strange for Lost and Pioneers.

Many Norms the visitors meet use the AR interface to origanize projects, chat, meet online....People on the street seem to gesticulate with arcane hand movements interacting with something no Lost or Pioneer can see - they have no Hive controller.

The meeting room in the castle was prepared. Someone got the "Emergency UN meeting props box" from the attic and decorated everything with large and small flags, placed cards on the table with UN logo, installed a UN logo on the big video screen in the room.

After all the characters arrived they are greeted by city officials who also do not know why everyone is here.

Perfect time for the characters to introduce each other before the UN meeting starts.

After a short time a UN official appears on the screen (from Addis Ababa, Ethiopia). Her name is Enku and she is experts for relics.

The characters receive a mission:
The next town "Albstadt" is a no-go area and a relic. It was destroyed when the caves under the area caved in. Thanks to low water levels over several years. A Lost family camped close by (the family of the characters in the meeting). Something strange happened. But the characters can tell the story themselves.

\subsection{Flashback for Lost}

(Here the Lost play a short session. This is what happened. Keep it short so the others do not get bored !)

Takeaways:

\begin{itemize}
    \item A person fled from Albstadt. It is still inhabited !
    \item another person chased this person. Obviously a professional fighter with illegal cyber augments (which means: replacement parts enhanced normal human abilities)
    \item The fleeing person claimed the project "Lifeguard" supports the town. But people vanish without a trace
\end{itemize}

The lost camp close to the Albstadt No-Go area. The camp is mobile and consists of diesel vehicles and tents. No Lost knows what happened in Albstadt 20 years ago. No one is allowed to go there. But they raid the outer ruins for stuff.

At the evening, while the "Risotto Rodenti" (containing rats, squirrels and especially invasive species) for the whole camp cooked they played Shakespear battle. Lost are educated and value history and old books a lot. The battle consists of one person starting to play a scene from an old theater play and handing over to another person who will have to continue without failing.

(The player characters are part of the battle-crowd. GM: Bring a copy of a page of a Shakespeare play so it can be played at the table).

While they played the guards at the perimeter stopped a person fleeing from Albstadt. The person wears rags. A Cyborg hunts this person. The Cyborg is quite high tech - which is illegal.

The guards at the perimeter stop the cyborg and do heavy damage with their guns. As soon as the cyborg loses the battle the cyberware self destructs and kills the person.

(The player characters are not art of the fight. But they can assist later: First aid, investigate the cyborg, ... This is why they know the essential info first hand).

The rescued person (named Kim) tells a strange story:

He was born in Albstadt. Which is almost 100 percent destroyed. The caved in area made the whole town very vertical. Some ruins of buildings attached to rocky spikes. Caves are accessible now where no caves have been before. The people build makeshift bridges and were waiting for help. 10 years ago it arrived. Projekt Lifeguard came. The Cyborg was one of them.

They visit once a month and bring the bare minimum. Food, medicine, water, power. The town is fully dependend on them (which is the plan of project Lifeguard).

People started to disappear. This is why Kim fled.

\subsection{Flashback for Norms}

The Norm people in the room got the UN task to investigate a few days ago. Their investigation happened in AR mode only while the characters were enjoying their free Latte Machiatto with Soy milk and cakes somewhere at a Danube beach in Sigmaringen.

By phoning people, going through online databases, checking web cams and more they can learn:

\begin{itemize}
    \item The containers with the project logo are from Sigmaringen
    \item A company called "Dumpit waste management inc" has the same ware house
    \item The containers are shipped by helicopter drones to Albstadt
    \item The project is not registered anywhere !
    \item Dumpit also ships waste container the same way
\end{itemize}

They feel observed ( a Dumpit investigator gets curious) and flee back to the "UN headquarter" which is a decorated room in the castle. The way back through the park adnd town can be made interesting. The observer will never directly interact. Dumpit just got curious who is investigating.

\subsection{Flashback for Pioneers}

The pioneers upgraded a building bot to a semi-autonomous bot. The bot is table sized, has 6 legs (with gecko grip), a nozzle to distribute a self hardening gooy substance for building and can either run tethered (power and substance comes through a hose) or with battery and goo storage for a short time span. It can be manually controlled or more interesting: run autonomously to print any kind of building, The inventor is in a wheelchair and decided that the autonomous mode only allows accessible structures: wide doors, no stairs.

This bot is called the "Beetle" and was painted by the Pioneer kids. This is why it looks like a multi colored real bug. And has glued on goggly eyes.

This tools is celebreated in the community and people challenge each other to build absurd things. This is happening as part of a building party where other projects are pushed as well.
 
At the end of the party the characters get a call from the UN to Sigmaringen. Something happened. They are neeeded. Bring the bot.

The inventor will not be able to join the adventure itself (wheelchair) but sends the characters.

The Beetle is the reason why this specific group of Pioneers have been summoned. The UN hopes it can be used to make parts of Albstad more accessible.

\section{Areas}

\subsection{Albstadt}

Albstadt was destroyed in a disaster. The karst area caved in. The landscape is more vertical with cliffs than horizontal now. Some buildings have be safed by structural fixes to keep them from totally falling into the ditches. There are improvised bridges everywhere. Caves have opened. Some are filled with water and need diving gear to reach the ground (where maybe some valuables from the earthquake ended up).
The people here did a good job at surviving in their broken town. But they did not dare to reach out for help. They were found by Project Lifeguard but the help sent through that made them even more dependent.

For the cave system: Check out "Blautopf" on wikipedia. 

\subsubsection{General}

Albstadt is a maze of spires, where the tops of the spires are at ground level. The canyon system is filled with water and caves (dry or filled with water) are everywhere.

Old buildings either survived at the top of the spires or are hanging at the side. With heavy damage and secured by ropes.

Between the spires there are improvised suspension bridges and supply lines to the supermarket (power, water). All of that needs repair or replacement.

At the beginning of the adventure the Sigmaringen hive will drop a radio repeater close to Albstadt. The Norms will have a connection to the Sigmaringen hive. No dice penalty but delivery time is +1h.
In addition they will get expenses of a few Resource Points to spend on whatever they need there.

\subsubsection{Forrest}

The forrest is dangerous. Sinkholes and boars make it almost inaccessible. It surrounds Albstadt and is the first reason this is a no-go area.

\textbf{Problem:} Wild boars and sinkholes
\textbf{Solutions:} Boars can be hunted, trapped or kept away. Sinkholes can be fixed by analysing the ground or building bridges and walkways between the trees. 
\textbf{Benefits:} Fertile soil for farming and tasty animals

\subsubsection{Supermarket}

The supermarket is the largest surviving structure. It is controlled by Ash and his goons. The parking space in front of the supermarket is where the containers are dropped by cargo zeppelins. Those are instantly secured by Ash and the goons.

\textbf{Problem:} It is controlled by Ash and his goons. He uses it to control the whole town by artificial scarcity
\textbf{Solutions:} Get rid of Ash, blackmail him, force him anotehr way, ....
\textbf{Benefits:} A building for storage, crafting, town center, ....

\subsubsection{Leo's}

\textbf{Leo} tried to fix his building and spire as good as possible. But was lacking material or technology. In his current state he can not help - but maybe after things are fixed somehow.

\textbf{Problem:} Leo does not have the technology to properly build what he wants to build
\textbf{Solutions:} Scale up, use Pioneer tech, order stuff from Sigmaringen
\textbf{Benefits:} A first hanging garden, proper bridges and supply lines. A showcase

\subsubsection{Water level maze}

This is where \textbf{Lara} spends her time: 200 m below the top of the spires. There are water ways with beaches, rapids, caves and large rocks in the water.

A small boat or canoes will be relevant for traveling.

There are a lot of dangerous caves. Plus one where the toxic waste was dumped. Staying there without protective equipment can be dangerous. Also: Do not eat the crabs there.

\textbf{Problem:} American crabs are eating the local life
\textbf{Solutions:} Build traps, eat the crabs
\textbf{Benefits:} Food

\subsubsection{Sams bunker}

\textbf{Sam} hides in a bunker like control structure. Built to house 20 people it was the home of 3 until recently (the cyborg and Sams collegue got killed).

Sam hides there now.

\textbf{Problem:} Sam is scared and hiding there.
\textbf{Solutions:} Talk, produce some prove the situation calmed down
\textbf{Benefits:} Workshop, drones and information

% What is left ? Church as city assembly. An old super market as source for tools, a museum for local history (really old farm tools),
% People life in shanty town, makeshift tents 
% How did the project replace essential things ? Power => Battery container, Food => Cans, Water => Bottles, Medicine => A doctor visits once a month
% Who is left ? Some people could build a sustainable life. Lifeguard is blocking that.
% How do they interact ? Some people love the dependency on PLG
% What can they find out ? Something is going on in the tunnels. The help is keeping people down. 
% How can they rebuild ? Start gardening, build water cisterns, 
% Who can be recruited & empowered ?
% How will PLG respond ? Sabotage and violence ? What before that ? Threats ?
% Supermarket owner gets gund by PLG to defend the last resources
% Person who wants to do medic only gets basic material and everyone has to wait for the monthly visit of PLG for real help
% There is a spy for PLG, but this spy can be turned and used against PLG

\subsection{People at Albstadt}

In Albstadt there are about 200-300 people. With some very active ones struggling, to keep everyone alive. But Project Life~-guard tricked them into non-sustainable ways to struggle and building more and more dependencies to Project Life~-guard. Those people could - using their skills smarter and more coordinated - build an in~-dependent Pioneer Community/~-Lost Family or Norm Hive. Until then they are a problem because they maintain the stupid structure tactically implemented by Project Lifeguard.

Those are the NPCs to interact with.

\section{NPCs}


\begin{npcBox}[title=Leo]

    \begin{aspects}
    \item \aspect[High Concept]{I create buildings}
    \item \aspect[Trouble]{I will hulk out if you make me angry}
    \item \aspect[Aspect]{I dream to build big but the situation keeps me small}    
    \end{aspects}
    
    \begin{skills}
        \item \skill{3} Academics
        \item \skill{1} Athletics
        \item \skill{0} Burglary
        \item \skill{1} Contacts
        \item \skill{4} Crafts
        \item \skill{0} Deceive
        \item \skill{0} Drive
        \item \skill{1} Empathy
        \item \skill{2} Fight
        \item \skill{0} Investigate
        \item \skill{1} Lore
        \item \skill{0} Notice
        \item \skill{3} Physique
        \item \skill{0} Provoke
        \item \skill{0} Rapport
        \item \skill{2} Resources
        \item \skill{0} Shoot
        \item \skill{0} Stealth
        \item \skill{2} Will
        % Lost: \item \skill{} Bushcraft
        % Norms: \item \skill{} Hive control
        % Pioneers \item \skill{} Prototyping
     \end{skills}
    
    \begin{stunts}
    \item \stunt{Name}{Description}
    \end{stunts}

\end{npcBox}
\newpage
\begin{npcBox}[title=Leo continued]
    
    \begin{stressSection}
    \stressLine{\stress{1}\stress{1}\stress{1}}{\stress{1}\stress{1}\stress{1}}
    \end{stressSection}
    \begin{tabularx}{\textwidth}{ XX }
    \end{tabularx}
    
    \begin{consequences}
    \item \consequence{2}
    \item \consequence{4}
    \item \consequence{6}
    \end{consequences}
    
    \begin{npcDescription}
    An impulsive architect, has a spire with a house on it. He tries to farm the side of the rock by attaching small boxes for plants.
    He was a part of the 5 minute rebellion where Kim fled. Being impulsive he threw rocks at the three people from project lifeguard. The Cyborg followed Kim. A rock thrown by Leo hit an engineer, throwing him of a bridge and killing him. The third engineer did hide in their container. Some antenna got destroyed by thrown rocks.
    After the rebellion turned bloody Leo withdrew. Shocked by the own action.
    If this gets fixed they can gain support by a skilled architect who just lacks the tools to build awesome vertical gardens and bridges.

    \textbf{Location:} His own spire. With a small house (repaired). And a small hanging garden at the side of the spire (just some flower pots, he did not have the right tools for proper hanging gardens)
    \textbf{Problem:} Has to deal with having killed someone. Maybe talking to Sam could help
    \textbf{When convinced:} Can help building stuff ( a real hanging garden, proper suspension bridges, traps for crabs)
    \end{npcDescription}
    
\end{npcBox}


\begin{npcBox}[title=Lara]

    \begin{aspects}
    \item \aspect[High Concept]{}
    \item \aspect[Trouble]{}    
    \item \aspect[Aspect]{}    
    \end{aspects}
    
    \begin{skills}
        \item \skill{0} Academics
        \item \skill{2} Athletics
        \item \skill{1} Burglary
        \item \skill{0} Contacts
        \item \skill{0} Crafts
        \item \skill{0} Deceive
        \item \skill{3} Drive
        \item \skill{0} Empathy
        \item \skill{2} Fight
        \item \skill{2} Investigate
        \item \skill{0} Lore
        \item \skill{4} Notice
        \item \skill{3} Physique
        \item \skill{0} Provoke
        \item \skill{0} Rapport
        \item \skill{0} Resources
        \item \skill{1} Shoot
        \item \skill{1} Stealth
        \item \skill{1} Will
        % Lost: \item \skill{} Bushcraft
        % Norms: \item \skill{} Hive control
        % Pioneers \item \skill{} Prototyping
     \end{skills}
    
    \begin{stunts}
    \item \stunt{Name}{Description}
    \end{stunts}
\end{npcBox}
\newpage
\begin{npcBox}[title=Lara continued]
    
    \begin{stressSection}
    \stressLine{\stress{1}\stress{1}\stress{1}}{\stress{1}\stress{1}\stress{1}}
    \end{stressSection}
    \begin{tabularx}{\textwidth}{ XX }
    \end{tabularx}
    
    \begin{consequences}
    \item \consequence{2}
    \item \consequence{4}
    \item \consequence{6}
    \end{consequences}
    
    \begin{npcDescription}
    Lara is a kind guide, knows all of Albstadt. She wants to uncover the truth about the cursed and haunted cave, the so called secret site, where the Project drops the poisonous garbage. It is not really haunted. But people going there died from illnesse soon after ...
    To investigate diving gear would be needed. Which is not available in Albstadt.
    When doing this investigation the guide would have to abandon an important duty: care for people she rescued after accidents. They need medical care and leaving for 1-2 days is not an option.

    \textbf{Location:} At home (needs repair) caring for the injured people. Some travelling.
    \textbf{Problem:} Needs someone to care for the injured people. 
    \textbf{When convinced:} Can help exploring Albstadt and the haunted cave.
    \end{npcDescription}
    
\end{npcBox}

\begin{npcBox}[title=Sam]

    \begin{aspects}
    \item \aspect[High Concept]{Scared good samaritan}
    \item \aspect[Trouble]{Are we the baddies ?}
    \item \aspect[Aspect]{Helper with evil masters}    
    \end{aspects}
    
    \begin{skills}
        \item \skill{4} Academics
        \item \skill{1} Athletics
        \item \skill{0} Burglary
        \item \skill{0} Contacts
        \item \skill{3} Crafts
        \item \skill{0} Deceive
        \item \skill{1} Drive
        \item \skill{1} Empathy
        \item \skill{0} Fight
        \item \skill{0} Investigate
        \item \skill{1} Lore
        \item \skill{2} Notice
        \item \skill{0} Physique
        \item \skill{0} Provoke
        \item \skill{3} Rapport
        \item \skill{2} Resources
        \item \skill{0} Shoot
        \item \skill{0} Stealth
        \item \skill{2} Will
        % Lost: \item \skill{} Bushcraft
        % Norms: \item \skill{} Hive control
        % Pioneers \item \skill{} Prototyping
     \end{skills}
    
    \begin{stunts}
    \item \stunt{Name}{Description}
    \end{stunts}

\end{npcBox}
\newpage
\begin{npcBox}[title=Sam continued]
    
    \begin{stressSection}
    \stressLine{\stress{1}\stress{1}\stress{1}}{\stress{1}\stress{1}\stress{1}}
    \end{stressSection}
    \begin{tabularx}{\textwidth}{ XX }
    \end{tabularx}
    
    \begin{consequences}
    \item \consequence{2}
    \item \consequence{4}
    \item \consequence{6}
    \end{consequences}
    
    \begin{npcDescription}
    Sam is the last of the three Project members. She wants to help people but is restricted by the company guidelines. Sam does not know anything about the poison dump. At the beginning she was helpful but is now scared after the short escalation, the "revolution". As engineer and medic she could care for the injured people. If she would not be scared.

    Also: Antenna to the HQ is damaged. Can not contact and ask for help.
    
    Are we the baddies ?
    
    Keep in mind: There is some good in this NPC. But she works for the baddies.

    With some convincing she can become a powerul asset and friend. Helping with medic and engineering. And also knowing (or guessing) some of the internals of Dumpit.

    \textbf{Location:} The company bunker. Or exploring the town via drones and getting into contact (the drones have speakers)
    \textbf{Problem:} Scared by the attack. Also doubts that the company is helping
    \textbf{When convinced:} Can offer medical help. Some engineering and insights into Project Lifeguard and the company "Dumpit"
    \end{npcDescription}
    
\end{npcBox}


\begin{npcBox}[title=Ash]

    \begin{aspects}
    \item \aspect[High Concept]{Brute with power}
    \item \aspect[Trouble]{I am totally dependend on project Lifeguard}
    \item \aspect[Aspect]{Obey I have the food, water and power}    
    \end{aspects}
    
    \begin{skills}
        \item \skill{0} Academics
        \item \skill{1} Athletics
        \item \skill{1} Burglary
        \item \skill{2} Contacts
        \item \skill{0} Crafts
        \item \skill{2} Deceive
        \item \skill{0} Drive
        \item \skill{0} Empathy
        \item \skill{3} Fight
        \item \skill{0} Investigate
        \item \skill{0} Lore
        \item \skill{1} Notice
        \item \skill{2} Physique
        \item \skill{3} Provoke
        \item \skill{0} Rapport
        \item \skill{4} Resources
        \item \skill{2} Shoot
        \item \skill{0} Stealth
        \item \skill{1} Will
        % Lost: \item \skill{} Bushcraft
        % Norms: \item \skill{} Hive control
        % Pioneers \item \skill{} Prototyping
     \end{skills}
    
    \begin{stunts}
    \item \stunt{Name}{Description}
    \end{stunts}

\end{npcBox}
\newpage
\begin{npcBox}[title=Ash continued]
    
    \begin{stressSection}
    \stressLine{\stress{1}\stress{1}\stress{1}}{\stress{1}\stress{1}\stress{1}}
    \end{stressSection}
    \begin{tabularx}{\textwidth}{ XX }
    \end{tabularx}
    
    \begin{consequences}
    \item \consequence{2}
    \item \consequence{4}
    \item \consequence{6}
    \end{consequences}
    
    \begin{npcDescription}
    The old supermarket is the central hub for food, resources and power. Everything is delivered there. The owner of the supermarket got recruited to distribute it. Power corrupted him. He controls the town.

    To get power over his masters (the Project) he wants to investigate whats in the secret site.
    
    This is the real baddie

    \textbf{Location:} Supermarket
    \textbf{Problem:} Dependent on the Project Lifeguard. Wants to blackmail them. Just in case.
    \textbf{When convinced:} Can not be convinced. Just blackmailed, forced, .... will then offer control over the whole settlement
    \end{npcDescription}
    
\end{npcBox}




\subsection{Project Lifeguard by Dumpit waste management inc.}

Project Lifeguars is a project of a relic company - and old company that will be dismantled as soon as the UN has some time. As it is small and seems non-relevant it is way down at the bottom of the backlog.
Dumit waste management Inc is officially there to recycle waste. Their dirty secret: Toxic waste is dumped into a natural chimney in the Albstadt territory. Where it falls down some hundred meters and ends up in a natural cave filled with water. This cave can be accessed from the side by skilled divers. And some people noticed the area is haunted. People spending some time there die. But no one in Albstadt connected the dots.

The reason is "Project Lifeguard". This is the cover story told to the people in Albstadt. A humanitarian aid sent there. But it is designed in a way to make the people dependent. At the same time they send a container with toxic waste by helicopter, three containers are dropped at the supermarket: One with food and water, one with power (hydrogen generators) and one with clothes and tools.
Those are barely aneough for the people there to survive. But it makes them dependent and breaks all endeavours to grow their own food, clean the water or install power generators.

A small outpost (manned with three people) is there to monitor everything. But the leave the people alone most of the time and use drones and cameras.

Before the adventure starts, one engineer is killed by Leo who was throwing a stone. The security cyborg followed the fleeing Kim and was killed by the Lost.
The last engineer and medic Sam is hiding in the bunker during the adventure.

One secret which is hard to spot are the two sattelite dishes for communication.

The obvious one is directed towards a stratosphere relay zeppelin to communicate with the company.

The more obscure dish is always directed towards the moon.

\begin{sidebarBox}[title=Kessler Syndrome]

    A mass crash of sattelites made the orbit around earth inacessible. There is are no weather satellites, GPS or map satellites left. No communication.
    This is called "Kessler Syndrome". Leaving earth is high risk and no one tries that anymore.

    As replacement people are using high altitude planes, baloons and zeppelins. They only cover a small region and must be started intentionally. But this is better than nothing.
\end{sidebarBox}


\begin{sidebarBox}[title=Moon base]

    There is no moon base. None that people would be aware of. Early in the Dirty Road to Eden phase the billionairs left earth. For the moon, Mars and some even digging tunnels below earth. No one is aware of that. Thanks to the Kessler Syndrome (which was caused by the last one leaving earth) they could not be followed even if people would realize what happened.
    In their secret bases they continue dreaming their capitalistic dream and influencing earth. Fighting to keep relic companies alive.

    But this will be handled in more details in another adventure. That will also cover the so called "Eat the rich festivals". The only hint in this adventure that there could be something big going on is the antenna pointed at the moon.
\end{sidebarBox}

Endgame: If enough of their evil plans are uncovered the UN can put this relic right at the top of the dismantle-todo-list. Lawyers and security forces will be raiding the HQ. The company assets will be seized.

\subsection{Sigmaringen Hive}

Sigmaringen survived the earthquake that killed Albstadt. It became a Norm hive. In the center there is a castle on a hill. The Danube in the town is used by solar powered ambibious busses to travel around. They installed solar power wherever possible to power the city indepenently. And they allowed the surrounding forrest to enter the city in a controlled way. They especially allowed the tame deer in. Those are curious, dumb, always hungry, cute and can knock people over witha  playful headbut. But the tourists love them.
The town is reachable by train.

% Investigation: Who is bsehind the Project Lifeguard ?
% How can it be stopped ?


\section{Player characters}

\begin{npcBox}[title=Chris - a Norm adventure therapist]

    \begin{aspects}
    \item \aspect[High Concept]{Adventure therapist}
    \item \aspect[Trouble]{Adrenaline is the best way to start a session}
    \item \aspect[Aspect]{Adrenaline junkie}
    \end{aspects}
    
    \begin{skills}
        \item \skill{1} Academics
        \item \skill{4} Athletics
        \item \skill{0} Burglary
        \item \skill{0} Contacts
        \item \skill{0} Crafts
        \item \skill{1} Deceive
        \item \skill{3} Drive
        \item \skill{3} Empathy
        \item \skill{0} Fight
        \item \skill{0} Investigate
        \item \skill{0} Lore
        \item \skill{0} Notice
        \item \skill{2} Physique
        \item \skill{0} Provoke
        \item \skill{2} Rapport
        \item \skill{1} Resources
        \item \skill{2} Shoot
        \item \skill{0} Stealth
        \item \skill{1} Will
        % Lost: \item \skill{} Bushcraft
        \item \skill{0} Hive control
        % Pioneers \item \skill{} Prototyping
     \end{skills}
    
    \begin{stunts}
    \item \stunt{Shock treatment}{When I start building rapport with an adrenaline filled person I will get +2}
    \end{stunts}
\end{npcBox}
\newpage
\begin{npcBox}[title=Chris continued]
    
    \begin{stressSection}
    \stressLine{\stress{1}\stress{1}\stress{1}\stress{1}}{\stress{1}\stress{1}\stress{1}\stress{1}}
    \end{stressSection}
    \begin{tabularx}{\textwidth}{ XX }
    \end{tabularx}
    
    \begin{consequences}
    \item \consequence{2}
    \item \consequence{4}
    \item \consequence{6}
    \end{consequences}
    
    \begin{npcDescription}
    Chris is a Norm adventure therapist. Even in the highly supportive and safe Norm society people sometimes need therapy. And a rush of adrenaline is a good start to reset the mind.
    For that Chris invites the people to some kind of (very often simulated) adventure and extreme sports. And adds a therapy session to the phase after the adrenaline rush.

    Many Norms do not need much to get their Adrenaline flow started as their standard living environment is very good ontrolled and safe.
    \end{npcDescription}
    
\end{npcBox}


\begin{npcBox}[title=Stef - a Norm investigator]

    \begin{aspects}
    \item \aspect[High Concept]{Lover of mysteries}
    \item \aspect[Trouble]{Have to keep my neurons running}
    \item \aspect[Aspect]{Everyone has his secrets - and I will find them}    
    \end{aspects}
    
    \begin{skills}
        \item \skill{2} Academics
        \item \skill{0} Athletics
        \item \skill{1} Burglary
        \item \skill{0} Contacts
        \item \skill{0} Crafts
        \item \skill{1} Deceive
        \item \skill{0} Drive
        \item \skill{0} Empathy
        \item \skill{0} Fight
        \item \skill{4} Investigate
        \item \skill{0} Lore
        \item \skill{2} Notice
        \item \skill{0} Physique
        \item \skill{0} Provoke
        \item \skill{3} Rapport
        \item \skill{0} Resources
        \item \skill{1} Shoot
        \item \skill{2} Stealth
        \item \skill{1} Will
        % Lost: \item \skill{} Bushcraft
        \item \skill{3} Hive control
        % Pioneers \item \skill{} Prototyping
     \end{skills}
    
    \begin{stunts}
    \item \stunt{Indexing databases}{Because I am an experienced investigator I can find common entries in totally different databases or data sources and can join them together - does not matter how absurd the combination is.}
    \end{stunts}

\end{npcBox}
\newpage
\begin{npcBox}[title=Stef continued]
    
    \begin{stressSection}
    \stressLine{\stress{1}\stress{1}\stress{1}}{\stress{1}\stress{1}\stress{1}\stress{1}}
    \end{stressSection}
    \begin{tabularx}{\textwidth}{ XX }
    \end{tabularx}
    
    \begin{consequences}
    \item \consequence{2}
    \item \consequence{4}
    \item \consequence{6}
    \end{consequences}
    
    \begin{npcDescription}
    Stef is a Norm investigator. Most of the time Stef sits in the park drinking coffe and digging through the data of the hive. Databases, logs, camera recordings. Most cases can be solved that way. But sometimes drones or even a personal visit are required. This is where lock picking skills start to become relevant.
    \end{npcDescription}
    
\end{npcBox}


\begin{npcBox}[title=Gutenberg - a Lost trapper and cook]

    \begin{aspects}
    \item \aspect[High Concept]{Just eat whats troubling you}
    \item \aspect[Trouble]{Gourmet cook for the barbarians}
    \item \aspect[Aspect]{Nature will feed you}
    \end{aspects}
    
    \begin{skills}
        \item \skill{0} Academics
        \item \skill{2} Athletics
        \item \skill{0} Burglary
        \item \skill{2} Contacts
        \item \skill{4} Crafts Cooking
        \item \skill{0} Deceive
        \item \skill{0} Drive
        \item \skill{0} Empathy
        \item \skill{1} Fight
        \item \skill{0} Investigate
        \item \skill{0} Lore
        \item \skill{3} Notice
        \item \skill{0} Physique
        \item \skill{0} Provoke
        \item \skill{0} Rapport
        \item \skill{1} Resources
        \item \skill{2} Shoot
        \item \skill{1} Stealth
        \item \skill{1} Will
        \item \skill{3} Bushcraft
        % Norms: \item \skill{} Hive control
        % Pioneers \item \skill{} Prototyping
     \end{skills}
    
    \begin{stunts}
    \item \stunt{Feeding dozens}{With some help I can instantly create a wonderful meal for a dozen people out of whatever we find in the wilderness or in the ruins.}
    \end{stunts}

\end{npcBox}
\newpage
\begin{npcBox}[title=Gutenberg continued]
    
    \begin{stressSection}
    \stressLine{\stress{1}\stress{1}\stress{1}}{\stress{1}\stress{1}\stress{1}\stress{1}}
    \end{stressSection}
    \begin{tabularx}{\textwidth}{ XX }
    \end{tabularx}
    
    \begin{consequences}
    \item \consequence{2}
    \item \consequence{4}
    \item \consequence{6}
    \end{consequences}
    
    \begin{npcDescription}
    Skilled at trapping and hunting animals, finding cans in ruins, plants and herbs in lost gardens and the woods. Specialised in hunting invasive species. Can cook delicious meals out of almost anything.
    \end{npcDescription}
    
\end{npcBox}



\begin{npcBox}[title=Indiana - a Lost looter]

    \begin{aspects}
    \item \aspect[High Concept]{Ruins are my home}
    \item \aspect[Trouble]{Always more than I can carry}
    \item \aspect[Aspect]{The ancients (lemmings) are fascinating}
    \end{aspects}
    
    \begin{skills}
        \item \skill{3} Academics
        \item \skill{3} Athletics
        \item \skill{2} Burglary
        \item \skill{0} Contacts
        \item \skill{0} Crafts
        \item \skill{1} Deceive
        \item \skill{0} Drive
        \item \skill{0} Empathy
        \item \skill{2} Fight
        \item \skill{0} Investigate
        \item \skill{2} Lore
        \item \skill{4} Notice
        \item \skill{1} Physique
        \item \skill{0} Provoke
        \item \skill{0} Rapport
        \item \skill{0} Resources
        \item \skill{0} Shoot
        \item \skill{0} Stealth
        \item \skill{1} Will
        \item \skill{1} Bushcraft
        % Norms: \item \skill{} Hive control
        % Pioneers \item \skill{} Prototyping
     \end{skills}
    
    \begin{stunts}
    \item \stunt{Sneak n Loot}{When I see a valuable item I want to loot I can use notice to sneak there and grab it.}
    \end{stunts}

\end{npcBox}
\newpage
\begin{npcBox}[title=Indiana continued]
    
    \begin{stressSection}
    \stressLine{\stress{1}\stress{1}\stress{1}\stress{1}}{\stress{1}\stress{1}\stress{1}\stress{1}}
    \end{stressSection}
    \begin{tabularx}{\textwidth}{ XX }
    \end{tabularx}
    
    \begin{consequences}
    \item \consequence{2}
    \item \consequence{4}
    \item \consequence{6}
    \end{consequences}
    
    \begin{npcDescription}
    Indiana loves the mysteries of the past. What did they eat in 2020 ? How did they live ? Did they really have 20 kilometers of traffic jam ?
    Ruins of the past are a wonderful and thrilling adventure leading to answers. And Indiana has the skills to survive them and find new insights in loot and treasures.
    \end{npcDescription}
    
\end{npcBox}



\begin{npcBox}[title=Static - a Pioneer Bionics architect]

    \begin{aspects}
    \item \aspect[High Concept]{I copy nature}
    \item \aspect[Trouble]{Technology must be art, must look natural}
    \item \aspect[Aspect]{I love to study nature}
    \end{aspects}
    
    \begin{skills}
        \item \skill{1} Academics
        \item \skill{2} Athletics
        \item \skill{0} Burglary
        \item \skill{0} Contacts
        \item \skill{3} Crafts
        \item \skill{0} Deceive
        \item \skill{1} Drive
        \item \skill{0} Empathy
        \item \skill{0} Fight
        \item \skill{0} Investigate
        \item \skill{1} Lore
        \item \skill{3} Notice
        \item \skill{0} Physique
        \item \skill{1} Provoke
        \item \skill{0} Rapport
        \item \skill{2} Resources
        \item \skill{0} Shoot
        \item \skill{2} Stealth
        \item \skill{0} Will
        % Lost: \item \skill{} Bushcraft
        % Norms: \item \skill{} Hive control
        \item \skill{4} Prototyping
     \end{skills}
    
    \begin{stunts}
    \item \stunt{Defies Gravity}{When building absurd constructions based on bionics gets a +2 on craft skill. But the design must be borderline insane physics wise. Normal people will shake their head and wonder how the magic trick works.}
    \end{stunts}

\end{npcBox}
\newpage
\begin{npcBox}[title=Static continued]
    
    \begin{stressSection}
    \stressLine{\stress{1}\stress{1}\stress{1}}{\stress{1}\stress{1}\stress{1}}
    \end{stressSection}
    \begin{tabularx}{\textwidth}{ XX }
    \end{tabularx}
    
    \begin{consequences}
    \item \consequence{2}
    \item \consequence{4}
    \item \consequence{6}
    \end{consequences}
    
    \begin{npcDescription}
    The best architect is nature. Learn from it by copying its structures and design.
    This leads Static to build awesome buildings looking like they grew where they are.
    The algorithms for that are homebrew. To help with the construction itself Static uses a bot that is a 3D printer for concrete on 6 legs.
    \end{npcDescription}
    
    \begin{equipment}
        \item Light source (OLED film: battery operation, can be cut to size and glued on. Colour controllable)
        \item Mobile computers, headphones, communication via radio (mesh network)
        \item A building construction bot (6 legs, insect style, painted by kids, size of a table)
    \end{equipment}
\end{npcBox}

\begin{npcBox}[title=Scriptit - a Pioneer Automationationeer]

    \begin{aspects}
    \item \aspect[High Concept]{Let my machines solve it: Automationeer}
    \item \aspect[Trouble]{This MUST be automated}
    \item \aspect[Aspect]{}
    \end{aspects}
    
    \begin{skills}
        \item \skill{2} Academics
        \item \skill{1} Athletics
        \item \skill{0} Burglary
        \item \skill{0} Contacts
        \item \skill{4} Crafts: Automation
        \item \skill{0} Deceive
        \item \skill{1} Drive
        \item \skill{0} Empathy
        \item \skill{2} Fight
        \item \skill{0} Investigate
        \item \skill{0} Lore
        \item \skill{2} Notice
        \item \skill{0} Physique
        \item \skill{0} Provoke
        \item \skill{0} Rapport
        \item \skill{3} Resources
        \item \skill{1} Shoot
        \item \skill{0} Stealth
        \item \skill{1} Will
        % Lost: \item \skill{} Bushcraft
        % Norms: \item \skill{} Hive control
        \item \skill{3} Prototyping
     \end{skills}
    
    \begin{stunts}
    \item \stunt{Scale it up}{Gets a +2 on crafting if an automation is programmed or built that can produce incedible amounts of the wanted product. It will be impossible to produce "just one". After a test run there will be at least dozens of this product.}
    \end{stunts}

\end{npcBox}
\newpage
\begin{npcBox}[title=Scriptit continued]
    
    \begin{stressSection}
    \stressLine{\stress{1}\stress{1}\stress{1}}{\stress{1}\stress{1}\stress{1}\stress{1}}
    \end{stressSection}
    \begin{tabularx}{\textwidth}{ XX }
    \end{tabularx}
    
    \begin{consequences}
    \item \consequence{2}
    \item \consequence{4}
    \item \consequence{6}
    \end{consequences}
    
    \begin{npcDescription}
    Scriptit would never do anything directly. If a machine, a computer or script could do it automatically. Others will have to be a bit patient when Scriptit does something for the first time. But quite often it can be scaled up afterwards.

    \end{npcDescription}
    
    \begin{equipment}
        \item Light source (OLED film: battery operation, can be cut to size and glued on. Colour controllable)
        \item Mobile computers, headphones, communication via radio (mesh network)
        \item A suitcase sized robotics kit
    \end{equipment}
\end{npcBox}



% Matze: Norm Schauspieler/Darsteller, der aus der Sicherheit ausbrechen will und sein Leben mit mehr Würze versehen will. Die Leute verstehen das aber nicht. Vater blieb in der Norm Siedlung

% Benedikt: Lost Entdecker, Höhlen~-forscher, hat unten Müll gefunden von einer fremden Nation. Co Taucher zwingt ihn zum umdrehen (Sauer~-stoff~-mangel). 
% Ausrüstung reicht nicht zum bergen. Mutter ist mit ihm in die Wildnis. Ist Bruder von Matze

% Selma:Pioneer, 

% Chris: Fliegender chaotischer Pioneer
