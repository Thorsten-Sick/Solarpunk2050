% Ziele: Story hooks
% Mehr Drama als eine Telenovella
% Man muss zig Abenteuer aus der Konstellation ziehen können
% Die Meisten Charaktere sind ü 60 und haben viel erlebt. Das aufzählen.

\chapter{The Lost flea market}

There are many Lost flea markets. This one is mobile. Travelling though the country and creating a meeting point for the different Lost groups. But also some Norms and Pioneers get attracted to it and are left confused.

A radio message notifies the Lost that a flea market is happening in their area. An imporant event. But to organise these markets lots of effort has to be invested. By a very special kind ot people.

The people will be introduced here.

\section{If it succeeds....}

If the flea markets is successful this time it is a mixture of the smell of food, unplugged music, trade and barter, theater and the noise of smiths and animals. Different dialects mix. It is a giant low tech festival where different - normally hostile - groups meet.

Friendships are made. Plans planned and goods, stories and rumours are traded.

Torches and camp fires light it at night. Chatting, songs and stories until late at night. People drink, smoke and eat together.

After about 2 weeks all the 500 people are gone.

In those 2 weeks great things can happen:

\subsection{Random events}

\begin{enumerate}
    \item The old Jacob wants to share his famous bathtub-gin receipe before he dies. The first three people bringing him special alcoholic beverage will learn this secret.
    \item At night there is a spontaneous fire juggling show
    \item Someone is running a quiz 'The 1980se'. You can win a bottle of Jacob's bathtub-gin.
    \item Two people engaged. To celebrate this a whole pig is being BBQed. Norms and Pioneers could face a culture shock.
    \item There is a shakespear competition. Two groups battle it out but still need some actors for some roles.
    \item Adults and kids older than 14 can learn shooting a semi automatic at the shooting range. Which could confuse some Norms.
\end{enumerate}


\chapter{The task}

The flea market team is travelling around, announces a flea market on th radion and stars preparing the event for about 500 people. Most of those Lost will contribute to the flea market. By food, craft shops or similar.

The base infrastructure is offered by the flea market team. They also try to connect hostile Lost groups. Because this is one of the reasons to create a flea market (others are: trade, knowledge, entertainment).

Player characters could arrive when the flea market is being set up. They can help and find new friends. After 2 weeks everyone will move on.

\section{The problems}

Building a flea market is never simple. Expect problems. NPCs, the protagonists and maybe some early arrivals will solve them.


\subsection{Random events during set up}

\begin{enumerate}
\item A guest group arrives to late - what could have happened ?
\item A guest group arrives to early
\item The field where the flea market is planned is soaked after heavy rain. It is a lake. With stinging nettles and scrub.
\item A burried Lemming ruin is found where the flea market should be built. The Jones-es want to loot it. This can lead to a dungeon crawl.
\item Creation of the fle market is slowed. Tasty food is keeping everyone distracted.
\item Curious kids from the nearby Pioneer camp come to visit int heir exosceletons - Lost are caught in between scared and protective (because of the technology)
\end{enumerate}

\section{Wendy and family}

The NPCs Wendy and her family have a telenovella drama level plot. If this is not the mood you want, remove it or modify it.

\chapter{The NPCs}

\section{Animal expert: Jonas Ohnesorg}

He organises shelter for the animals (from rabbits to cattle). Doing that he also cares for the health of the bartered animals and feed supply. Local local butchery is also his resort. When not organising the flea market he is trainer for birds of prey. But sometimes those catch rabbits - including those owned by the farmers.


\newpage
\begin{npcBox}[title=Jonas Ohnesorg]

    \begin{aspects}
    \item \aspect[High Concept]{He loves animals more than they love him}
    \item \aspect[Trouble]{Others are always tempted to handle his animals without care}
    \item \aspect[Relationship]{Loves Antigone}
    \item \aspect[Aspect]{Loves eating hot food - he needs kicks in his life}
    \end{aspects}

    \begin{skills}
    \item \nskill{Academics}{0}
    \item \nskill{Athletics}{1}
    \item \nskill{Burglary}{0}
    \item \nskill{Contacts}{1}
    \item \nskill{Crafts}{3}
    \item \nskill{Deceive}{0}
    \item \nskill{Drive}{0}
    \item \nskill{Empathy}{1}
    \item \nskill{Fight}{3}
    \item \nskill{Investigate}{0}
    \item \nskill{Lore}{0}
    \item \nskill{Notice}{1}
    \item \nskill{Physique}{2}
    \item \nskill{Provoke}{0}
    \item \nskill{Rapport}{2}
    \item \nskill{Resources}{0}
    \item \nskill{Shoot}{0}
    \item \nskill{Stealth}{0}
    \item \nskill{Will}{2}
    \item \nskill{Bushcraft}{4}
    \end{skills}

    \begin{stunts}
    \item \stunt{Animal whisperer}{Gets +2 on Bushcraft when teaching and commanding animals.}
    \end{stunts}

    \begin{stressSection}
    \stressLine{\stress{1}\stress{1}\stress{1}\stress{1}\stress{1}\stress{1}}{\stress{1}\stress{1}\stress{1}\stress{1}}
    \end{stressSection}
    \begin{tabularx}{\textwidth}{ XX }
    \end{tabularx}

    \begin{consequences}
    \item \consequence{2}
    \item \consequence{4}
    \item \consequence{6}
    \end{consequences}

    \begin{npcDescription}
    TODO
    \end{npcDescription}


    \begin{equipment}
    \item His eagle "Igel" (Hedgehog in German). He still thinks this is funny.
    \item Equipment for the different animals.
    \end{equipment}
\end{npcBox}


\subsection{Random events involving animals}

\begin{enumerate}
\item An animal escapes
\item An animals is ill. There is a risk of an outbreak - can someone find a vetinary ?
\item Animals give birth to their young
\item Norms or Pioneers are confused by the animals. They want to pet them, keep them or film - which turns into a disaster
\item The animals lack food. Someone must get a truck load of food
\item His birds of prey start hunting small animals
\end{enumerate}

\newpage
\section{Kitchen and military tactics: Gustav Müller}

Seine Kernthesen: Ordnung, Disziplin und die Saucen sind zentral

Lieblings-doofer Spruch: 'Kein Mampf, kein Kampf'


Zwischen dem Führen einer Küche und einer Schlacht Organisation gibt es kaum Unterschiede'. Für ihn ist beides das Führen kleiner Einheiten. Seine Einstellung bringt ihn zur Verteidigung des Lost Lagers oder zur Großküche für einen Flohmarkt. Mit seiner herrisch militärischen Art kann er zwar Erfolge verbuchen. Doch er macht sich auch Feinde.

Info für SL: So sind Großküchen wirklich entstanden. Aus militärischen Strukturen: \href{https://de.wikipedia.org/wiki/K%C3%BCchenbrigade}{Küchenbrigade}


\newpage
\begin{npcBox}[title=Gustav Müller]

    \begin{aspects}
    \item \aspect[High Concept]{Perfectly organized kitchen general}
    \item \aspect[Trouble]{Hart to himself - 'till he drops}
    \item \aspect[Relationship]{He does not have any closer relationsships to other humans}
    \item \aspect[Aspect]{Perfection is in the details - and beautiful ornaments}
    \end{aspects}

    \begin{skills}
    \item \nskill{Academics}{0}
    \item \nskill{Athletics}{1}
    \item \nskill{Burglary}{0}
    \item \nskill{Contacts}{0}
    \item \nskill{Crafts - Cooking}{3}
    \item \nskill{Deceive}{0}
    \item \nskill{Drive}{1}
    \item \nskill{Empathy}{0}
    \item \nskill{Fight}{1}
    \item \nskill{Investigate}{0}
    \item \nskill{Lore}{2}
    \item \nskill{Notice}{1}
    \item \nskill{Physique}{2}
    \item \nskill{Provoke}{2}
    \item \nskill{Rapport}{0}
    \item \nskill{Resources}{0}
    \item \nskill{Shoot}{3}
    \item \nskill{Stealth}{0}
    \item \nskill{Will}{4}
    \end{skills}

    \begin{stunts}
    \item \stunt{Structured}{He gets +2 on Craft and shooting if there is a clear hierarchy. -2 if the org chart is chaotic.}
    \end{stunts}

    \begin{stressSection}
    \stressLine{\stress{1}\stress{1}\stress{1}\stress{1}\stress{1}\stress{1}}{\stress{1}\stress{1}\stress{1}\stress{1}}
    \end{stressSection}
    \begin{tabularx}{\textwidth}{ XX }
    \end{tabularx}

    \begin{consequences}
    \item \consequence{2}
    \item \consequence{4}
    \item \consequence{6}
    \end{consequences}

    \begin{npcDescription}
    TODO
    \end{npcDescription}


    \begin{equipment}
    \item Großkochgeräte. Ganze Anhänger umgebaut zu Küchenelementen
    \item Eine Desert Eagle
    \end{equipment}
\end{npcBox}


\subsection{Zufallsereignisse}

\begin{enumerate}
\item He has a truck with a herb garden. At the moment it is not close to the kitchen tents. Someone forgot that. Now someone has to drive it through the flea market to the kitchen.
\item Someone has to improvise a giant mixxer using a diesel generator and some gear
\item A long table is built through the whole fle market and everyone is invited to a banquet - if it works
\item Gustav leaves the kitchen to his sous chef for a few hours. He has to attend to combat tactics exercises in the nearby forest.
\item He is asked if he can offer shooting training for children and adults. He can, if he gets help with the kitchen and the shooting range that still needs to be built.
\item People are being rudely thrown out of the kitchen. There's trouble.
\end{enumerate}

\newpage

\section{Funk: Antigone Freitag}

She knows everyone on the radio waves – and speaks many languages. Her goal is to complete her collection of Lemmings figurines: to this end, she regularly checks in with her radio contacts and would even send out a rescue team to retrieve one of the missing figurines. Unfortunately, she cannot go herself, as she is confined to a wheelchair. But the whole world comes to her via radio.

\newpage
\begin{npcBox}[title=Antigone]

    \begin{aspects}
    \item \aspect[High Concept]{Radio operator and social encyclopaedia}
    \item \aspect[Trouble]{Knows almost everyone closely - but often has no idea what people look like.}
    \item \aspect[Relationship]{Because she knows everyone very well, she doesn't have anyone special.}
    \item \aspect[Aspect]{My trained little dog makes me more flexible and mobile.}
    \end{aspects}

    \begin{skills}
    \item \nskill{Academics}{3}
    \item \nskill{Athletics}{1}
    \item \nskill{Burglary}{0}
    \item \nskill{Contacts}{3}
    \item \nskill{Crafts - Radio}{2}
    \item \nskill{Deceive}{0}
    \item \nskill{Drive}{0}
    \item \nskill{Empathy}{2}
    \item \nskill{Fight}{0}
    \item \nskill{Investigate}{1}
    \item \nskill{Lore}{1}
    \item \nskill{Notice}{2}
    \item \nskill{Physique}{0}
    \item \nskill{Provoke}{0}
    \item \nskill{Rapport}{4}
    \item \nskill{Resources}{0}
    \item \nskill{Shoot}{0}
    \item \nskill{Stealth}{0}
    \item \nskill{Will}{1}
    \end{skills}

    \begin{stunts}
    \item \stunt{Seeing with my ears}{She is so accustomed to radio contact that she gets a +2 to empathy with her eyes closed, just by listening.}
    \end{stunts}

    \begin{stressSection}
    \stressLine{\stress{1}\stress{1}\stress{1}\stress{1}\stress{1}\stress{1}}{\stress{1}\stress{1}\stress{1}\stress{1}}
    \end{stressSection}
    \begin{tabularx}{\textwidth}{ XX }
    \end{tabularx}

    \begin{consequences}
    \item \consequence{2}
    \item \consequence{4}
    \item \consequence{6}
    \end{consequences}

    \begin{npcDescription}
    TODO
    \end{npcDescription}


    \begin{equipment}
    \item Her little mixed-breed dog “Snack”, who can fetch things on command.
    \item Radio equipment
    \item Soldering station
    \end{equipment}
\end{npcBox}

\subsection{Zufallsereignisse}

\begin{enumerate}
\item She found someone on the radio who has collectible figures. And that person is at the flea market. Since there is no radio contact, the protagonists have to find the person based on the description.
\item An arriving troop is in danger and the PCs must guide them here.
\item Radio repeaters must be built and distributed to groups that will be leaving soon.
\item The radio fails. An amplifier and electronics are quickly needed from the flea market.
\item She wants to travel through the flea market too. But her wheelchair is impractical in the mud. She needs help. On the trip, she will see many of the people she usually only talks to for the first time.
\item She wants to broadcast a flea market radio programme. Receivers need to be distributed and content created.
\end{enumerate}

\newpage

\section{Gaia Priester}

Laura, Gaianist. She is an inexperienced priestess in a religion that is just establishing itself. As she lives in isolation, she is very grateful for any spiritual input (books, conversations). She often tries to explain Gaia to others, but quickly stumbles when things get more complex.

\newpage
\begin{npcBox}[title=Laura, Gaianistin]

    \begin{aspects}
    \item \aspect[High Concept]{Inexperienced Gaianist}
    \item \aspect[Trouble]{The shoes are too big.}
    \item \aspect[Relationship]{She just joned the flea market - and does not know many people there}
    \item \aspect[Aspect]{Just found the way, highly enthusiastic and still stumbling}
    \end{aspects}

    \begin{skills}
    \item \nskill{Academics}{3}
    \item \nskill{Athletics}{1}
    \item \nskill{Burglary}{0}
    \item \nskill{Contacts}{1}
    \item \nskill{Crafts}{0}
    \item \nskill{Deceive}{0}
    \item \nskill{Drive}{0}
    \item \nskill{Empathy}{3}
    \item \nskill{Fight}{0}
    \item \nskill{Investigate}{2}
    \item \nskill{Lore}{2}
    \item \nskill{Notice}{1}
    \item \nskill{Physique}{0}
    \item \nskill{Provoke}{0}
    \item \nskill{Rapport}{4}
    \item \nskill{Resources}{1}
    \item \nskill{Shoot}{0}
    \item \nskill{Stealth}{0}
    \item \nskill{Will}{2}
    \end{skills}

    \begin{stunts}
    \item \stunt{Mandala}{With the right materials and time, she can lay an inspiring mandala that gives attentive observers +2 to Willpower. Even with care, the mandala lasts a maximum of 2 days.}
    \end{stunts}

    \begin{stressSection}
    \stressLine{\stress{1}\stress{1}\stress{1}\stress{1}\stress{1}\stress{1}}{\stress{1}\stress{1}\stress{1}\stress{1}}
    \end{stressSection}
    \begin{tabularx}{\textwidth}{ XX }
    \end{tabularx}

    \begin{consequences}
    \item \consequence{2}
    \item \consequence{4}
    \item \consequence{6}
    \end{consequences}

    \begin{npcDescription}
    TODO
    \end{npcDescription}


    \begin{equipment}
    \item Tea herbs - fresh and dried ones
    \item A tea tent
    \end{equipment}
\end{npcBox}


Gaia priests have dedicated themselves to the living Earth – Gaia. To the unity of the eco/techno and sociosphere. They particularly try to encourage other people to build bridges. Bridges between hostile groups, for example, and to implement joint projects.

\newpage


\section{Heavy machines: Wendy}

Chef Logistikerin, Heavy machines und Truckerin: Wendy (früher Wilhelm). Ihre Ehe mit Doris hat sich vor der Katastrophe zerrüttet. Sie wusste auch nicht warum. Später hat sie festgestellt, dass sie Trans und damit eine Frau ist. Sie hat ein Foto aus der Alten Zeit hinter der Sonnenblende. Er zusammen mit Doris, seiner Frau. Sie waren vor den Katastrophen mit dem Truck durch Europa unterwegs. Die Ehe scheiterte, weil Wilhelm/Wendy immer unterwegs war und Doris eher sesshaft. Aus der Ehe stammt ein Kind, 'Flash'. Dass Wendy mal Wilhelm war, wissen die Lost nicht.

\newpage
\begin{npcBox}[title=Wendy]

    \begin{aspects}
    \item \aspect[High Concept]{Große Maschinen machen mich glücklich}
    \item \aspect[Trouble]{Auf der Straße zuhause}
    \item \aspect[Relationship]{Gescheiterte Ehe mit Doris, weil Doris zu sesshaft ist.}
    \item \aspect[Aspect]{Tägliches Training ist nötig für meine Psyche}
    \end{aspects}

    \begin{skills}
    \item \nskill{Academics}{0}
    \item \nskill{Athletics}{1}
    \item \nskill{Burglary}{0}
    \item \nskill{Contacts}{0}
    \item \nskill{Crafts}{3}
    \item \nskill{Deceive}{0}
    \item \nskill{Drive}{4}
    \item \nskill{Empathy}{2}
    \item \nskill{Fight}{0}
    \item \nskill{Investigate}{0}
    \item \nskill{Lore}{0}
    \item \nskill{Notice}{2}
    \item \nskill{Physique}{3}
    \item \nskill{Provoke}{1}
    \item \nskill{Rapport}{1}
    \item \nskill{Resources}{0}
    \item \nskill{Shoot}{0}
    \item \nskill{Stealth}{0}
    \item \nskill{Will}{2}
    \item \nskill{Bushcraft}{1}
    \end{skills}

    \begin{stunts}
    \item \stunt{Schwergewicht}{Wendy bekommt +2 auf Fahren, wenn das Fahrzeug mehr als 5 Tonnen wiegt.}
    \end{stunts}

    \begin{stressSection}
    \stressLine{\stress{1}\stress{1}\stress{1}\stress{1}\stress{1}\stress{1}}{\stress{1}\stress{1}\stress{1}\stress{1}}
    \end{stressSection}
    \begin{tabularx}{\textwidth}{ XX }
    \end{tabularx}

    \begin{consequences}
    \item \consequence{2}
    \item \consequence{4}
    \item \consequence{6}
    \end{consequences}

    \begin{npcDescription}
    TODO
    \end{npcDescription}


    \begin{equipment}
    \item Zugriff auf Bagger, Lastwagen und Stapler der Flohmarkt Organisation
    \item Ein Satz Gewichte fürs Training
    \item Schminkset
    \end{equipment}
\end{npcBox}
\newpage

\section{Norm Dokumentatorin Doris}

Sie ist auf Abenteuer Trip (ohne Sanitäre Einrichtung und ohne Kunstfleisch, Drohnen oder Vernetzung): War früher mit Wilhelm verheiratet. Trägt immer noch den Ring in ihrer Tasche, weiss aber nicht, dass es Wendy ist. Kommen sie wieder zusammen ?
Den Abenteuer Trip unternimmt sie, weil sie mit Freunden gewettet hat. Die Freunde wissen, dass ihr Ex dort irgendwo unter den Lost ist und hoffen, die kommen wieder zusammen. Denn ihnen hat Doris oft von der gemeinsamen Zeit erzählt.

\newpage
\begin{npcBox}[title=Doris]

    \begin{aspects}
    \item \aspect[High Concept]{Stadtverwöhnt und auf Abenteuer}
    \item \aspect[Trouble]{Auf der Suche nach verlorener Vergangenheit}
    \item \aspect[Relationship]{Ihre Freund wissen besser als sie, wer ihr fehlt}
    \item \aspect[Aspect]{Sesshaft - mit starken Verbindungen}
    \end{aspects}

    \begin{skills}
    \item \nskill{Academics}{3}
    \item \nskill{Athletics}{0}
    \item \nskill{Burglary}{0}
    \item \nskill{Contacts}{1}
    \item \nskill{Crafts}{0}
    \item \nskill{Deceive}{0}
    \item \nskill{Drive}{0}
    \item \nskill{Empathy}{3}
    \item \nskill{Fight}{0}
    \item \nskill{Investigate}{1}
    \item \nskill{Lore}{0}
    \item \nskill{Notice}{2}
    \item \nskill{Physique}{0}
    \item \nskill{Provoke}{0}
    \item \nskill{Rapport}{4}
    \item \nskill{Resources}{2}
    \item \nskill{Shoot}{0}
    \item \nskill{Stealth}{1}
    \item \nskill{Will}{1}
    \item \nskill{Hive Kontrolle}{2}
    \end{skills}

    \begin{stunts}
    \item \stunt{Den Wald sehen}{In fremder Umgebung ist sie besonders aufmerksam. Wenn sie Lost oder Pioneer Eigenheiten beobachtet, bekommt sie +2 auf Wahrnehmung.}
    \end{stunts}

    \begin{stressSection}
    \stressLine{\stress{1}\stress{1}\stress{1}\stress{1}\stress{1}\stress{1}}{\stress{1}\stress{1}\stress{1}\stress{1}}
    \end{stressSection}
    \begin{tabularx}{\textwidth}{ XX }
    \end{tabularx}

    \begin{consequences}
    \item \consequence{2}
    \item \consequence{4}
    \item \consequence{6}
    \end{consequences}

    \begin{npcDescription}
    TODO
    \end{npcDescription}


    \begin{equipment}
    \item Ihren Hive Controller
    \item Einen Hive Funk Repeater, der nur manchmal funktioniert. Kaputt ? Sollte er tun kann sie Dinge aus dem Hive per Drohne bestellen
    \end{equipment}
\end{npcBox}
\newpage

\section{Pioneer Tochter 'Flash'}

Die Tochter von Doris und Wilhelm/Wendy. Ca. 35 Jahre alt. Sie ist sehr aktiv und unstet. Flash ist mit ihrer Mutter leicht zerstritten (wegen ihr kein Kontakt zum Vater, und die Mutter ist eine langweilige Norm). Weiss aber, dass der Vater Wendy heisst und bei den Lost ist, Leider hat sie keine Ahnung, wie sie sich ihm vorstellen soll. Für sie ist eine originelle Geschlechtsidentität absolut üblich in ihrem Umfeld.

\newpage
\begin{npcBox}[title=Flash]

    \begin{aspects}
    \item \aspect[High Concept]{Zukunfts gestaltende Genetik Expertin}
    \item \aspect[Trouble]{Zerstritten mit der Familie, die sie liebt}
    \item \aspect[Relationship]{Sie findet es normaler, eine Frau als Vater zu haben, als diese selbst}
    \item \aspect[Aspect]{E-Bike süchtig}
    \end{aspects}

    \begin{skills}
    \item \nskill{Academics}{3}
    \item \nskill{Athletics}{1}
    \item \nskill{Burglary}{0}
    \item \nskill{Contacts}{0}
    \item \nskill{Crafts}{3}
    \item \nskill{Deceive}{1}
    \item \nskill{Drive}{1}
    \item \nskill{Empathy}{0}
    \item \nskill{Fight}{0}
    \item \nskill{Investigate}{2}
    \item \nskill{Lore}{2}
    \item \nskill{Notice}{0}
    \item \nskill{Physique}{0}
    \item \nskill{Provoke}{0}
    \item \nskill{Rapport}{1}
    \item \nskill{Resources}{0}
    \item \nskill{Shoot}{0}
    \item \nskill{Stealth}{0}
    \item \nskill{Will}{2}
    \item \nskill{Prototyping}{4}
    \end{skills}

    \begin{stunts}
    \item \stunt{Kleinvieh}{Bevorzugt kleine und wendige Fahrzeuge. Bei allen Fahrzeugen, die leichter sind als sie, bekommt sie +2 auf Fahren.}
    \end{stunts}

    \begin{stressSection}
    \stressLine{\stress{1}\stress{1}\stress{1}\stress{1}\stress{1}\stress{1}}{\stress{1}\stress{1}\stress{1}\stress{1}}
    \end{stressSection}
    \begin{tabularx}{\textwidth}{ XX }
    \end{tabularx}

    \begin{consequences}
    \item \consequence{2}
    \item \consequence{4}
    \item \consequence{6}
    \end{consequences}

    \begin{npcDescription}
    TODO
    \end{npcDescription}


    \begin{equipment}
    \item Ein E-Bike dass jenseits aller Vernunft getuned ist
    \item Ein Gentech Kit, mit dem man Diagnosen machen kann und einfachste Eingriffe
    \end{equipment}
\end{npcBox}
\newpage

\chapter{Andere Gruppen}

Lost Flohmärkte wollen gezielt mehrere Lost Gruppen zusammen bringen. Doch oft findet man auch leicht verwirrte Pioneers und Norms.

\section{Jones-es}

So nennt sich eine Gruppe, die in den Ruinen der Lemminge nach Technologie, Büchern und anderen wertvollen oder interessante Dingen sucht. Sie bieten ihre letzten Funde an. Dafür benötigen sie Medikamente und Reperaturen.

\subsection{Zufallsfunde aus den Ruinen}

\begin{enumerate}
    \item Eine Kiste alter und gut erhaltener Weinflaschen
    \item Perry Rhodan Silberbände
    \item Private Video Kasetten aus den 1990er - was da wohl drauf ist ?
    \item Eine alte Landkarte dieser Region. Doch viele der Orte sind nach den Katastrophen nicht mehr vorhanden. Lohnt sich eine Expedition ?
    \item Ein Kinder-Aufsatz 'So stelle ich mir die Zukunft vor'. Was da wohl drin steht ? Was aus dem Kind geworden ist ?
    \item Eine Kiste voller alter Brettspiele
\end{enumerate}

Am wertvollsten sind die Artefakte aus der alten Welt, die die Lost bei Expeditionen geborgen haben. Die Artefakte der Lemminge. Insbesondere die Lost sind auf alte Technologie angewiesen, um ihre Fahrzeuge und Gebäude zu reparieren. Bücher selbst haben aber einen fast spirituellen Wert für die Lost und seltene Exemplare würden auf dem Flohmarkt heimlich gehandelt. Bis sie jemand um Ruhm zu ernten nach 'Alexandria' bringt, wo sie kopiert und vervielfältigt werden.

Lost handeln übrigens im Tauschhandel, Norms bezahlen mit digitalem Geld und bei Pioneers finden viele Transaktionen eher über Ruf und Ruhm der Personen statt. Auf einem Lost Flohmarkt sind also gerade für nicht-Lost Kulturschocks und chaotische Ketten an Tauschhandel vorprogrammiert.

\section{Farmer}

Suchen Tiere zur Zucht und Pflanzensamen. Sie sind bereit mit Naturalien zu zahlen. Spontane Schlachtungen für den Festbraten sind auch zu erwarten. Die vielen Tiere können ein Mist-Entsorgungs Problem verursachen und massiven Gestank, wenn man das nicht im Vorfeld einplant.


\section{Handwerker}

Sie bieten Reperaturen, benötigen aber Rohstoffe.

\section{Der Jahrmarkt}

Natürlich gibt es noch hunderte Lost aus andere Gruppen, die kochen, reparieren, verkaufen, kaufen, vorlesen, musizieren, suchen und einen Jahrmarkt betreiben mit aller Art von Buden und Attraktionen. Bei Kinder beliebt: Wetten auf eine Maus in einem Labyrinth: Welchen Ausgang wird sie nehmen ?


