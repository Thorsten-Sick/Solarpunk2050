%% This is a template for default adventures

\chapter{Pyramid}
\label{ch:adventure pyramid}

At the end of this story the players decide the mental, spiritual and physical fate of Yussuf, Farouk and tricked believers - or they will allow a dead Billionair to do that.

This adventure will add some Indiana Jones flavour to the game.

\section{Topic}

The pioneers from three different factions with different philosophies encounter an isolated group of people with their own religion who need saving. Those are split into several groups. The big questions are:

\begin{itemize}
    \item Who's body can you save ?
    \item Who's soul can be saved?
    \item Is the price to high to shatter someone's world view ?
\end{itemize}

\section{What really happened}

This story is happening close to Berlin, at Zehdenick/Gransee area. With minor adjustments you can place it somewhere else.

Maybe 20 or thirty years ago An unknown Billionair died in a car crash. His last will and testament stated that he wants a pyramid being built at the location of his death. Including a temple, a temple area, some believers and his own religion praising the "New Ra". Maybe the drugs he took while writing it were involved.

Of course a pyramid needs traps.

This all got transferred into a contract and everyone hired hat to follow it to the letter. Everyone agreed that the money had better been spent on helping people and not building pyramids. And no one liked the billionair (they just needed his money) - the people involved used every available freedom in the contract to sabotage the plan.

\begin{itemize}
    \item The high priest Farouk forgot the name of the billionair and removed it from the cartouche of all obelisks
    \item The engineers building traps were used to safety engineering. The traps are as safe as possible.
\end{itemize}

Farouk created the religion (as written in his contract) but did only put minimal effort into it. He has to lead the regular prayers. Or the people living close to the pyramid would not get food from the automated dispenser in the pyramid and starve.

As he is getting older he teaches Yussuf to be the priest after he dies. Yussuf is a true believer and he does not know the background story. He would not believe it even facing evidence.

The pyramid was found by the UN just now and they got confused. A pyramid close to Berlin ? So they are sending the protagonists to investigate.

At the same time the ceremony staff of Farouk ran out of power (despite him constantly recharging it at the central chamber in the pyramid). The Firmware got corrupted and he can not access the pyramid in a safe way any more ! The next cycle of rituals can nor proceed. The food will not be dispensed and everyone will starve.

The protagonists will have to enter the pyramid Indiana Jones style. With all the traps active. And recharge and re-flash the staff at the central chamber. Yussuf and Farouk will accompany them. And on this trip they will learn the origin of this pyramid.

At the end they will decide the fate of many people.

\section{Character introduction}

\subsection{Lost}

Lost characters will start their game in a camp with their extended families somewhere in the swamps around Berlin. They make a living hunting animals with their swamp boats and raiding ruins. Living a simple life. Close to nature and family. While eating German Gumbo and playing on their Banjos their radio operator receives a UN call asking for assistance and promising Resource points as payment. Everyone wants those points and a small delegation is sent to the meeting point in the Norm town Gransee (and the Lost never dared to go there before. The Norms are a strange group).
The protagonists will hop into their diesel powered old trucks and head for the city.

\subsection{Pioneers}

A long time ago there was a hacker camp with thousands of nerds. It was meant to last just one week. But the disasters in the rest of the world motivated many of them to just stay there and build a Pioneer settlement. Kids were born and raised between soldering irons and tech experiments. Not knowing the world outside. The disaster about 10 years ago with the gene edited chillies (a group of people wanted to process them to ice cream using liquid nitrogen. There was a combustion.) killed dozens and left a contaminated part of the camp inaccessible without protective gear. They still try to make it accessible. Resource points to pay for non-renewables would be required.

Pioneers have a construction set to build prototype Modular Individual Vehicles in a very short time. So they will arrive in a very strange electric vehicle in Gransee.

\subsection{Norms}

Norms live in fully automated cities called Hives. The Gransee Hive is quite small. The old buildings are still existing but turned into a kind of museum while the modern architecture was built around it. The Hive seed AI is large enough to coordinate all the people and their Hive controllers. But the range of the wifi is quite limited and will never reach to the pyramid !
The ceres providing lab grown food is also small scale. They have only one processing line producing fresh fruit pulp from stem cells. This is why there is currently the notification going around to eat more banana. Food is free anyway. But there are now still more than 50  5l banana pulp containers that need to be distributed before the production line can switch to the next fruit (raspberry). So this is an emergency !
Transportation is public transport (3 kinds of coaches: Those with a café, some with restaurants type three has playground and a small space for the parents) and individual vehicles have to be parked outside the city.
Everyone has a Hive controller to be always connected to everyone else and the AI. Being able to access all information and control everything. For visitors there are guest controllers. But those will start up in kid mode first and will require a tutorial before all features are available. And a medical exam before alcohol and coffee can be ordered.

The Norm players will be asked to host the UN meeting. It will take place in the Ceres. Please decorate a room (UN decoration is in a box stored at the attic). And pick up the guests. They will be confused. Distribute the Hive controllers and make sure the Lost and Pioneers do ot bump to much against the infrastructure.


\section{UN meeting}

Before leaving they might remember to order a Hive Repeater from the Library of Things. After the adventure they will have to bring it back - undamaged and unmodified. But the repeater would extend the range of the Hive up to the pyramid when placed between the Hive and the Pyramid. Ordering a pizza by flying drone will take about 1h to the Pyramid - but it can be done. In the pyramid there will be no Hive reception. The walls are thick.

\section{Finding the pyramid}

The biosphere reservation east of Gransee is without any trees. After the disasters it is muddy with some tree stumps peaking skywards. The pyramid is in a valley, surrounded by some "ancient egyptian" houses and a temple. Some fields grow rice (not enough). A canal for reed boats leads from the pyramid through the village. Automated food distribution is transported on this waterway (after each ceremony).

The pyramid is in the center and 1m higher than the Cheops pyramid - if anyone should be asking this question.

\section{Entering the area}

Arriving there they encounter a ancient egyptian slum. The simulated "Nil" idly flowing between the few fields and the houses. Normal people living there are shocked by the appearance of the visitors. If the high priest Faroul (80 years old) sees them he will try to guide them fast and sneaky into his temple and the attached living room. He wants to brief them before they start to talk to the people of Yussuf.

With the visitors he speaks normal and knows plenty about technology - having lived in Berlin before disasters struck. When doing rituals he will speak Swabian - the sacred language.

(He made this whole religion up lots of years ago. With his Swabian origin and a focus on hacking this religion together as fast as possible this was the best choice. Fulfilling the contract with minimal effort.)

Farouk needs help. The storage in the pyramid will deliver food after a ritual is celebrated. But his ceremonical staff is broken. He guesses the battery is broken (but fixing it will also identify a corrupted firmware). Without the staff he can not properly enter the pyramid. And in there there is the charging/flashing station. Close to the burial chamber.

\section{Into the pyramid}

The characters can sneak through the main entry into the pyramid. Oe be introduced by Farouk as pilgrims from far away paying respect toe the New Ra. No matter what: Farouk and Yussuf will enter the pyramid with them. Farouk has already been in the inner sanctum. But for Yussuf this is the enlightenment journey. Which will very likely end in confusion and a shattered world view. Making him the end boss.

Entering the pyramid Farouk will notice differences from his normal trips into it. Without the bluetooth handshake from his staff the direct paths are blocked. Instead the adventurers will learn that traps got activated - soon.

After crossing the large stone doors they will close automatically. Radio frequencies (to the Hive !) will be blocked by the thick stone walls.

A few more steps and someone will trigger the first trap.

\section{A warning}

A trap door at the ceiling suddenly opens and a skeleton falls out. Dangling in front of a shocked adventurer. The skeleton is made of plastics. With fake and real spider webs. A sign in it's hands reads "Traps activated. Turn around or die". The locked doors only leave option 2.

Investigating the small room in the ceiling it was stored in thes will find a small badge: "Ticket ID: 234786".

\section{The trip through the pyramid}

While sneaking through the labyrinth of the pyramid (Farouk: "There was a straight slope upwards ! Where did this labyrinth come from ?") they will encounter several traps. The engineers built them as safe as possible while maintaining the contract. The Ticket IDs on every track can later be linked to a project management software description.

Take as many traps as you want. Maybe skip some.

\subsection{Trap: rolling stones}

The adventurers will decent into a pit by ladder. This pit leads forward and up. This must be the way to the center of the pyramid (and it is). Directly to their right there is a strange system. There is a small room behind iron bars containing a giant boulder. And a broken trigger mechanism. If a protagonist should repair it it will turn out to be a lift for the boulder. And the boulder will be lifted upwards.
A minute later the bars will open. The lift is back. And rolling-crashing-dropping sounds can be heard coming closer. This is a marble track and the boulder will crush them. For safety reasons there are flashing warning signs on the ceiling. And alcoves every few meters of the track where the adventurers can evade the boulder.
When the boulder reaches the end of the track it will roll into the lift. The bars will close and the boulder is lifted back to the start.
The trap is simple to handle as soon as the protagonists understand it. But a trap with warning signs and escape routes should confuse them.

The badge in the lift reads "Ticket ID: 98798123"

By the way: Not repairing the broken lift makes this part of the trip a piece of cake

\subsection{Trap: arrow trap}

The floor for the next 10 meters is covered with large stone slabs. On each stone a letter of the alphabet is engraved. To cross it they will have to jump over the stones in a specific pattern. If they step on a wrong letter they will trigger the mechanism. Being struck be dozens of arrows - foam arrows Nerf style.

(The specifications did not go into details which arrows to take. So the engineers picked soft ones)

The trick to identify the right letters: All letters are written in Comic Sans. Except the letter X. Which is in a strange font. An expert in typography can help here. Jump from X to X. Or be hit by annoying arrows. On the other side there is a badge: "Ticket ID: 1254356"

\subsection{Hint: The New Ra}

A painting (egyptian style) at the wall shows a guy driving a sports car at a beach. With palm trees in the background and the sea. Long, blond, curly hair floating in the wind.

Farouk, as a matter of fact: This is the New Ra. While he was human
Yussuf, praying: In his vessel of flesh and blood. Like us mortals. Before he shed his hull and ascended

\subsection{Trap: trapdoor and spikes}

Suddenly a trap door opens under the protagonists. They will fall 10 cm. Maybe sprain an ankle. The drop stops at wooden spikes. Tennis balls are impaled on the wooden spikes. For safety reasons.

A badge reads: "Ticket ID: 987345"

\subsection{Hint: The rise of the New Ra}

A new picture at the wall (old egyptian style) shows the car of Ra crashing into a pine tree. A fireball and the lifeless body of Ra rising to the sky. His spirit soaring to the sun.

Yussuf (filled with devotion): This is where the New Ra sacrificed himself at a tree for us.

Farouk: It must have happened here somewhere. The pyramid was built at the place of the ...event.


\subsection{Trap: Fire Columns}

Walking around a corner the protagonists are facing flickering columns of fire shooting out of the floor up to the ceiling. They seem to be dancing to a silent music.
It might be possible to walk trough - if you know the pattern.

Someone with history skills might identify the silent song - playing it loud or singing it will make dancing through simple.

Maybe no one is surprised anymore that the song is "Never gonna give you up".

The Badge at the other side reads: "Ticket ID: 97867234"

There is also a red emergency button to switch if off. And a fire extinguisher. And a first aid kit.

\subsection{Entrance to the burial chamber}

The adventurers should now be quite confused by the strange traps playing nice. And soon they will learn the truth.

They are standing at the end of the corridor. In front of a majestic door. To the left there is a socket where the Staff of Ra can be placed for re-charging and re-flashing. Which Farouk will do immediately. The door does not open yet - as the process is not finished.
To the other side of the corridor is a tapestry. Which is the first time they encounter fabric. Pushing this aside there is a boring metal door behind. A sign reads "Control room".

\subsection{Control room}

The hidden control room was central for the final stage of building the pyramid (especially the traps). And was used to control everything burial the burial ceremony.

But it was also the home of nerdy tech folk. Bored. Disgusted by their tasks. Trying to create psychological vents.

There is a dart board with a photograph of a guy with curly blond hair as a target. The face is unrecognizable - to many holes.
An expensive coffee machine, a fridge and a stack of soft drinks.

The computers are still operational and can be switched on. The passwords are on sticky notes under the keyboard or at a whiteboard. No one cared - it seems.

Investigating the computer you will find lots of todos in the ticket system. All can be identified by the ticket numbers on the traps. Reading the discussion no one was ever motivated to build real traps. Just fulfil the contract and keep it safe - so no one gets harmed.
The protagonists can learn new swear words and insults from the entries. No one was interested to serve the "F*** Ra".

In a cardboard box in the corner you can find new Staffs of Ra. Still wrapped in plastic film. Just in case the update on the original staff fails.

You can also log in to a computer and open the burial chamber. Default music to play is set to "Also sprach Zarathustra". Of course.

Yussuf should already be close to snapping and seeing this room could trigger it.

\subsection{Burial chamber}

The center of the room is a golden sarcophagus. Opening the lid will reveal a well conserved mummy in astronaut gear. The face (destroyed by the crash) is covered with a golden mask. Looking a lot like the visor of a marvel superhero. There are golden sunbeams around his head. Looking like hair.

There are rows of chairs like for a presentation or speech. It seems in the old days this is how the Lemmings celebreated in the old days. A screen is rolled down for a powerpoint presentation. Which auto starts.
It shows the life of the New Ra. Lots of money (the older ones will recognize that). Yachts. Rockets. Private submarines. Pools. But somehow no one will learn more about the deceased.
The end of the powerpoint seems to be an acknowledgement: A man in a suite (with golden hair) lifting rockets and other men in suits to the stars.

If the protagonists turn around they will see lots of glass cabinets with merch from SF shows. Space ship models. Uniforms. A Bat'leth from Star Trek, daggers, phasers. Everything is fake. The bladed weapons are still dangerous. Older protagonists can recognize it. Maybe some Lost with their historical tendencies or Norms from long running series.

This is where Yussuf will snap. If no one intervenes. He will grab a bladed weapon and attack everyone. Trying to get the Staff of Ra. His world view just shattered.

Fighting is an option. But also using the SFX equipment from the tech room as a voice of god.

\section{Back to the Hive}

This adventure will end in the Ceres of the Gransee Hive. There is still left over banana. But also already the next fruit raspberry and ice cream. The UN got their post mortem and everyone can relax. Maybe some people like Farouk and Yussuf are joining on their first trip to a new civilisation.

\section{NPCs}


\begin{npcBox}[title=Farouk]

    \begin{aspects}
    \item \aspect[High Concept]{Aging priest of a made up religion}
    \item \aspect[Trouble]{I do not believe, but I act - for the people}
    \item \aspect[Aspect]{Yussuf is like my own son}
    \end{aspects}

    \begin{skills}
        \item \nskill{Academics}{4}
        \item \nskill{Athletics}{0}
        \item \nskill{Burglary}{0}
        \item \nskill{Contacts}{1}
        \item \nskill{Crafts}{1}
        \item \nskill{Deceive}{2}
        \item \nskill{Drive}{0}
        \item \nskill{Empathy}{2}
        \item \nskill{Fight}{0}
        \item \nskill{Investigate}{0}
        \item \nskill{Lore}{1}
        \item \nskill{Notice}{2}
        \item \nskill{Physique}{0}
        \item \nskill{Provoke}{0}
        \item \nskill{Rapport}{3}
        \item \nskill{Resources}{1}
        \item \nskill{Shoot}{0}
        \item \nskill{Stealth}{0}
        \item \nskill{Will}{3}
     \end{skills}

    \begin{stressSection}
    \stressLine{\stress{1}\stress{1}\stress{1}}{\stress{1}\stress{1}\stress{1}\stress{1}\stress{1}\stress{1}}
    \end{stressSection}
    \begin{tabularx}{\textwidth}{ XX }
    \end{tabularx}

    \begin{consequences}
    \item \consequence{2}
    \item \consequence{4}
    \item \consequence{6}
    \end{consequences}

    \begin{npcDescription}
    Farouk is 80 years old and the high priest of the New Ra religion. He also created it. Now he just wants everyone to be safe. For that the rituals have to continue. His plan (until the PC arrive) is to make Yussuf the new high priest. The name he was born with is "Martin Eisele", from the south German Swabian region.
    he is a nice guy and a mentor/father figure for Yussuf.

    \end{npcDescription}

\end{npcBox}



\begin{npcBox}[title=Leo]

    \begin{aspects}
    \item \aspect[High Concept]{True believer and almost ready to become the High Priest}
    \item \aspect[Trouble]{My world view is rock solid and will shattered when questioned}
    \item \aspect[Aspect]{Doubts that Farouk is a true believer}
    \end{aspects}

    \begin{skills}
        \item \nskill{Academics}{0}
        \item \nskill{Athletics}{4}
        \item \nskill{Burglary}{0}
        \item \nskill{Contacts}{0}
        \item \nskill{Crafts}{0}
        \item \nskill{Deceive}{0}
        \item \nskill{Drive}{0}
        \item \nskill{Empathy}{1}
        \item \nskill{Fight}{3}
        \item \nskill{Investigate}{0}
        \item \nskill{Lore}{1}
        \item \nskill{Notice}{1}
        \item \nskill{Physique}{2}
        \item \nskill{Provoke}{0}
        \item \nskill{Rapport}{2}
        \item \nskill{Resources}{0}
        \item \nskill{Shoot}{2}
        \item \nskill{Stealth}{1}
        \item \nskill{Will}{0}
     \end{skills}

    \begin{stressSection}
    \stressLine{\stress{1}\stress{1}\stress{1}\stress{1}\stress{1}\stress{1}}{\stress{1}\stress{1}\stress{1}}
    \end{stressSection}
    \begin{tabularx}{\textwidth}{ XX }
    \end{tabularx}

    \begin{consequences}
    \item \consequence{2}
    \item \consequence{4}
    \item \consequence{6}
    \end{consequences}

    \begin{npcDescription}
    Yussuf is a 20 year old believer in the New Ra. He grew up in the "Egyptian" village and does not know the outside world. For Farouk he is more than the apprentice. Farouk treats him like a son. Yussuf does not deal well with his world view being shattered.
    \end{npcDescription}

\end{npcBox}













\section{potential story hooks}

How to go on from this adventure ?

\section{Experience with player behaviour}

\begin{itemize}
    \item One group picked the evil path: They made the believers in the New Ra depending on constant deliveries from the Hive getting regular Resource Points from the UN. In a campaign I would switch those characters into NPC mode and make them the bad guys for the next few adventures.
    \item Another group integrated the believers into the Hive. After a celebratory procession with the sarcophagus to the Hive. A year later: The door bell rings "Do you want to speak about the New Ra?". Now there is a mummy cult in Northern Germany.
    \item One group skipped the boss fight. When Rocky realized Yussuf is going off the rails and managed a therapeutical bear hug. Preventing the escalation.
\end{itemize}


\section{Customizing this adventure}

\subsection{Location}
This story is happening close to Berlin, at Zehdenick/Gransee area. With minor adjustments you can place it somewhere else.

\subsection{Languages}
The sacred language is the Swabian dialect. If you can speak it: perfect. It is the dialect right from the other side of Germany - far away from Berlin. For people in Berlin it will sound rural. Maybe you can use some equivalent.

\subsection{Motivation}

* Potential incentives for the characters to enter this adventure


\section{Handouts}

\section{Helpers}

* NSC Liste (Name, Beschreibung, Beziehung, Rolle im Abenteuer)
* Beziehungsdiagramm (Intrigen Abenteuer)
* Flussdiagramm (lineare Abenteuer)
* Links auf spezielle Regeln
* Links auf spezielle Hintergrund Infos für die Welt

%% Chandler Regeln für Krimis

%%    Glaubwürdigkeit des Abenteuers
%%    Technische Perfektion (der Mordmethode, z.B.)
%%    Realismus
%%    Muss spielerischen Wert auch ohne das Geheimnis haben
%%    Fokus auf ein Thema/Stimmung/Schwerpunkt
%%    Keine zu frühe Aufklärung
%%    Einfache Struktur, klare Lösung
%%    Ehrlichkeit bei den Hinweisen
%%    Logische Lösung
%%    Bestrafung des Täters (Justiz, Selbstjustiz, …)

%%%%%%%%%%%  sidebar box with links
\begin{sidebarBox}[title=Relics]
\hyperref[sec:Relic]{Relics} are objects or organisations from the past. They do not fit into the new world and will soon be dismantled. The list of relics to be dismantled is long. And some just keep on going until it is their turn. Or they fight back.
\end{sidebarBox}

%%%%%%%%%%%% For the index at the end:

\index{Billionaires from outer space}

%%%%%%%%%%%%%   For links:

\label{ch:Label name}

%%%%%%%%%%%  NPC


Farouk: 80
Yussuf: 20


\begin{npcBox}[title=Leo]

    \begin{aspects}
    \item \aspect[High Concept]{An architect}
    \item \aspect[Trouble]{I will hulk out if you make me angry}
    \item \aspect[Aspect]{I dream of building big, but the situation keeps me small}
    \end{aspects}

    \begin{skills}
        \item \nskill{Academics}{3}
        \item \nskill{Athletics}{1}
        \item \nskill{Burglary}{0}
        \item \nskill{Contacts}{1}
        \item \nskill{Crafts}{4}
        \item \nskill{Deceive}{0}
        \item \nskill{Drive}{0}
        \item \nskill{Empathy}{1}
        \item \nskill{Fight}{2}
        \item \nskill{Investigate}{0}
        \item \nskill{Lore}{1}
        \item \nskill{Notice}{0}
        \item \nskill{Physique}{3}
        \item \nskill{Provoke}{0}
        \item \nskill{Rapport}{0}
        \item \nskill{Resources}{2}
        \item \nskill{Shoot}{0}
        \item \nskill{Stealth}{0}
        \item \nskill{Will}{2}
     \end{skills}

    \begin{stunts}
    \item \stunt{Improvising}{Leo gets a +2 to crafting when building architecture with alternative materials or scrap.}
    \end{stunts}



    \begin{stressSection}
    \stressLine{\stress{1}\stress{1}\stress{1}\stress{1}\stress{1}\stress{1}}{\stress{1}\stress{1}\stress{1}\stress{1}}
    \end{stressSection}
    \begin{tabularx}{\textwidth}{ XX }
    \end{tabularx}

    \begin{consequences}
    \item \consequence{2}
    \item \consequence{4}
    \item \consequence{6}
    \end{consequences}

    \begin{npcDescription}
    An impulsive architect, he has a spire with a house on it. He is trying to cultivate the side of the rock by attaching small boxes for plants.
    He was part of the 5 minute riot that Kim escaped from. Being impulsive, he threw rocks at the three Project Lifeguard people. The cyborg followed Kim. A rock thrown by Leo hit an engineer, throwing him off a bridge and killing him. Sam, the third engineer, was hiding in their bunker. An antenna was destroyed by rocks thrown.
    After the rebellion turned bloody, Leo retreated. Shocked by his own actions.
    If this can be fixed, they can gain the support of a skilled architect who only lacks the tools to build fantastic vertical gardens and bridges.

    Leo is 60 years old and remembers the time before the disaster. He moved to Albstadt 2 years before the disaster. Leo studied engineering and architecture before moving there.

    \textbf{Location:} His own spire. With a small house (repaired). And a small hanging garden at the side of the spire (just some flower pots, he did not have the right tools for proper hanging gardens).

    \textbf{Problem:} Has to deal with killing someone. Maybe talking to Sam will help.

    \textbf{When convinced:} Can help build things (a real hanging garden, real suspension bridges, crab traps)
    \end{npcDescription}

\end{npcBox}