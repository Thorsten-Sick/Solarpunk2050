\chapter{Settlers}
\label{ch:settlers}

This adventure is less detailed and is intended for GMs and players with knowledge of the Solarpunk 2050 world. It is more suitable for improvisation.

It is best played after \hyperref[ch:the name of the rose]{"The Name of the Rose"} in \hyperref[subsec:Ravensburg]{Ravensburg}. Perhaps the players have been given a hint to visit Überlingen, but cannot travel directly via Zepelin (\hyperref[sec:Jaime and Elisabeth]{Elisabeth}: "Last time we were in Überlingen there were some ... misunderstandings and ... complications. But we can take you to Salem").

\section{Topic}

This adventure around Salem has two different biomes. The burnt down town of Salem and the forests between Salem and Überlingen.
\subsection{Salem}

Since 2035, Salem has been a burnt-out ruin.
The rich billionaires who fled to the Moon used Salem as a training ground for their children and potential servants. When the training was over, they destroyed the evidence by blowing up the whole town.

\begin{itemize}
    \item The insane mindset of billionaires who see people as mere tools to achieve their goals.
    \item A respected institution that has gone bad
\end{itemize}

\subsection{Forest}

10 years ago they came from the remote area of the Deggenhauser Valley: Some Lost ignored the burnt down Salem and moved directly into the forest as settlers. They build a sprawling settlement (despite the monkeys) and start an economy. The player characters are asked to carry the various products to the others, thus playing the logistical role of the wanderers in the Settlers computer game.

\begin{itemize}
    \item Logistics for a small settlement
    \item Simple and without a big goal has its own advantages. It is somehow idyllic.
    \item Despite the threat of monkeys, everything here feels welcoming and positive.
    \item All the Eiseles are nice people and have plenty of good food for the visitors
\end{itemize}

\section{Summary}

In both biomes, wild monkeys that have escaped from the Affenberg Salem zoo are a threat. The wild monkeys must be fended off or fought. They are very fast and can climb. So be careful. A dead monkey nailed to a wall or post can be used to scare them away. This makes it necessary to hunt one and carry it with you at all times. ....

The city has been burnt to the ground after the Billionaires blew it up. But as they used extra explosives at the crime scenes - and blew the fuses when they tried to detonate them, preventing the explosion - the evidence can still be found. If you want to search near the piles of C4 and fertiliser. The two locations are the sports centre (where there is a centrifuge for astronaut training and a pool for zero-gravity EVA training). And the centre of the school, the old school/abbey, where you can find a lot of documents.

Leaving the town to the south, they will soon become part of a small economy and will meet many Lost Settlers, all with the family name "Eisele" (common in this area) - and they claim not to be related.
They will need some help distributing their goods and also their carrier pigeons.

The famous \hyperref[sec:Gene Dealer]{Gene Dealer} is currently active in this area.

\section{Getting started}

\section{Salem}

Salem was blown up and burned to the ground. This crime was committed to hide evidence. The ruins are still there. Outside of Salem, people were busy fighting other disasters, so after an initial scan, the official explanation was accepted as a "gas leak".

\subsection{Fritz Eisele}
The only resident is "Fritz Eisele", who may be the first of the Eiseles the characters meet. He is a fisherman with a small hut made of rubble near the lake in Salem.
He protects himself with a monkey nailed to the fence. As the monkey is rotting, he will ask the characters for a fresh one, paying with guns and smoked fish.
When questioned, he will tell more about the monkey threat and why he keeps a dead monkey around.

He can also tell you more about the other Eiseles in the forest and will need help distributing fish and his carrier pigeons to the other Eiseles (the birds are their way of communicating).

\subsection{Sports center}

The whole sports centre is filled with piles of C4 explosives and fertiliser. The only reason it didn't go off when the bad guys left: The fuse blew. I do not recommend fixing the electricity.

In the basement of the sports centre there is a pool with a space habitate structure in it to train EVA. Spacesuits can be found in the lockers.

In the upper area there is a large centrifuge for training rocket launches and the expected G-forces.

\subsection{School}

The school was once very famous, of incredibly high quality, and the boarding school, set in historic buildings, was impressive. Early in their plans, the Billionairs began a hostile takeover (starting in 2027). They replaced the management and soon everyone who did not fit into their evil plans. The old headmaster committed "suicide" soon after.
They secretly changed the whole curriculum to space. Teaching engineering, training in the centrifuge and EVAs in pools.
Evidence of this can be found on 200 year old wooden desks and in expensive leather books.
But first the protagonists must enter the old school building (the monkeys on the roof defend their territory by throwing faeces and dead birds at the visitors). And once inside the building, they are challenged with a high-speed fight in a dark 3D environment of the marble staircase. And unexploded C4 everywhere.

The whole school building is filled with piles of C4 explosives and fertiliser. The only reason it didn't go off when the bad guys left: The fuse blew. I do not recommend fixing the electricity.

You can also find the bodies of the staff who were shot in 2035 to hide evidence. The murderers expected to cover up the crime by blowing up the building.

If the investigation uncovers all the documents, the protagonists will learn that there were three groups of children in the astronaut training:

\begin{itemize}
    \item The rich kids. Their destiny: rules in space
    \item The average children whose bodies have tolerated the brain implant: Their fate is slavery in space.
    \item The average child without the implant: Killed in 2035 when everyone else went to the space centre for the rocket launch.
\end{itemize}

\section{Woods}

The forest lies between Salem in the north and Überlingen in the south. Plants have reclaimed the area.

\subsection{Restaurant "Goldener Anker"}

The Goldener Anker is a renovated and active restaurant on the destroyed road between Salem and Überlingen. It is run by "Herr und Frau Eisele", as they call themselves. Without using their first names. Despite the low technology, they have managed to make it idyllic and cosy. They both cook, garden and do repairs.
Around the restaurant there is enough of a small farm to feed them and any visitors:

\begin{itemize}
    \item Rabbit hutches
    \item Chicken coops
    \item Beehives
    \item Biodiverse gardens (8 different potato varieties, several apple varieties, herbs...)
    \item Distillery
    \item A bakehouse
    \item Simple and with no big goal has its own benefits. It is somehow idyllic
\end{itemize}

At the moment, they prefer books as payment. Shocking any Lost: They use the pages to fold small bags for the seeds they harvest.

They will pay with shelter, excellent food and Frau Eisele's famous scratchy knitted sweaters.

A guest in their restaurant is the \hyperref[sec:Gene Dealer]{Gene Dealer}

\subsection{The Goat Eisele}

About 1-2 hours on foot from the restaurant, the so-called goat Eisele has his shepherd's cart on a pasture. There he lives a happy life with his sheep and goats.
He is not intelligent, he is not educated and he cannot read. But he knows how to live out there. He knows all about sheep, weather, cheese, milk, herbs and his sheepdogs.
As he cannot read, his communication with the pigeons is limited to drawings.
Usually the Eiseles in the restaurant want meat and cheese (with or without herbs). And send bread instead. Transport is, of course, provided by the protagonists in exchange for food and shelter.

Several times a year he shears wool. He spins it into yarn, dyes it and sends it to the restaurant where Frau Eisele makes the famous scratchy knitted sweaters.

\subsection{Pia Eisele}

Pia Eisele lives on the other side of the restaurant. Also about 1-2 hours away. She has pitched several tents on the side of a hill. But she lives in a cave with her mushroom farm. She grows about a dozen different edible mushrooms in the wood and in the tunnels.
These tunnels were naturally cooled beer storage from around 1800.

In case you were wondering whether the Eiseles reproduce or are an inbreed:
Pia Eisele is a joyful woman and has 5 children. These children, of different ages and obviously from different fathers (all travellers who happened to visit Pia), look after themselves. They are very competent in the woods and can hunt their own monkeys with their bows.

They grow up happy, but certainly not according to any pedagogical standards. If you want to put a label on it, it is a kind of anti-authoritarian forest kindergarten. If your yardstick is the ability to survive: The education is a complete success.

Interacting with the children: They are curious and want to learn to write or read their names.

Everyone is wearing Frau Eisele's warm and itchy knitted jumpers.

If there are good-looking men in the group of protagonists: This is the chance to become a father without any responsibilities (playing this was a fun moment of ethical discussion in my group).

Pia is going to pay for the logistics or some education for the children in mushrooms. These are also the goods they want to transport to the restaurant for Frau Eisele's famous mushroom goulash.

Of course: She also lacks pigeons.


\section{Deggenhauser Tal}

Not part of the area or the adventure. But all the Eiseles come from there. So here is some background.

Between Ravensburg and Salem there is a valley called "Deggenhauser Tal". Even in 2020 this valley is remote. It is easy to get to, but all the main roads somehow avoid it and nobody visits it by chance. Some hills and rivers also make it a bit difficult to visit. Not impossible. But inconvenient. This has created a special mentality and made people like the Eiseles self-sufficient and independent.

\section{The monkeys}

\begin{npcBox}[title=Monkey]

    \begin{aspects}
    \item \aspect[High Concept]{Always angry and wild}
    \item \aspect[Trouble]{Popcorn is for tame monkeys or tames monkeys}
    \item \aspect[Aspect]{Taking back what is mine}
    \end{aspects}

    \begin{skills}
    \item \nskill{Academics}{0}
    \item \nskill{Athletics}{4}
    \item \nskill{Burglary}{1}
    \item \nskill{Contacts}{0}
    \item \nskill{Crafts}{0}
    \item \nskill{Deceive}{0}
    \item \nskill{Drive}{0}
    \item \nskill{Empathy}{0}
    \item \nskill{Fight}{3}
    \item \nskill{Investigate}{0}
    \item \nskill{Lore}{0}
    \item \nskill{Notice}{1}
    \item \nskill{Physique}{3}
    \item \nskill{Provoke}{1}
    \item \nskill{Rapport}{0}
    \item \nskill{Resources}{0}
    \item \nskill{Shoot (throwing stuff)}{2}
    \item \nskill{Stealth}{0}
    \item \nskill{Will}{2}
    \end{skills}

    \begin{stunts}
    \item \stunt{Double speed}{Monkeys are incredibly fast and can move across two zones. Even when climbing.}
    \end{stunts}

    \begin{stressSection}
    \stressLine{\stress{1}\stress{1}\stress{1}}{\stress{1}\stress{1}\stress{1}}
    \end{stressSection}
    \begin{tabularx}{\textwidth}{ XX }
    \end{tabularx}

    \begin{consequences}
    \item \consequence{2}
    \item \consequence{4}
    \item \consequence{6}
    \end{consequences}

    \begin{npcDescription}
    Just a bunch of wild monkeys. But their strength and speed are intimidating.
    \end{npcDescription}

\end{npcBox}